\section{Vektorska analiza}

\subsection{Skalarna in vektorska polja}

\begin{definicija}
Naj bo $D \subseteq \R^3$ odprta. Funkcijam oblike
$U \colon D \to \R$ pravimo
\emph{skalarno polje}\index{Polje}. Preslikavam
$\vv{R} \colon D \to \R^3$ oblike pravimo \emph{vektorsko polje}.
\end{definicija}

\begin{definicija}
\emph{Standardna baza}\index{Prostor!Standardna baza} je množica
\[
\set{\vv{i}, \vv{j}, \vv{k}} = \set{\vv{e_1}, \vv{e_2}, \vv{e_3}}.
\]
\end{definicija}

\begin{definicija}
Pravimo, da je baza $\set{\vv{p}, \vv{q}, \vv{r}}$
\emph{pozitivno orientirana}\index{Prostor!Orientacija baze}, če je
\[
[\vv{p}, \vv{q}, \vv{r}] > 0.
\]
Če je mešani produkt negativen, pravimo, da je baza
\emph{negativno orientirana}.
\end{definicija}

\begin{opomba}
Standardna baza je pozitivno orientirana.
\end{opomba}

\begin{opomba}
Baza je pozitivno orientirana natanko tedaj, ko je
\[
\vv{p} \times \vv{q} = \vv{r}.
\]
\end{opomba}

\datum{2022-3-2}

\begin{definicija}
\emph{Smerni odvod}\index{Polje!Smerni odvod} skalarnega polja
$U$ v smeri vektorja $\vv{s}$ v točki $p$ je limita
\[
\lim_{t \to 0}
\frac{U(\vv{p} + t \vv{s}) - U(\vv{p})}{t} =
\frac{\partial U}{\partial \vv{s}}(\vv{p}),
\]
če obstaja.
\end{definicija}

\begin{opomba}
Če je $U \in \mathcal{C}^1(D)$, velja
\[
\frac{\partial U}{\partial \vv{s}}(\vv{p}) =
(DU)(\vv{p}) \cdot \vv{s} =
\grad U \cdot \vv{s}.
\]
\end{opomba}

\begin{definicija}
Operator \emph{nabla}\index{Polje!Nabla} je operator
\[
\vv{\nabla} = \left(
\frac{\partial}{\partial x},
\frac{\partial}{\partial y},
\frac{\partial}{\partial z}
\right).
\]
\end{definicija}

\begin{trditev}
Naj bo $U \in \mathcal{C}^1(D)$. V točki $\vv{p} \in D$
skalarno polje najhitreje narašča v smeri gradienta, najhitreje pa
pada v nasprotni smeri.
\end{trditev}

\obvs

\begin{opomba}
V smereh, pravokotnih na gradient, se $U$ najpočasneje
spreminja.
\end{opomba}

\begin{definicija}
Naj bo $\vv{R}$ vektorsko polje.
\emph{Divergenca}\index{Polje!Divergenca} polja je sled odvoda,
oziroma
\[
\dv \vv{R} = X_x + Y_y + Z_z = \vv{\nabla} \cdot \vv{R}.
\]
\end{definicija}

\begin{definicija}
Naj bo $\vv{R}$ vektorsko polje. \emph{Rotor}\index{Polje!Rotor}
polja je produkt\footnote{Abuse of notation, razlike odvodov
komponent.}
\[
\rot \vv{R} = \vv{\nabla} \times \vv{R}.
\]
\end{definicija}

\begin{trditev}
Naj bo $D$ odprta podmnožica $\R^3$,
$U \in \mathcal{C}^2(D)$ skalarno in
$\vv{R} \in \mathcal{C}^2(D)$ vektorsko polje. Tedaj velja

\begin{enumerate}[i)]
\item $\rot(\grad U) =
\vv{\nabla} \times \vv{\nabla} U = \vv{0}$ in
\item $\dv(\rot \vv{R}) =
\vv{\nabla} \cdot \left(\vv{\nabla} \times \vv{R}\right) = 0$.
\end{enumerate}
\end{trditev}

\begin{proof}
Velja
\[
\vv{\nabla} \times \vv{\nabla} \cdot U =
(u_{zy} - u_{yz}, u_{xz} - u_{zx}, u_{yx} - u_{xy}) =
\vv{0}
\]
in
\[
\vv{\nabla} \cdot \left(\vv{\nabla} \times \vv{R}\right) =
Z_{yx} - Y_{zx} + X_{zy} - Z_{xy} + Y_{xz} - X_{yz} =
0. \qedhere
\]
\end{proof}

\begin{definicija}
Vektorsko polje je \emph{potencialno}\index{Polje!Potencialno}, če
obstaja tako skalarno polje $U \in \mathcal{C}^1(D)$, da
je $\vv{R} = \grad U$. Polju $U$ pravimo
\emph{potencial}.
\end{definicija}

\begin{definicija}
Naj bo $U$ skalarno polje.
\emph{Laplaceov operator}\index{Polje!Laplaceov operator} je
\[
\Delta U =
\dv \grad U =
U_{xx} + U_{yy} + U_{zz}.
\]
\end{definicija}

\begin{definicija}
Funkcijam, ki rešijo enačbo
\[
\Delta U = 0,
\]
pravimo \emph{harmonične funkcije}.
\end{definicija}

\begin{definicija}
Množica $D \subseteq \R^3$ je
\emph{konveksna}\index{Množica!Konveksna}, če za poljubni točki
$\vv{a}, \vv{b} \in D$ in $t \in [0,1]$ tudi
\[
t \vv{a} + (1-t) \vv{b} \in D.
\]
\end{definicija}

\begin{definicija}
Množica $D \subseteq \R^3$ je
\emph{zvezdasta}\index{Množica!Zvezdasta}, če obstaja taka točka
$\vv{a} \in D$, da je za vse $\vv{b} \in D$ in $t \in [0,1]$ tudi
\[
t \vv{a} + (1-t) \vv{b} \in D.
\]
\end{definicija}

\datum{2022-3-7}

\begin{trditev}
Naj bo $D \subseteq \R^3$ zvezdasto območje.\footnote{Povezana
odprta množica.} Naj bo $\vv{R} \in \mathcal{C}^1(D)$ vektorsko
polje.

\begin{enumerate}[i)]
\item Če je $\rot \vv{R} = 0$, je $\vv{R}$ potencialno.
\item Če je $\dv \vv{R} = 0$, obstaja tako vektorsko polje
$\vv{F} \in \mathcal{C}^2(D)$, da je $\vv{R} = \rot \vv{F}$.
\end{enumerate}
\end{trditev}

\begin{proof}
Označimo $\vv{R} = (X, Y, Z)$ in $D$ zvezdasto glede na točko
$(0, 0, 0)$.

\begin{enumerate}[i)]
\item Naj bo
\[
U(x,y,z) = \int_0^1 \left(
x \cdot X(tx, ty, tz) +
y \cdot Y(tx, ty, tz) +
z \cdot Z(tx, ty, tz)
\right)\;dt.
\]
Sledi, da je
\begin{align*}
U_x(x,y,z)
&=
\int_0^1 (X + tx \cdot X_x + ty \cdot Y_x + tz \cdot Z_x )\;dt
\\
&=
\int_0^1 (X + tx \cdot X_x + ty \cdot X_y + tz \cdot X_z)\;dt
\\
&=
\eval{t \cdot X(tx, ty, tz)}{0}{1}
\\
&= X.
\end{align*}
\item Označimo
\[
\alpha(x,y,z) = \int_0^1 t \cdot X(tx, ty, tz)\;dt.
\]
Simetrično definiramo še $\beta$ in $\gamma$. Opazimo, da velja
\[
\alpha_x + \beta_y + \gamma_z = 0,
\]
saj je $X_x + Y_y + Z_z = 0$. Sedaj naj bo
\[
\vv{F} =
(\alpha, \beta, \gamma) \times (x, y, z) =
(z \beta - y \gamma, x \gamma - z \alpha, y \alpha - x \beta).
\]
Sledi, da je prva komponenta $\rot \vv{F}$ enaka
\begin{align*}
\frac{\partial}{\partial y} (y \alpha - x \beta) -
\frac{\partial}{\partial z} (x \gamma - z \alpha)
&=
\alpha + y \alpha_y - x \beta_y - x \gamma_z + \alpha + z \alpha_z
\\
&=
2 \alpha + y \alpha_y + z \alpha_z - x(\beta_y + \gamma_z) =
\\
&=
\int_0^1 \left(2t X + t^2x X_x + t^2y X_y + t^2z X_z\right)\;dt
\\
&=
\eval{t^2 X(tx, ty, tz)}{0}{1}
\\
&= X(x, y, z). \qedhere
\end{align*}
\end{enumerate}
\end{proof}

\begin{opomba}
Potencial vektorskega polja je določen do konstante natančno.
\end{opomba}

\begin{proof}
Za $U = U - \mathcal{V}$ je množica
\[
A = \setb{(x,y,z) \in D}{U(x,y,z) = U(x_0,y_0,z_0)}
\]
odprta in zaprta.
\end{proof}

\begin{opomba}
Če je $\dv \vv{R} = 0$, so vse rešitve enačbe
$\vv{R} = \rot \vv{F}$ oblike $\vv{F} + \grad U$.
\end{opomba}

\begin{opomba}
Vsako vektorsko polje $\vv{R} \in \mathcal{C}^1(D)$ na zvezdastem
območju $D$ lahko zapišemo v obliki
\[
\vv{R} = \rot \vv{F} + \grad U.
\]
\end{opomba}

\newpage

\subsection{Orientacija krivulj in ploskev}

\begin{definicija}
Naj bo $\Gamma \subseteq \R^3$ gladka krivulja.
\emph{Orientacija}\index{Krivulja!Orientacija} krivulje $\Gamma$ je
zvezen izbor enotskega tangentnega vektorja vzdolž $\Gamma$.
\end{definicija}

\begin{opomba}
Če je $\Gamma$ povezana, ima natanko dve orientaciji.
\end{opomba}

\begin{definicija}
\emph{Odsekoma gladka krivulja}\index{Krivulja!Odsekoma gladka}
$\Gamma$ je vsaka končna unija gladkih krivulj, ki se ne sekajo,
razen v zaporednih robnih točkah.
\end{definicija}

\begin{definicija}
\emph{Orientacija} odsekoma gladke krivulje
\[
\Gamma = \bigcup_{i=1}^n \Gamma_i
\]
je tak izbor orientacij $\Gamma_i$, da so presečišča začetna točka
ene in končna točka druge krivulje.
\end{definicija}

\datum{2022-3-9}

\begin{definicija}
Naj bo $\Sigma \subseteq \R^3$ gladka ploskev.
\emph{Orientacija}\index{Ploskev!Orientacija} ploskve $\Sigma$ je
zvezen izbor enotske normale na $\Sigma$. Ploskvi z orientacijo
pravimo \emph{orientabilna}.
\end{definicija}

\begin{opomba}
Vsaka orientabilna povezana ploskev ima natanko dve orientaciji.
\end{opomba}

\begin{opomba}
Vsaka regularna parametrizacija $\vv{r} \colon D \to \R^3$ poda
orientacijo
\[
\vv{N} =
\frac{\vv{r_u} \times \vv{r_v}}{\norm{\vv{r_u} \times \vv{r_v}}}.
\]
\end{opomba}

\begin{definicija}
\emph{Odsekoma gladka ploskev}\index{Ploskev!Odsekoma gladka}
$\Sigma$ je vsaka končna unija gladkih omejenih ploskev z robom,
pri čemer je presek vsakih dveh prazen ali del robnih krivulj,
presek vsakih treh pa je prazen ali točka.
\end{definicija}

\begin{opomba}
Orientacija ploskve določa orientacijo roba
$\vv{T} = \vv{N} \times \vv{n}$, kjer je $\vv{n}$ normala na rob,
ki kaže izven ploskve.
\end{opomba}

\begin{definicija}
\emph{Orientacija} odsekoma gladke ploskve
\[
\Sigma = \bigcup_{i=1}^n \Sigma_i
\]
je tak izbor orientacij $\Sigma_i$, da so njihovi robovi
orientirani nasprotno.
\end{definicija}

\newpage

\subsection{Krivuljni integral}

\begin{definicija}
Naj bo $\Gamma$ gladka krivulja z regularno parametrizacijo
$\vv{r} \colon [\alpha, \beta] \to \Gamma$. Naj bo
$U \colon \Gamma \to \R$ zvezno skalarno polje.
\emph{Krivuljni integral skalarnega polja}\index{Krivulja!Integral}
je definiran kot
\[
\lint_\Gamma U\;ds =
\int_\alpha^\beta U(\vv{r}(t)) \abs{\dot{\vv{r}}(t)}\;dt.
\]
\end{definicija}

\begin{opomba}
Za odsekoma gladko krivuljo
\[
\Gamma = \bigcup_{i=1}^n \Gamma_i
\]
je integral definiran kot
\[
\lint_\Gamma U\;ds =
\sum_{i=1}^n \lint_{\Gamma_i} U\;du.
\]
\end{opomba}

\begin{definicija}
Naj bo $\vv{\Gamma}$ gladka orientirana krivulja z regularno
parametrizacijo $\vv{r} \colon [\alpha, \beta] \to \Gamma$, ki je
usklajena z orientacijo. Naj bo $\vv{R} \colon \Gamma \to \R^3$
zvezno skalarno polje. \emph{Krivuljni integral vektorskega polja}
je definiran kot
\[
\lint_{\vv{\Gamma}} \vv{R}\;d\vv{r} =
\int_\alpha^\beta \vv{R}(\vv{r}(t)) \cdot \dot{\vv{r}}(t)\;dt.
\]
\end{definicija}

\begin{opomba}
Integral je enak za vse parametrizacije, ki so usklajene z
orientacijo, saj je
\[
\lint_{\vv{\Gamma}} \vv{R}\;d\vv{r} =
\lint_\Gamma \left(\vv{R} \cdot \vv{T}\right)ds.
\]
\end{opomba}

\begin{opomba}
Pišemo tudi
\[
\lint_{\vv{\Gamma}} \vv{R}\;d\vv{r} =
\lint_\Gamma X\;dx + Y\;dy + Z\;dz.
\]
Izraz $X\;dx + Y\;dy + Z\;dz$ je \emph{diferencialna forma}.
\end{opomba}

\begin{trditev}
Naj bo $\vv{R} \colon D \to \R^3$, $\vv{R} = \grad U$ zvezno
potencialno vektorsko polje. Naj bo $\Gamma$ orientirana odsekoma
gladka krivulja v $D$ z začetno točko $A$ in končno točko $B$.
Tedaj je
\[
\lint_{\vv{\Gamma}} \vv{R}\;d\vv{r} =
U(B) - U(A).
\]
\end{trditev}

\begin{proof}
Velja
\[
\lint_{\vv{\Gamma}} \vv{R}\;d\vv{r} =
\int_\alpha^\beta (Du) \dot{\vv{r}}\;dt =
\int_\alpha^\beta \frac{d}{dt} \left(U(\vv{r}(t)\right)dt =
U(B) - U(A). \qedhere
\]
\end{proof}

\begin{izrek}
Naj bo $D \subseteq \R^3$ odprta in $\vv{R}$ zvezno vektorsko polje
na $D$. Naslednje izjave so ekvivalentne:

\begin{enumerate}[i)]
\item $\vv{R}$ je potencialno.
\item Integral $\vv{R}$ po odsekoma gladkih krivuljah v $D$ je
neodvisen od poti.
\item Integral $\vv{R}$ po vsaki sklenjeni odsekoma gladki krivulji
v $D$ je enak $0$.
\end{enumerate}
\end{izrek}

\begin{proof}
Implikaciji iz prve točke v drugi dve sta očitni.
\end{proof}

\begin{opomba}
Krivulja je sklenjena, če začetna in končna točka sovpadata. Tedaj
pišemo
\[
\lint_{\vv{\Gamma}} \vv{R}\;d\vv{r} =
\olint_{\vv{\Gamma}} \vv{R}\;d\vv{r}.
\]
\end{opomba}
