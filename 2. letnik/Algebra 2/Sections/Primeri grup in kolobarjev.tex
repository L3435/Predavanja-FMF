\section{Primeri grup in kolobarjev}

\subsection{Cela števila}

\begin{izrek}[Osnovni izrek o deljenju]
\index{Izrek!Osnovni izrek o deljenju}
Naj bo $m \in \Z$ in $n \in \N$. Potem obstajata taki enolični
števili $q$ in $r$, za kateri je
\[
m = qn + r \quad \text{in} \quad 0 \leq r < n.
\]
\end{izrek}

\begin{proof}
Naj bo
\[
S = \setb{k \in \Z}{kn \leq m}.
\]
$S$ je navzgor omejena, zato ima največji element $q$, ki ustreza
zgornjim pogojem.
\end{proof}

\begin{posledica}
Podmnožica $H$ aditivne grupe $\Z$ je podgrupa natanko tedaj, ko je
$H$ oblike $n \Z$ za $n \in \N_0$.
\end{posledica}

\begin{proof}
$n \Z$ je očitno grupa za vsak $n$, opazimo pa, da najmanjši
naravni element $H$ deli vse ostale.
\end{proof}

\begin{definicija}
$d \in \N$ je
\emph{največji skupni delitelj}\index{Cela števila!Največji skupni delitelj}
celih števil $m$ in $n$, če $d \mid n$, $d \mid m$ in vsak skupni
delitelj $m$ in $n$ deli tudi $d$. Označimo $d = \gcd(m,n)$.
\end{definicija}

\begin{trditev}
Naj bo $G$ aditivna grupa in $H, K \leq G$. Potem je tudi
\[
H + K = \setb{h + k}{h \in H \land k \in K}
\]
podgrupa $G$.
\end{trditev}

\obvs

\begin{posledica}
Za vse pare celih števil $m$ in $n$, ki nista obe $0$, obstaja
enoličen največji skupni delitelj, ki je oblike
\[
d = mx + ny
\]
za neka $x, y \in \Z$.
\end{posledica}

\begin{proof}
Grupa $n \Z + m \Z$ je grupa oblike $d \Z$.
\end{proof}

\begin{definicija}
Če je $\gcd(m,n) = 1$ pravimo, da sta si $m$ in $n$
\emph{tuji}\index{Cela števila!Tujost}.
\end{definicija}

\begin{lema}[Evklid]\index{Izrek!Evklidova lema}
Naj bo $p \in \mathbb{P}$ in $m, n \in \Z$. Potem velja
\[
p \mid m \cdot n \implies p \mid n \lor p \mid m.
\]
\end{lema}

\begin{proof}
Če $p \nmid m$, je $\gcd(p,m) = 1$, zato obstajata taka $x$ in $y$,
da je
\[
px + my = 1.
\]
Sledi, da je
\[
p \cdot \left(nx + \frac{mn}{p} \right) = n. \qedhere
\]
\end{proof}

\datum{2021-11-12}

\begin{izrek}[Osnovni aritmetike]\index{Izrek!Osnovni aritmetike}
Vsako naravno število lahko zapišemo kot produkt praštevil na
enoličen način do vrstnega reda natančno.
\end{izrek}

\begin{proof}
Obstoj faktorizacije dokažemo z indukcijo. Če ima $n$ dve različni
faktorizaciji, pa lahko pokrajšamo skupne faktorje in pridemo do
protislovja.
\end{proof}

\begin{izrek}[Evklid]
Praštevil je neskončno.
\end{izrek}

\begin{proof}
V nasprotnem primeru števila
\[
\prod_{p \in \mathbb{P}} p + 1
\]
ne moremo faktorizirati.
\end{proof}

\newpage

\subsection{Modularna aritmetika}

\begin{okvir}
\begin{definicija}
Pravimo, da sta celi števili $a$ in $b$
\emph{kongruentni po modulu $n$}\index{Kongruence}, če velja
$n \mid a-b$. Pišemo $a \equiv b \pmod{n}$.
\end{definicija}
\end{okvir}

\begin{trditev}
Kongruentnost je ekvivalenčna.
\end{trditev}

\obvs

\begin{definicija}
Z
\[
\Z_n = \setb{[x]}{x \in \Z}
\]
označimo množico ekvivalenčnih razredov relacije kongruentnosti
po modulu $n$.
\end{definicija}

\begin{trditev}
Če je $a \equiv b \pmod{n}$ in $c \equiv d \pmod{n}$, je tudi
$a + c \equiv b + d \pmod{n}$ in $ac \equiv bd \pmod{n}$.
\end{trditev}

\begin{proof}
Velja
\[
n \mid (a - b) + (c - d)
\quad \text{in} \quad
n \mid c \cdot (a - b) + b \cdot (c - d). \qedhere
\]
\end{proof}

\begin{definicija}
Na $\Z_n$ vpeljemo operaciji\footnote{Po prejšnji trditvi sta
operaciji dobro definirani.}
\[
+ \colon ([x],[y]) \mapsto [x+y]
\quad \text{in} \quad
\cdot \colon ([x],[y]) \mapsto [xy].
\]
\end{definicija}

\begin{opomba}
S tema operacijama je $\Z_n$ komutativen kolobar.
\end{opomba}

\begin{definicija}
Komutativnemu kolobarju brez deliteljev niča pravimo
\emph{cel kolobar}\index{Algebrska struktura!Kolobar!Cel}.
\end{definicija}

\begin{lema}
Končen cel kolobar je polje.
\end{lema}

\begin{proof}
Preslikava $x \mapsto ax$ za $a \ne 0$ je injektivna, zato je
surjektivna.
\end{proof}

\begin{posledica}
Za praštevila $p$ je $\Z_p$ polje.
\end{posledica}

\newpage

\subsection{Obseg kvaternionov}

\datum{2021-11-19}

\begin{okvir}
\begin{definicija}
S $\HH$ označimo štiri razsežen vektorski prostor z bazo
$\set{1,i,j,k}$. Z vpeljanim množenjem
\[
i^2=j^2=k^2=ijk=-1
\]
je $\HH$ realna algebra. Tej množici pravimo
\emph{kvaternioni}\index{Kvaternioni}.\footnote{Odkril jih je
W.\ R.\ Hamilton, od tod tudi oznaka $\HH$.}
\end{definicija}
\end{okvir}

\begin{trditev}
$\HH$ je obseg.
\end{trditev}

\begin{proof}
Vidimo, da je
\[
\frac{\overline{h}}{\abs{h}^2}
\]
inverz $h$.
\end{proof}

\begin{opomba}
$\HH$ lahko ekvivalentno definiramo kot $\R \times \R^3$ z vpeljanim
množenjem
\[
(\alpha,\vv{u}) \cdot (\beta, \vv{v}) =
(\alpha \beta - \skl{\vv{u},\vv{v}},
\alpha \vv{v} + \beta \vv{u} + \vv{u} \times \vv{v}).
\]
\end{opomba}

\begin{definicija}
Grupi
\[
Q = \set{\pm 1, \pm i, \pm j, \pm k}
\]
pravimo \emph{kvaternionska grupa}\index{Grupa!Kvaternionska}.
\end{definicija}

\newpage

\subsection{Matrični kolobarji in linearne grupe}

\begin{definicija}
Element $e$ kolobarja $K$ je
\emph{idempotent}\index{Kolobar!Idempotent, nilpotent}, če velja
$e^2 = e$.
\end{definicija}

\begin{definicija}
Element $a$ kolobarja $K$ je \emph{nilpotent}, če obstaja tak
$k \in \N$, da je $a^k = 0$.
\end{definicija}

\begin{definicija}
Grupi
\[
\GL_n(\F) = M_n^*(\F) =
\setb{A \in M_n(\F)}{\det A \ne 0}
\]
pravimo
\emph{splošna linearna grupa}\index{Grupa!Splošna linearna}.
\end{definicija}

\begin{definicija}
Definiramo še naslednje grupe:

\begin{description}[labelwidth=\widthof{\bfseries Posebna ortogonalna:}]
\item[Posebna linearna:]
$\operatorname{SL}_n(\F) = \setb{A \in \GL_n(\F)}{\det A = 1}$
\item[Ortogonalna:]
$\operatorname{O}_n(\F) = \setb{A \in \GL_n(\F)}{A^{-1} = A^\top}$
\item[Posebna ortogonalna:]
$\operatorname{SO}_n(\F) =
\setb{A \in \operatorname{O}_n(\F)}{\det A = 1}$
\item[Unitarna:]
$\operatorname{U}_n = \setb{A \in \GL_n(\C)}{A^{-1} = A^*}$
\item[Posebna unitarna:]
$\operatorname{SU}_n =
\setb{A \in \operatorname{U}_n}{\det A = 1}$
\end{description}
\end{definicija}

\newpage

\subsection{Kolobarji funkcij}

\begin{definicija}
Naj bo $X$ množica. Množica
\[
K = \setb{f}{f \colon X \to \R}
\]
s seštevanjem in množenjem po točkah postane kolobar. Z množenjem
s skalarjem postane tudi algebra nad $\R$.
\end{definicija}

\begin{definicija}
V primeru $X = \N$ s $c$ označimo podalgebro konvergentnih
zaporedij, z $\ell^\infty$ pa podalgebro omejenih zaporedij.
\end{definicija}

\newpage

\subsection{Diedrske grupe}

\datum{2021-11-26}

\begin{definicija}
\emph{Diedrska grupa}\index{Grupa!Diedrska} reda $2n$ je grupa
simetrij pravilnega $n$-kotnika. Generirana je z rotacijo $r$ za
kot $\frac{2\pi}{n}$ in zrcaljenjem $z$, za katera velja
\[
r^n = z^2 = (rz)^2 = 1.
\]
Označimo jo z
\[
D_{2n} = \setb{r^a z^b}{0 \leq a < n \land 0 \leq b \leq 1}.
\]
\end{definicija}

\newpage

\subsection{Kolobarji polinomov}

\begin{definicija}
Naj bo $K$ kolobar. \emph{Polinom}\index{Kolobar!Polinom} s
koeficienti iz $K$ je formalna vsota
\[
f(x) = \sum_{k=0}^n a_k X^k.
\]
Množica $K[X]$ vseh polinomov s koeficienti iz $K$ postane kolobar
s seštevanjem
\begin{align*}
\sum_{k=0}^n a_k X^k + \sum_{k=0}^m b_k X^k
&= \sum_{k=0}^{\max(n,m)} \left(a_k + b_k\right) X^k
\intertext{in množenjem}
\left(\sum_{k=0}^n a_k X^k\right) \cdot
\left(\sum_{k=0}^m b_k X^k\right)
&= \sum_{k=0}^{n + m} \left(\sum_{l=0}^k a_l b_{k-l}\right) X^k.
\end{align*}
\end{definicija}

\begin{opomba}
Kolobar polinomov lahko razširimo na kolobar formalnih potenčnih
vrst $K[[X]]$.
\end{opomba}

\begin{trditev}
Če $K$ nima deliteljev niča, za vse $f,g \in K[X]$ velja
\[
\deg(f \cdot g) = \deg f + \deg g.
\]
\end{trditev}

\begin{definicija}
Naj bo $K$ komutativen kolobar.
\emph{Vrednost polinoma}\index{Kolobar!Polinom!Vrednost} v točki
$x \in K$ je element
\[
f(x) = \sum_{k=0}^n a_k x^k.
\]
Ustrezna polinomska funkcija je preslikava $x \mapsto f(x)$. $x$ je
\emph{ničla}\index{Kolobar!Polinom!Ničla} ali \emph{koren}
polinoma, če velja $f(x) = 0$.
\end{definicija}

\begin{definicija}
Naj bo $K$ kolobar. \emph{Polinom več spremenljivk} je
\[
K[X_1,\dots,X_n] = (K[X_1,\dots,X_{n-1}])[X_n].
\]
Če je $K$ komutativen, definiramo vrednost v točki $x \in K^n$
podobno kot pri polinomih v eni spremenljivki.
\end{definicija}
