\section{Kombinatorika}

\subsection{Osnovna načela kombinatorike}

\datum{2022-2-14}

\begin{trditev}[Načelo produkta]
\index{Načelo produkta, vsote, enakosti}
Naj bodo $A_1, \dots, A_n$ končne množice. Tedaj je
\[
\abs{\prod_{i=1}^n A_i} = \prod_{i=1}^n \abs{A_i}.
\]
\end{trditev}

\begin{trditev}[Načelo vsote]
Naj bodo $A_1, \dots, A_n$ končne, paroma disjunktne množice. Tedaj
je
\[
\abs{\bigcup_{i=1}^n A_i} = \sum_{i=1}^n \abs{A_i}.
\]
\end{trditev}

\begin{trditev}[Načelo enakosti]
Če obstaja bijekcija med končnima množicama $A$ in $B$, je
\[
\abs{A} = \abs{B}.
\]
\end{trditev}

\begin{definicija}
Označimo
\[
[n] = \setb{i \in \N}{i \leq n}.
\]
\end{definicija}

\begin{trditev}
Za \emph{Eulerjev fi}\index{Eulerjev fi}
\[
\varphi(n) = \abs{\setb{i \in [n]}{(i,n) = 1}}
\]
velja rekurzivna formula
\[
\sum_{d \mid n} \varphi(d) = n.
\]
\end{trditev}

\begin{proof}
Na dva načina izračunamo moč množice
\[
\setb{\frac{i}{n}}{i \in [n]}. \qedhere
\]
\end{proof}

\begin{izrek}[Dirichletovo načelo]
\index{Izrek!Dirichletovo načelo}
Če je $n > m$, potem ne obstaja injektivna preslikava
$f \colon [n] \to [m]$.
\end{izrek}

\obvs

\begin{trditev}[Načelo dvojnega preštevanja]
\index{Načelo dvojnega preštevanja}
Če dva izraza predstavljata število elementov iste množice, sta
enaka.
\end{trditev}

\begin{definicija}
Definiramo padajočo in naraščajočo potenco
\[
k^{\underline{n}} = \prod_{i=0}^{n-1} (k - i)
\quad \text{in} \quad
k^{\overline{n}} = \prod_{i=0}^{n-1} (k + i).
\]
\end{definicija}

\begin{trditev}
Za množico $N$ z $n$ elementi in množico $K$ s $k$ elementov velja

\begin{enumerate}[i)]
\item $\abs{K^N} = k^n$
\item $\abs{\setb{f \colon N \to K}{\text{$f$ je injektivna}}} =
k^{\underline{n}}$
\item Število bijekcij med $N$ in $K$ je $n!$, če je $n = k$, sicer
pa $0$.
\end{enumerate}
\end{trditev}

\obvs

\begin{opomba}
Če imata končni množici enako moč, je bijektivnost preslikave med
njima ekvivalentna tako injektivnosti kot surjektivnosti.
\end{opomba}
