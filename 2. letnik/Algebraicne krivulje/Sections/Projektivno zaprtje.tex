\section{Projektivno zaprtje}

\subsection{Projektivna ravnina}

\datum{2022-3-3}

\begin{definicija}
Naj bo $K$ polje. \emph{Afina ravnina}\index{Afina ravnina} je
množica $A_2(K) = K^2$.
\end{definicija}

\begin{definicija}
Naj bo $K$ polje.
\emph{Projektivna ravnina}\index{Projektivna ravnina} je množica
vseh premic v $K^3$, ki potekajo skozi izhodišče. Označimo jo z
$P_2(K)$.
\end{definicija}

\begin{definicija}
\emph{Projektivne koordinate}\index{Projektivna ravnina!Koordinate}
projektivne točke je razmerje
\[
(x : y : z).
\]
\end{definicija}

\begin{opomba}
Vsakim projektivnim koordinatam, različnim od $[0 : 0 : 0]$, ustreza
natanko ena projektivna točka.
\end{opomba}

\begin{opomba}
Projektivno ravnino lahko identificiramo z afino ravnino, ki ji
dodamo \emph{točke v neskončnosti}. Točkam v projektivni ravnini,
ki so oblike $(x : y : 1)$, identificiramo s točko $(x, y)$ v afini
ravnini in jim pravimo \emph{končne točke}.

Točke $(x : y : 0)$ ustrezajo \emph{točkam v neskončnosti}, ki jih
identificiramo s snopi vzporednic.
\end{opomba}

\begin{opomba}
Projektivno ravnino $P_2(\R)$ lahko identificiramo tudi s sfero
$S^2$.
\end{opomba}

\begin{definicija}
\emph{Projektivna premica}\index{Projektivna ravnina!Premica} je
vsaka ravnina, ki gre skozi izhodišče. Identificiramo jo z afino
premico, ki ji dodamo pripadajočo točko v neskončnosti, oziroma
premico v neskončnosti.
\end{definicija}

\begin{opomba}
V sferičnem modelu so premice glavni krogi.
\end{opomba}

\begin{opomba}
Vsaki dve različni projektivni premici se sekata v natanko eni
projektivni točki. Skozi vsaki dve različni projektivni premici
poteka natanko ena projektivna premica.
\end{opomba}

\newpage

\subsection{Projektivne algebraične krivulje}

\begin{definicija}
Polinom $F \in \C[x,y,z]$ je \emph{homogen}\index{Polinom!Homogen},
če so vsi njegovi monomi iste stopnje.
\end{definicija}

\begin{opomba}
$F$ je homogen polinom stopnje $n$ natanko tedaj, ko za vse $x$,
$y$, $z$ in $\lambda$ velja
\[
F(\lambda x, \lambda y, \lambda z) = \lambda^n F(x, y, z).
\]
\end{opomba}

\begin{definicija}
Množica projektivnih ničel homogenega polinoma $F \in \C[x, y, z]$
je
\[
V_h(F) = \setb{(a : b : c) \in P_2(\C)}{F(a, b, c) = 0}.
\]
\end{definicija}

\begin{definicija}
Podmnožica $\mathcal{C} \subseteq P_2(\C)$ je
\emph{projektivna algebraična krivulja}\index{Algebraična krivulja!Projektivna},
če obstaja tak nekonstanten homogen polinom $F \in \C[x, y, z]$,
da velja
\[
\mathcal{C} = V_h(F).
\]
\end{definicija}

\begin{opomba}
Množico $V_n(F)$ si lahko predstavljamo kot unijo množice
$V(F(x,y,1))$ in nekaj točk v neskončnosti, ki jih dobimo iz
faktorizacije polinoma $F(x,y,0)$. Podobno lahko tudi afino
algebraično krivuljo predstavimo s projektivno krivuljo
polinoma\footnote{Takemu polinomu pravimo \emph{homogenizacija}
polinoma $f$.}
\[
F(x, y, z) =
z^{\deg f} \cdot f\left(\frac{x}{z}, \frac{y}{z} \right).
\]
\end{opomba}
