\section{Algebraične krivulje}

\subsection{Definicija}

\datum{2022-2-17}

\begin{definicija}
Polinom $P \in K[x_1, \dots, x_n]$ je
\emph{nerazcepen}\index{Polinom!Nerazcepen}, če se ga ne da zapisati
kot produkt dveh nekonstantnih polinomov iz $K[x_1, \dots, x_n]$.
\end{definicija}

\begin{definicija}
Za polinom $F \in K[x,y]$ označimo njegovo množico ničel
\[
V(F) = \setb{(a,b) \in K^2}{F(a,b) = 0}.
\]
\end{definicija}

\begin{opomba}
Množicam oblike $V(f)$ pravimo
\emph{(afine) algebraične množice}\index{Algebraična množica}.
\end{opomba}

\begin{okvir}
\begin{definicija}
Množica $\mathcal{C} \subseteq K^2$ je
\emph{algebraična krivulja}\index{Algebraična krivulja}, če obstaja
tak nekonstanten polinom $F \in K[x,y]$, da je
\[
\mathcal{C} = V(F).
\]
Pravimo, da je krivulja
\emph{nerazcepna}\index{Algebraična krivulja!Nerazcepna}, če je v
zgornji definiciji $F$ nerazcepen polinom.
\end{definicija}
\end{okvir}

\begin{definicija}
\emph{Afina preslikava}\index{Afina preslikava} je kompozitum
linearne preslikave in translacije. Če je ta linearna preslikava
obrnljiva, je tudi afina preslikava obrnljiva in ji pravimo
\emph{afina transformacija}.
\end{definicija}

\begin{trditev}
Kompozitum afinih transformacij je afina transformacija.
\end{trditev}

\begin{proof}
Afine transformacije so natanko preslikave
\[
(x,y) \mapsto (ax + by + \alpha, cx + dy + \beta),
\]
kjer je $ad \ne bc$.
\end{proof}

\begin{definicija}
Krivulji $\mathcal{C}$ in $\mathcal{D}$ sta
\emph{afino ekvivalentni}\index{Algebraična krivulja!Afino ekvivalentna},
če obstaja afina transformacija $\Phi$, za katero je
$\Phi(\mathcal{C}) = \mathcal{D}$.
\end{definicija}

\begin{opomba}
Afina ekvivalenca je ekvivalenčna relacija.
\end{opomba}

\newpage

\subsection{Studyjeva lema}

\datum{2022-2-24}

\begin{definicija}
\emph{Minimalni polinom}\index{Algebraična množica!Minimalni polinom}
algebraične množice $V(f)$ je produkt nerazcepnih faktorjev $f$.
\end{definicija}

\begin{definicija}
\emph{Stopnja}\index{Algebraična množica!Stopnja} algebraične
množice je stopnja njenega minimalnega polinoma.
\end{definicija}

\begin{definicija}
Naj bo $A$ komutativen kolobar in $f, g \in A[x]$. Označimo
\[
f = \sum_{i=0}^m a_i x^{m-i}
\quad \text{in} \quad
g = \sum_{i=0}^n b_i x^{n-i}.
\]
\emph{Rezultanto}\index{Polinom!Rezultanta} polinomov $f$ in $g$
definiramo kot
\[
\Res(f,g) =
\det \begin{bmatrix}
& \tikzmark{l1} a_0 & a_1 & \dots & a_m & & & \\
& & \ddots & \ddots & \ddots & \ddots & & \\
& & & a_0 & a_1 & \dots & a_m \tikzmark{r1} & \vspace{12pt} \\
&\tikzmark{l2} b_0 & b_1 & \dots & b_n & & & \\
& & \ddots & \ddots & \ddots & \ddots & & \\
& & & b_0 & b_1 & \dots & b_n \tikzmark{r2} &
\end{bmatrix}
\DrawBox[thick, blue,fill=yellow!20, fill opacity=0.2]{l1}{r1}{\textcolor{black}{\scriptsize $(n+m) \times n$}}
\DrawBox[thick, blue,fill=yellow!20, fill opacity=0.2]{l2}{r2}{\textcolor{black}{\scriptsize $(n+m) \times m$}}
\]
\end{definicija}

\begin{izrek}
Naj bo $A$ komutativen kolobar brez deliteljev niča z enolično
faktorizacijo. Za nekonstantna polinoma $f, g \in A[x]$ sta
naslednji trditvi ekvivalentni:

\begin{enumerate}[i)]
\item $\Res(f,g) = 0$
\item $f$ in $g$ imata skupen nekonstanten faktor.
\end{enumerate}
\end{izrek}

\begin{proof}
Dokazali bomo, da sta obe trditvi ekvivalentni temu, da obstajata
$\varphi, \psi \in A[x]$, ne oba enaka $0$, za katera velja
\[
\varphi f + \psi g = 0,
\quad
\deg \varphi < \deg g
\quad \text{in} \quad
\deg \psi < \deg f.
\]
Rezultanta je enaka nič natanko tedaj, ko so vrstice linearno
odvisne, od koder dobimo polinoma $\varphi$ in $\psi$. Zaradi
pogoja s stopnjami dobimo, da imata $f$ in $g$ skupen faktor.

Za obratno smer preprosto izberemo
\[
\varphi = \frac{g}{\gcd(f,g)}
\quad \text{in} \quad
\psi = -\frac{f}{\gcd(f,g)}. \qedhere
\]
\end{proof}

\begin{lema}[Study]\index{Lema!Study}
Naj bo $f \in \C[x,y]$ nerazcepen nekonstanten polinom. Tedaj za
vsak polinom $g \in \C[x,y]$ velja
\[
f \mid g \iff V(f) \subseteq V(g).
\]
\end{lema}

\begin{proof}
Naj bo
\[
f = \sum_{i=0}^m a_i x^{m-i}
\quad \text{in} \quad
g = \sum_{i=0}^n b_i x^{n-i},
\]
kjer so $a_i, b_i \in \C[y]$.\footnote{Ker je $\C$ komutativen,
velja $\C[x,y] = \C[y][x]$.} Brez škode za splošnost naj bo
$m \geq 1$. Ker je $a_0 \ne 0$, obstaja tak $y_0$, da je
$a_0(y_0) \ne 0$.

Oglejmo si polinom $f_{y_0}(x) = f(x,y_0)$. Ker je $\C$ algebraično
zaprto polje, ima ta polinom ničlo $x_0$. Sledi, da je
$f(x_0, y_0) = 0$, zato $(x_0, y_0) \in V(g)$, zato je tudi
\[
g_{y_0}(x_0) = 0.
\]
Sledi, da imata polinoma $f_{y_0}$ in $g_{y_0}$ skupni faktor
$x - x_0$ in je njuna rezultanta enaka $0$. Sledi, da je $y_0$
ničla rezultante $\Res(f,g)$. Ker to velja za skoraj vse $y_0$, je
$\Res(f,g) = 0$, oziroma, da imata $f$ in $g$ skupni faktor, to je
$f$.
\end{proof}

\begin{opomba}
Zgornja lema je znana tudi pod imenom \emph{Nullstellensatz}.
\end{opomba}

\begin{posledica}
Za vsak nekonstanten polinom $f \in \C[x,y]$ velja
$V(f) \ne \emptyset$.
\end{posledica}

\begin{proof}
Naj bo $h$ nerazcepen faktor $f$. Tedaj za vsak $g \in \C[x,y]$
velja $\emptyset = V(h) \subseteq V(g)$, zato $h \mid g$, kar je
protislovje.
\end{proof}

\begin{posledica}
Vsaka algebraična množica enolično določa nerazcepne faktorje
pripadajočega polinoma. Vsako algebraično množico lahko na enoličen
način zapišemo kot unijo nerazcepnih.
\end{posledica}

\begin{proof}
Naj bo
\[
f = c \cdot \prod_{i=1}^k f_i^{n_i}.
\]
Sledi, da je
\[
V(f) = \bigcup_{i=1}^k V(f_i).
\]
Če je $V(f) = V(g)$, od tod sledi, da $f_i \mid g$ za vse $i$.
Simetrično dobimo $g_i \mid f$.
\end{proof}
