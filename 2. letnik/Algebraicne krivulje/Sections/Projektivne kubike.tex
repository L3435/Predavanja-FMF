\section{Projektivne kubike}

\subsection{Nesingularne kubike}

\datum{2022-4-28}

\begin{trditev}
Vsaka nesingularna krivulja je nerazcepna.
\end{trditev}

\begin{proof}
Razcepne krivulje imajo dve komponenti, ki se sekata v singularni
točki.
\end{proof}

\begin{definicija}
\emph{Weierstrassova normalna forma}\index{Kubika!Weierstrassova forma}
je
\[
y^2z = x^3 + axz^2 + bz^3.
\]
\end{definicija}

\begin{izrek}
Vsaka nesingularna kuubika je projektivno ekvivalentna neki
Weierstrassovi normalni formi.
\end{izrek}

\begin{proof}
Vsaka nesingularna kubika ima vsaj en prevoj. Velja namreč, da je
\[
\mathcal{C} \cap H(\mathcal{C}) = \Flex(\mathcal{C}),
\]
presek $\mathcal{C} \cap H(\mathcal{C})$ pa je neprazen.

S projektivnostjo prevoj preslikamo v $(0 : 1 : 0)$, tangento v tej
točki pa v $z = 0$. Naj bo
\[
F(x,y,z) =
ax^3 + by^3 + cz^3 +
dx^2y + ey^2x + fy^2z +
gz^2y + hz^2x + ix^2z +
jxyz =
0
\]
enačba dobljene kubike. Očitno je $b = 0$. Ker je tangenta oblike
$z=0$, ima $F(x,y,0)$ trojno ničlo v točki $(0, 1)$. Velja pa
\[
F(x,y,0) = ax^3 + dx^2y + ey^2x,
\]
zato je tudi $d = e = 0$. Enačba kubike je tako
\[
F(x,y,z) =
ax^3 + cz^3 + fy^2z +
gz^2y + hz^2x + ix^2z +
jxyz =
0.
\]
Opazimo, da je $a \ne 0$, saj v nasprotnem primeru velja $z \mid F$
in je kubika sisngularna. Opazimo še, da je
$F_x(0,1,0) = F_y(0,1,0) = 0$. Ker je $(0 : 1 : 0)$ regularna
točka, je zato $F_z(0,1,0) \ne 0$, oziroma $f \ne 0$.

Naj bo
\[
y' = y + \frac{g}{2f} z + \frac{i}{2f} x.
\]
Sledi, da je
\[
fy^2z + gz^2y + jxyz =
fzy'^2 - \frac{f}{4} z \cdot
\left(\frac{g}{f} z + \frac{i}{f} x\right)^2.
\]
Sledi, da je krivulja projektivno ekvivalentna kubiki
\[
ax^3 + cz^3 + hz^2x + ix^2z + fzy^2 -
\frac{f}{4} z \cdot \left(\frac{g}{f} z + \frac{i}{f} x\right)^2,
\]
oziroma
\[
zy^2 = Ax^3 + Bx^2z + Cxz^2 + Dz^3,
\]
kjer je $A \ne 0$. Sedaj definiramo še
\[
x' = \sqrt[3]{A} * x + B/(3 \sqrt[3]{A}^2) z.
\]
Dobimo želeno obliko :)).
\end{proof}

\begin{trditev}
Weierstrassova normalna forma je singularna natanko tedaj, ko je
$4a^3 + 27b^2 = 0$.
\end{trditev}

\begin{proof}
Izračunamo parcialne odvode
\begin{align*}
F_x &= -3x^2 - az^2,
\\
F_y &= 2yz,
\\
F_z &= y^2 - 2axz - 3bz^2.
\end{align*}
Denimo, da so vsi enaki $0$. Če je $z = 0$, velja $x = y = 0$, kar
ni projektivna točka. Sledi, da je $y = 0$, torej
\[
3x^2 + az^2 = 2ax + 3bz = 0,
\]
saj je $z \ne 0$. Enačba $2ax = -3bz$ ima rešitev natanko tedaj, ko
jo ima enačba
\[
4a^2 x^2 = 9b^2 z^2.
\]
Dobimo sistem
\[
\begin{bmatrix}
4a^2 & -9b^2 \\
 3   &   a
\end{bmatrix}
\cdot
\begin{bmatrix}
x^2 \\
z^2
\end{bmatrix}
=
0,
\]
ki ima netrivialno rešitev natanko tedaj, ko je determinanta
zgornje matrike enaka $0$.
\end{proof}

\begin{opomba}
Polinom $x^3 + ax + b$ ima večkratno ničlo natanko tedaj, ko je
$4a^3 + 27b^2 = 0$.\footnote{Oboje je ekvivalentno
$\Res(f, f') = 0$.}
\end{opomba}

\datum{2022-5-5}

\begin{izrek}
Vsaka nesingularna kubika ima natanko 9 prevojev.
\end{izrek}

\begin{proof}
Vsaka nesingularna kubika je projektivno ekvivalentna krivulji
\[
y^2z = x^3 + axz^2 + bz^3,
\]
kjer je $4a^3 + 27b^2 \ne 0$. Množica prevojev je enaka množici
presečišč $\mathcal{C} \cap H(\mathcal{C})$. Polinom krivulje
$\mathcal{C}$ je enak
\[
F(x,y,z) = y^2z - x^3 - axz^2 - bz^3,
\]
Hessejeva matrika polinoma pa
\[
H_F = \begin{bmatrix}
-6x  & 0  &     -2az   \\
  0  & 2z &      2y    \\
-2az & 2y & -2ax - 6bz
\end{bmatrix}.
\]
Njena determinanta je enaka
\[
\det H_F = 8 \cdot (3ax^2 z + 9bxz^2 - a^2z^3 + 3xy^2).
\]
Denimo, da je $z = 0$. Sledi, da je tudi $x = 0$, zato dobimo prvi
prevoj $(0 : 1 : 0)$. V nasprotnem primeru lahko predpostavimo
$z = 1$.

Velja torej
\[
ax^2 + 3bx - \frac{1}{3} a^2 + x (x^3 + ax + b).
\]
Preverimo lahko, da je $y \ne 0$, zato nam vsaka rešitev za $x$ da
dve rešitvi za $y$. Dovolj je tako preveriti, da zgornji polinom
nima večkratnih ničel, oziroma, da je $\Res(f, f') \ne 0$.
Opazimo,\footnote{Just trust me lol} da je rezultanta enaka
\[
-\frac{(4a^3 + 27b^2)^2}{27}. \qedhere
\]
\end{proof}

\begin{izrek}
Vsaka premica, ki gre skozi dva prevoja nesingularne kubike, gre
skozi še natanko en prevoj.
\end{izrek}

\begin{proof}
Neki
\end{proof}

\datum{2022-5-12}

\begin{definicija}
\emph{Hessejeva normalna forma}\index{Kubika!Hessejeva forma}
je
\[
x^3 + y^3 + z^3 = 3kxyz.
\]
\end{definicija}

\begin{trditev}
Ta kubika je singuarna natanko tedaj, ko je $k = 0$ ali $k^3 = 1$.
\end{trditev}

\begin{proof}
Naj bo $F = x^3 + y^3 + z^3 - 3kxyz$. Ni težko dobiti, da za
singularne točke te krivulje velja
\[
(k^3 - 1) x^2 y^2 z^2 = 0.
\]
Če je $k^3 \ne 1$, hitro dobimo $x=y=z=0$, torej singularnih točk
ni. Sicer imamo singularno točko $(k : k : 1)$.
\end{proof}

\begin{trditev}
Prevoji nesingularne Hessejeve normalne forme so
\[
(0 : -\omega^m : 1)
\]
in ciklične permutacije koordinat.
\end{trditev}

\begin{proof}
Determinanta Hessejeve matrike Hessejeve normalne forme je enaka
\[
-54k^2 \cdot (x^3 + y^3 + z^3) + 54 \cdot (4 - k^3)xyz.
\]
Dobimo $xyz = 0$, od koder dobimo zgornje ničle.
\end{proof}

\begin{izrek}
Vsaka konfiguracija 9 točk je projektivno ekvivalentna standardni
konfiguraciji.
\end{izrek}

\begin{proof}
Najprej dokažimo projektivno ekvivalenco s konfiguracijo
\[
\begin{array}{ccc}
(-1 :    1   : 1) & ( \alpha :  1 : 1) & (1 :     1   : 1) \\
(-1 : \alpha : 1) & (    0   :  0 : 1) & (1 : -\alpha : 1) \\
(-1 :   -1   : 1) & (-\alpha : -1 : 1) & (1 :    -1   : 1).
\end{array}
\]
Po lemi o štirih točkah lahko izberemo štiri točke, ki se slikajo
v štiri želene točke. Sledi, da se tudi presečišče premic skozi te
točke slika kamor se mora. Izberemo še parameter $\alpha$, ki
določa, kam se preslika ena izmed preostalih štirih točk. Pri tem
dobimo, da je $\alpha^2 = -3$.
\end{proof}

\datum{2022-5-19}

\begin{izrek}
Vsaka nesingularna kubika je projektivno ekvivalentna Hessejevi
normalni fomri.
\end{izrek}

\begin{proof}
Denimo, da je standardna konfiguracija devetih točk množica
prevojev neke nesingularne kubike. Če v kubiko vstavimo $x = 0$ in
$z = 1$, dobimo
\[
by^3 + c + fy^2 + gy = 0.
\]
Ker so $-\omega^i$ ničle tega polinoma, je ta do skalarja natančno
enak $y^3 + 1$, zato je $b = c$ in $f = g = 0$. Podobno sklepamo za
ostale ciklične permutacije. Sledi, da je kubika oblike
\[
x^3 + y^3 + z^3 = 3kxyz,
\]
saj je nesingularna.
\end{proof}

\newpage

\subsection{Singularne kubike}

\begin{izrek}
Vsaka nerazcepna singularna kubika je projektivno ekvivalentna
krivulji
\[
x^3 + y^3 + 3xyz = 0
\quad \text{ali} \quad
x^3 = y^2z.
\]
\end{izrek}

\begin{proof}
Po izreku\footnote{Citation needed} vemo, da ima taka kubika
natanko eno singularno točko, v kateri ima kvečjemu dve tangenti.
Ločimo dva primera:

\begin{enumerate}[i)]
\item V singularni točki ima kubika dve tangenti. S projektivnostjo
jo lahko preslikamo tako, da sta ti tangenti ravno $x=0$ in $y=0$.
Z upoštevanjem pogoja singularnosti točke $(0 : 0 : 1)$ dobimo, da
je enačba kubike
\[
ax^3 + by^3 + dx^2y + exy^2 + fy^2z + ix^2z + jxyz.
\]
Tangento $y = 0$ parametriziramo z $(t : 0 : 1)$. Presečna
večkratnost te tangente je enaka $3$, zato velja
\[
t^3 \mid at^3 + it^2z,
\]
oziroma $i = 0$. Simetrično je $f = 0$. Enačba krivulje je tako
\[
ax^3 + by^3 + dx^2y + exy^2 + jxyz.
\]
Opazimo, da so $a$, $b$ in $j$ neničelni, saj je v nasprotnem
primeru krivulja razcepna. Predpostavimo lahko $a = b = 1$ in
$j = 1$, saj lahko posamične koordinate pomnožimo s skalarjem.
S preslikavo $z \mapsto z - \frac{d}{3} x - \frac{e}{3} y$ dobimo
obliko
\[
x^3 + y^3 + 3xyz = 0.
\]
\item V singularni točki ima kubika dvojno tangento. Znova lahko
predpostavimo, da je singularna točka $(0 : 0 : 1)$, tangenta pa
premica $y = 0$. Znova dobimo, da je enačba kubike
\[
ax^3 + by^3 + dx^2y + exy^2 + fy^2z + ix^2z + jxyz.
\]
Naj bo $(pt : qt : 1)$ tangenta. Velja, da je $t$ trojna ničla
polinoma
\[
fq^2t^2 + ip^2t^2 + jpqt^2.
\]
Ker je $y = 0$ edina ničla, je $i = j = 0$. Enačba krivulje je tako
\[
ax^3 + by^3 + dx^2y + exy^2 + fy^2z = 0.
\]
Opazimo, da sta $a$ in $f$ neničelna, zato lahko $x$ preslikamo
tako, da je $d = 0$, $z$ pa tako, da je $b = e = 0$. Z množenjem s
skalarjem lahko privzamemo še, da je $a = 1$ in $f = -1$. \qedhere
\end{enumerate}
\end{proof}

\begin{izrek}
Vsaka razcepna kubika, ki razpade na linearen in kvadratni faktor,
je projektivno ekvivalentna krivulji
\[
(y+z)(xy+yz+zx) = 0
\quad \text{ali} \quad
z(xy+yz+zx.
\]
\end{izrek}

\begin{proof}
Ločimo dva primera:

\begin{enumerate}[i)]
\item Premica je tangentna na stožnico. Naj bo $P$ njuno
presečišče, $Q$ in $R$ pa točki na stožnici. Brez škode za
splošnost naj bo $P = (1 : 0 : 0)$, $Q = (0 : 1 : 0)$ in
$R = (0 : 0 : 1)$. Naj bo enačba krivulje
\[
(ax^2 + by^2 + cz^2 + dxy + eyz + fzx)(gx + hy + iz).
\]
Z zgornjimi predpostavkami lahko najprej enačbo poenostavimo v
\[
(dxy + eyz + fzx)(hy + iz).
\]
Opazimo, da so $d$, $e$ in $f$ neničelna, saj bi drugače lahko
krivuljo faktorizirali dalje. Z množenjem s skalarjem lahko
predpostavimo, da je $d = e = f = 1$. Ni težko dobiti, da je
tangenta v $P$ premica $y + z = 0$.
\end{enumerate}
\end{proof}

\newpage

\subsection{Grupa kubike}

\datum{2022-5-26}

\begin{definicija}
Naj bo $\mathcal{C}$ kubika. Operacija $*$ na točkah $\mathcal{C}$
je definirana tako, da je $P * Q$ tretje presečišče premice $PQ$ s
kubiko $\mathcal{C}$. Pri tem primerno upoštevamo večkratnosti.
\end{definicija}

\begin{izrek}
Če se dve kubiki sekata v devetih točkah, gre vsaka kubika, ki gre
skozi 8 izmed teh točk, tudi skozi deveto.
\end{izrek}

\begin{trditev}
Za operacijo $*$ veljajo naslednje lastnosti:

\begin{enumerate}[i)]
\item Operacija je komutativna.
\item Za vsaki točki $P$ in $Q$ velja $(P * Q) * P = Q$.
\item Za vse točke $P$, $Q$, $R$ in $S$ velja
$((P * Q) * R) * S = P * ((Q * S) * R)$.
\end{enumerate}
\end{trditev}

\begin{definicija}
Naj bo $\mathcal{C}$ nerazcepna kubika, $O$ pa neka točka na
$\mathcal{C}$. Množica regularnih točk $\Reg \mathcal{C}$,
opremljena z operacijo
\[
P + Q = (P * Q) * O,
\]
je \emph{grupa kubike}\index{Kubika!Grupa}.
\end{definicija}

\begin{trditev}
Z zgornjo operacijo je $\Reg \mathcal{C}$ res grupa.
\end{trditev}

\obvs

\begin{trditev}
Vse grupe iste kubike so izomorfne.
\end{trditev}

\begin{proof}
Med grupama imamo preslikavo
$\varphi \colon \Reg \mathcal{C} \to \Reg \mathcal{C}$, podano s
predpisom
\[
\varphi(P) = (O * O') * P.
\]
Naj bo $A = O * O'$. Velja
\[
\varphi(P + Q) = A * ((P * Q) * O)
\]
in
\[
\varphi(P) +' \varphi(Q) = (A * P) +' (A * Q) =
((A * P) * (A * Q)) * O' =
A * ((P * O') * (A * Q)).
\]
Ker pa je
\[
(A * Q) * (P * O') =
((O * O') * Q) * (P * O') =
O * ((O' * (P * O')) * Q) =
O * (P * Q),
\]
je $\varphi$ homomorfizem. Ker je inverzen sam sebi, je
izomorfizem.
\end{proof}

\begin{trditev}
Če sta kubiki $\mathcal{C}$ in $\mathcal{C}'$ projektivno
ekvivalentni, sta njuni grupi izomorfni.
\end{trditev}

\begin{trditev}
Grupa kubike je izomorfna $(\C, +)$,
$(\C \setminus \set{0}, \cdot)$ ali $S^1 \times S^1$.
\end{trditev}
