\section{Tangente in singularnosti}

\subsection{Tangente}

\begin{definicija}
Naj bo $f$ polinom v treh spremenljivkah in $(a, b)$ njegova ničla.
\emph{Tangenta}\index{Algebraična krivulja!Tangenta} na $f$ v
$(a, b)$ je premica
\[
\frac{\partial f}{\partial x} (x - a) +
\frac{\partial f}{\partial y} (y - b) =
0,
\]
če nista oba odvoda enaka $0$.
\end{definicija}

\begin{definicija}
Točka $(a, b)$ na $f(x, y) = 0$ je
\emph{regularna}\index{Točka!Regularna}, če velja
\[
\frac{\partial f}{\partial x} (a, b) \ne 0
\quad \text{ali} \quad
\frac{\partial f}{\partial y} (a, b) \ne 0,
\]
sicer je \emph{singularna}\index{Točka!Singularna}.
\end{definicija}

\begin{definicija}
Naj bo $F$ nekonstanten homogen polinom v treh spremenljivkah in
$(a : b : c)$ njegova ničla.
\emph{Tangenta}\index{Algebraična krivulja!Tangenta} na $F$ v
$(a : b : c)$ je premica
\[
\frac{\partial F}{\partial x}(a,b,c) \cdot (x - a) +
\frac{\partial F}{\partial y}(a,b,c) \cdot (y - b) +
\frac{\partial F}{\partial z}(a,b,c) \cdot (z - c) =
0,
\]
če niso vsi odvodi enaki $0$.
\end{definicija}

\begin{izrek}[Eulerjeva identiteta]
\index{Izrek!Eulerjeva identiteta}
Če je $F$ homogen polinom v treh spremenljivkah stopnje $n$, je
\[
x \frac{\partial F}{\partial x} +
y \frac{\partial F}{\partial y} +
z \frac{\partial F}{\partial z} =
n F.
\]
\end{izrek}

\begin{proof}
Velja
\[
F(tx, ty, tz) = t^n F(x, y, z),
\]
zato je
\[
x \frac{\partial F}{\partial x}(tx, ty, tz) +
y \frac{\partial F}{\partial y}(tx, ty, tz) +
z \frac{\partial F}{\partial z}(tx, ty, tz) =
n \cdot t^{n-1} F(x, y, z).
\]
Sedaj preprosto vstavimo $t = 1$.
\end{proof}

\begin{posledica}
Enačba tangente se poenostavi v
\[
\frac{\partial F}{\partial x}(a,b,c) \cdot x +
\frac{\partial F}{\partial y}(a,b,c) \cdot y +
\frac{\partial F}{\partial z}(a,b,c) \cdot z =
0.
\]
\end{posledica}

\begin{definicija}
Naj bo $(a, b)$ singularna točka na $f(x, y) = 0$. \emph{Tangente}
v točki $(a, b)$ na krivuljo $f(x, y) = 0$ so tiste premice, za
katere je presečna večkratnost večja od minimalne. Minimumu
presečnih večkratnosti pravimo \emph{red}\index{Točka!Red} točke
$(a, b)$.
\end{definicija}
