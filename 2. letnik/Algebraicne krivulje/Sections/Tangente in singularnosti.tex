\section{Tangente in singularnosti}

\subsection{Tangente}

\begin{definicija}
Naj bo $f$ polinom v treh spremenljivkah in $(a, b)$ njegova ničla.
\emph{Tangenta}\index{Algebraična krivulja!Tangenta} na $f$ v
$(a, b)$ je premica
\[
\frac{\partial f}{\partial x} (x - a) +
\frac{\partial f}{\partial y} (y - b) =
0,
\]
če nista oba odvoda enaka $0$.
\end{definicija}

\begin{definicija}
Točka $(a, b)$ na $f(x, y) = 0$ je
\emph{regularna}\index{Točka!Regularna}, če velja
\[
\frac{\partial f}{\partial x} (a, b) \ne 0
\quad \text{ali} \quad
\frac{\partial f}{\partial y} (a, b) \ne 0,
\]
sicer je \emph{singularna}\index{Točka!Singularna}.
\end{definicija}

\begin{definicija}
Naj bo $F$ nekonstanten homogen polinom v treh spremenljivkah in
$(a : b : c)$ njegova ničla.
\emph{Tangenta}\index{Algebraična krivulja!Tangenta} na $F$ v
$(a : b : c)$ je premica
\[
\frac{\partial F}{\partial x}(a,b,c) \cdot (x - a) +
\frac{\partial F}{\partial y}(a,b,c) \cdot (y - b) +
\frac{\partial F}{\partial z}(a,b,c) \cdot (z - c) =
0,
\]
če niso vsi odvodi enaki $0$.
\end{definicija}

\begin{izrek}[Eulerjeva identiteta]
\index{Izrek!Eulerjeva identiteta}
Če je $F$ homogen polinom v treh spremenljivkah stopnje $n$, je
\[
x \frac{\partial F}{\partial x} +
y \frac{\partial F}{\partial y} +
z \frac{\partial F}{\partial z} =
n F.
\]
\end{izrek}

\begin{proof}
Velja
\[
F(tx, ty, tz) = t^n F(x, y, z),
\]
zato je
\[
x \frac{\partial F}{\partial x}(tx, ty, tz) +
y \frac{\partial F}{\partial y}(tx, ty, tz) +
z \frac{\partial F}{\partial z}(tx, ty, tz) =
n \cdot t^{n-1} F(x, y, z).
\]
Sedaj preprosto vstavimo $t = 1$.
\end{proof}

\begin{posledica}
Enačba tangente se poenostavi v
\[
\frac{\partial F}{\partial x}(a,b,c) \cdot x +
\frac{\partial F}{\partial y}(a,b,c) \cdot y +
\frac{\partial F}{\partial z}(a,b,c) \cdot z =
0.
\]
\end{posledica}

\begin{definicija}
Naj bo $p$ singularna točka na $f(x, y) = 0$. \emph{Tangente}
v točki $p$ na krivuljo $f(x, y) = 0$ so tiste premice, za
katere je presečna večkratnost večja od minimalne. Minimumu
presečnih večkratnosti pravimo \emph{red}\index{Točka!Red} točke
$p$ in ga označimo z $\ord_p(f)$.
\end{definicija}

\newpage

\subsection{Singularne točke}

\datum{2022-3-31}

\begin{trditev}
Če je $f$ minimalen polinom krivulje, velja
\[
\ord_p(f) = \min \setb{i + j}
{\frac{\partial^{i+j} f}{\partial x^i \partial y^j}(p) \ne 0}.
\]
\end{trditev}

\begin{proof}
Naj bo
\[
g(t) = f(x_0 + at, y_o + bt).
\]
Tedaj za največjo potenco $t^k$, ki deli $g$, velja
\[
g(0) = g'(0) = \dots = g^{(k-1)}(0) = 0
\quad \text{in} \quad
g^{(k)} \ne 0.
\]
Velja pa
\[
g^{(s)}(0) = \sum_{i + j = s} \binom{s}{i}
\frac{\partial^s f}{\partial x^i \partial y^j} (x_0, y_0) a^ib^j.
\qedhere
\]
\end{proof}

\begin{opomba}
Če je singularna točka izhodišče, je red enak najmanjši stopnji
monoma v $f$.
\end{opomba}

\begin{posledica}
Za tuja polinoma $f$ in $g$ velja
\[
\ord_p(fg) = \ord_p(f) + \ord_p(g).
\]
\end{posledica}

\obvs

\begin{posledica}
Če je $\ord_p(\mathcal{C}) = \deg \mathcal{C} - 1$, ima
$\mathcal{C}$ racionalno parametrizacijo.
\end{posledica}

\begin{proof}
Naj $f_i$ označuje vsoto monomov stopnje $i$. Tedaj je
\[
f(x, tx) =
f_{d-1}(x, tx) + f_d(d,tx) =
x^{d-1} f_{d-1}(1, t) + x^d f_d(1, t).
\]
Sledi, da je
\[
x = -\frac{f_{d-1}(1, t)}{f_d(1, t)}
\quad \text{in} \quad
y = -\frac{t f_{d-1}(1, t)}{f_d(1, t)}. \qedhere
\]
\end{proof}

\begin{posledica}
Naj bo $L$ premica, ki seka $\mathcal{C}$ v končno mnogo točkah.
Tedaj velja
\[
\sum_{p \in \mathcal{C} \cap L} \ord_p \leq \deg \mathcal{C}.
\]
\end{posledica}

\begin{proof}
Uporabimo Bezoutov izrek.
\end{proof}

\newpage

\subsection{Tangente v singularnih točkah}

\begin{trditev}
Naj bo $(0, 0)$ točka na algebraični krivulji z minimalnim
polinomom $f$. Tedaj je $(at, bt)$ tangenta natanko tedaj, ko je
$f_m(a,b) = 0$.
\end{trditev}

\obvs


