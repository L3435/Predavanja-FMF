\section{Tangente in singularnosti}

\subsection{Tangente}

\begin{definicija}
Naj bo $f$ polinom v treh spremenljivkah in $(a, b)$ njegova ničla.
\emph{Tangenta}\index{Algebraična krivulja!Tangenta} na $f$ v
$(a, b)$ je premica
\[
\frac{\partial f}{\partial x} (x - a) +
\frac{\partial f}{\partial y} (y - b) =
0,
\]
če nista oba odvoda enaka $0$.
\end{definicija}

\begin{definicija}
Točka $(a, b)$ na $f(x, y) = 0$ je
\emph{regularna}\index{Točka!Regularna}, če velja
\[
\frac{\partial f}{\partial x} (a, b) \ne 0
\quad \text{ali} \quad
\frac{\partial f}{\partial y} (a, b) \ne 0,
\]
sicer je \emph{singularna}\index{Točka!Singularna}.
\end{definicija}

\begin{definicija}
Naj bo $F$ nekonstanten homogen polinom v treh spremenljivkah in
$(a : b : c)$ njegova ničla.
\emph{Tangenta}\index{Algebraična krivulja!Tangenta} na $F$ v
$(a : b : c)$ je premica
\[
\frac{\partial F}{\partial x}(a,b,c) \cdot (x - a) +
\frac{\partial F}{\partial y}(a,b,c) \cdot (y - b) +
\frac{\partial F}{\partial z}(a,b,c) \cdot (z - c) =
0,
\]
če niso vsi odvodi enaki $0$.
\end{definicija}

\begin{izrek}[Eulerjeva identiteta]
\index{Izrek!Eulerjeva identiteta}
Če je $F$ homogen polinom v treh spremenljivkah stopnje $n$, je
\[
x \frac{\partial F}{\partial x} +
y \frac{\partial F}{\partial y} +
z \frac{\partial F}{\partial z} =
n F.
\]
\end{izrek}

\begin{proof}
Velja
\[
F(tx, ty, tz) = t^n F(x, y, z),
\]
zato je
\[
x \frac{\partial F}{\partial x}(tx, ty, tz) +
y \frac{\partial F}{\partial y}(tx, ty, tz) +
z \frac{\partial F}{\partial z}(tx, ty, tz) =
n \cdot t^{n-1} F(x, y, z).
\]
Sedaj preprosto vstavimo $t = 1$.
\end{proof}

\begin{posledica}
Enačba tangente se poenostavi v
\[
\frac{\partial F}{\partial x}(a,b,c) \cdot x +
\frac{\partial F}{\partial y}(a,b,c) \cdot y +
\frac{\partial F}{\partial z}(a,b,c) \cdot z =
0.
\]
\end{posledica}

\begin{definicija}
Naj bo $p$ singularna točka na $f(x, y) = 0$. \emph{Tangente}
v točki $p$ na krivuljo $f(x, y) = 0$ so tiste premice, za
katere je presečna večkratnost večja od minimalne. Minimumu
presečnih večkratnosti pravimo \emph{red}\index{Točka!Red} točke
$p$ in ga označimo z $\ord_p(f)$.
\end{definicija}

\newpage

\subsection{Singularne točke}

\datum{2022-3-31}

\begin{trditev}
Če je $f$ minimalen polinom krivulje, velja
\[
\ord_p(f) = \min \setb{i + j}
{\frac{\partial^{i+j} f}{\partial x^i \partial y^j}(p) \ne 0}.
\]
\end{trditev}

\begin{proof}
Naj bo
\[
g(t) = f(x_0 + at, y_o + bt).
\]
Tedaj za največjo potenco $t^k$, ki deli $g$, velja
\[
g(0) = g'(0) = \dots = g^{(k-1)}(0) = 0
\quad \text{in} \quad
g^{(k)} \ne 0.
\]
Velja pa
\[
g^{(s)}(0) = \sum_{i + j = s} \binom{s}{i}
\frac{\partial^s f}{\partial x^i \partial y^j} (x_0, y_0) a^ib^j.
\qedhere
\]
\end{proof}

\begin{opomba}
Če je singularna točka izhodišče, je red enak najmanjši stopnji
monoma v $f$.
\end{opomba}

\begin{posledica}
Za tuja polinoma $f$ in $g$ velja
\[
\ord_p(fg) = \ord_p(f) + \ord_p(g).
\]
\end{posledica}

\obvs

\begin{posledica}
Če je $\ord_p(\mathcal{C}) = \deg \mathcal{C} - 1$, ima
$\mathcal{C}$ racionalno parametrizacijo.
\end{posledica}

\begin{proof}
Naj $f_i$ označuje vsoto monomov stopnje $i$. Tedaj je
\[
f(x, tx) =
f_{d-1}(x, tx) + f_d(d,tx) =
x^{d-1} f_{d-1}(1, t) + x^d f_d(1, t).
\]
Sledi, da je
\[
x = -\frac{f_{d-1}(1, t)}{f_d(1, t)}
\quad \text{in} \quad
y = -\frac{t f_{d-1}(1, t)}{f_d(1, t)}. \qedhere
\]
\end{proof}

\begin{posledica}
Naj bo $L$ premica, ki seka $\mathcal{C}$ v končno mnogo točkah.
Tedaj velja
\[
\sum_{p \in \mathcal{C} \cap L} \ord_p \leq \deg \mathcal{C}.
\]
\end{posledica}

\begin{proof}
Uporabimo Bezoutov izrek.
\end{proof}

\newpage

\subsection{Tangente v singularnih točkah}

\begin{trditev}
Naj bo $(0, 0)$ točka na algebraični krivulji z minimalnim
polinomom $f$. Tedaj je $(at, bt)$ tangenta natanko tedaj, ko je
$f_m(a,b) = 0$.
\end{trditev}

\obvs

\datum{2022-4-7}
% neke singularnosti al neki idk
\newpage

\datum{2022-4-14}

\begin{lema}
Če krivulji $\mathcal{C}_1$ in $\mathcal{C}_2$ nimata skupne
komponente, velja
\[
\Sing(\mathcal{C}_1 \cup \mathcal{C}_2) =
\Sing(\mathcal{C}_1) \cup \Sing(\mathcal{C}_2) \cup
(\mathcal{C}_1 \cap \mathcal{C}_2).
\]
\end{lema}

\begin{proof}
Velja
\[
\frac{\partial (F_1 F_2)}{\partial x}(p) = 0 \iff
F_1(p) \cdot \frac{\partial F_2}{\partial x}(p) +
F_2(p) \cdot \frac{\partial F_1}{\partial x}(p) = 0.
\]
Če je $p \in \mathcal{C}_1$ singularna točka unije, je zato
$F_2(p) = 0$ ali pa je $p$ singularna točka $\mathcal{C}_1$, in
obratno.
\end{proof}

\begin{posledica}
Krivulja stopnje $n$ ima kvečjemu $\frac{n(n-1)}{2}$ singularnih
točk.
\end{posledica}

\begin{proof}
Posledico dokažemo z indukcijo po stopnji krivulje. Če je krivulja
nerazcepna, to sledi iz zgornjega izreka. Če je $\mathcal{C}$
razcepna, jo lahko zapišemo kot $\mathcal{C}_1 \cup \mathcal{C}_2$,
kjer je $\mathcal{C}_1$ nerazcepna. Naj bo
$n_1 = \deg \mathcal{C}_1$ in $n_2 = \deg \mathcal{C}_2$. Sledi, da
ima $\mathcal{C}$ kvečjemu
\[
\frac{(n_1-1)(n_1-2)}{2} + \frac{n_2 (n_2-1)}{2} + n_1 n_2 =
\frac{n_1^2 + 2n_1n_2 + n_2^2 - 3n_1 - n_2 + 2}{2} \leq
\frac{n (n-1)}{2}
\]
singularnih točk.
\end{proof}

\newpage

\subsection{Prevoji}

\begin{definicija}
Točka $p \in \mathcal{C}$ je \emph{prevoj}\index{Točka!Prevoj}, če
je $p$ regularna točka in obstaja tangenta $\ell$ v točki $p$, za
katero je
\[
\mult_p(\mathcal{C} \cap \ell) \geq 3.
\]
Množico vseh prevojev označimo z $\Flex(\mathcal{C})$.
\end{definicija}

\begin{opomba}
Če je ena izmed komponent krivulje premica $\ell$, so vse točke na
$\ell$ prevoji.
\end{opomba}

\begin{definicija}
Če je $\mult_p(\mathcal{C} \cap \ell) = \infty$, pravimo točki $p$
\emph{nepravi prevoj}, če pa je
$\mult_p(\mathcal{C} \cap \ell) = 3$, pa ji pravimo \emph{običajen
prevoj}.
\end{definicija}

\begin{definicija}
Naj bo $\mathcal{C} = V_h(F)$ in $H_F$ Hessejeva matrika polinoma
$F$. Krivulji $H(\mathcal{C}) = V_h(\det H_F)$ pravimo
\emph{Hessejeva krivulja}\index{Algebraična krivulja!Hessejeva}.
\end{definicija}

\begin{izrek}
Za vsako krivuljo $\mathcal{C}$ velja
\[
\mathcal{C} \cap H(\mathcal{C}) =
\Flex(\mathcal{C}) \cup \Sing(\mathcal{C}).
\]
\end{izrek}

\begin{proof}
Dovolj je pokazati, da je
$\Sing(\mathcal{C}) \subseteq H(\mathcal{C})$ in
$\Reg(\mathcal{C}) \cap H(\mathcal{C}) = \Flex(\mathcal{C})$.

\begin{lema*}
Velja
\[
z^2 \det H_F =
(n-1)^2 \det \begin{bmatrix}
F_{xx} & F_{xy} & F_x             \\
F_{xy} & F_{yy} & F_y             \\
F_x    & F_y    & \frac{n}{n-1} F
\end{bmatrix}.
\]
\end{lema*}

\begin{proof}
Velja
\[
z^2 \det H_F =
\det \begin{bmatrix}
F_{xx}   & F_{xy}   & z F_{xz}   \\
F_{xy}   & F_{yy}   & z F_{yz}   \\
z F_{xz} & z F_{zy} & z^2 F_{zz}
\end{bmatrix}.
\]
Ker pa je
\[
x (F_x)_x + y (F_x)_y + z (F_x)_z = (n-1) F_x,
\]
lahko s prištevanjem večkratnikov vrstic in stolpcev dobimo matriko
\[
\begin{bmatrix}
F_{xx}       & F_{xy} & (n-1) F_x      \\
F_{xy}       & F_{yy} & (n-1) F_y      \\
(n-1) F_x    & (n-1) F_y    & n(n-1) F
\end{bmatrix}. \qedhere
\]
\end{proof}

Naj bo $p$ singularna točka, za katero je brez škode za splošnost
$z \ne 0$. Po zgornji lemi je zato $(z^2 \det H_F)(p) = 0$. Sledi,
da je $\Sing(\mathcal{C}) \subseteq H(\mathcal{C})$.

Naj bo $p \in \Reg \mathcal{C}$. Brez škode za splošnost naj bo
$p = (0 : 0 : 1)$, njena tangenta $\ell$ pa naj ima enačbo $y = 0$.
Naj bo $f(x,y) = F(x,y,1)$. Očitno $f$ nima konstantnega člena,
linearni člen pa je oblike $\lambda y$.

Izračunajmo presečno večkratnost. Opazimo, da je
\[
F(x,0,z) = a_2 x^2 z^{n-2} + a_3 x^3 z^{n-3} + \dots.
\]
Točka $p$ je prevoj natanko tedaj, ko je zgornji polinom deljiv z
$x^3$. Velja
\begin{align*}
F(x,y,z) &= F(x,0,z) + F(x,y,z) - F(x,0,z)
\\
&=
x^2 G(x,z) + y \cdot H(x,y,z).
\end{align*}
Sedaj izračunamo parcialne odvode
\begin{align*}
F_x &= 2x G + x^2 G_x + y H_x,
\\
F_y &= H + y H_y,
\\
F_z &= x^2 G_z + y H_z.
\end{align*}
Ker v točki $p$ velja $F_x = F_z = 0$, točka $p$ pa je regularna,
sledi $H(0,0,1) \ne 0$. Hessejeva determinanta je enaka
\[
\det H_F = \frac{(n-1)^2}{z^2} \det \begin{bmatrix}
2G + 4x G_x + x^2 G_{xx} + y H_{xx} &
H_x + y H_{xy} &
2x G + x^2 G_x + y H_x
\\
H_x + y H_{xy} &
2H_y &
H + y H_y
\\
2x G + x^2 G_x + y H_x &
H + y H_y &
\frac{n}{n-1} F
\end{bmatrix}.
\]
Sledi, da je
\[
\det H_F (0 : 0 : 1) =
\det \begin{bmatrix}
2G  & H_x   & 0 \\
H_x & 2 H_y & H \\
0   & H     & 0
\end{bmatrix} =
-2 G(0,1) \cdot H(0,0,1)^2.
\]
Opazimo, da je determinanta enaka $0$ natanko tedaj, ko je
$G(0,1) = 0$, oziroma, ko $x^3 \mid F(x, 0, z)$.
\end{proof}

\begin{izrek}\label{iz:1}
Naj bo $\mathcal{C}'$ komponenta krivulje $\mathcal{C}$. Krivulja
$\mathcal{C}'$ je premica natanko tedaj, ko je
$\mathcal{C}' \subseteq H(\mathcal{C})$.
\end{izrek}

\begin{proof}
Če je $\mathcal{C}'$ premica, naj bo brez škode za splošnost njena
enačba $z=0$. Sledi, da je $F = z \cdot G$. Velja
\[
z^2 \det H_F =
(n-1)^2 \det \begin{bmatrix}
F_{xx} & F_{xy} & F_x             \\
F_{xy} & F_{yy} & F_y             \\
F_x    & F_y    & \frac{n}{n-1} F
\end{bmatrix}.
\]
Desna stran je deljiva z $z^3$, zato $z \mid H_F$.

\begin{lema*}
Naj bo $p$ regularna točka krivulje $\mathcal{C}$, $\ell$ pa
tangenta v tej točki. Tedaj je
\[
\mult_p(\mathcal{C} \cap \ell) =
\mult_p(H(\mathcal{C} \cap \ell) + 2.
\]
\end{lema*}

\begin{proof}
Brez škode za splošnost naj bo $p = (0 : 0 : 1)$, enačba $\ell$ pa
$y = 0$. Sledi, da za $k = \mult_p(\mathcal{C} \cap \ell)$ velja
$x^k \parallel F(x, 0, z)$, zato lahko zapišemo
\[
F(x,y,z) = x^k G(x,z) + y \cdot H(x,y,z),
\]
kjer je $G(0,1) \ne 0$ in $H(0,0,1) \ne 0$. To vstavimo v
\[
z^2 \det H_F =
(n-1)^2 \det \begin{bmatrix}
F_{xx} & F_{xy} & F_x             \\
F_{xy} & F_{yy} & F_y             \\
F_x    & F_y    & \frac{n}{n-1} F
\end{bmatrix}.
\]
Po približno dveh straneh dobimo, da je zgornja determinanta enaka
\[
x^{k-2} \varphi(x,z) + \psi(x,y,z),
\]
kjer je $\varphi(0,1) \ne 0$. Ker $p$ ne leži na premici $z=0$,
sledi, da je $\mult_p(H(\mathcal{C}) \cap \ell) = k-2$.
\end{proof}

Naj bo $p \in \mathcal{C}'$ točka, ki ne leži na nobeni drugi
komponenti. Če bi veljalo $\mathcal{C}' \subseteq H(\mathcal{C})$,
bi sledilo
\[
\mult_p(\mathcal{C}' \cap \ell) \leq
\mult_p(H(\mathcal{C}) \cap \ell),
\]
kar je seveda protislovje.
\end{proof}

\begin{trditev}
Naj bo $f$ število prevojev na krivulji $\mathcal{C}$ stopnje $n$
in $s$ število njenih singularnih točk. Če $\mathcal{C}$ nima
linearne komponente, velja
\[
f + 2s \leq 3n(n-2).
\]
\end{trditev}

\begin{proof}
Po izreku~\ref{iz:1} sledi, da sta $\mathcal{C}$ in
$H(\mathcal{C})$ tuji. Opazimo, da je
$\deg H(\mathcal{C}) = 3 \cdot (n-2)$, velja pa
\begin{align*}
\deg \mathcal{C} \cdot \deg H(\mathcal{C})
&=
\sum_{p \in \mathcal{C} \cap H(\mathcal{C})}
\mult_p(\mathcal{C} \cap H(\mathcal{C})
\\
&=
\sum_{p \in \Flex(\mathcal{C})}
\mult_p(\mathcal{C} \cap H(\mathcal{C}) +
\sum_{p \in \Sing(\mathcal{C})}
\mult_p(\mathcal{C} \cap H(\mathcal{C})
\\
&\geq
f + 2s. \qedhere
\end{align*}
\end{proof}
