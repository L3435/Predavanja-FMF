\section{Osnove verjetnosti}

\subsection{Izidi, dogodki, verjetnosti}

\datum{2022-10-4}

\begin{definicija}
\emph{Množico vseh možnih izidov}\index{Množica izidov} označujemo
z $\Omega$.
\end{definicija}

\begin{definicija}
\emph{Dogodek}\index{Dogodek} je vsaka podmnožica množice $\Omega$.
\end{definicija}

\begin{definicija}
Družina dogodkov $\mathcal{F}$ ima naslednje lastnosti:

\begin{enumerate}[i)]
\item velja $\Omega \in \mathcal{F}$ in
$\emptyset \in \mathcal{F}$,
\item če je $A \in \mathcal{F}$, je tudi
$A^\mathsf{c} \in \mathcal{F}$ in
\item če so $A_1, A_2, \ldots \in \mathcal{F}$, je tudi
\[
\bigcup_{i=1}^\infty A_i \in \mathcal{F}.
\]
\end{enumerate}

Družini množic z zgornjimi lastnostmi pravimo
\emph{$\sigma$-algebra}\index{Sigma-algebra@$\sigma$-algebra}.
\end{definicija}

\begin{definicija}
\emph{Verjetnost}\index{Verjetnost} je preslikava
$P \colon \mathcal{F} \to [0,1]$ z naslednjimi
lastnostmi:\footnote{To so aksiomi Kolmogorova.}

\begin{enumerate}[i)]
\item Velja $P(\Omega) = 1$ in $P(\emptyset) = 0$.
\item Če so dogodki $A_1, A_2, \dots$ disjunktni, je
\[
P\left(\bigcup_{n=1}^\infty A_n\right) = \sum_{n=1}^\infty P(A_n).
\]
\end{enumerate}
\end{definicija}

\begin{trditev}
Veljajo naslednje trditve:

\begin{enumerate}[i)]
\item Velja $P(A) + P(A^\mathsf{c}) = 1$.
\item Velja $P(A \cup B) = P(A) + P(B) - P(A \cap B)$.
\end{enumerate}
\end{trditev}

\begin{proof}
Trditev dokažemo po točkah:

\begin{enumerate}[i)]
\item Ker je $A \cup A^\mathsf{c} = \Omega$ in
$A \cap A^\mathsf{c} = \emptyset$, je
\[
P(A) + P(A^\mathsf{c}) = P(\Omega).
\]
\item Obe strani lahko zapišemo kot
\[
P(A \cap B^\mathsf{c}) + P(A^\mathsf{c} \cap B) + P(A \cap B).
\qedhere
\]
\end{enumerate}
\end{proof}

\datum{2022-10-5}

\begin{izrek}[Formula za vključitve in izključitve]
\index{Izrek!Vključitve in izključitve}
Naj bodo $A_1, A_2, \dots A_n$ dogodki. Tedaj je
\[
P\left(\bigcup_{i=1}^n A_i\right) =
\sum_{\substack{S \subseteq \set{1, \dots, n} \\ S \ne \emptyset}}
(-1)^{\abs{S}+1} \cdot P\left(\bigcap_{i=1}^n A_i\right).
\]
\end{izrek}

\begin{proof}
Formulo dokažemo z indukcijo. Trditev za $n=2$ smo že dokazali.
Naj bo
\[
A = \bigcup_{i=1}^n A_i
\]
in $B = A_{n+1}$.
Tedaj je
\begin{align*}
P(A \cup B) &= P(A) + P(B) - P(A \cap B)
\\
&=
\sum_{\substack{S \subseteq \set{1, \dots, n} \\ S \ne \emptyset}}
(-1)^{\abs{S}+1} \cdot P\left(\bigcap_{i=1}^n A_i\right) + P(B) -
\sum_{\substack{S \subseteq \set{1, \dots, n} \\ S \ne \emptyset}}
(-1)^{\abs{S}+1} \cdot P\left(\bigcap_{i=1}^n A_i \cap B\right)
\\
&=
\sum_{\substack{S \subseteq \set{1,\dots, n+1} \\ S \ne \emptyset}}
(-1)^{\abs{S}+1} \cdot P\left(\bigcap_{i=1}^n A_i\right). \qedhere
\end{align*}
\end{proof}

\begin{izrek}
Naj bodo $A_1 \subseteq A_2 \subseteq \dots$ naraščajoči dogodki.
Tedaj je
\[
P\left(\bigcup_{n=1}^\infty A_n\right) =
\lim_{n \to \infty} P(A_n).
\]
\end{izrek}

\begin{proof}
Velja
\[
\bigcup_{n=1}^\infty A_n =
A_1 \cup \bigcup_{n=2}^\infty (A_n \setminus A_{n-1}).
\]
Velja torej
\[
P\left(\bigcup_{n=1}^\infty A_n\right) =
P(A_1) + \sum_{n=2}^\infty (P(A_n) - P(A_{n-1})) =
\lim_{n \to \infty} P(A_n). \qedhere
\]
\end{proof}

\begin{opomba}
Iz De Morganovih pravil sledi analog s preseki -- če so
$A_1 \supseteq A_2 \supseteq \dots$ padajoči dogodki, je
\[
P\left(\bigcap_{n=1}^\infty A_n\right) =
\lim_{n \to \infty} P(A_n).
\]
\end{opomba}

\begin{izrek}[Borel-Cantellijeva lema]
\index{Izrek!Borel-Cantellijeva lema}
Naj bodo $A_1, A_2, \dots$ dogodki in
\[
A = \bigcap_{n=1}^\infty \bigcup_{k=n}^\infty A_k.
\]
Če je
\[
\sum_{k=1}^\infty P(A_k) < \infty,
\]
je $P(A) = 0$.
\end{izrek}

\begin{proof}
Opazimo, da so
\[
\bigcup_{k=n}^\infty A_k
\]
padajoči dogodki, zato je
\[
P(A) =
\lim_{n \to \infty} P\left(\bigcup_{k=n}^\infty A_k\right) \leq
\lim_{n \to \infty} \sum_{k=n}^\infty P(A_k) = 0. \qedhere
\]
\end{proof}

\newpage

\subsection{Pogojne verjetnosti}

\datum{2022-10-11}

\begin{definicija}
Naj bo $B$ dogodek z verjetnostjo $P(B) > 0$.
\emph{Pogojno verjetnost}\index{Verjetnost!Pogojna verjetnost}
dogodka $A$ pri pogoju $B$ definiramo kot
\[
P(A \mid B) = \frac{P(A \cap B)}{P(B)}.
\]
\end{definicija}

\begin{definicija}
Nabor dogodkov $\set{H_1, H_2, \dots}$ je
\emph{particija}\index{Množica izidov!Particija} $\Omega$, če je
\[
\bigcup_i H_i = \Omega
\quad \text{in} \quad
\forall i \ne j \colon H_i \cap H_j = \emptyset.
\]
\end{definicija}

\begin{izrek}[Formula za popolno verjetnost]
Naj bo $\set{H_1, H_2, \dots}$ particija $\Omega$ in $A$ dogodek.
Tedaj velja
\[
P(A) = \sum_i P(A \mid H_i) \cdot P(H_i).
\]
\end{izrek}

\begin{proof}
Velja
\[
P(A) = \sum_i P(A \cap H_i) = \sum_i P(A \mid H_i) \cdot P(H_i).
\qedhere
\]
\end{proof}

\begin{opomba}
Velja tudi
\[
P\left(\bigcap_{i=1}^n A_i\right) =
\prod_{k=1}^n P\left(A_k~\middle|~\bigcap_{i=1}^{k-1} A_i\right).
\]
\end{opomba}

\datum{2022-10-12}

\begin{definicija}
Dogodki $A_1, \dots, A_n$ so
\emph{neodvisni}\index{Dogodek!Neodvisen}, če za vse
$S \subseteq \set{1,\dots,n}$ velja
\[
P\left(\bigcap_{i \in S} A_i\right) = \prod_{i \in S} P(A_i).
\]
Dogodki $\set{A_n}_{i \in I}$ so neodvisni, če je vsaka njihova
končna poddružina neodvisna.
\end{definicija}

\begin{opomba}
Ekvivalentno sta dogodka neodvisna natanko tedaj, ko zanju velja
$P(A \mid B) = P(A)$.
\end{opomba}

\datum{2022-10-18}

\begin{izrek}[Borel-Cantellijeva lema]
\index{Izrek!Borel-Cantellijeva lema}
Naj bodo dogodki $A_1, A_2, \dots$ neodvisni in
\[
A = \bigcap_{n=1}^\infty \bigcup_{k=n}^\infty A_k.
\]
Če je
\[
\sum_{i=1}^\infty P(A_i) = \infty,
\]
je $P(A) = 1$.
\end{izrek}

\begin{proof}
Velja
\[
A^{\mathsf{c}} =
\bigcup_{n=1}^\infty \bigcap_{k=n}^\infty A_k^{\mathsf{c}}.
\]
Dovolj je pokazati, da imajo preseki verjetnost $0$. Ker so tudi
komplementi neodvisni, velja
\[
P\left(\bigcap_{k=n}^\infty A_k^{\mathsf{c}}\right) \leq
P\left(\bigcap_{k=n}^N A_k^{\mathsf{c}}\right) =
\prod_{k=n}^N (1 - P(A_k)) \leq
\prod_{k=n}^N e^{-P(A_k)},
\]
kar konvergira proti $0$.
\end{proof}

\begin{definicija}
Družina dogodkov $\mathcal{P}$ je \emph{$\pi$-sistem}, če je za
vsaka $A, B \in \mathcal{P}$ tudi $A \cap B \in \mathcal{P}$.
\end{definicija}

\begin{izrek}
Naj bo družina $\set{B_1, \dots, B_n}$ $\pi$-sistem, $A$ pa
dogodek, ki je neodvisen od vsakega $B_i$. Potem je $A$ neodvisen
od vsakega dogodka, ki ga lahko sestavimo iz dogodkov $B_i$ s
komplementi, preseki in unijami.
\end{izrek}

\begin{proof}
Vsak dogodek, ki ga lahko sestavimo iz dogodkov $B_i$, je
disjunktna unija dogodkov oblike
\[
\bigcap_{i=1}^n B_i^*,
\]
kjer je $B_i^* \in \set{B_i, B_i^{\mathsf{c}}}$. Ker delamo v
$\pi$-sistemu, je brez škode za splošnost dovolj preveriti dogodke
oblike
\[
\bigcap_{i=1}^m B_i^{\mathsf{c}} \cap B_{m+1}.
\]
Velja pa
\begin{align*}
P\left(A \cap \bigcap_{i=1}^m B_i^{\mathsf{c}} \cap B_{m+1}\right)
&=
P\left(A \cap \left(\bigcup_{i=1}^m B_i\right)^{\mathsf{c}} \cap
B_{m+1}\right)
\\
&=
P(A \cap B_{m+1}) -
P\left(\bigcup_{i=1}^m B_i \cap B_{m+1} \cap A\right)
\\
&=
P(A) \cdot P(B_{m+1}) -
P\left(\bigcup_{i=1}^m (B_i \cap B_{m+1} \cap A)\right)
\intertext{Sedaj lahko unijo razvijemo po načelu vključitev in
izključitev, nato pa izpostavimo $P(A)$, ki je neodvisen z vsemi
peseki. Po ponovni uporabi vključitev in izključitev dobimo}
P\left(A \cap \bigcap_{i=1}^m B_i^{\mathsf{c}} \cap B_{m+1}\right)
&=
P(A) \cdot \left(P(B_{m+1}) -
P\left(\bigcup_{i=1}^m B_i \cup B_{m+1}\right)\right)
\\
&= P(A) \cdot
P\left(\bigcap_{i=1}^m B_i^{\mathsf{c}} \cap B_{m+1}\right).
\qedhere
\end{align*}
\end{proof}

\begin{izrek}
Naj bosta $\set{A_1, \dots, A_m}$ in $\set{B_1, \dots, B_n}$
$\pi$-sistema dogodkov. Če so vsi dogodki $A_i$ neodvisni z vsemi
dogodki $B_j$, so neodvisni vsi dogodki, ki jih lahko sestavimo s
komplementi, unijami in preseki množic iz posameznega sistema.
\end{izrek}
