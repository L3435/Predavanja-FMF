\section{Osnovne navadne parcialne enačbe}

\subsection{Diferencialne enačbe}

\datum{2022-10-13}

\begin{definicija}
\emph{Splošna diferencialna enačba I.~reda}\index{Diferencialna enačba!I.~reda}
je enačba oblike
\[
\dot{x}(t) = F(t, x(t)),
\]
kjer je $F$ dovolj lepa funkcija.
\end{definicija}

\begin{definicija}
\emph{Splošna diferencialna enačba $n$-tega reda}\index{Diferencialna enačba!$n$-tega reda}
je enačba oblike
\[
F(t, x(t), \dot{x}(t), \dots, x^{(n)}(t)) = 0.
\]
Rešitev enačbe so vse funkcije $\widetilde{x}(t)$, ki rešijo enačbo
za vsak $t$ iz intervala, ki nas zanima.
\end{definicija}

\begin{definicija}
\emph{Splošen sistem}\index{Diferencialna enačba!Sistem} $n$ enačb
reda $m$ je sistem oblike
\begin{align*}
F_1(t, x(t), \dot{x}(t), \dots, x^{(m)}(t)) &= 0,
\\
F_2(t, x(t), \dot{x}(t), \dots, x^{(m)}(t)) &= 0,
\\
& ~~\! \vdots
\\
F_n(t, x(t), \dot{x}(t), \dots, x^{(m)}(t)) &= 0,
\end{align*}
kjer $x \colon \R \to \R^n$ označuje vektorsko funkcijo.
\end{definicija}

\begin{trditev}
Vsako enačbo reda $n$ lahko zapišemo v obliki sistema $n$ enačb
I.~reda.
\end{trditev}

\begin{proof}
Naj bo
\[
x^{(n)} = F(t, x, \dot{x}, \dots, x^{(n-1)}).
\]
Vpeljimo neodvisne spremenljivke $x_1(t), \dots, x_n(t)$ kot
\[
x_i(t) = x^{(i-1)}(t).
\]
Enačbo lahko sedaj prepišemo v sistem
\begin{align*}
\dot{x}_1 &= x_2
\\
\dot{x}_2 &= x_3
\\
&~~\! \vdots
\\
\dot{x}_{n-1} &= x_n
\\
\dot{x}_n &= F(t, x_1, \dots, x_n). \qedhere
\end{align*}
\end{proof}

\newpage

\subsection{Enačbe I.~reda in vektorska polja}

\begin{definicija}
\emph{Vektorsko polje}\index{Vektorsko polje} $F$ na območju
$\Omega \subseteq \R^n$ je gladka preslikava
$F \colon \Omega \to \R^n$.
\end{definicija}

\begin{definicija}
Naj bo $F \colon \Omega \to \R^n$ vektorsko polje in naj bo
$\vv{x}_0 \in \Omega$.
\emph{Integralska krivulja}\index{Integralska krivulja} polja $F$
skozi točko $\vv{x}_0$ je krivulja
$\gamma \colon t \mapsto \gamma(t) \in \Omega$, za katero velja
$\dot{\gamma}(t) = F(\gamma(t))$ in $\gamma(0) = \vv{x}_0$.
\end{definicija}

\datum{2022-10-20}

\begin{opomba}
Sistem $\dot{\gamma}(t) = F(\gamma(t))$ je \emph{avtonomen} --
neodvisna spremenljivka $t$ na desni ne nastopa eksplicitno.
\end{opomba}

\begin{definicija}
Sistem $n$ diferencialnih enačb I.~reda oblike
\[
\dot{x}(t) = F(t, x(t))
\]
je
\emph{neavtonomen sistem}\index{Diferencialna enačba!Sistem!Neavtonomen}.
\end{definicija}

\begin{opomba}
Neavtonomne sisteme lahko obravnavamo s pomočjo razširjenega
faznega prostora -- dodamo dimenzijo $x_{n+1}(t) = t$. V tem
sistemu je polje translacijsko invariantno za translacije v smeri
$t$.
\end{opomba}

\newpage

\subsection{Primeri diferencialnih enačb}

\begin{primer}
Rešitev enačbe
\[
\dot{x} = g(t)
\]
je funkcija
\[
x(t) = \int_0^t g(\tau)\,d\tau + C.
\]
\end{primer}

\begin{primer}
Na podoben način lahko rešimo enačbo
\[
\dot{x} = h(x).
\]
Sistem lahko namreč preoblikujemo v $\xi'(x) = \frac{1}{h(x)}$,
kjer je $\xi = x^{-1}$. Za
\[
G(x) = \int_0^x \frac{1}{h(\zeta)}\,d\zeta
\]
tako velja
\[
x = G^{-1}(t-C).
\]
\end{primer}

\begin{primer}
Enačbi oblike
\[
\dot{x} = f(t) \cdot g(x)
\]
pravimo \emph{splošna enačba z ločljivima spremenljivkama}. Če je
$H(x)$ primitivna funkcija funkcije $\frac{1}{g(x)}$, lahko enačbo
preoblikujemo v
\[
\dot{H}(x(t)) = f(x).
\]
Za primitivno funkcijo $F$ funkcije $f$ tako velja
\[
x(t) = H^{-1}(F(t) + C).
\]
\end{primer}

\begin{primer}
Enačba
\[
\dot{x} = f(t,x)
\]
je \emph{homogena}, če za vsak $\lambda \ne 0$ velja
\[
f(\lambda t, \lambda x) = f(t,x).
\]
Naj bo $v = \frac{x}{t}$. Tedaj velja
\[
\dot{v} = \frac{t \dot{x} - x}{t^2} =
\frac{1}{t} (f(1,v) - v).
\]
To je enačba z ločljivima spremenljivkama, ki jo rešimo tako kot
prejšnjo.
\end{primer}

\newpage

\subsection{Splošne in partikularne rešitve}

\begin{definicija}
\emph{Splošna rešitev sistema} je $n$-parametrična družina funkcij
\[
t \mapsto \gamma(t, c_1, \dots, c_n),
\]
ki rešijo sistem.
\end{definicija}

\begin{opomba}
Parametri pridobimo iz začetnih pogojev sistema.
\end{opomba}
