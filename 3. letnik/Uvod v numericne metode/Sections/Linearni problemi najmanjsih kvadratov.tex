\section{Linearni problemi najmanjših kvadratov}

\subsection{QR razcep}

\begin{definicija}
Naj bo $A \in \R^{m \times n}$ in $b \in \R^m$, kjer je $m \geq n$.
\emph{Rešitev po metodi najmanjših kvadratov}\index{Metoda najmanjših kvadratov}
sistema $Ax = b$ je vektor $x \in \R^n$, za katerega je
$\norm{Ax - b}$ minimalna.
\end{definicija}

\begin{opomba}
Predpostavimo, da je $\rang A = n$, sicer ne dobimo enolične
rešitve.
\end{opomba}

\begin{izrek}
Naj bo $A \in \R^{m \times n}$ in $b \in \R^m$, kjer je $m \geq n$
in $\rang A = n$. Rešitev normalnega sistema
$A^\top A x = A^\top b$ je rešitev sistema $Ax = b$ po metodi
najmanjših kvadratov.
\end{izrek}

\begin{proof}
Vrednost $\norm{Ax - b}$ je minimalna, ko je $Ax - b \perp \im A$.
To je ekvivalentno
\[
\skl{Ax - b, Ay} = 0
\]
za vsak $y \in \R^n$, oziroma
\[
\skl{A^\top Ax - A^\top b, y} = 0. \qedhere
\]
\end{proof}

\begin{opomba}
Sistemu $A^\top Ax = A^\top b$ pravimo \emph{normalni sistem}.
\end{opomba}

\begin{opomba}
Normalnen sistem lahko rešimo z razcepom Choleskega. Časovna
zahtevnost celotnega postopka je $O(n^2 m)$.
\end{opomba}

\begin{definicija}
\emph{QR razcep}\index{Matrika!QR razcep} matrike
$A \in \R^{m \times n}$ je razcep $A = Q \cdot R$, kjer je
$Q \in \R^{m \times n}$ matrika z ortonormiranimi stolpci in
$R \in \R^{n \times n}$ zgornjetrikotna matrika s pozitivno
diagonalo.
\end{definicija}

\begin{izrek}
Naj bo $A \in \R^{m \times n}$, kjer je $m \geq n$ in
$\rang A = n$. Tedaj ima matrika $A$ enoličen QR razcep.
\end{izrek}

\begin{proof}
Če je $A = QR$, je $A^\top A = R^\top Q^\top Q R = R^\top R$, zato
je $R$ enolično določena z razcepom Choleskega. Od tod lahko
izračunamo še $Q = AR^{-1}$. Ker je
\[
Q^\top Q =
\br{R^\top}^{-1} A^\top A R^{-1} =
\br{R^\top}^{-1} R^\top R R^{-1} =
I,
\]
tako dobljeni matriki ustrezata pogojem QR razcepa.
\end{proof}

\begin{opomba}
Podobno velja za kompleksne matrike, pri čemer zahtevamo, da je
$Q$ unitarna.
\end{opomba}

\begin{opomba}
QR razcep matrike izračunamo z Gram-Schmidtovo ortogonalizacijo.
Algoritem je naslednji:

\begin{algorithmic}[1]
\For{$k=1$ to $n$}
  \State $q_k = a_k$
  \For{$i=1$ to $k-1$}
    \State $r_{i,k} = \skl{q_i, a_k}$
    \label{alg:gs:mod}
    \State $q_k \gets q_k - r_{i,k} q_i$
  \EndFor
  \State $r_{k,k} = \norm{q_k}$
  \State $q_k \gets \frac{1}{r_{k,k}} q_k$
\EndFor
\end{algorithmic}
\end{opomba}

\begin{opomba}
V praksi raje uporabljamo \emph{modificirano Gram-Schmidt metodo}.
Razlika je v vrstici~\ref{alg:gs:mod}, kjer ukaz nadomestimo z
$r_{i,k} = \skl{q_i, q_k}$. Metoda je stabilnejša.
\end{opomba}

\begin{trditev}
Naj bo
\[
\begin{bmatrix}
A & b
\end{bmatrix}
=
\begin{bmatrix}
Q & q_{n+1}
\end{bmatrix}
\cdot
\begin{bmatrix}
R &  z   \\
0 & \rho
\end{bmatrix}
\]
QR razcep. Rešitev po metodi najmanjših kvadratov sistema $Ax = b$
je vektor $x$, za katerega je $Rx = z$.
\end{trditev}

\begin{proof}
Izračunamo lahko
\[
Ax - b =
\begin{bmatrix}
A & b
\end{bmatrix}
\cdot
\begin{bmatrix}
x \\ -1
\end{bmatrix}
=
Q(Rx - z) - \rho q_{n+1}.
\]
Ker je $q_{n+1} \perp \im Q$, je minimum dosežen, ko je
$Q(Rx - z) = 0$.
\end{proof}

\begin{opomba}
Enostavno je videti, da z razcepom $A = QR$ za rešitev po metodi
najmanjših kvadratov velja tudi $Rx = Q^\top b$. Izkaže se, da je
zgornji postopek stabilnejši od reševanja tega sistema.
\end{opomba}

\begin{opomba}
Poznamo tudi razširjeni QR razcep -- pri tem dobimo
$A = \widetilde{Q} \cdot \widetilde{R}$, kjer je $\widetilde{Q}$
kvadratna ortogonalna matrika in $\widetilde{R}$ zgornjetrapezna
matrika. Tedaj za
\[
A =
\widetilde{Q} \cdot \widetilde{R} =
\begin{bmatrix}
Q & Q_1
\end{bmatrix}
\cdot
\begin{bmatrix}
R \\ 0
\end{bmatrix}
\]
velja
\[
Ax - b =
\begin{bmatrix}
Rx - Q^\top b \\ -Q_1^\top b
\end{bmatrix}.
\]
Minimum je tako dosežen pri $Rx = Q^\top b$ in je enak
$\norm{Q_1^\top b}$.
\end{opomba}

\begin{definicija}
\emph{Givensova rotacija}\index{Matrika!Givensova rotacija} je
matrika
\[
R_{i,j}(c,s)^\top =
\begin{bmatrix}
I &    &   &   &   \\
  &  c &   & s &   \\
  &    & I &   &   \\
  & -s &   & c &   \\
  &    &   &   & I
\end{bmatrix},
\]
kjer je $(c, s) = (\cos \varphi, \sin \varphi)$.
\end{definicija}

\begin{opomba}
Če izberemo
\[
(c,s) =
\frac{1}{\norm{(a_{i,j}, a_{j,j}}} \cdot \br{a_{j,j}, a_{i,j}},
\]
ima matrika
\[
R_{i,j}(c,s)^\top \cdot A
\]
na mestu $(i,j)$ element $0$. Z zaporednimi uporabami takih
transformacij lahko dobimo modificiran QR razcep matrike $A$.
\end{opomba}

\begin{opomba}
Algoritem za modificiran QR razcep z Givensovimi rotacijami je
naslednji:

\begin{algorithmic}[1]
\State $Q = I$
\For{$i=1$ to $n$}
  \For{$k=i+1$ to $m$}
    \State $r = \norm{(a_{i,i}, a_{k,i})}$
    \State $c = \frac{a_{i,i}}{r}$
    \State $s = \frac{a_{k,i}}{r}$
    \State $A \gets R_{i,k}(c,s)^\top \cdot A$
    \label{alg:gv:updA}
    \State $b \gets R_{i,k}(c,s)^\top \cdot b$
    \label{alg:gv:updb}
    \State $Q \gets R_{i,k}(c,s)^\top \cdot Q$
    \label{alg:gv:updQ}
  \EndFor
\EndFor
\State $Q \gets Q^\top$
\end{algorithmic}
Pri tem upoštevamo, v
vrsticah~\ref{alg:gv:updA},~\ref{alg:gv:updb} in~\ref{alg:gv:updQ}
zadošča posodobiti le zadnje stolpce dveh vrstic, kar bistveno
zmanjša število potrebnih operacij. Časovna zahtevnost algoritma je
$O(6m^2n - n^3)$. Če izpustimo vrstico~\ref{alg:gv:updQ}, se
zahtevnost zmanjša na $O(3mn^2 - n^3)$.\footnote{To naredimo, kadar
nas matrika $\widetilde{Q}$ ne zanima.}
\end{opomba}

\begin{definicija}
Naj bo $w \in \R^n$ neničeln vektor. Preslikavi
\[
P = I - \frac{2}{\norm{w}^2} w w^\top
\]
pravimo
\emph{Householderjevo zrcaljenje}\index{Matrika!Householderjevo zrcaljenje}.
\end{definicija}

\begin{trditev}
Velja $P = P^\top$ in $P^2 = I$.
\end{trditev}

\obvs

\begin{trditev}
Naj bo $x = \alpha w + u$, kjer je $u \perp w$. Tedaj je
$Px = - \alpha w + u$.
\end{trditev}

\begin{proof}
Dovolj je opaziti, da je $w^\top w = \norm{w}^2$ in $w^\top u = 0$.
\end{proof}

\begin{posledica}
Če je $\norm{x} = \norm{y} \ne 0$, za $w = x-y$ velja $Px = y$.
\end{posledica}

\begin{proof}
Zapišemo lahko
\[
x = \frac{x+y}{2} + \frac{w}{2}
\quad \text{in} \quad
y = \frac{x+y}{2} - \frac{w}{2}.
\]
Sedaj je dovolj opaziti, da je $\skl{x+y,w} = 0$.
\end{proof}

\begin{opomba}
Householderjeva zrcaljenja lahko uporabimo za izračun QR razcepa.
Postopoma v želeno obliko nastavljamo stolpce. Če imamo v $i$-tem
stolpcu elemente $a_1, \dots, a_n$, vzamemo\footnote{Tu je
izjemoma $\sgn(0) = 1$. S popravkom predznaka se izognemo numerični
nestabilnosti v primeru, ko je
$\abs{a_i} \approx \norm{(a_i, \dots, a_n)}$.}
\[
w =
\br{a_i + \sgn(a_i) \norm{(a_i, \dots, a_n)}, a_{i+1}, \dots, a_n}
\]
in matriko $A$ z leve zmnožimo z
\[
\begin{bmatrix}
I & 0 \\
0 & P
\end{bmatrix}.
\]
\end{opomba}

\begin{opomba}
QR razcep s Householderjevimi zrcaljenji dobimo z naslednjim
algoritmom:

\begin{algorithmic}[1]
\State $Q=I$
\For{$i=1$ to $\min(n, m-1)$}
  \State $w = \br{a_{i,i} + \sgn(a_{i,i})
  \norm{(a_{i,i}, \dots, a_{n,i})}, a_{i+1,i}, \dots, a_{n,i}}$
  \State $P_i = I - \frac{2}{\norm{w}^2} ww^\top$
  \label{alg:hh:zrcdef}
  \State $A \gets P_i \cdot A$
  \label{alg:hh:updA}
  \State $b \gets P_i \cdot b$
  \label{alg:hh:updb}
  \State $Q \gets P_i \cdot Q$
  \label{alg:hh:updQ}
\EndFor
\end{algorithmic}
V vrstici~\ref{alg:hh:zrcdef} $P_i$ dopolnimo zgoraj levo z
identiteto do dimenzije $m \times m$. V
vrsticah~\ref{alg:hh:updA},~\ref{alg:hh:updb} in~\ref{alg:hh:updQ}
zadošča posodobiti zadnjih $i$ vrstic, kar bistveno zmanjša
število operacij. Časovna zahtevnost algoritma je
$O \br{4m^2n - \frac{2}{3} n^3}$. Če izpustimo
vrstico~\ref{alg:hh:updQ}, se zahtevnost zmanjša na
$O \br{2mn^2 - \frac{2}{3} n^3}$.
\end{opomba}

\begin{opomba}
Občutljivost rešitve predoločenega sistema po metodi najmanjših
kvadratov je odvisna od $K(A) + \norm{Ax - b} K(A)^2$.\footnote{Tu
za matrično normo seveda vzamemo kar $\norm{\cdot}_2$.} Pri
normalnem sistemu dobimo faktor $K(A^\top A) = K(A)^2$, pri
reševanju s QR razcepom pa samo $K(R) = K(A)$.
\end{opomba}
