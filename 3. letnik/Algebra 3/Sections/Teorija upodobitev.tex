\section{Teorija upodobitev}

\epigraph{">Pa kemiki to uporabljajo ko opazujejo strukture
kristalov. Samo da oni tega ne znajo."<}{-- prof.~dr.~Primož
Moravec}

\subsection{Upodobitve}

\begin{definicija}
Naj bo $R$ komutativen kolobar z enoto. $R$-modul $A$ je
\emph{$R$-algebra}\index{$R$-algebra}, če je kolobar in za vse
$r \in R$ ter $a_1, a_2 \in A$ velja
\[
r(a_1 a_2) = (r a_1) a_2 = a_1 (r a_2).
\]
\end{definicija}

\begin{definicija}
Naj bo $A$ $R$-algebra.
\emph{Upodobitev}\index{$R$-algebra!Upodobitev} algebre $A$ je
homomorfizem $R$-algeber $\rho \colon A \to \End_R(V)$, kjer je $V$
$R$-modul.
\end{definicija}

\begin{opomba}
Če je $V$ prost, lahko pišemo $\rho \colon A \to M_n(R)$. Številu
$n$ pravimo \emph{stopnja upodobitve}.
\end{opomba}

\begin{opomba}
Na vsako upodobitev $\rho$ $R$-algebre $A$ lahko gledamo kot
$R$-modul $V$, ki je $A$-modul. Definiramo lahko namreč
$a \cdot v = \rho(a)v$. Velja tudi obratno --
$a \mapsto (v \mapsto av)$ je upodobitev $R$-algebre $A$.
\end{opomba}

\begin{definicija}
Naj bo $G$ grupa. \emph{Grupna algebra}\index{Grupna algebra} $RG$
je prosti $R$-modul nad množico $G$ z množenjem
\[
\br{\sum_{g \in G} \lambda_g g} \cdot \br{\sum_{h \in G} \mu_h h} =
\sum_{g \in G} \br{\sum_{hk = g} \lambda_h \mu_k} g.
\]
\end{definicija}

\begin{definicija}
Naj bo $G$ grupa. \emph{upodobitev grupe}\index{Upodobitev}
$G$ nad $R$ je homomorfizem $R$-algeber
$\rho \colon RG \to \End_R(V)$, kjer je $V$ $R$-modul.
\end{definicija}

\begin{trditev}
Zožitev $\eval{\rho}{G}{}$ je homomorfizem grup $G$ in $\GL_R(V)$.
\end{trditev}

\obvs

\begin{opomba}
Vsak homomorfizem grup $G$ in $\GL_R(V)$ lahko razširimo do
homomorfizma $R$-algeber $RG$ in $\End_R(V)$. Tako lahko
ekvivalentno definiramo upodobitev grupe $G$ nad $R$ kot
homomorfizem grup $\rho \colon G \to \GL_R(V)$.
\end{opomba}

\begin{definicija}
Naj bo $\rho \colon G \to \GL_R(V)$ upodobitev. Podmodul $W$
$R$-modula $V$ je \emph{$G$-invarianten}, če za vsaka $g \in G$ in
$w \in W$ velja $\rho(g) w \in W$.
\end{definicija}

\begin{opomba}
Ekvivalentno je $W$ podmodul $RG$-modula $V$.
\end{opomba}

\begin{definicija}
Naj bo $W \leq V$ $G$-invarianten podmodul. Upodobitvi
$\rho_W \colon G \to \GL_R(W)$ s predpisom
$g \mapsto (w \mapsto \rho(g) w)$ pravimo
\emph{podupodobitev}\index{Upodobitev!Podupodobitev} upodobitve
$\rho$.
\end{definicija}

\begin{definicija}
Upodobitvi $\rho_1 \colon G \to \GL_R(V_1)$ in
$\rho_2 \colon G \to \GL_R(V_2)$ sta
\emph{ekvivalentni}\index{Upodobitev!Ekvivalentna}, če sta $V_1$ in
$V_2$ izomorfna kot $RG$-modula.
\end{definicija}

\begin{opomba}
Ekvivalentno obstaja izomorfizem $T \colon V_1 \to V_2$
$R$-modulov, za katerega je $T \circ \rho_1(g) = \rho_2(g) \circ T$
za vse $g \in G$.
\end{opomba}

\begin{definicija}
\emph{Direktna vsota}\index{Upodobitev!Direktna vsota} upodobitev
$\rho_1 \colon G \to \GL_R(V_1)$ in
$\rho_2 \colon G \to \GL_R(V_2)$ je upodobitev
$\rho_1 \oplus \rho_2 \colon G \to \GL_R(V_1 \oplus V_2)$ s
predpisom
\[
g \mapsto \br{(v_1, v_2) \mapsto (\rho_1(g) v_1, \rho_2(g) v_2)}.
\]
\end{definicija}

\begin{definicija}
\emph{Tenzorski produkt}\index{Upodobitev!Tenzorski produkt}
upodobitev $\rho_1 \colon G \to \GL_R(V_1)$ in
$\rho_2 \colon G \to \GL_R(V_2)$ je upodobitev
$\rho_1 \otimes \rho_2 \colon G \to \GL_R(V_1 \otimes_R V_2)$ s
predpisom
\[
g \mapsto
\br{v_1 \otimes v_2 \mapsto \rho_1(g) v_1 \otimes \rho_2(g) v_2}.
\]
\end{definicija}

\newpage

\subsection{Polenostavni moduli}

\begin{definicija}
Naj bo $V$ $A$-modul.
\begin{enumerate}[i)]
\item Modul $V$ je \emph{enostaven}\index{Modul!Enostaven}
(nerazcepen), če sta edina njegova $A$-podmodula $\set{0}$ in $V$.
\item Modul $V$ je \emph{polenostaven}\index{Modul!Polenostaven}
(povsem razcepen), če je direktna vsota enostavnih $A$-podmodulov.
\end{enumerate}
\end{definicija}

\begin{izrek}[Maschke]\index{Izrek!Maschke}
Naj bo $G$ končna grupa in $F$ polje, za katerega
$\chr F \nmid \abs{G}$. Naj bo $\rho \colon G \to \GL_F(V)$
upodobitev grupe $G$ nad $F$ za končnorazsežen vektorski prostor
$V$ in $W$ $G$-invarianten podprostor v $V$. Potem obstaja
$G$-invarianten podprostor $U$ v $V$, za katerega je
$V = W \oplus U$.
\end{izrek}

\begin{proof}
Naj bo $\pi \colon V \to W$ projektor. Definirajmo preslikavo
$\pi' \colon V \to V$ s predpisom
\[
\pi' = \frac{1}{\abs{G}} \sum_{g \in G} \rho(g) \pi \rho(g)^{-1}.
\]
Tudi $\pi'$ je projektor na $W$. Za $w \in W$ namreč velja
\[
\pi' w =
\frac{1}{\abs{G}} \sum_{g \in G} \rho(g) \pi \rho^{-1}(g) w =
\frac{1}{\abs{G}} \sum_{g \in G} \rho(g) \rho^{-1}(g) w =
w,
\]
saj je $\rho^{-1}(g) w \in W$, poleg tega pa je seveda
$\im \pi' = W$. Sledi, da je $V = W \oplus \ker \pi'$. Dovolj je
tako pokazati, da je $\ker \pi'$ $G$-invarianten. Zadošča torej, da
je $\pi' \colon V \to W$ homomorfizem $G$-modulov. Velja pa
\begin{align*}
\pi'(g \cdot v) &=
\frac{1}{\abs{G}}
\sum_{g \in G} \rho(g) \pi \rho^{-1}(g) \rho(h) v
\\
&=
\frac{1}{\abs{G}}
\sum_{g \in G} \rho(h) \rho(h^{-1} g) \pi \rho(g^{-1} h) v
\\
&=
\frac{1}{\abs{G}}
\rho(h) \sum_{g \in G} \rho(g) \pi \rho^{-1}(g) v
\\
&=
h \cdot (\pi' v). \qedhere
\end{align*}
\end{proof}

\begin{posledica}
Naj bo $F$ polje in $G$ končna grupa, za katero
$\chr F \nmid \abs{G}$. Potem je vsak končnorazsežen $FG$-modul
polenostaven.
\end{posledica}

\begin{lema}\label{rep:lm:dir_sum}
Naj bo $U = S_1 + \dots + S_n$ $A$-modul, pri čemer so $S_i$
enostavni $A$-moduli. Naj bo $V$ podmodul v $U$. Potem obstaja
podmnožica $I \subseteq \setb{i \in \N}{i \leq n}$, za katero je
\[
U = V \oplus \bigoplus_{i \in I} S_i.
\]
\end{lema}

\begin{proof}
Naj bo $I$ maksimalna množica, za katero je
\[
W = V \oplus \bigoplus_{i \in I} S_i
\]
direktna vsota. Če obstaja $j$, za katerega je
$S_j \not \subseteq W$, zaradi enostavnosti sledi
$S_j \cap W = \set{0}$, kar je v prostislovju z maksimalnostjo
množice $I$. Sledi, da je $W = U$.
\end{proof}

\begin{definicija}
Naj bo $U$ $A$-modul.
\emph{Kompozicijska vrsta}\index{Modul!Kompozicijska vrsta} modula
$U$ je zaporedje
\[
\set{0} = U_0 \leq U_1 \leq \dots \leq U_n = U,
\]
pri čemer je za vsak $i$ modul $U_i$ maksimalen podmodul v
$U_{i+1}$. Modulom, za katere obstaja končna kompozicijska vrsta,
pravimo \emph{moduli s končno kompozicijsko dolžino}.
\end{definicija}

\begin{trditev}
Naj bo $U$ $A$-modul. Naslednje trditve so ekvivalentne:

\begin{enumerate}[i)]
\item $U$ je direktna vsota končno mnogo enostavnih podmodulov.
\item $U$ je direktna vsota končno mnogo enostavnih $A$-modulov.
\item $U$ ima končno kompozicijsko dolžino in vsak podmodul v $U$
je direktni sumand $U$-ja.
\end{enumerate}

Če drži katera od zgornjih trditev, velja tudi za vse podmodule in
kvociente $U$-ja.
\end{trditev}

\begin{proof}
Prvi trditvi sta ekvivalentni po lemi~\ref{rep:lm:dir_sum}. Ti
trditvi po isti lemi implicirata tretjo.

Predpostavimo, da ima $U$ končno kompozicijsko dolžino in da je
vsak podmodul v $U$ njegov direktni sumand. Opazimo, da to velja
tudi za vse podmodule $V \leq U$, saj lahko za $W \leq V$ zapišemo
$V = W \oplus (X \cap V)$. Nadaljujemo z indukcijo po dolžini
kompozicijske vrste.

Če je dolžina vrste enaka $1$, je $U$ enostaven, zato velja prva
trditev. Če je dolžina vrste daljša, lahko zapišemo
$U = V \oplus W$ in uporabimo indukcijsko predpostavko na $V$ ter
$W$.

Dokažimo še dedovanje na podmodule in kvociente. Na podmodule se
deduje tretja trditev. Za vsak kvocient $\kvoc{U}{V}$ lahko po
tretji trditvi zapišemo $U = V \oplus W$. Ker je $W$ podmodul v
$U$, zadošča prvi trditvi. Sledi, da ji zadošča tudi
$\kvoc{U}{V} \cong W$.
\end{proof}

\newpage

\subsection{Artin-Wedderburnov izrek}

\begin{izrek}[Schurova lema]\index{Schurova lema}
Naj bo $A$ končnorazsežna algebra nad poljem $F$. Naj bosta $S_1$
in $S_2$ enostavna $A$-modula.

\begin{enumerate}[i)]
\item Če je $S_1 \not \cong S_2$, je $\Hom_A(S_1, S_2) = \set{0}$.
\item $\End_A(S_1)$ je obseg.
\item Če je $F$ algebraično zaprto, je $\End_A(S_1) \cong F$.
\end{enumerate}
\end{izrek}

\begin{proof}
\phantom{a}
\begin{enumerate}[i)]
\item Naj bo $\varphi \colon S_1 \to S_2$ neničeln homomorfizem
$A$-modulov. Ker je $\im \varphi$ neničeln podmodul v $S_2$, je
$\im \varphi = S_2$. Ker je $\ker \varphi \ne S_1$, je
$\ker \varphi = \set{0}$. Sledi, da je $\varphi$ izomorfizem.
\item Vsi neničelni homomorfizmi so po istem argumentu kot v prvi
točki izomorfizmi.
\item Naj bo $\lambda \in F$ lastna vrednost endomorfizma
$\varphi$. Sledi, da $\varphi = \lambda I$ ni injektiven, zato je
$\varphi = \lambda I$. \qedhere
\end{enumerate}
\end{proof}

\begin{definicija}
Za kolobar $A$ definiramo
\emph{nasprotni kolobar}\index{Kolobar!Nasprotni} $A^{\mathsf{op}}$
kot kolobar, katerega elementi so elementi $A$, seštevanje sovpada
s seštevanjem v $A$, množenje pa je definirano z $a \cdot b = ba$.
\end{definicija}

\begin{lema}
Naj bo $A$ kolobar z enoto. Če na $A$ gledamo kot na $A$-modul,
velja $\End_A(A) \cong A^{\mathsf{op}}$.
\end{lema}

\begin{proof}
Preslikavi $\varphi \mapsto \varphi(1)$ in
$x \mapsto (a \mapsto ax)$ sta inverzna homomorfizma.
\end{proof}

\begin{definicija}
Naj bo $A$ kolobar z enoto. $A$ je
\emph{polenostaven}\index{Kolobar!Polenostaven}, če so vsi
$A$-moduli polenostavni.
\end{definicija}

\begin{lema}
Če je
\[
U = \bigoplus_{i=1}^r U_i
\]
$A$-modul, je $\End_A(U)$ izomorfen algebri $r \times r$ matrik
$[\varphi_{i,j}]$, kjer je $\varphi_{i,j} \in \Hom_A(U_i, U_j)$.
\end{lema}

\begin{proof}
Ni težko videti, da so endomorfizmi $\varphi \in \End_A(U)$ v
bijektivni korespondenci z zgornjimi matrikami. Kratek izračun
pokaže, da velja
\[
\varphi \circ \psi(v_j) =
\sum_{i=1}^r \br{\varphi \circ \psi}_{i,j}(v_j). \qedhere
\]
\end{proof}

\begin{izrek}[Artin-Wedderburn]\index{Izrek!Artin-Wedderburn}
Naj bo $A$ končnorazsežna algebra nad poljem $F$, za katero velja,
da je vsak končnogeneriran $A$-modul polenostaven. Potem je $A$
direktna vsota matričnih algeber nad obsegi. Natančneje, če je
\[
A = \bigoplus_{i=1}^k S_i^{n_i},
\]
kjer so $S_i$ paroma neizomorfni enostavni $A$-moduli, je $A$ kot
$F$-algebra izomorfna
\[
\bigoplus_{i=1}^k M_{n_i}(D_i),
\]
kjer je $D_i = \End_A(S_i)^{\mathsf{op}}$. Če je $F$ algebraično
zaprto, je $D_i = F$.
\end{izrek}

\begin{proof}
Po Schurovi lemi je $\Hom_A \br{S_j^{n_j}, S_i^{n_i}} = \set{0}$ za
vse $i \ne j$. Po lemi sledi, da je $\End_A(U)$ izomorfen algebri
$n \times n$ matrik, ki pa so bločno diagonalne. Posebej, velja
\[
\End_A(A) \cong \bigoplus_{i=1}^k \End_A \br{S_i^{n_i}}.
\]
Po zgornji lemi pa je $\End_A \br{S_i^{n_i}}$ izomorfna algebri
$n_i \times n_i$ matrik z elementi iz
$\End_A(S_i) = D_i^{\mathsf{op}}$. Velja torej
\[
A^{\mathsf{op}} \cong
\End_A(A) \cong
\bigoplus_{i=1}^k M_{n_i} \br{D_i^{\mathsf{op}}}.
\]
Ker je
$M_{n_i} \br{D_i^{\mathsf{op}}}^{\mathsf{op}} \cong M_{n_i}(D_i)$,
je izrek dokazan. Enakost $D_i = F$ v primeru algebraične zaprtosti
sledi iz Schurove leme.
\end{proof}

\begin{opomba}
Vsaka direktna vsota matričnih algeber je polenostavna.
\end{opomba}

\begin{opomba}
Algebra $M_n(D)$ je enostavna.
\end{opomba}

\begin{opomba}
Matrične algebre, ki nastopajo v razcepi, so enolično določene do
izomorfizma natančno.
\end{opomba}

\begin{posledica}
Naj bo $A$ končnorazsežna $F$-algebra. Če je $A$ kot $A$-modul enak
\[
A = \bigoplus_{i=1}^r S_i^{n_i},
\]
kjer so $S_i$ paroma neizomorfni enostavni $A$-moduli, je
$\setb{S_i}{1 \leq i \leq r}$ množica vseh predstavnikov izomorfnih
razredov enostavnih $A$-modulov. Če je $F$ algebraično zaprto, je
$n_i = \dim_F S_i$ in
\[
\dim_F A = \sum_{i=1}^r n_i^2.
\]
\end{posledica}

\begin{proof}
Najprej opazimo, da je vsak enostaven $A$-modul izomorfen
$\kvoc{A}{I}$. Res, naj bo $u \in U$ neničeln element modula.
Ker je $Au$ netrivialen podmodul v $U$, je $Au = U$. Po izreku o
izomorfizmu sledi $\kvoc{A}{\ker \varphi} \cong U$ za homomorfizem
$\varphi \colon a \mapsto au$. Če $\ker \varphi$ ni maksimalen
ideal, modul $U$ ni enostaven.

Za vsak enostaven $A$-modul $S$ tako sledi, da je kvocient modula
\[
\bigoplus_{i=1}^r S_i^{n_i}.
\]
Ker je kompozitum inkluzije $S_i \hookrightarrow A$ in kvocientne
projekcije izomorfizem enostavnih modulov, sledi, da za nek $i$
velja $S \cong S_i$.

Naj bo sedaj $F$ algebraično zaprto. Po Schurovi lemi tako velja
\[
A \cong \bigoplus_{i=1}^r M_{n_i}(F),
\]
od koder sledi
\[
\dim_F A = \sum_{i=1}^r n_i^2.
\]
Preostane še dokaz enakosti $n_i = \dim_F S_i$. Pokažimo, da je
$S_i^{n_i} \cong M_{n_i}(F)$.

Iz dokaza Artin-Wedderburnovega izreka vidimo, da za vsaka
$s \in S_i^{n_i}$ in $t \in M_{n_j}(F)$ ob pogoju $i \ne j$ velja
$st = 0$. Opazimo, da je $M_{n_i}(F) \cong \kvoc{A}{V}$, kjer je
$V$ podmodul direktne vsote matričnih algeber, ki ga $M_{n_i}(F)$
anihilira. Kompozitum inkluzije $S_i^{n_i}$ v $A$ in kvocientne
projekcije je tako surjektiven, zato je
$\dim_F S_i^{n_i} \geq \dim_F M_{n_i}(F)$. Ker je
\[
\dim_F A = \sum_{i=1}^r \dim_F S_i^{n_i},
\]
sledi, da velja enakost.
\end{proof}

\begin{posledica}
Naj bo $G$ končna grupa in $F$ polje, za katerega
$\chr F \nmid \abs{G}$.

\begin{enumerate}[i)]
\item Kot kolobar je $FG$ izomorfen direktni vsoti matričnih
algeber nad obsegi.
\item Če je $F$ algebraično zaprto in je $S_1, \dots, S_r$
kompleten nabor predstavnikov izomorfnih razredov enostavnih
$FG$-modulov, za $d_i = \dim_F S_i$ velja, da je $d_i$ večkratnost,
s katero $S_i$ nastopa v razcepu $FG$-modula $FG$ in je
\[
\abs{G} = \sum_{i=1}^r d_i^2.
\]
\end{enumerate}
\end{posledica}

\newpage

\subsection{Karakterji}

\begin{definicija}
Naj bo $\rho \colon G \to \GL_\C(V)$ končnorazsežna upodobitev
končne grupe $G$. Preslikavi $\chi \colon G \to \C$ s predpisom
\[
\chi(g) = \tr \rho(g)
\]
pravimo \emph{karakter}\index{Upodobitev!Karakter} upodobitve
$\rho$.
\end{definicija}

\begin{trditev}
Naj bo $\chi$ karakter upodobitve $\rho$.

\begin{enumerate}[i)]
\item Velja $\chi(1) = \deg \rho$.
\item Za vsak $g \in G$ je $\chi (x^{-1}) = \oline{\chi(g)}$.
\item Za vse $g, h \in G$ je $\chi(hgh^{-1}) = \chi(g)$.
\end{enumerate}
\end{trditev}

\begin{proof}
Dokažimo drugo točko. Ker je $\rho(g)^n = 1$, so lastne vrednosti
$\rho(g)$ koreni enote. Ker velja $\lambda^{-1} = \oline{\lambda}$,
sledi
\[
\rho(g^{-1}) = \sum_{i=1}^n \lambda_i^{-1} =
\sum_{i=1}^n \oline{\lambda_i} = \oline{\rho(g)}. \qedhere
\]
\end{proof}

\begin{definicija}
Naj bo $V$ končnorazsežen $\C G$-modul. S $\chi_V$ označimo
karakter upodobitve $\rho \colon G \to \GL_\C(V)$.
\end{definicija}

\begin{opomba}
Ta karakter je dobro definiran. Če sta $V$ in $W$ izomorfna
$\C G$-modula, velja namreč
\[
\rho_V(g) = T^{-1} \circ \rho_W(g) \circ T,
\]
zato sta sledi enaki.
\end{opomba}

\begin{trditev}
Naj bosta $V$ in $W$ končnorazsežna $\C G$-modula. Tedaj je

\begin{enumerate}[i)]
\item $\chi_{V \oplus W} = \chi_V + \chi_W$,
\item $\chi_{V \otimes W} = \chi_V \times \chi_W$ in
\item $\chi_{V^*} = \oline{\chi_V}$.
\end{enumerate}
\end{trditev}

\begin{proof}
Preprosto izračunamo sled pripadajočičh matrik. Upodobitev
$\rho^* \colon G \to \GL_\C(V^*)$ deluje s predpisom
$\rho(g) = (\varphi \mapsto \varphi \circ \rho(g^{-1}))$.
\end{proof}

\begin{lema}
Naj bosta $V$ in $W$ končnorazsežna $\C G$-modula. Tedaj sta
$G$-modula $\Hom_\C(V, W)$ in $V^* \otimes_\C W$ izomorfna.
\end{lema}

\begin{proof}
Naj bo $\alpha \colon V^* \otimes_\C W \to \Hom_\C(V, W)$
preslikava, podana s predpisom
$\varphi \otimes w \mapsto (v \mapsto \varphi(v) w)$. Očitno je
$\C$-linearna, ker pa velja
\[
\alpha(g (\varphi \otimes w)) =
\alpha(g \varphi \otimes gw) =
(v \mapsto (g \varphi)(v) gw) =
(v \mapsto \varphi(g^{-1}v) gw) =
g \cdot \alpha(\varphi \otimes w).
\]
Zaradi enakosti dimenzij je dovolj, da dokažemo surjektivnost. Naj
bo $f \colon V \to W$ poljuben homomorfizem. Za bazo
$\setb{v_i}{i \leq n}$ naj bo $f(v_i) = w_i$. Naj bo
$\setb{\varphi_i}{i \leq n}$ dualna baza in
\[
\widetilde{f} = \sum_{i=1}^n \varphi_i \otimes f(v_i).
\]
Tedaj je
\[
\alpha \br{\widetilde{f}} v_j = f(v_j),
\]
zato je $\alpha \br{\widetilde{f}} = f$.
\end{proof}

\begin{lema}
Naj bosta $V$ in $W$ končnorazsežna $\C G$-modula. Tedaj je
$\Hom_\C(V, W)^G = \Hom_{\C G}(V, W)$.
\end{lema}

\begin{proof}
Naj bo $\varphi \in \Hom_\C(V, W)$. Sledi, da je
\begin{align*}
\varphi \in \Hom_\C(V,W)^G &\iff
\forall g \in G \colon g \cdot \varphi = \varphi
\\
&\iff
\forall g \in G \colon \forall v \in V \colon
g \varphi(g^{-1} v) = \varphi(v)
\\
&\iff
\forall g \in G \colon \forall v \in V \colon
\varphi(g v) = g \varphi(v)
\\
&\iff
\varphi \in \Hom_{\C G}(V, W). \qedhere
\end{align*}
\end{proof}

\begin{lema}
Naj bo $V$ $\C G$-modul. Tedaj je preslikava $\pi \colon V \to V$ s
predpisom
\[
\pi = \frac{1}{\abs{G}} \sum_{g \in G} g
\]
homomorfizem $G$-modulov in projektor na $V^G$ in velja
\[
\tr \br{\frac{1}{\abs{G}} \sum_{g \in G}} = \dim V^G.
\]
\end{lema}

\begin{proof}
Očitno je $\pi$ linearna. Ker je
\begin{align*}
\pi(hv) &=
\br{\frac{1}{\abs{G}} \sum_{g \in G} g} hv
\\
&=
\br{\frac{1}{\abs{G}} \sum_{g \in G} gh} v
\\
&=
\pi(v)
\\
&=
\br{\frac{1}{\abs{G}} \sum_{g \in G} hg} v
\\
&=
h \pi(v),
\end{align*}
je tudi homomorfizem $G$-modulov. Seveda je $\pi(v) \in V^g$. Ker
je očitno $\eval{\pi}{V^G}{} = \id$, je to res projektor. Ker je
sled projektorja enaka dimenziji slike, velja tudi zgornja enakost.
\end{proof}

\begin{definicija}
Naj bosta $\varphi, \psi \colon G \to \C$ funkciji, konstantni na
konjugiranostnih razredih. Definiramo
\emph{skalarni produkt}\index{Upodobitev!Skalarni produkt}
\[
\skl{\varphi, \psi} =
\frac{1}{\abs{G}} \sum_{g \in G} \oline{\varphi(g)} \psi(g).
\]
\end{definicija}

\begin{trditev}
Za vse $\varphi, \psi, \varphi_1$ in $\varphi_2$ velja:

\begin{enumerate}[i)]
\item $\skl{\varphi, \varphi} \geq 0$,
\item $\skl{\varphi, \varphi} = 0 \iff \varphi = 0$,
\item $\skl{\varphi, \psi} = \oline{\skl{\psi, \varphi}}$,
\item $\skl{\varphi_1 + \varphi_2, \psi} =
\skl{\varphi_1, \psi} + \skl{\varphi_2, \psi}$ in
\item $\skl{\varphi_1 \varphi_2, \psi} =
\skl{\varphi_1, \oline{\varphi_2} \psi}$.
\end{enumerate}
\end{trditev}

\obvs

\begin{trditev}
Če sta $\chi$ in $\psi$ karakterja, velja
$\skl{\chi, \psi} = \skl{\psi, \chi}$.
\end{trditev}

\begin{proof}
Velja
\[
\skl{\chi, \psi} =
\frac{1}{\abs{G}} \sum_{g \in G} \oline{\chi(g)} \psi(g) =
\frac{1}{\abs{G}} \sum_{g \in G} \chi(g^{-1}) \psi(g) =
\frac{1}{\abs{G}} \sum_{g \in G} \chi(g) \psi(g^{-1}) =
\skl{\psi, \chi}. \qedhere
\]
\end{proof}

\begin{trditev}
Veljata naslednji trditvi:

\begin{enumerate}[i)]
\item Če je $\chi$ karakter nerazcepne upodobitve $G$, je
$\skl{\chi, \chi} = 1$.
\item Če sta $\chi$ in $\psi$ karakterja neizomorfnih upodobitev
$G$, je $\skl{\chi, \psi} = 0$.
\end{enumerate}
\end{trditev}

\begin{proof}
Naj bosta $V$ in $W$ pripadajoča $\C G$-modula. Na elemente $g$
glejmo kot na delovanje na prostoru $\Hom_\C(V,W)$. Po Schurovi
lemi je
\[
\tr \br{\frac{1}{\abs{G}} \sum_{g \in G} g} =
\dim \Hom_\C(V, W)^G =
\dim \Hom_{\C G}(V, W) =
\begin{cases}
0, & V \not \cong W, \\
1, & V \cong W.
\end{cases}
\]
Ker je
\[
\skl{\chi, \psi} =
\frac{1}{\abs{G}} \sum_{g \in G} \br{\oline{\chi} \psi}(g)
\]
in je $\oline{\chi} \psi$ karakter $\Hom_\C(V, W)$, je
\[
\skl{\chi, \psi} =
\tr \br{\frac{1}{\abs{G}} \sum_{g \in G} g}. \qedhere
\]
\end{proof}

\begin{posledica}
Naj bo
\[
V = \bigoplus_{i=1}^r S_i^{n_i}
\]
končnorazsežen $\C G$-modul, pri čemer so $S_i$ paroma neizomorfni
enostavni $\C G$-moduli. Tedaj je
\[
n_i = \skl{\chi_V, \chi_i},
\]
kjer je $\chi_i$ karakter upodobitve $S_i$.
\end{posledica}

\begin{proof}
Velja
\[
\chi_V = \sum_{i=1}^r n_i \chi_i. \qedhere
\]
\end{proof}

\begin{posledica}
Naj bosta $V$ in $W$ končnorazsežna $\C G$-modula. Tedaj je
$V \cong W$ natanko tedaj, ko je $\chi_V = \chi_W$.
\end{posledica}

\begin{proof}
Uporabimo prejšnjo posledico.
\end{proof}

\begin{posledica}
Če je $V$ končnorazsežen $\C G$-modul, za njegov karakter $\chi$
velja $\skl{\chi, \chi} = 1$ natanko tedaj, ko je enostaven.
\end{posledica}

\begin{proof}
Razpišemo lahko
\[
\skl{\chi, \chi} = \sum_{i=1}^r n_i^2. \qedhere
\]
\end{proof}
