\section{Teorija kategorij}

\epigraph{">Ups, niste videli..."<}{-- prof.~dr.~Primož Moravec}

\subsection{Definicija, izomorfizmi, začetni in končni objekti}

\datum{2023-4-14}

\begin{definicija}
\emph{Kategorija}\index{Kategorija} $\cat{C}$ je struktura, ki ima

\begin{itemize}
\item razred objektov $\Ob \cat{C}$,
\item za vsaka $A, B \in \Ob \cat{C}$ množico morfizmov
$\cat{C}(A, B)$,
\item operacijo komponiranja
$\circ \colon \cat{C}(A,B) \times \cat{C}(B,C) \to \cat{C}(A,C)$,
pri čemer je komponiranje asociativno in unitalno.\footnote{Za vsak
$A \in \cat{C}$ obstaja $1_A \in \cat{C}(A,A)$, za katerega je
$f \circ 1_A = f$ in $1_A \circ g$.}
\end{itemize}
\end{definicija}

\begin{definicija}
Naj bo $\cat{C}$ kategorija in $f \in \cat{C}(A,B)$ morfizem. $f$
je \emph{izomorfizem}\index{Kategorija!Izomorfizem}, če ima inverz
$g \in \cat{C}(B,A)$, za katerega je
\[
f \circ g = 1_B
\quad \text{in} \quad
g \circ f = 1_A.
\]
Objekta $A$ in $B$ kategorije $\cat{C}$ sta \emph{izomorfna}, če
v $\cat{C}(A,B)$ obstaja izomorfizem. Pišemo $A \cong B$.
\end{definicija}

\begin{opomba}
Inverz je enolično določen.
\end{opomba}

\begin{definicija}
Morfizem $f \in \cat{C}(A,B)$ je
\index{Kategorija!Prerez, retrakt}
\index{Kategorija!Monomorfizem, epimorfizem}

\begin{enumerate}[i)]
\item \emph{prerez}, če ima levi inverz,
\item \emph{retrakt}, če ima desni inverz,
\item \emph{monomorfizem}, če za vsaka $g, h \in \cat{C}(C,A)$ iz
\[
f \circ g = f \circ h
\]
sledi $g = h$,
\item \emph{epimorfizem}, če za vsaka $g, h \in \cat{C}(B,C)$ iz
\[
g \circ f = h \circ f
\]
sledi $g = h$.
\end{enumerate}
\end{definicija}

\begin{definicija}
Objekt $Z \in \Ob \cat{C}$ je
\emph{začeten objekt}\index{Kategorija!Začetni, končni objekt}, če
za vsak objekt $C \in \Ob \cat{C}$ obstaja natanko en morfizem v
$\cat{C}(Z,C)$.

Objekt $K \in \Ob \cat{C}$ je končen objekt, če za vsak objekt
$C \in \Ob \cat{C}$ obstaja natanko en morfizem v $\cat{C}(C,K)$.
\end{definicija}

\begin{trditev}
Poljubna začetna objekta v $\cat{C}$ sta izomorfna.
\end{trditev}

\begin{proof}
Naj bosta $Z_1$ in $Z_2$ začetna objekta. Sledi, da je $1_{Z_1}$
edini morfizem v $\cat{C}(Z_1, Z_1)$. Obstajata še morfizma
$f \in \cat{C}(Z_1, Z_2)$ in $g \in \cat{C}(Z_2, Z_1)$. Ker je
$g \circ f \in \cat{C}(Z_1, Z_1)$, je $g \circ f = 1_{Z_1}$.
Simetrično dobimo $f \circ g = 1_{Z_2}$.
\end{proof}

\begin{opomba}
Enako velja za končne objekte.
\end{opomba}

\begin{definicija}
Naj bo $\cat{C}$ kategorija.
\emph{Dualna kategorija}\index{Kategorija!Dualna}
$\cal{C}^{\mathsf{op}}$ kategorije $\cat{C}$ je kategorija, v
kateri je

\begin{enumerate}[i)]
\item $\Ob \cat{C}^{\mathsf{op}} = \Ob \cat{C}$,
\item $	\cat{C}^{\mathsf{op}}(A,B) = \cat{C}(B,A)$ in
\item za komponiranje $*$ velja $g * f = f \circ g$.\footnote{Tu
$\circ$ označuje komponiranje v $\cat{C}$.}
\end{enumerate}
\end{definicija}

\begin{opomba}
Morfizem $f$ je monomorfizem v $\cat{C}$ natanko tedaj, ko je
$f$ epimorfizem v $\cat{C}^{\mathsf{op}}$.
\end{opomba}

\begin{opomba}
Objekt $Z$ je začetni objekt v $\cat{C}$ natanko tedaj, ko je
končen objekt v $\cat{C}^{\mathsf{op}}$.
\end{opomba}

\newpage

\subsection{Funktorji in naravne transformacije}

\datum{2023-4-21}

\begin{definicija}
Naj bosta $\cat{C}$ in $\cat{D}$ kategoriji.
\emph{Funktor}\index{Funktor} $F$ iz $\cat{C}$ v $\cat{D}$ je
predpis, sestavljen iz

\begin{enumerate}[i)]
\item predpisa $\Ob \cat{C} \to \Ob \cat{D}$, $A \mapsto F(A)$,
\item za vsaka objekta $A$ in $B$ iz $\cat{C}$ imamo predpis
$\cat{C}(A,B) \to \cat{D}(F(A), F(B))$, $f \mapsto F(f)$.
\end{enumerate}

Pri tem zahtevamo $F(1_A) = 1_{F(A)}$ za vsak $A \in \Ob \cat{C}$
in $F(f \circ g) = F(f) \circ F(g)$ za vse smiselne $f$ in $g$.
\end{definicija}

\begin{definicija}
Naj bosta $\cat{C}$ in $\cat{D}$ kategoriji.
\emph{Kontravariantni funktor}\index{Kategorija!Kofunktor}
(\emph{kofunktor}) $F \colon \cat{C} \to \cat{D}$ je predpis,
sestavljen iz

\begin{enumerate}[i)]
\item predpisa $\Ob \cat{C} \to \Ob \cat{D}$, $A \mapsto F(A)$,
\item za vsaka objekta $A$ in $B$ iz $\cat{C}$ imamo predpis
$\cat{C}(A,B) \to \cat{D}(F(B), F(A))$, $f \mapsto F(f)$.
\end{enumerate}

Pri tem zahtevamo $F(1_A) = 1_{F(A)}$ za vsak $A \in \Ob \cat{C}$
in $F(f \circ g) = F(g) \circ F(f)$ za vse smiselne $f$ in $g$.
\end{definicija}

\begin{opomba}
Kofunktor $F \colon \cat{C} \to \cat{D}$ je funktor
$F \colon \cat{C}^{\mathsf{op}} \to \cat{D}$.
\end{opomba}

\begin{opomba}
Funktorji ohranjajo izomorfizme, prereze in retrakte, ne pa nujno
monomorfizmov, epimorfizmov, začetnih in končnih objektov.
\end{opomba}

\begin{definicija}
Naj bosta $F, G \colon \cat{C} \to \cat{D}$ funktorja.
\emph{Naravna transformacija}\index{Kategorija!Naravna transformacija}
$\mu \colon F \to G$ je nabor morfizmov
$\mu_C \colon F(C) \to G(C)$ za $C \in \cat{C}$, za katere
diagram
\[
\begin{tikzcd}[column sep=large, row sep=large]
F(C) \arrow[r, "F(f)"] \arrow[d, "\mu_C"'] &
F(D) \arrow[d, "\mu_D"] \\
G(C) \arrow[r, "G(f)"'] & G(D)
\end{tikzcd}
\]
komutira za vse $f \in \cat{C}(C,D)$.
\end{definicija}

\newpage

\subsection{Univerzalne konstrukcije}

\begin{definicija}
Naj bosta $A$ in $B$ objekta kategorije $\cat{C}$.
\emph{Produkt}\index{Kategorija!Produkt} $A$ in $B$ je tak objekt
$C$ z morfizmoma $p \colon C \to A$ in $q \colon C \to B$, da za
vsak $X \in \Ob \cat{C}$ in morfizma $f \colon X \to A$ ter
$g \colon X \to B$ obstaja natanko en morfizem $h \colon X \to C$,
da velja $p \circ h = f$ in $q \circ h = g$.
\[
\begin{tikzcd}[column sep=large, row sep=large]
A & C \arrow[l, "p"'] \arrow[r, "q"] & B \\
  & X \arrow[ul, "f"] \arrow[ur, "g"']
\arrow[u, dashrightarrow, "h"]
\end{tikzcd}
\]
\end{definicija}

\begin{trditev}
Naj bosta $A$ in $B$ objekta kategorije $\cat{C}$ s produktoma
\[
\begin{tikzcd}
A & C \arrow[l, "p"'] \arrow[r, "q"] & B
\end{tikzcd}
\]
in
\[
\begin{tikzcd}
A & C' \arrow[l, "p'"'] \arrow[r, "q'"] & B.
\end{tikzcd}
\]
Tedaj je $C \cong C'$.
\end{trditev}

\begin{proof}
Obstaja morfizem $h \colon C' \to C$, za katerega je
$p \circ h = p'$ in $q \circ h = q'$. Podobno obstaja morfizem
$h' \colon C \to C'$, za katerega je $p' \circ h' = p$ in
$q' \circ h' = q$. Sledi, da je
\[
p \circ (h \circ h') = p \circ \id_C
\quad \text{in} \quad
q \circ (h \circ h') = q \circ \id_C
\]
Po enoličnosti dobimo $h \circ h' = \id_C$ in simetrično
$h' \circ h = \id_{C'}$.
\end{proof}

\begin{opomba}
Produkte v kategoriji $\cat{C}$ lahko definiramo tudi tako, da
za fiksna $A, B \in \Ob \cat{C}$ definiramo kategorijo $\cat{D}$,
ki ima za objekte diagrame
\[
\begin{tikzcd}
A & C \arrow[l, "p"'] \arrow[r, "q"] & B
\end{tikzcd}
\]
in morfizme definirane na naraven način. Produkt $A$ in $B$ je
končen objekt v kategoriji $\cat{D}$.
\end{opomba}

\begin{definicija}
Naj bosta $A$ in $B$ objekta kategorije $\cat{C}$.
\emph{Koprodukt}\index{Kategorija!Koprodukt} $A$ in $B$ je produkt
$A$ in $B$ v $\cat{C}^{\mathsf{op}}$.
\end{definicija}

\begin{opomba}
Ekvivalentno, koprodukt $A$ in $B$ je tak objekt $C$ z morfizmoma
$p \colon A \to C$ in $q \colon B \to C$, da za vsak
$X \in \Ob \cat{C}$ in morfizma $f \colon A \to X$ ter
$g \colon B \to X$ obstaja natanko en morfizem $h \colon C \to X$,
da velja $h \circ p = f$ in $h \circ q = g$.
\[
\begin{tikzcd}[column sep=large, row sep=large]
A \arrow[r, "p"] \arrow[dr, "f"'] &
C \arrow[d, dashrightarrow, "h"] &
B \arrow[l, "q"'] \arrow[dl, "g"] \\
  & X
\end{tikzcd}
\]
\end{opomba}

\begin{definicija}
Kategorija $\cat{C}$ je
\emph{konkretna}\index{Kategorija!Konkretna}, če imajo njeni
objekti strukturo množice, morfizmi pa so preslikave.
\end{definicija}

\begin{definicija}
Naj bo $\cat{C}$ konkretna kategorija in $X$ množica.
\emph{Prost objekt}\index{Kategorija!Prost objekt} v $\cat{C}$ nad
množico $X$ je objekt $F_X$ skupaj s strukturno preslikavo
$\iota \colon X \to F_X$, za katera velja, da za vsak objekt
$C \in \Ob \cat{C}$ in preslikavo $j \colon X \to C$ obstaja
natanko en morfizem $f \colon F_x \to C$, da je
$f \circ \iota = j$.
\[
\begin{tikzcd}[column sep = large, row sep=large]
X \arrow[r, "j"] \arrow[d, "\iota"'] & C \\
F_X \arrow[ur, dashrightarrow, "f"']
\end{tikzcd}
\]
\end{definicija}

\begin{trditev}
Prost objekt je enolično določen do izomorfizma natančno.
\end{trditev}

\begin{proof}
Naj bosta $F$ in $F'$ s preslikavama $\iota \colon X \to F$ in
$\iota \colon X \to F'$ prosta objekta. Sledi, da obstaja
preslikava $f \colon F \to F'$, za katero je
$f \circ \iota = \iota'$, in preslikava $f' \colon F' \to F$, za
katero je $f' \circ \iota' = \iota$. Sledi, da je
\[
(f' \circ f) \circ \iota = \id \circ \iota,
\]
zato je zaradi enoličnosti $f' \circ f = \id$. Simetrično dobimo
$f \circ f' = \id$.
\end{proof}

\begin{opomba}
Proste objekte v kategoriji $\cat{C}$ lahko definiramo tudi tako,
da definiramo kategorijo $\cat{D}$, ki ima za objekte diagrame
\[
\begin{tikzcd}
X \arrow[r, "\iota"] & A
\end{tikzcd}
\]
in morfizme definirane na naraven način. Prost objekt je začetni
objekt v kategoriji $\cat{D}$.
\end{opomba}

\newpage

\subsection{Izomorfizem in ekvivalenca kategorij}

\datum{2023-5-5}

\begin{definicija}
Funktor $F \colon \cat{C} \to \cat{D}$ je
\emph{izomorfizem kategorij}\index{Kategorija!Izomorfizem}, če
obstaja tak funktor $G \colon \cat{D} \to \cat{C}$, da je
\[
F \circ G = 1_{\cat{D}}
\quad \text{in} \quad
G \circ F = 1_{\cat{C}}.
\]
Pišemo $\cat{C} \cong \cat{D}$.
\end{definicija}

\begin{definicija}
Naj bosta $F, G \colon \cat{C} \to \cat{D}$ funktorja in
$\mu \colon F \to G$ naravna transformacija. Funktorja $F$ in $G$
sta
\emph{naravno izomorfna}\index{Funktor!Naravno izomorfen}, če so
vsi morfizmi $\mu_C \colon F(C) \to G(C)$ izomorfizmi.
\end{definicija}

\begin{definicija}
Kategorija $\cat{C}$ je \emph{majhna}\index{Kategorija!Majhna}, če
sta $\Ob \cat{C}$ in razred vseh morfizmov množici.
\end{definicija}

\begin{opomba}
Naj bo $\cat{C}$ majhna kategorija in $\cat{D}$ poljubna
kategorija. Naj bo $\cat{D}^{\cat{C}}$ kategorija, katere objekti
so funktorji $F \colon \cat{C} \to \cat{D}$, morfizmi med $F$ in
$G$ pa naravne transformacije. Tedaj sta $F$ in $G$ naravno
izomorfna natanko tedaj, ko sta izomorfna kot objekta
$\cat{D}^{\cat{C}}$.
\end{opomba}

\begin{definicija}
Kategoriji $\cat{C}$ in $\cat{D}$ sta
\emph{ekvivalentni}\index{Kategorija!Ekvivalentna}, če obstajata
taka funktorja $F \colon \cat{C} \to \cat{D}$ in
$G \colon \cat{D} \to \cat{C}$, da je $F \circ G$ naravno izomorfen
$1_{\cat{D}}$, $G \circ F$ pa naravno izomorfen $1_{\cat{C}}$.
\end{definicija}

\begin{definicija}
Naj bo $F \colon \cat{C} \to \cat{D}$ funktor.

\begin{enumerate}[i)]
\item $F$ je \emph{zvest}\index{Funktor!Zvest, poln, gost}, če je
injektiven na morfizmih -- za vsaka $X, Y \in \Ob \cat{C}$ je
$F_{X,Y} \colon \cat{C}(X,Y) \to \cat{D}(F(X), F(Y))$ injektivna.
\item $F$ je \emph{poln}, če je surjektiven na morfizmih.
\item $F$ je \emph{gost}, če za vsak $Y \in \Ob \cat{D}$ obstaja
tak $X \in \Ob \cat{C}$, da je $F(X) \cong Y$.
\end{enumerate}
\end{definicija} 

\begin{opomba}
Pravimo, da je funktor $F$ \emph{ekvivalenca}, če je zvest, gost in
poln.
\end{opomba}

\begin{definicija}
Naj bo $\cat{C}$ majhna kategorija. Definirajmo funktor
$\mathcal{Y} \colon \cat{C} \to
\uline{\operatorname{Set}}^{\cat{C}^{\mathsf{op}}}$ na naslednji
način:

\begin{enumerate}[i)]
\item Vsak Objekt $A$ preslikamo v funktor
$F_A \colon \cat{C}^{\mathsf{op}} \to \uline{\operatorname{Set}}$,
ki deluje po predpisu $F_A(X) = \cat{C}(X,A)$ na objektih in
$F_A(f) = (\varphi \mapsto \varphi \circ f)$ na morfizmih.
\item Vsak morfizem $f \colon A \to B$ preslikamo v naravno
transformacijo $\mathcal{Y}(f) \colon F_A \to F_B$, z naborom
morfizmov $\mathcal{Y}(f)_X \colon \cat{C}(X, A) \to \cat{C}(X,B)$
s predpisom $\mathcal{Y}(f)_X(\varphi) = f \circ \varphi$.
\end{enumerate}
\end{definicija}

\begin{izrek}[Lema Yonede]\index{Izrek!Lema Yonede}
Naj bo $\cat{C}$ majhna kategorija. Tedaj je funktor
$\mathcal{Y} \colon \cat{C} \to
\uline{\operatorname{Set}}^{\cat{C}^{\mathsf{op}}}$ zvest in poln.
\end{izrek}

\begin{proof}
Naj bosta $A$ in $B$ objekta kategorije $\cat{C}$. Denimo, da za
$f, g \colon A \to B$ velja $\mathcal{Y}(f) = \mathcal{Y}(g)$.
Sledi, da je
\[
f =
f \circ 1_{A} =
\mathcal{Y}(f)_A(1_A) =
\mathcal{Y}(g)_A(1_A) =
g \circ 1_A =
g.
\]
Sledi, da je $F$ zvest.

Pokažimo še polnost. Naj bo $\eta \colon F_A \to F_B$ naravna
transformacija. Naj bo $e = \eta_A(1_A) \in \cat{C}(A,B)$.
Pokažimo, da je $\eta = \mathcal{Y}(e)$. Za poljuben morfizem
$f \colon C \to A$ vemo, da komutira naslednji diagram:
\[
\begin{tikzcd}[column sep=large, row sep=large]
\cat{C}(A,A) \arrow[r, "F_A(f)"] \arrow[d, "\eta_A"'] &
\cat{C}(C,A) \arrow[d, "\eta_C"] \\
\cat{C}(A,B) \arrow[r, "F_B(f)"'] &
\cat{C}(C,B)
\end{tikzcd}
\]
Opazimo, da je
\[
F_B(f) \circ \eta_A(1_A) =
F_B(f)(e) =
e \circ f =
\mathcal{Y}(e)_C(f),
\]
po drugi strani pa je zaradi komutativnega diagrama
\[
F_B(f) \circ \eta_A(1_A) =
\eta_C \circ F_A(f)(1_A) =
\eta_C(f). \qedhere
\]
\end{proof}

\begin{opomba}
Ekvivalentno, vsako majhno kategorijo $\cat{C}$ lahko vložimo v
kategorijo kovariantnih funktorjev
$\cat{C} \to \uline{\operatorname{Set}}$.
\end{opomba}

\begin{posledica}[Cayleyev izrek]\index{Izrek!Cayley}
Vsaka grupa $G$ je izomorfna podgrupi grupe $\Sym_G$.
\end{posledica}

\begin{proof}
Grupi $G$ priredimo kategorijo $\uline{G}$, katere element je $G$,
morfizmi pa njeni elementi. Po lemi Yonede je preslikava
$x \mapsto \mathcal{Y}(x)$ injektivna. Za preslikavo
$\rho \colon G \to G^G$, ki deluje po predpisu
$\rho(x) = \mathcal{Y}(x)_G$, zato velja $\rho(x) \ne \rho(y)$ za
vse $x \ne y$. Injektivna je torej tudi preslikava $\rho$. Ni težko
preveriti, da je $\rho \colon G \to \Sym_G$ homomorfizem grup.
\end{proof}
