\section{Galoisova teorija}

\epigraph{">Jezik se mi zapleta. Ne vem, ali bi moral več spit ali
manj spit."<}{-- prof.~dr.~Primož Moravec}

\subsection{Normalne in separabilne razširitve}

\datum{2023-2-17}

\begin{definicija}
Enačba je \emph{rešljiva z radikali}\index{Enačba!Rešljiva}, če
lahko rešitev izrazimo z operacijami $+$, $-$, $\cdot$, $\div$,
$\sqrt[n]{}$ iz podanih parametrov.
\end{definicija}

\begin{definicija}
Polje $E$ je
\emph{radikalska razširitev}\index{Razširitev polj!Radikalska}
polja $F$, če obstaja tak $a \in F$, da je $E = F(\sqrt[n]{a})$ za
nek $n \in \N$.
\end{definicija}

\begin{opomba}
Polinomska enačba $p(X) = 0$ za $p(X) \in F[X]$ je rešljiva z
radikali natanko tedaj, ko obstaja veriga
\[
F \subseteq F_1 \subseteq F_2 \subseteq \dots \subseteq F_k = F(p)
\]
radikalskih razširitev.
\end{opomba}

\begin{definicija}
Razširitev $E$ polja $F$ je
\emph{normalna}\index{Razširitev polj!Normalna}, če za vsak
nerazcepen polinom $p(X) \in F[X]$ velja, da $E$ vsebuje bodisi vse
bodisi nobene ničle polinoma $p$.
\end{definicija}

\begin{izrek}
Naj bo $E/F$ končna razširitev. Tedaj je $E/F$ normalna natanko
tedaj, ko je $E$ razpadno polje nekega polinoma $p(X) \in F[X]$.
\end{izrek}

\begin{proof}
Predpostavimo, da je $E/F$ normalna. Naj bo
$E = F(a_1, \dots, a_r)$ in naj bo $p_i$ minimalni polinom za
$a_i$. Naj bo
\[
p(X) = \prod_{i=1}^r p_i(X).
\]
Ker je $p_i$ nerazcepen, so vse ničle polinoma $p_i$ vsebovane v
$E$ za vsak $i$. Posledično so tudi vse ničle polinoma $p$
vsebovane v $E$. Sledi, da je $F(p) \subseteq E$. Ker je očitno
$E \subseteq F(p)$, je $E = F(p)$.

Predpostavimo sedaj, da je $E = F(p)$ za polinom $p$. Naj bo
$q$ poljuben nerazcepen polinom z ničlo $a \in E$. Naj bo $b$
poljubna ničla polinoma $q$. Opazimo, da je $q$ minimalen polinom
za $a$ in $b$, zato je $F(a) \sim F(b)$ z izomorfizmom
$\sigma \colon a \mapsto b$. Sledi, da obstaja izomorfizem
$\tau \colon (F(a))(p) \to (F(b))(p)$, ki se na $F(a)$ ujema s
$\sigma$:
\[
\begin{tikzcd}
& {F(a)} & {(F(a))(p) = F(a)(b_1, \dots, b_r)} \\
F \\
& {F(b)} & {(F(b))(p) = F(b)(b_1, \dots, b_r)}
\arrow[hook, from=2-1, to=1-2]
\arrow[hook, from=2-1, to=3-2]
\arrow["\sigma", from=1-2, to=3-2]
\arrow[hook, from=1-2, to=1-3]
\arrow[hook, from=3-2, to=3-3]
\arrow["\tau", from=1-3, to=3-3]
\end{tikzcd}
\]
Ker je $a \in F(p)$, je $a$ racionalna funkcija ničel polinoma $p$,
zato je tudi $b = \sigma(a)$ racionalna funkcija ničel polinoma
$p$, zato je $b \in F(p)$.
\end{proof}

\begin{definicija}
Polinom $p(X) \in F[X]$ je
\emph{separabilen}\index{Polinom!Separabilen}, če ima same
enostavne ničle.
\end{definicija}

\begin{definicija}
Končna razširitev $E/F$ je
\emph{separabilna}\index{Razširitev polj!Separabilna}, če je za
vse $a \in E$ minimalni polinom elementa $a$ separabilen.
\end{definicija}

\begin{izrek}[O primitivnem elementu]
\index{Izrek!O primitivnem elementu}
\label{gal:iz:1}
Naj bo $\chr F = 0$. Če je $E/F$ končna razširitev, je enostavna.
\end{izrek}

\datum{2023-2-24}

\begin{proof}
Dovolj je pokazati, da za poljubna $a, b \in E$ velja
$F(a,b) = F(a + \lambda b)$ za nek $\lambda \in F$. Naj bosta $p$
in $q$ minimalna polinoma elementov $a$ in $b$ zaporedoma. Naj bodo
njune ničle zaporedoma $a_1, a_2, \dots, a_m$ in
$b_1, b_2, \dots b_n$, kjer je $a = a_1$ in $b = b_1$. Ker je
$\chr F = 0$, so ničle posameznega polinoma paroma različne.

Naj bo
\[
\lambda \in F \bigsetminus \setb{\frac{a-a_i}{b_j-b}}{j > 1}.
\]
Sedaj definiramo
$\tilde{p}(X) = p(a+\lambda b-\lambda X) \in F(a + \lambda b)[X]$.
Opazimo, da je $b$ edina skupna ničla polinomov $\tilde{p}$ in $q$.
Sledi, da je $\gcd(\tilde{p},q) = X - b$. Po Bezoutovi lemi
obstajata taka polinoma $r,s \in F(a+\lambda b)[X]$, da velja
\[
r \tilde{p} + sq = X-b,
\]
zato je tudi $X-b \in F(a+\lambda b)[X]$. Sledi, da je
$b \in F(a+\lambda b)$ in posledično $a \in F(a+\lambda b)$.
\end{proof}

\begin{izrek}[O primitivnem elementu]
\index{Izrek!O primitivnem elementu}
Če je $E/F$ končna separabilna razširitev, je enostavna.
\end{izrek}

\begin{proof}
Če je $F$ neskončno polje, lahko naredimo enak razmislek kot pri
dokazu izreka~\ref{gal:iz:1}.

Naj bo $F$ sedaj končno polje. Sledi, da je tudi
$E = F(a_1, \dots, a_k)$ končno. Sledi, da je
$(E \setminus \set{0}, \cdot)$ ciklična grupa -- naj bo $a$ njen
generator. Očitno je tedaj $E = F(a)$.
\end{proof}

\newpage

\subsection{Galoisova grupa}

\begin{definicija}
Avtomorfizem $\sigma$ polja $E$ je
\emph{$F$-avtomorfizem}\index{Preslikava!$F$-avtomorfizem}, če je
$\eval{\sigma}{F}{} = \id$. Množici $F$-avtomorfizmov pravimo
\emph{Galoisova grupa}\index{Grupa!Galoisova} razširitve $E/F$ in
jo označimo z $\Gal(E/F)$.
\end{definicija}

\begin{opomba}
Galoisova grupa $\Gal(E/F)$ je podgrupa grupe $\Aut E$.
\end{opomba}

\begin{lema}
Naj bo $E/F$ razširitev polja. Naj bo $a \in E$ ničla polinoma
$p \in F[X]$. Potem je za poljuben $\sigma \in \Gal(E/F)$ tudi
$\sigma(a)$ ničla polinoma $p$.
\end{lema}

\obvs

\begin{zgled}
Velja $\Gal(\C/\R) \cong \kvoc{\Z}{2\Z}$. Velja namreč
$\Gal(\C/\R) = \set{\id, z \mapsto \oline{z}}$.
\end{zgled}

\begin{zgled}
Grupa $\Gal(\Q(\sqrt[3]{2})/\Q)$ je trivialna.
\end{zgled}

\begin{trditev}
Če je $E/F$ končna normalna separabilna razširitev, je
\[
\abs{\Gal(E/F)} = [E : F].
\]
\end{trditev}

\begin{proof}
Obstaja tak $a \in E$, da je $E = F(a)$. Naj bo $p$ minimalen
polinom za $a$. Vse ničle polinoma $p$ so enostavne in vsebovane v
$E$, velja pa $\deg p = [E : F]$. Da določimo
$\sigma \in \Gal(E/F)$, je dovolj določiti $\sigma(a)$, za
kar imamo natanko $\deg p$ možnosti -- za poljubno ničlo $b$
polinoma $p$ preslikava $a \mapsto b$ inducira izomorfizem
$E = F(a) \to F(b) \cong E$.
\end{proof}

\begin{definicija}
Naj bo $p \in F[X]$.
\emph{Galoisova grupa}\index{Grupa!Galoisova} polinoma $p$ je
grupa
\[
\Gal(p) = \Gal(F(p)/F).
\]
\end{definicija}

\begin{opomba}
Preslikava $\sigma \in \Gal(p)$ je permutacija ničel
$a_1, \dots, a_k$ polinoma $p$, zato jo lahko vložimo v $S_k$.
\end{opomba}

\newpage

\subsection{Galoisova korespondenca}

\begin{definicija}
Naj bo $G$ podgrupa v $\Gal(E/F)$.
\emph{Fiksno polje}\index{Polje!Fiksno} glede na grupo $G$ je
množica
\[
E^G = \setb{a \in E}{\forall \sigma \in G \colon \sigma(a) = a}.
\]
\end{definicija}

\begin{opomba}
Če so $F \subseteq L \subseteq E$ polja, je $\Gal(E/L)$ podgrupa v
$\Gal(E/F)$.
\end{opomba}

\datum{2023-3-3}

\begin{lema}
Naj bodo $F \subseteq L \subseteq E$ polja.

\begin{enumerate}[i)]
\item Če je $E/F$ končna razširitev, je taka tudi $E/L$.
\item Če je $E/F$ normalna razširitev, je taka tudi $E/L$.
\item Če je $E/F$ separabilna razširitev, je taka tudi $E/L$.
\end{enumerate}
\end{lema}

\begin{proof}
\phantom{a}
\begin{enumerate}[i)]
\item Ker je $E$ končnorazsežen vektorski prostor nad $F$, je
končnorazsežen tudi nad poljem $L$.
\item Naj bo $a \in E$ ničla nerazcepnega polinoma $p(x) \in L[x]$
in naj bo $q$ minimalni polinom $a$ nad $F$. Opazimo, da velja
$\gcd(p, q) = p$, saj je nekonstanten. Tako dobimo $p \mid q$. Ker
so zaradi normalnosti razširitve $E/F$ vse ničle polinoma $q$
vsebovane v $E$, enako velja za ničle $p$.
\item Podobno kot pri drugi točki iz separabilnosti polinoma $q$
sledi separabilnost minimalnega polinoma $p$ za $a \in E$. \qedhere
\end{enumerate}
\end{proof}

\begin{definicija}
Normalnim separabilnim razširitvam pravimo
\emph{Galoisove razširitve}\index{Galoisova razširitev}.
\end{definicija}

\begin{lema}
Naj bo $E/F$ končna Galoisova razširitev. Tedaj je
$E^{\Gal(E/F)} = F$.
\end{lema}

\begin{proof}
Naj bo $a \in E \setminus F$ in $p$ minimalni polinom za $a$.
Dovolj je dokazati, da obstaja izomorfizem $\sigma \in \Gal(E/F)$,
za katerega je $\sigma(a) \ne a$. Ker je $a \not \in F$, je
$\deg p > 1$. Zaradi separabilnosti ima $p$ tako še ničlo
$b \ne a$. Ker imata $a$ in $b$ skupni minimalni polinom, obstaja
izomorfizem $T \colon F(a) \to F(b)$, za katerega je $F(a) = b$.

Ker je polje $F(p)$ razpadno polje nad $F(a)$ in $F(b)$, se
izomorfizem $T$ razširi\footnote{Algebra 2.} do avtomorfizma
$\widehat{T}$ polja $F(p)$. Ker pa je tudi $E / F(p)$ normalna, se
$\widehat{T}$ razširi do izomorfizma $\sigma$ polja $E$.
\[
\begin{tikzcd}[row sep=large, column sep=large]
&
F(a)
\arrow[r, hookrightarrow] \arrow[dd, "\cong", "T"']
&
F(a)(p) = F(p)
\arrow[r, hookrightarrow] \arrow[dd, "\cong", "\widehat{T}"']
&
E
\arrow[dd, "\cong", "\sigma"']
\\
F
\arrow[ur, hookrightarrow] \arrow[dr, hookrightarrow]
\\
&
F(b)
\arrow[r, hookrightarrow]
&
F(b)(p) = F(p)
\arrow[r, hookrightarrow]
&
E
\end{tikzcd}
\]
Pri tem velja $\sigma(a) = b \ne a$, zato je
$a \not \in E^{\Gal(E/F)}$.
\end{proof}

\begin{izrek}[Fundamentalni Galoisove teorije]
\index{Izrek!Fund.~Galoisove teorije}
Naj bo $E/F$ končna Galoisova razširitev.

\begin{enumerate}[i)]
\item Korespondenca $L \mapsto \Gal(E/L)$, $G \mapsto E^G$ med
vmesnimi polji in podgrupami je bijektivna, ti preslikavi pa sta si
inverzni.
\item Za $F \subseteq L \subseteq M \subseteq E$ velja
\[
[\Gal(E/L) : \Gal(E/M)] = [M : L].
\]
\item Za $F \subseteq L \subseteq E$ je $L/F$ normalna razširitev
natanko tedaj, ko je $\Gal(E/L) \edn \Gal(E/F)$. V tem primeru je
\[
\kvoc{\Gal(E/F)}{\Gal(E/L)} \cong \Gal(L/F).
\]
\end{enumerate}
\end{izrek}

\begin{proof}
\phantom{a}
\begin{enumerate}[i)]
\item Naj bo $L$ vmesna razširitev polja $F$. Sledi, da je tudi
$E/L$ Galoisova razširitev. Sledi, da je $E^{\Gal(E/L)} = L$.

Naj bo sedaj $G \leq \Gal(E/F)$. Naj bo $\sigma \in G$ in
$a \in E^G$. Po definiciji je $\sigma(a) = a$, zato je
$\sigma \in \Gal(E / E^G)$ in posledično $G \leq \Gal(E/E^G)$.
Pokažimo, da je $\abs{\Gal(E/E^G)} \leq \abs{G}$.

Ker so je $E/F$ končna separabilna razširitev, obstaja tak
$a \in E$, da je $E = F(a)$. Sedaj definiramo
\[
p(x) = \prod_{T \in G} (x - T(a)).
\]
Ni težko opaziti, da za vsak $\sigma \in G$ velja
\[
\sigma(p) = \prod_{T \in G} (X - \sigma(T(a))) = p,
\]
zato je $p \in E^G[x]$. Sledi, da minimalni polinom $q$ za $a$ nad
$E^G$ deli $p$. Tako dobimo
\[
\abs{\Gal(E/E^G)} = [E : E^G] = \deg q \leq \deg p = \abs{G}.
\]
\item Ker velja $[E : L] = [E : M] \cdot [M : L]$, sledi
\[
[M : L] = \frac{[E : L]}{[E : M]} =
\frac{\abs{\Gal(E/L)}}{\abs{\Gal(E/M)}}.
\]
\item Naj bo $a$ ničla minimalnega polinoma $p \in F[x]$. Po dokazu
prejšnje leme vidimo, da za vsako ničlo $b$ polinoma $p$ obstaja
izomorfizem $\sigma \in \Gal(E/F)$, za katerega je $\sigma(a) = b$.
Poleg tega je za vsak $\sigma \in \Gal(E/F)$ element $\sigma(a)$
ničla polinoma $p$, zato $\sigma(a)$ preteče natanko vse ničle
polinoma $p$.

Pokažimo, da je $L/F$ normalna razširitev natanko tedaj, ko je
$\sigma(L) = L$ za vsak $\sigma \in \Gal(E/F)$. Če je $L$ normalna,
lahko zapišemo $L = F(p) = F(a_1, \dots, a_n)$. Ker je za vsak
$\sigma \in \Gal(E/F)$ element $\sigma(a_i)$ ničla minimalnega
polinoma elementa $a_i$, je $\sigma(a_i) \in L$ zaradi normalnosti.
Tako je res $\sigma(L) \subseteq L$ in zato $\sigma(L) = L$ zaradi
enakosti dimenzij. Če je $\sigma(L) = L$, pa za vsaki ničli $a$ in
$b$ nerazcepnega polinoma $p \in F[x]$ obstaja izomorfizem
$\sigma$, za katerega je $\sigma(a) = b$. Če $L$ vsebuje ničlo $a$,
vsebuje torej tudi vse ostale ničle.

Pokažimo sedaj, da velja
$\Gal(E/\sigma(L)) = \sigma \cdot \Gal(E/L) \cdot \sigma^{-1}$.
Res, velja
\begin{align*}
\tau \in \Gal(E/\sigma(L)) &\iff
\forall a \in L \colon \tau(\sigma(a)) = \sigma(a)
\\
&\iff
\forall a \in L \colon \sigma^{-1}(\tau(\sigma(a))) = a
\\
&\iff
\sigma^{-1} \tau \sigma \in \Gal(E/L).
\end{align*}

Sedaj lahko dokažemo trditev. Velja namreč
\begin{align*}
\text{$E/F$ je normalna razširitev} &\iff
\forall \sigma \in \Gal(E/F) \colon \sigma(L) = L
\\
&\iff
\forall \sigma \in \Gal(E/F) \colon
\Gal(E/L) = \sigma \Gal(E/L) \sigma^{-1}
\\
&\iff
\Gal(E/L) \edn \Gal(E/F).
\end{align*}

Denimo, da je $L$ res normalna razširitev. Naj bo
$\Phi \colon \Gal(E/F) \to \Gal(L/F)$ homomorfizem s predpisom
$\sigma \mapsto \eval{\sigma}{L}{}$. Ker je $E/L$ separabilna,
lahko zapišemo $E = L(a)$. Poljuben izomorfizem
$\tau \in \Gal(L/F)$ lahko razširimo do izomorfizma
$\sigma \in \Gal(E/F)$ tako, da definiramo $\sigma(a) = a$. V tem
primeru je seveda $\Phi(\sigma) = \tau$, zato je $\Phi$
surjektiven. Očitno je $\ker \Phi = \Gal(E/L)$, zato po izreku o
izomorfizmu sledi
\[
\kvoc{\Gal(E/F)}{\Gal(E/L)} \cong \Gal(L/F). \qedhere
\]
\end{enumerate}
\end{proof}

\datum{2023-3-10}

\begin{opomba}
Naj bo $E/F$ končna razširitev in $H \leq \Gal(E/F)$ podgrupa,
$L = E^H$ pa vmesna razširitev. Najmanjšo normalno razširitev
$\widetilde{L}/F$, za katero je $L \subseteq \widetilde{L}$, dobimo
kot $\widetilde{L} = E^{\widetilde{H}}$, kjer je
\[
\widetilde{H} =
\bigcap_{\sigma \in \Gal(E/F)} \sigma H \sigma^{-1}.
\]
\end{opomba}

\begin{definicija}
Polju $\widetilde{L}$ pravimo
\emph{normalno zaprtje}\index{Polje!Normalno zaprtje} polja $L$.
\end{definicija}

\newpage

\subsection{Rešljivost grup}

\begin{definicija}
Grupa $G$ je \emph{rešljiva}\index{Grupa!Rešljiva}, če obstaja
končno zaporedje podgrup
\[
1 = G_0 \leq G_1 \leq \dots \leq G_k = G,
\]
za za katerega za vsak $i < k$ velja $G_i \edn G_{i+1}$ in je
$\kvoc{G_{i+1}}{G_i}$ abelova.
\end{definicija}

\begin{trditev}
Naj bo $G$ rešljiva grupa.

\begin{enumerate}[i)]
\item Če je $H \leq G$, je tudi $H$ rešljiva.
\item Če je $N \edn G$, je tudi $\kvoc{G}{N}$ rešljiva.
\end{enumerate}
\end{trditev}

\begin{proof}
Po definiciji obstaja zaporedje
\[
1 = G_0 \leq G_1 \leq \dots \leq G_k = G,
\]
ki ustreza pogojem rešljivosti. Ni težko videti, da za zaporedje
$H_i = H \cap G_i$ velja $H_i \edn H_{i+1}$. Velja še
\[
\kvoc{G_{i+1} \cap H}{G_i \cap H} =
\kvoc{G_{i+1}}{G_{i+1} \cap H \cap G_i}.
\]
Opazimo, da je po drugem izreku o izomorfizmu
\[
\kvoc{G_{i+1}}{G_{i+1} \cap H \cap G_i} \cong
\kvoc{(G_{i+1} \cap H) G_i}{G_i} \leq
\kvoc{G_{i+1}}{G_i},
\]
zato so kvocienti abelovi. Sledi, da je $H$ rešljiva.

Naj bo sedaj $N \edn G$. Tedaj je
\[
1 = \kvoc{G_0 N}{N} \leq \kvoc{G_1 N}{N} \leq \dots \leq
\kvoc{G_k N}{N} = \kvoc{G}{N}.
\]
Opazimo, da velja $\kvoc{G_i N}{N} \edn \kvoc{G_{i+1} N}{N}$. Za
$a \in G_i$, $b \in G_{i+1}$ in $n, m \in N$ namreč velja
\[
(bmN)(anN)(bmN)^{-1} =
\underbrace{bmb^{-1}}_N
\underbrace{bab^{-1}}_{G_i}
\underbrace{b(nm^{-1})b^{-1}}_N N \in
\kvoc{G_i N}{N}.
\]
Opazimo še
\[
\kvoc{G_{i+1} N}{G_i N} =
\kvoc{G_{i+1} G_i N}{G_i N} \cong
\kvoc{G_{i+1}}{G_{i+1} \cap G_i N} \cong
\kvoc{\kvoc{G_{i+1}}{G_i}}{\kvoc{G_{i+1} \cap G_i N}{G_i}},
\]
kar je abelova grupa.
\end{proof}

\begin{opomba}
Grupa $S_5$ ni rešljiva, saj $A_5$ ni rešljiva (je enostavna in ni
abelova).
\end{opomba}

\begin{izrek}[Feit-Thompson]\index{Izrek!Feit-Thompson}
Vsaka končna grupa lihe moči je rešljiva.
\end{izrek}

\newpage

\subsection{Reševanje polinomskih enačb z radikali}

V nadaljevanju predpostavimo $\chr F = 0$.

\begin{definicija}
Naj bo $p(X) \in F[X]$. Enačba $p(X) = 0$ je
\emph{rešljiva z radikali}\index{Polinom!Rešljivost z radikali}, če
obstaja tako zaporedje polj
\[
F = E_0 \subseteq E_1 \subseteq \dots \subseteq E_n = E,
\]
da so razširitve $E_{i+1}/E_i$ radikalske in velja
$F(p) \subseteq E$.
\end{definicija}

\begin{lema}
Naj bo $F$ polje, ki vsebuje $n$-te korene enote in $a \in F$.
Tedaj je grupa $\Gal(X^n-a)$ ciklična.
\end{lema}

\begin{proof}
Naj bo $\omega \in F$ primitiven $n$-ti koren enote. Tedaj so vse
rešitve enačbe $X^n - a = 0$ ravno $b \omega^k$. Razpadno polje
polinoma $X^n - a$ je tako kar $F(b)$.

Naj bo $\sigma \in \Gal(X^n - a)$ in $\sigma(b) = b \omega^\ell$.
Opazimo, da je preslikava $\Gal(X^n - a) \to \Z_n$,
$\sigma \mapsto \ell$, injektiven homomorfizem.
\end{proof}

\begin{izrek}
Recimo, da je enačba $p(X) = 0$ za $p(X) \in F[X]$ rešljiva z
radikali. Potem je $\Gal_F(p)$ rešljiva grupa.
\end{izrek}

\begin{proof}
Naj bo
\[
F = E_0 \subseteq E_1 \subseteq \dots \subseteq E_m = E
\]
zaporedje iz definicije rešljivosti z radikali. Naj bo še
$a_i^{r_i} \in E_i$ za vsak $i$ in
\[
n = \prod_{i=1}^{m-1} r_i.
\]
Naj bo $\omega$ primitiven $n$-ti koren enote. Naj bo $\Omega$
Galoisova razširitev polja $F$, ki vsebuje $E$ in $\omega$. Naj bo
$\widetilde{E}$ normalno zaprtje $E(\omega)$ v $\Omega$. Opazimo,
da je
\[
\widetilde{E} =
F(\omega, a_1, \dots, a_{m-1}, \sigma_1(a_1), \dots).
\]
Tako lahko najdemo zaporedje
\[
F \subseteq F(\omega) \subseteq F(\omega, a_1) \subseteq
\dots \subseteq \widetilde{E}.
\]
Z Galoisovo korespondenco dobimo zaporedje
\[
\Gal \br{\widetilde{E}/\widetilde{E}} \subseteq \dots \subseteq
\Gal \br{\widetilde{E}/F(\omega, a_1)} \subseteq
\Gal \br{\widetilde{E}/F(\omega)} \subseteq
\Gal \br{\widetilde{E}/F}.
\]
Po zgornji lemi so kvocienti zaporednih grup ciklični. Sledi, da je
$\Gal \br{\widetilde{E}/F}$ rešljiva grupa. Sledi, da je
\[
\Gal(E/F) \cong
\kvoc{\Gal \br{\widetilde{E}/F}}{\Gal \br{\widetilde{E}/E}}
\]
rešljiva grupa.
\end{proof}

\begin{opomba}
Velja tudi obratno -- če je $\Gal_F(p)$ rešljiva grupa, je enačba
$p(X) = 0$ rešljiva z radikali.
\end{opomba}
