\section{Riemann surfaces}

\subsection{Definition and holomorphic maps}

\datum{2025-2-20}

\begin{definicija}
A \emph{surface}\index{surface} is a manifold of complex dimension
$1$.
\end{definicija}

\begin{definicija}
A \emph{Riemann surface}\index{Riemann!surface} is a connected
complex surface.
\end{definicija}

\begin{definicija}
The \emph{Riemann sphere}\index{Riemann!sphere} is defined as
$\rs = \proj{\C}{1}$ with the usual complex
structure.\footnote{Also denoted by ${}\proj{}{1}$.}
\end{definicija}

\begin{definicija}
A \emph{complex torus}\index{complex torus} is given by a quotient
\[
T = \kvoc{\C}{a\Z \oplus b\Z},
\]
where $a, b \in \C$ are $\R$-linearly independent. The
parallelogram bounded by $0$, $a$, $b$ and $a+b$ is called the
\emph{fundamental domain}\index{fundamental domain} of $T$.
\end{definicija}

\begin{izrek}[Identity]
\index{identity theorem}
Let $X$ and $Y$ be Riemann surfaces and $f, g \colon X \to Y$ be
holomorphic maps. If the set $A = \setb{x \in X}{f(x) = g(x)}$ has
an accumulation point, then $f = g$ on $X$.
\end{izrek}

\begin{proof}
We prove that the set of accumulation points is open. Take an
accumulation point $a \in X$. Note that, by continuity,
$f(a) = g(a)$. Consider charts $\varphi \colon U \to V$ on $X$ and
$\psi \colon W \to Z$ on $Y$ such that $a \in U$, $f(a) \in W$ and
$f(U) \subseteq W$. Applying the identity theorem for holomorphic
functions on the function $\psi \circ f \circ \varphi^{-1}$, we
find that $f$ and $g$ agree on $U$, which is a neighbourhood of
$a$. All such points are accumulation points of $A$.

Note that this means that the set of accumulation points of $A$ is
both open and closed. As $X$ is connected and this set is
non-empty, $A = X$. By continuity, $f = g$ on $A = X$.
\end{proof}

\begin{izrek}[Riemann's removable singularity theorem]
\index{Riemann's removable singularity theorem}
Let $X$ be a Riemann surface, $U \subseteq X$ an open set and
$a \in U$. Suppose that $f \colon U \setminus \set{a} \to \C$ is a
holomorphic function that is bounded on $U \setminus \set{a}$. Then
$f$ can be extended uniquely to a holomorphic function
$\tilde{f} \colon U \to \C$ with
$\eval{\tilde{f}}{U \setminus \set{a}}{} = f$.
\end{izrek}

\begin{proof}
First note that we can shrink $U$ down to obtain a chart
$\varphi \colon U \to \C$, then apply Riemann's removable
singularity theorem to the function $f \circ \varphi^{-1}$ in the
point $\varphi(a)$ and define
$\tilde{f}(a) =
\widetilde{\br{f \circ \varphi^{-1}}} \br{\varphi(a)}$.
As it's a composition of holomorphic functions, is is itself
holomorphic. By continuity, the extension is unique.
\end{proof}

\datum{2025-2-27}

\begin{definicija}
Let $X$ be a Riemann surface. A
\emph{meromorphic}\index{meromorphic funcion} function $f$ on $X$
is a function $f \colon X \setminus A \to \C$ such that
$\eval{f}{X \setminus A}{}$ is holomorphic, $A$ is a closed set of
isolated points, and
\[
\lim_{\substack{z \to a \\ z \in X \setminus A}} \abs{f(z)} =
\infty
\]
for all $a \in A$. We denote the set of meromorphic functions on
$X$ by $\mathcal{M}(X)$.
\end{definicija}

\begin{opomba}
The set $\mathcal{M}(X)$ is a field.
\end{opomba}

\begin{izrek}
Let $X$ be a Riemann surface and $f \in \mathcal{M}(X)$. For each
pole $p$ of $f$ we set $f(p) = \infty \in \rs$. Then $f$ is a
holomorphic map from $X$ to $\rs$. Conversely, every holomorphic
map $f \colon X \to \rs$ that is not identically $\infty$ defines
a meromorphic function on $X$.
\end{izrek}

\begin{proof}
Note that $f$ is clearly continuous. It therefore suffices to show
that $f$ is holomorphic at every pole. Recall that a chart around
$\infty$ is given by $\varphi \colon z \mapsto \frac{1}{z}$.
Let $U$ be a neighbourhood of $p$ that contains no other pole, and
define $g \colon U \setminus \set{p} \to \C$ by
$g(z) = \br{f \circ \varphi^{-1}(z)}^{-1}$. Using Riemann's
removable singularity theorem, this map has a unique holomorphic
extension with $g(p) = 0$ by continuity. But that means that the
proposed extension of $f$ is indeed holomorphic at $p$.

Suppose $f \colon X \to \rs$ is holomorphic. Define
$A = \setb{z \in X}{f(z) = \infty}$. Then
$\eval{f}{X \setminus A}{}$ is clearly a meromorphic function.
\end{proof}

\begin{izrek}
Let $X$ and $Y$ be Riemann surfaces and $f \colon X \to Y$ a
holomorphic map. For any point $p \in X$ there exist charts
$\varphi \colon U \to V$ and $\psi \colon Z \to W$ such that
$p \in U$, $f(p) \in Z$, $\varphi(p) = 0 = \psi(f(p))$,
$f(U) \subseteq Z$ and
\[
\psi \circ f \circ \varphi^{-1}(z) = z^k
\]
for some integer $k \in \N_0$. This integer is determined uniquely.
\end{izrek}

\begin{proof}
Let $\tilde{\varphi} \colon U \to V$ be a chart on $X$ with
$p \in U$ such that $\tilde{\varphi}(p) = 0$. Furthermore, let
$\psi \colon Z \to W$ be a chart on $Y$ with $f(U) \subseteq Z$ and
$\psi(f(p)) = 0$. Define
$g = \psi \circ f \circ \tilde{\varphi}^{-1}$. Then
$g(z) = z^k \cdot h(z)$, where $k \geq 1$, $h(0) \ne 0$ and $h$ is
a holomorphic function. Locally, $h$ has a $k$-th root, hence
\[
g(z) = \br{z \cdot \sqrt[k]{h(z)}}^k = w(z)^k.
\]
Taking $\varphi = w \circ \tilde{\varphi}$, we get the sought
charts on small enough domains. As $k$ is equal to the number of
preimages of points distinct from $p$, it is unique.
\end{proof}

\begin{definicija}
Such integer $k$ is called the
\emph{multiplicity}\index{multiplicity} of $f$ in $p$.
\end{definicija}

\begin{posledica}
Every non-constant holomorphic map $f \colon X \to Y$ is open.
\end{posledica}

\begin{proof}
In the charts from the above theorem, disks around $\varphi{p}$ map
to disks.
\end{proof}

\begin{posledica}
Let $X$ and $Y$ be Riemann surfaces and $f \colon X \to Y$ a
bijective holomorphic map. Then $f^{-1} \colon Y \to X$ is
holomorphic.
\end{posledica}

\begin{proof}
As $f$ is not constant, it is open, hence $f^{-1}$ is continuous.
In local coordinates, $f$ is of the form $z \mapsto z^k$. As $f$ is
bijective, $k=1$, hence the inverse is locally $z \mapsto z$, which
is holomorphic.
\end{proof}

\begin{posledica}[Maximum principle]
\index{maximum principle}
Let $X$ be a Riemann surface and $f \colon X \to \C$ a non-constant
holomorphic function. Then $\abs{f}$ does not attain its maximum.
\end{posledica}

\begin{proof}
The map $f$ is open.
\end{proof}

\begin{izrek}
\label{rsf:thm:sur}
Let $X$ and $Y$ be Riemann surfaces and $f \colon X \to Y$ a
non-constant holomorphic map. If $X$ is compact, then $Y$ is
compact and $f$ is surjective.
\end{izrek}

\begin{proof}
Note that $f(X)$ is an open and closed subset of $Y$, which is
connected.
\end{proof}

\begin{posledica}
If $f \colon X \to \C$ is a holomorphic function for a compact
Riemann surface $X$, then $f$ is constant.
\end{posledica}

\obvs

\begin{izrek}[Liouville]
\index{Liouville's theorem}
Every bounded holomorphic function on complex numbers is constant.
\end{izrek}

\begin{proof}
We can extend $f$ to a function $\rs \to \C$ by Riemann's removable
singularity theorem. Applying the above theorem, we get that $f$ is
constant as $\C$ is not compact.
\end{proof}

\begin{izrek}[Fundamental theorem of algebra]
\index{fundamental theorem of algebra}
Every non-constant polynomial with complex coefficients has a
complex root.
\end{izrek}

\begin{proof}
The polynomial can be extended to a holomorphic map
$p \colon \rs \to \rs$, which is surjective by
theorem~\ref{rsf:thm:sur}. As $p(\infty) = \infty$, the set
$p^{-1}(0)$ contains a complex number.
\end{proof}

\begin{izrek}
Every holomorphic function $f \colon \rs \to \rs$ is either a
rational function or $f \equiv \infty$.
\end{izrek}

\begin{proof}
If $f \not \equiv \infty$, the set
$A = \setb{z \in \C}{f(z) = \infty}$ is finite -- otherwise, it'd
have an accumulation point. Let $A = \setb{a_i}{i \leq n}$. If
needed, replace $f$ by $\frac{1}{f}$ so that $\infty \not \in A$,
and repeat the argument. Now consider the function
\[
g =
f - \sum_{i=1}^n \sum_{k=1}^{N_i}
d_{i,k} \cdot \frac{1}{(z - a_i)^k},
\]
where $d_{i,k}$ are obtained from principal parts of Laurent series
around $a_i$. As $g \colon \rs \to \C$ is a holomorphic function,
it is constant, hence $f$ is a rational function.
\end{proof}

\newpage

\subsection{Homotopy and the fundamental group}

\begin{definicija}
Denote the \emph{homotopic}\index{homotopic} relation by $\sim$.
\end{definicija}

\begin{trditev}
Not all paths in $\C^*$ are homotopic.
\end{trditev}

\begin{proof}
Consider $\gamma(t) = e^{2 \pi i t}$ and
$\delta(t) = e^{-2 \pi i t}$. Recall that
\[
\frac{1}{2 \pi i} \cdot \lint_{\gamma(t)} \frac{1}{z}\,dz = 1
\]
and
\[
\frac{1}{2 \pi i} \cdot \lint_{\delta(t)} \frac{1}{z}\,dz = -1.
\]
As integrals are a homotopy invariant, $\gamma$ and $\delta$ are
not homotopic.
\end{proof}
