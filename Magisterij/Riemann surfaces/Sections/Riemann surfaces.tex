\section{Riemann surfaces}

\subsection{Definition and holomorphic maps}

\datum{2025-2-20}

\begin{definicija}
A \emph{surface}\index{surface} is a manifold of complex dimension
$1$.
\end{definicija}

\begin{definicija}
A \emph{Riemann surface}\index{Riemann!surface} is a connected
complex surface.
\end{definicija}

\begin{definicija}
The \emph{Riemann sphere}\index{Riemann!sphere} is defined as
$\rs = \proj{\C}{1}$ with the usual complex
structure.\footnote{Also denoted by ${}\proj{}{1}$.}
\end{definicija}

\begin{definicija}
A \emph{complex torus}\index{complex torus} is given by a quotient
\[
T = \kvoc{\C}{a\Z \oplus b\Z},
\]
where $a, b \in \C$ are $\R$-linearly independent. The
parallelogram bounded by $0$, $a$, $b$ and $a+b$ is called the
\emph{fundamental domain}\index{fundamental domain} of $T$.
\end{definicija}

\begin{izrek}[Identity]
\index{identity theorem}
Let $X$ and $Y$ be Riemann surfaces and $f, g \colon X \to Y$ be
holomorphic maps. If the set $A = \setb{x \in X}{f(x) = g(x)}$ has
an accumulation point, then $f = g$ on $X$.
\end{izrek}

\begin{proof}
We prove that the set of accumulation points is open. Take an
accumulation point $a \in X$. Note that, by continuity,
$f(a) = g(a)$. Consider charts $\varphi \colon U \to V$ on $X$ and
$\psi \colon W \to Z$ on $Y$ such that $a \in U$, $f(a) \in W$ and
$f(U) \subseteq W$. Applying the identity theorem for holomorphic
functions on the function $\psi \circ f \circ \varphi^{-1}$, we
find that $f$ and $g$ agree on $U$, which is a neighbourhood of
$a$. All such points are accumulation points of $A$.

Note that this means that the set of accumulation points of $A$ is
both open and closed. As $X$ is connected and this set is
non-empty, $A = X$. By continuity, $f = g$ on $A = X$.
\end{proof}

\begin{izrek}[Riemann's removable singularity theorem]
\index{Riemann's removable singularity theorem}
Let $X$ be a Riemann surface, $U \subseteq X$ an open set and
$a \in U$. Suppose that $f \colon U \setminus \set{a} \to \C$ is a
holomorphic function that is bounded on $U \setminus \set{a}$. Then
$f$ can be extended uniquely to a holomorphic function
$\tilde{f} \colon U \to \C$ with
$\eval{\tilde{f}}{U \setminus \set{a}}{} = f$.
\end{izrek}

\begin{proof}
Shrink $U$ down to obtain a chart $\varphi \colon U \to \C$, then
apply Riemann's removable singularity theorem to the function
$f \circ \varphi^{-1}$ in the point $\varphi(a)$ and define
$\tilde{f}(a) =
\widetilde{\br{f \circ \varphi^{-1}}} \br{\varphi(a)}$.
As it's a composition of holomorphic functions, is is itself
holomorphic. By continuity, the extension is unique.
\end{proof}
