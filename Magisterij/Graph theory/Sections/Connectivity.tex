\section{Connectivity}

\subsection{Connectivity number}

\begin{definicija}
The \emph{connectivity number}\index{connectivity number}
$\kappa(G)$ is the minimum number of vertices such that we get
either a disconnected graph or one vertex upon removing them. We
say that $G$ is \emph{$k$-connected}\index{$k$-connected graph} if
$\kappa(G) \geq k$.
\end{definicija}

\begin{opomba}
We see that $\kappa(G) \leq \delta(G)$.
\end{opomba}

\begin{opomba}
As an independent set is always disconnected (or just one vertex),
we see that
\[
\kappa(G) \leq n - \alpha(G) = \beta(G).
\]
\end{opomba}

\begin{izrek}
The minimal number of edges in a $k$-connected graph of order $n$
is $\ceil{\frac{nk}{2}}$.
\end{izrek}

\begin{proof}
We see that $k \leq \kappa(G) \leq \delta(G)$, hence
\[
m(G) =
\frac{1}{2} \sum_{v \in V} \deg(v) \geq
\frac{nk}{2}.
\]
It remains to show that the bound $\ceil{\frac{nk}{2}}$ is
achievable. We consider the following graphs:

\begin{enumerate}[i)]
\item If $k$ is even, take
$H_{n,k} = C_n^{\frac{k}{2}}$.\footnote{Here $G^k$ is the graph
with the same vertices as $G$, and $xy \in V(G^k)$ if and only if
$d(x,y) \leq k$ in $G$.}
\item If $k$ is odd and $n$ is even, take $H_{n,k}$ to be
$C_n^{\frac{k-1}{2}}$ with additional edges between every pair of
diametrically opposite vertices.
\item If both $n$ and $k$ are odd, take $H_{n,k}$ to be
$C_n^{\frac{k-1}{2}}$ with additional edges between
$v_i$ and $v_{i+\frac{n-1}{2}}$ for $i \leq \frac{n+1}{2}$.
\end{enumerate}

It is clear that $m(H_{n,k}) = \ceil{\frac{nk}{2}}$. Next, we prove
that each of these graphs is $k$-connected. Consider the graph
$H_{n,k}$ with $k-1$ vertices removed.

\begin{enumerate}[i)]
\item Note that we can always go from one vertex to the next one
left in the cycle, unless we removed $\frac{k}{2}$ consecutive
vertices. But that can only happen once in the whole cycle, meaning
we can just take the other way around.

\item We can again try to go to the next vertex in the cycle. To
have two breaks in the cycle, all $k-1$ removed vertices must be in
the breaks. But the two components are still connected by a
diameter.

\item Same as the previous case. \qedhere
\end{enumerate}
\end{proof}

\newpage

\datum{2024-10-24}

\subsection{Edge connectivity}

\begin{definicija}
A set $F \subseteq E$ is a
\emph{disconnecting set}\index{disconnecting set} if
$G \setminus F$ is disconnected.
\end{definicija}

\begin{definicija}
Let $A \subseteq V$. The set of edges
$E(A, A^{\mathsf{c}})$ is called an
\emph{edge cut}\index{edge!cut}.
\end{definicija}

\begin{opomba}
Every nontrivial edge cut is a disconnecting set. Every minimal
disconnecting set is an edge cut.
\end{opomba}

\begin{definicija}
The \emph{edge connectivity number}\index{edge!connectivity number}
of $G$ is the minimum number of edges in a disconnecting set in an
edge cut. We denote it by $\kappa'(G)$.
\end{definicija}

\begin{definicija}
A graph $G$ is \emph{$k$-edge-connected}\index{$k$-edge-connected}
if the removal of less than $k$ edges results in a connected graph.
Equivalently, $k \leq \kappa'(G)$.
\end{definicija}

\begin{izrek}
Let $G$ be a simple graph with $n \geq 2$ with $n \geq 2$. Then
\[
\kappa(G) \leq \kappa'(G) \leq \delta(G).
\]
\end{izrek}

\begin{proof}
The second inequality results from the edge cut with $A = \set{v}$,
where $v$ is a vertex of minimal degree.

Let $F \subseteq E(G)$ be an edge cut in $G$ with minimal
cardinality, that is $\abs{F} = \kappa'(G)$. We consider two cases:

\begin{enumerate}[i)]
\item If $F$ forms a complete bipartite graph, then
\[
\kappa'(G) =
\abs{A} \cdot \abs{A^{\mathsf{c}}} =
\abs{A} \cdot \br{n - \abs{A}} \geq
n-1 \geq
\kappa(G).
\]
\item If $F$ does not form a complete bipartite graph, consider
vertices $x \in A$ and $y \in A^{\mathsf{c}}$ with $xy \not \in E$.
For each edge in $F$, choose an endpoint that is different from $x$
and $y$. They clearly form a vertex cut of cardinality at most
$\abs{F}$, hence $\kappa(G) \leq \kappa'(G)$. \qedhere
\end{enumerate}
\end{proof}

\begin{posledica}
The minimal number of edges in a $k$-edge-connected graph on $n$
vertices is $\ceil{\frac{kn}{2}}$ when $n > k \geq 2$.
\end{posledica}

\begin{proof}
Note that $k \leq \kappa'(G) \leq \delta(G)$. By the handshake
lemma, we find that $m(G) \geq \frac{nk}{2}$. As $H_{n,k}$ is
$k$-connected, it is also $k$-edge-connected.
\end{proof}

\newpage

\subsection{\texorpdfstring{$2$}{2}-connected graphs}

\begin{izrek}[Whitney]
\index{Whitney theorem}
If $G$ is a $2$-connected graph, then for every distinct vertices
$u, v \in G$ there exist two internally disjoint $uv$-paths.
\end{izrek}

\begin{proof}
Suppose the statement is false. Take a counterexample with minimal
$k = d(u,v)$. Note that $k \geq 2$, as otherwise $G$ is not
$2$-edge-connected, and hence is not $2$-connected.

Let $w$ be a vertex with $d(u,w) = k-1$ and $d(v,w) = 1$. Note that
there exists a $uv$-path $P$ not containing $w$ since $G$ is
$2$-connected. Now consider two $uw$-paths, which exist by
minimality. If $v$ is in this cycle, we trivially get two
$uv$-paths. Otherwise, we get three $uv$-paths. To get two disjoint
paths, travel along $P$ until the first intersection with one of
the other paths, then switch to that one.
\end{proof}

\begin{lema}[Expansion]
\index{Expansion lemma}
Let $G$ be a $k$-connected graph. If we construct a graph by adding
a new vertex and connecting it to at least $k$ vertices of $G$, the
resulting graph is again $k$-connected.
\end{lema}

\obvs

\datum{2024-11-7}

\begin{izrek}
If $G$ is a graph with $n \geq 3$, the following statements are
equivalent:

\begin{enumerate}[i)]
\item The graph $G$ is $2$-connected.
\item The graph $G$ is connected with no cut-vertex.
\item For every vertices $u$ and $v$ there exist at least two
internally vertex-disjoint paths between them.
\item There exists a cycle through any two vertices.
\item There exists a cycle through any two edges and
$\delta(G) \geq 1$.
\end{enumerate}
\end{izrek}

\begin{proof}
Using Whitney's theorem, we see that the first four statements are
clearly equivalent. Suppose now that $G$ is $2$-connected 
consider two distinct edges $e$ and $f$. Expand $G$ by adding new
vertices $w$ and $w'$, where $w$ is connected to the vertices of
$e$ and $w'$ is connected to the vertices of $f$. By the expansion
lemma, there exists a cycle through $w$ and $w'$, which induces the
sought cycle in $G$. If $e = f$, take another edge $e'$. By the
above argument, there exists a cycle through $e$ and $e'$, which is
the required cycle.

Suppose now that the last condition holds. In particular, $G$ has
no isolated edges. For any vertices $u, v \in G$, we can therefore
take distinct edges $e, f \in E$ with $u \in e$ and $v \in f$.
Since any cycle through $e$ and $f$ is also a cycle through $u$ and
$v$, $G$ is $2$-connected.
\end{proof}

\begin{lema}[Subdivision]
\index{subdivision lemma}
Let $G'$ be a graph from $G$ that is obtained by subdividing an
edge with a vertex. Then $G'$ is $2$-connected if and only if $G$
is $2$-connected.
\end{lema}

\begin{proof}
For any two edges in $G'$, take the corresponding edges in $G$
(instead of taking subdivisions, take the whole edge). Cycles in
$G'$ containing these two edges correspond precisely with cycles in
$G$ containing the corresponding edges.
\end{proof}

\newpage

\subsection{Ear decomposition of graphs}

\begin{definicija}
In a graph $G$, a path $P$ is an \emph{open ear}\index{ear} if all
internal vertices of $P$ are of degree $2$, while the endpoints
have degree at least $3$ in $G$.
\end{definicija}

\begin{definicija}
An \emph{open ear decomposition}\index{ear decomposition} of $G$ is
a sequence $P_0, P_1, \dots, P_k$, where $P_0$ is a cycle in $G$
and $P_i$ is an ear for $i > 0$ in the graph
\[
G_i = \bigcup_{j=0}^i P_j
\]
and $G_k = G$. Furthermore, we require that $P_i$ be
edge-disjoint.\footnote{Not stated in the lectures, but removes the
edge case where $P_k = P_{k+1}$, which sounds annoying.}
\end{definicija}

\begin{izrek}
\label{con:thm:ear}
A graph $G$ is $2$-connected if and only if it admits an ear
decomposition.
\end{izrek}

\begin{proof}
Suppose that $G$ has an ear decomposition. By induction, we can
prove that $G_i$ is $2$-connected, as we can apply the expansion
and subdivision\footnote{Possibly the converse!} lemmas.

Now suppose that $G$ is $2$-connected. Set $P_0$ to be an arbitrary
cycle in $G$. If $G_i$ is not an induced graph of $G$, let
$P_{i+1}$ be a missing edge. Otherwise, choose a vertex $u$ not in
$G_i$. Take two edges, one in $G_i$ and one with vertex $u$. These
lie in a cycle, which includes an ear containing $u$, which is our
$P_{i+1}$. As we cover some edges on each step, the process is
finite.
\end{proof}

\begin{trditev}
A graph $G$ is $2$-edge-connected if and only if it is connected
and every edge of $G$ lies in a cycle.
\end{trditev}

\obvs

\begin{definicija}
In a graph $G$, a cycle $P$ is a \emph{closed ear}\index{ear} if
all but one vertex of $P$ are of degree $2$, while the last one has
degree at least $4$ in $G$.
\end{definicija}

\begin{definicija}
A \emph{closed ear decomposition}\index{ear decomposition} of $G$
is a sequence $P_0, P_1, \dots, P_k$, where $P_0$ is a cycle in $G$
and $P_i$ is an open or closed ear for $i > 0$ in the graph
\[
G_i = \bigcup_{j=0}^i P_j
\]
and $G_k = G$. Furthermore, we require that $P_i$ be edge-disjoint.
\end{definicija}

\begin{izrek}
A graph $G$ is $2$-edge-connected if and only if it has a closed
ear decomposition.
\end{izrek}

\begin{proof}
Analogous as theorem~\ref{con:thm:ear}.
\end{proof}

\datum{2024-11-14}

\begin{definicija}
A directed graph $\vv{G}$ is
\emph{strongly connected}\index{strongly connected} if for every
$u, v \in V \br{\vv{G}}$ there exists a directed path from $u$ to
$v$. A \emph{strong orientation}\index{strong orientation} of a
graph $G$ is a directed graph $\vv{G}$ which is strongly connected.
\end{definicija}

\begin{izrek}[Robbin]
\index{Robbin's theorem}
A graph $G$ has a strong orientation if and only if it is $2$-edge
connected.
\end{izrek}

\begin{proof}
If $G$ has a strong orientation, it is clearly connected and every
edge lies in a cycle. Now suppose that $G$ is $2$-edge connected.
Let $P_0, P_1, \dots, P_k$ be a closed ear decomposition of $G$.
Direct the edges of the cycle in a cycle and along each ear in a path.
It is clear that the resulting orientation is strong.
\end{proof}

\newpage

\subsection{Minimal cuts}

\begin{definicija}
Let $x, y \in V$ be non-adjacent vertices in $G$. A set
$S \subseteq V$ is an \emph{$x,y$-cut}\index{edge!cut} if $x$ and
$y$ belong to different components of $G \setminus S$. We denote
the minimum size of an $x,y$-cut by $\kappa_G(x,y)$.
\end{definicija}

\begin{definicija}
For $x, y \in V$, we denote by $\lambda_G(x,y)$ the maximal number
of pairwise internally vertex-disjoint $x,y$-path.
\end{definicija}

\begin{izrek}[Menger]
\index{Menger's theorem}
Suppose that $x$ and $y$ are non-adjacent vertices in $G$. Then
$\kappa_G(x,y) = \lambda_G(x,y)$.
\end{izrek}

\begin{proof}
For convenience, we denote the above numbers by $\kappa$ and
$\lambda$ respectively. Clearly $\kappa \geq \lambda$, as we need
to select at least one vertex from each disjoint $x,y$-paths to
disconnect them.

To prove the reverse inequality, we induct on $n$. For $n = 2$, we
clearly have $\kappa = \lambda = 0$.

Suppose now that $n \geq 3$ and consider two cases:

\begin{enumerate}[i)]
\item There exists a minimum $x,y$-cut $S$ such that $S \ne N(x)$
and $S \ne N(y)$. Let $V_x$ denote the set of vertices that can be
reached from $x$ by a path with no internal vertices from $S$, and
define $V_y$ analogously. By the minimality of $S$, we find that
$S = V_x \cap V_y$.

Let $G_x$ be the graph obtained from $G[V_1]$ by adding a vertex
$y'$ that is adjacent to precisely the vertices in $S$. Note that,
as $S \ne N(y)$, the number of vertices decreased, and that $S$ is
a minimum $x,y'$-cut in $G_x$. It follows that
$\kappa = \kappa_G(x,y) = \kappa_{G_x}(x,y') = \lambda_{G_x}(x,y')$
by the induction hypothesis. Analogously,
$\kappa = \lambda_{G_y}(x',y)$. By pairing up the $x,y'$-paths with
$x',y$ paths according to the visited vertex in $S$, we obtain
$\kappa$ internally vertex-disjoint $x,y$-paths, hence
$\lambda \geq \kappa$.

\item The only minimum $x,y$-cuts are $N(x)$ and/or $N(y)$. If $x$
and $y$ have a neighbour $z$ in common, we can remove it from $G$
and apply the induction hypothesis. Note that removing $z$ reduced
both the number of $x,y$-paths and the minimum size of an $x,y$-cut
by $1$, hence equality holds for $G$ as well.

Suppose then that $N(x)$ and $N(y)$ are disjoint. If
$N(x) \cup N(y) \cup \set{x,y} = V$, we can construct a bipartite
graph $H$ with sets $N(x)$ and $N(y)$ (we disregard internal edges
in both $N(x)$ and $N(y)$). The number of internally
vertex-disjoint $x,y$-paths is clearly equal to the size of the
maximum matching in $H$. Without loss of generality suppose that
$N(x)$ is a minimum $x,y$-cut. Note that for every
$A \subseteq N(x)$, we have $\abs{N_H(A)} \geq \abs{A}$, as
otherwise we could obtain a smaller $x,y$-cut by replacing the
vertices in $A$ with those in $N_H(A)$. By Hall's theorem, there
exists a perfect matching, hence
$\lambda = \alpha'(H) = N(x) = \kappa$.

Finally, if there exists a vertex $v \ne x, y$ with
$v \not \in N(x) \cup N(y)$, then $v$ does not belong to any
minimum $x,y$-cuts, therefore
$\kappa_{G \setminus v}(x,y) = \kappa$. Applying the induction
hypothesis, we can find $\kappa$ internally vertex-disjoint
$x,y$-paths in $G \setminus v$. Since these are also valid in $G$,
we conclude $\lambda \geq \kappa$. \qedhere
\end{enumerate}
\end{proof}

\begin{definicija}
Let $x, y \in V$ be vertices in $G$. A set $R \subseteq E$ is an
\emph{$x,y$-edge cut}\index{edge!cut} if $x$ and $y$ belong to
different components of $G \setminus R$.
\end{definicija}

\datum{2024-11-21}

\begin{definicija}
For $x, y \in V$, we denote by $\kappa_G'(x,y)$ the minimal
cardinality of an $x,y$-edge cut in $G$, and by
$\lambda_G'(x,y)$ the maximal number of edge-disjoint $x,y$-path.
\end{definicija}

\begin{definicija}
The \emph{line graph}\index{line graph} $L(G)$ of a graph $G$ has
vertices representing the edges of $G$. Two vertices in $L(G)$ are
connected if and only if they share a vertex in $G$.
\end{definicija}

\begin{izrek}[Menger]
\index{Menger's theorem}
For every $x, y \in V$ we have $\kappa_G'(x,y) = \lambda_G'(x,y)$.
\end{izrek}

\begin{proof}
Define a new graph $G'$ by adding vertices $u$ and $v$ to $G$,
which are connected to $x$ and $y$ respectively, and consider its
line graph. Note that any path between the new edges in $L(G')$
corresponds to a path between $x$ and $y$ in $G$. In particular,
vertex-disjoint paths in $L(G')$ correspond to edge-disjoint paths
in $G$. Hence
\[
\lambda_{L(G')}(xu, yv) = \lambda_G'(x,y).
\]
By Menger's theorem, we know that
\[
\lambda_{L(G')}(xu, yv) = \kappa_{L(G')}(xu, yv).
\]
Finally, by definition of a line graph, a vertex cut in $L(G')$
that separates $xu$ and $yv$ corresponds to an edge cut in $G$ that
separates $x$ and $y$, hence
\[
\kappa_{L(G')}(xu, yv) = \kappa_G(x,y). \qedhere
\]
\end{proof}

\begin{lema}
For each edge $e \in E$, we have
\[
\kappa(G) - 1 \leq \kappa \br{G \setminus \set{e}} \leq \kappa(G).
\]
\end{lema}

\begin{proof}
The second inequality follows from the fact that each vertex cut in
$G$ is also a vertex cut in $G \setminus \set{e}$.

Suppose that $\kappa \br{G \setminus \set{e}} < \kappa(G)$. Let $S$
be a minimum vertex cut in $G' = G \setminus \set{e}$. If any of
vertices $x$ and $y$ has degree at least two, we can add it to $S$
to get a vertex cut in $G$. Otherwise, we find that
$\abs{S} = n-2$, hence $S \cup \set{x}$ is a vertex cut in $G$.
\end{proof}

\begin{izrek}[Menger]
\index{Menger's theorem}
In any graph $G$ with at least two vertices, the following
statements hold:

\begin{enumerate}[i)]
\item We have
$\kappa'(G) = \displaystyle \min_{x \ne y} \lambda_G'(x,y)$.
\item We have
$\kappa(G) = \displaystyle \min_{x \ne y} \lambda_G(x,y)$.
\end{enumerate}
\end{izrek}

\begin{proof}
The only non-trivial part is showing that we can take the minimum
over all $x \ne y$ in ii), not just non-adjacent
ones.\footnote{Note that the equality clearly holds for complete
graphs.} It suffices to show that for every adjacent vertices $x$
and $y$ we have $\lambda_G(x,y) \geq \kappa(G)$. Denote
$G' = G \setminus \set{xy}$ and note that
\[
\lambda_G(x,y) = \lambda_{G'}(x,y) + 1.
\]
Applying Menger's theorem and the above lemma, we get
\[
\lambda_G(x,y) = \kappa_{G'}(x,y) + 1 \geq \kappa(G). \qedhere
\]
\end{proof}
