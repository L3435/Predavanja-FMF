\section{Colourings}

\subsection{Vertex colourings}

\begin{definicija}
Let $G$ be a simple graph. A
\emph{$k$-colouring}\index{colouring} of $G$ is a map
$\varphi \colon V \to [k]$ such that for all $xy \in E$ we have
$\varphi(x) \ne \varphi(y)$. The graph $G$ is
\emph{$k$-colourable}\index{colourable}\footnote{Also $k$-partite.}
if it has a $k$-colouring.
\end{definicija}

\begin{definicija}
The \emph{chromatic number}\index{chromatic number} $\chi(G)$ is
the smallest integer $k$ such that $G$ is $k$-colourable.
\end{definicija}

\begin{definicija}
Denote by $\omega(G)$ the order of the largest clique in $G$.
\end{definicija}

\begin{trditev}
In a graph $G$, the inequality
\[
\omega(G) \leq \chi(G) \leq \Delta(G) + 1
\]
holds.
\end{trditev}

\obvs

\begin{trditev}
In a graph $G$, we have
\[
\frac{n}{\alpha(G)} \leq \chi(G).
\]
\end{trditev}

\begin{proof}
As every colour class is an independent set, it contains at most
$\alpha(G)$ vertices.
\end{proof}

\begin{izrek}[Welsh-Powell]
\index{Welsh-Powell theorem}
If $d_1 \geq d_2 \geq \dots \geq d_n$ are the degrees of vertices
in $G$, then
\[
\chi(G) \leq 1 + \max_{i \leq n} \br{\min(d_i, i-1)}.
\]
\end{izrek}

\begin{proof}
Colour the vertices in sequence $v_1, v_2, \dots, v_n$, always
using the smallest possible number at each step.
\end{proof}

\datum{2024-11-28}

\begin{trditev}
We have the following characterisations:

\begin{enumerate}[i)]
\item For a graph $G$, $\chi(G) = 1$ if and only if
$E = \emptyset$.
\item For a graph $G$, $\chi(G) = 2$ if and only if $G$ is
bipartite and $E \ne \emptyset$.
\end{enumerate}
\end{trditev}

\obvs

\begin{izrek}[Brooks]
\index{Brooks' theorem}
Suppose $G$ is a connected graph that is neither a complete graph
nor an odd cycle. Then $\chi(G) \leq \Delta(G)$.
\end{izrek}

\begin{proof}
Denote $\Delta(G) = k$. Suppose first that $G$ is not regular and
let $\deg(v) \leq k-1$. Colour the vertices greedily in decreasing
order of distance from $v$. As we coloured at most $k-1$ neighbours
of a vertex in each step, it is clear we need at most $k$ colours.

Now consider regular graphs. For $k \leq 2$, excluding the special
cases, the inequality clearly holds. Now assume $k \geq 3$. We
consider three cases:

\begin{enumerate}[i)]
\item We have $\kappa(G) = 1$, that is, there exists a cut vertex
$x$ that splits $G$ into parts $V_1$ and $V_2$. Denote
$G_i = G[V_i \cup \set{x}]$. Note that $G_i$ is not $k$-regular,
as $\deg_{G_i}(x) \leq k-1$. We can colour both $V_1$ and $V_2$
with $k$ colours. By joining them at $x$, we find a $k$-colouring
of $G$.

\item Suppose that $G \setminus \set{x,y}$ is disconnected. Again,
denote $G_i = G[V_i \cup \set{x,y}]$ and note that $G_i$ is not
$k$-regular. As above, we colour both graphs and attempt to join
the colourings. This is not possible only when every colouring of
$G_1$ assigns the same colour to both vertices, while every
colouring of $G_2$ assigns different colours to them, or
vice-versa. In particular, $G_1 + xy$ is not $k$-colourable. We
deduce that $\Delta(G_1 + xy) = k$, hence it is a $k$-regular
graph. But then $\deg_{G_2}(x) = \deg_{G_2}(y) = 1$. As $k \geq 3$,
we can colour $x$ and $y$ with the same colour in $G_2$, hence
this case is not possible.

\item Finally, consider $\kappa(G) \geq 3$. As $G \ne K_n$, we can
find vertices $x$ and $y$ such that $d(x,y) = 2$. Let
$z \in N(x) \cap N(y)$. Note that $G \setminus \set{x,y}$ is
connected, hence there exists a path which contains neither from
$z$ to any other vertex. We proceed to colour vertices greedily.
First, colour $x$ and $y$ with the same colour. Then proceed to
colour the vertices in decreasing order of distance from $z$ in
$G \setminus{x,y}$. As we coloured at most $k-1$ neighbours of a
vertex in each step, we need at most $k$ colours for every vertex.
The exception is $z$, but two of its neighbours are already
coloured with the same colour. \qedhere
\end{enumerate}
\end{proof}

\begin{definicija}
A \emph{Mycielski construction}\index{Mycielski construction} of a
graph $G$ with $V = \setb{v_i}{i \leq n}$ is a graph $M(G)$ with
\begin{align*}
V(M(G)) &= V \cup \setb{u_i}{i \leq n} \cup \set{z}
\intertext{and}
E(M(G)) &=
E \cup \setb{u_i v_j}{v_i v_j \in E} \cup \setb{zu_i}{i \leq n}.
\end{align*}
\end{definicija}

\begin{izrek}
If $G$ is a graph with at least one edge, then
$\chi(M(G)) = \chi(G) + 1$ and $\omega(M(G)) = \omega(G)$.
\end{izrek}

\begin{proof}
Let $\chi(G) = k$. Note first that $M(G)$ is $(k+1)$-colourable.
Indeed, we can copy a $k$-colouring of $G$ to both $v_i$ and $u_i$,
then colour $z$ with a new colour. It is clear that this satisfies
the conditions of a colouring.

Suppose that $\chi(M(G)) \leq k$ and consider a $k$-colouring
$\varphi$ of $M(G)$ where $\varphi(z)=k$. Denote
$S = \setb{v_i}{\varphi(v_i) = k}$. We can define a new colouring
on $G$ as
\[
\psi(v_i) =
\begin{cases}
\varphi(u_i), & v_i \in S, \\
\varphi(v_i), & v_i \not \in S.
\end{cases}
\]
It is easy to see that this is a $(k-1)$-colouring of $G$, as
$\varphi(u_i) \ne k$ for all $i$. This is a contradiction as
$\chi(G) = k$, hence $\chi(M(G)) = k+1$.

As $G$ is a subgraph of $M(G)$, we obviously have
$\omega(G) \leq \omega(M(G))$. If $z$ is in a clique of $M(G)$,
then it has order at most $2$. Otherwise, if it contains a vertex
$u_i$, it contains neither $v_i$ nor any other vertex $u_j$.
By replacing $u_i$ with $v_i$, we preserve the clique. For every
clique in $M(G)$, we found a corresponding clique in $G$ of the
same size, hence $\omega(M(G)) \leq \omega(G)$.
\end{proof}

\datum{2024-12-5}

\begin{izrek}
If $G$ is a graph with $\chi(G) = k$, then
$m(G) \geq \binom{k}{2}$.
\end{izrek}

\begin{proof}
There is at least one edge between each pair of colours.
\end{proof}

\newpage

\subsection{Turan's theorem and chordal graphs}

\begin{definicija}
A graph $G$ is a
\emph{complete $k$-partite graph}\index{complete $k$-partite graph}
if all pairs of vertices from different colour classes are
connected. We denote it by $K_{n_1, \dots, n_k}$, where $n_i$ are
sizes of the partite classes.
\end{definicija}

\begin{definicija}
The \emph{Turan graph}\index{Turan graph} $T_{n,k}$ is the complete
$k$-partite graph on $n$ vertices such that each of the partite
classes is of size $\floor{\frac{n}{k}}$ of $\ceil{\frac{n}{k}}$.
\end{definicija}

\begin{izrek}[Turan]
If $G$ is a graph of order $n$ with $\omega(G) \leq r$, then
\[
m \leq m(T_{n,r}).
\]
\end{izrek}

\begin{proof}
We induct on $r$. If $r = 1$, the inequality clearly holds. Now
suppose $r \geq 2$ and denote $\Delta(G) = k$. In particular, let
$\deg(v) = k$.

Now let $G' = G[N(v)]$. We can see that $\omega(G') \leq r-1$. By
the induction hypothesis, there are at most
$m(T_{k,r-1})$ edges in $G'$.

Construct another graph $H$ as follows -- add $n-k$ vertices to
$T_{k,r-1}$ and connect each of these vertices to each vertex in
$T_{k,r-1}$. Note that $n(H) = n$ and $\omega(H) = r$. Observe that
\[
m \leq
\sum_{v \not \in G'} \deg(v) + m(G') \leq
(n-k) \cdot k + m(T_{k,r-1}) =
m(H).
\]
Furthermore, $H$ is a complete $r$-partite graph, hence
\[
E(H) =
\sum_{i \ne j} \abs{V_i} \cdot \abs{V_j} =
\frac{1}{2} \cdot \br{n^2 - \sum_{i=1}^r \abs{V_i}^2}.
\]
By Karamata's inequality, this is greatest when $H$ is a Turan
graph.\footnote{For non-Turan graphs, the number of edges increases
upon moving a vertex from a class $V_i$ into a class $V_j$ if
$\abs{V_i} \geq \abs{V_j}+2$.}
\end{proof}

\begin{opomba}
The bound is sharp with equality if and only if $G \cong T_{n,r}$.
\end{opomba}

\begin{posledica}
If $G$ is a graph of order $n$ with $\chi(G) = r$, then
\[
m \leq m(T_{n,r}).
\]
The equality holds if and only if $G \cong T_{n,r}$.
\end{posledica}

\obvs

\begin{definicija}
Denote by $\ex(n, F)$ the maximum number of edges in a graph $G$
with $n(G) = n$ such that $G$ does not contain $F$ as a subgraph.
\end{definicija}

\begin{definicija}
A graph $G$ is a \emph{chordal graph}\index{chordal graph} if
it has no induced subgraph that is isomorphic to a cycle $C_k$ with
$k \geq 4$.
\end{definicija}

\begin{definicija}
A vertex $v$ is a \emph{simplicial vertex}\index{simplicial vertex}
in $G$ if $N(v)$ is a clique.
\end{definicija}

\begin{definicija}
A
\emph{simplicial elimination ordering}\index{simplicial elimination ordering}
in $G$ is an order $v_1, \dots, v_n$ of vertices such that
$N(v_i) \cap \setb{v_j}{j \geq i}$ induces a clique.
\end{definicija}

\begin{izrek}[Voloshin's lemma]
\index{Voloshin's lemma}
If $G$ is a chordal graph, then for every $x \in V(G)$ there exists
a simplicial vertex among the ones farthest from $x$.
\end{izrek}

\begin{proof}
We induct on $n$. For $n = 1$, the statement trivially holds. For
$n \geq 2$, consider an arbitrary vertex $x$. If $x$ is a universal
vertex in $G$, apply the induction hypothesis to $G \setminus x$.
Otherwise, let $T$ be the set of vertices farthest from $x$. Denote
by $H$ a component of $G[T]$ and let $S = N(H) \setminus H$.
Finally, let $Q$ be the component of $G \setminus S$ containing
$x$.

We claim that $S$ induces a clique. Let $u, v \in S$ be distinct.
Each vertex in $S$ clearly has neighbours both in $H$ and in $Q$.
Since both $H$ and $Q$ induce connected subgraphs, we can find
$u, v$-paths with internal vertices in $H$ and one with internal
vertices in $Q$. Consider shortest such paths. These paths form
a cycle of order $k \geq 4$, hence it also contains a chord. But by
the above conditions, the only possible chord is $uv$. Applying
this argument to all possible pairs of vertices in $S$, we conclude
that $S$ is a clique.

We now apply the induction hypothesis to $G[S \cup H]$. If this
graph is a clique, then every vertex in $H$ is simplicial.
Otherwise, take a vertex $u \in S$ such that
$H \not \subseteq N(u)$. Then the induction hypothesis supplies us
with a simplicial vertex in $H$. Since any simplicial vertex in
$G[S \cup H]$ is also simplicial in $G$, we found a simplicial
vertex in $T$.
\end{proof}

\datum{2024-12-12}

\begin{izrek}
A graph $G$ is chordal if and only if there exists a simplicial
elimination ordering of the vertices in $G$.
\end{izrek}

\begin{proof}
Suppose first that $G$ is chordal. Applying Voloshin's lemma, we
find that $G$ has a simplicial vertex $v_1$. We can then apply
Voloshin's lemma to $G \setminus v_1$. Repeating this process $n$
times, we get a simplicial elimination ordering
$v_1, v_2, \dots, v_n$.

Suppose now that $G$ is not chordal -- that is, it contains an
induced cycle $C_k$ with $k \geq 4$. Then no vertex in this cycle
can be the first one to appear in a simplicial elimination
ordering, therefore it cannot exist.
\end{proof}

\begin{izrek}
If a graph $G$ is chordal, then $\chi(G) = \omega(G)$.
\end{izrek}

\begin{proof}
Recall that $\chi(G) \geq \omega(G)$, hence we need only prove the
reverse inequality. Let $v_1, v_2, \dots, v_n$ be a simplicial
elimination ordering of $G$. Consider the greedy colouring of
vertices in reverse order. It is clear that we only need
$\omega(G)$ colours, as we coloured at most $\omega(G)-1$ vertices
of the considered vertex in each step.
\end{proof}

\newpage

\subsection{Perfect graphs and chromatic index}

\begin{definicija}
A graph $G$ is \emph{perfect}\index{perfect graph} if
$\chi(H) = \omega(H)$ holds for every induced subgraph $H$ of $G$.
\end{definicija}

\begin{izrek}
Every chordal graph is perfect.
\end{izrek}

\obvs

\begin{izrek}
Bipartite graphs are perfect.
\end{izrek}

\begin{proof}
All induced subgraphs of $G$ are clearly bipartite as well, hence
we only need to show that $\chi(G) = \omega(G)$, which is clear --
they are both $1$ if $E = \emptyset$, otherwise they are both equal
to $2$.
\end{proof}

\begin{definicija}
An \emph{edge colouring}\index{edge!colouring} of $G$ is a function
$c \colon E(G) \to \N$ such that every two distinct edges $e$ and
$f$ with a vertex in common satisfy $c(e) \ne c(f)$. The
\emph{chromatic index}\index{chromatic index} $\chi'(G)$ is the
minimum number of colours needed for an edge colouring.
\end{definicija}

\begin{izrek}[Vizing]
\index{Vizing's theorem}
For every graph $G$ we have
$\Delta(G) \leq \chi'(G) \leq \Delta(G)+1$.
\end{izrek}

\begin{izrek}
If $G$ is a bipartite graph, then $\chi'(G) = \Delta(G)$.
\end{izrek}

\begin{proof}
We consider two cases:

\begin{enumerate}[i)]
\item If $G$ is regular, then we can find a perfect matching. We
colour all the edges in this matching with one colour and remove
them. We can repeat this process $\Delta(G)$ times to obtain the
desired colouring.

\item Suppose that $G$ is not regular. First, add vertices to $G$
in such a way that both its parts have the same size. Repeat the
following process until the graph is regular: Choose a vertex in
each part with degree less than $k = \Delta(G)$. Add the graph
$K_{k,k}$ with one missing edge to $G$, then connect the chosen
vertices to the endpoints of the missing edge. As
\[
\sum_{v \in G} (k - \deg(v))
\]
is a decreasing monovariant, the process eventually stops. We can
thus colour the edges in the resulting graph with $k$ colours,
which induces a colouring of our initial graph. \qedhere
\end{enumerate}
\end{proof}

\begin{izrek}
If $G$ is a bipartite graph, then its line graph is perfect.
\end{izrek}

\begin{proof}
Note that $\chi(L(G)) = \chi'(G)$. Since $G$ is bipartite, we have
\[
\chi'(G) = \Delta(G) = \omega(L(G)),
\]
since there are no $K_3$ subgraphs in $G$. Thus
$\chi(L(G)) = \omega(L(G))$. Note that every induced subgraph of a
line graph is a linegraph of a subgraph of the original graph. In
particular, any induced subgraph $H$ of $L(G)$ is a line graph of a
subgraph of $G$, hence it is a line graph of a bipartite graph as
well. As above, $\chi(H) = \omega(H)$, hence $L(G)$ is perfect.
\end{proof}

\begin{izrek}[Perfect graph theorem]
\index{Perfect graph theorem}
A graph $G$ is perfect if and only if $\oline{G}$ is perfect.
\end{izrek}

\begin{izrek}[Strong perfect graph theorem]
\index{Strong perfect graph theorem}
A graph $G$ is perfect if and only if neither $G$ not $\oline{G}$
has an induced odd cycle of length $k \geq 5$.
\end{izrek}

\begin{definicija}
A graph $G$ is
\emph{$(\beta, \alpha')$-perfect}\index{$(\beta, \alpha')$-perfect}
if every induced subgraph $H$ of $G$ satisfies
$\beta(H) = \alpha'(H)$.
\end{definicija}

\begin{izrek}
A graph $G$ is $(\beta, \alpha')$-perfect if and only if $G$ is
bipartite.
\end{izrek}

\begin{proof}
Bipartite graphs are clearly $(\beta, \alpha')$-perfect by König's
theorem. Suppose that $G$ is not bipartite and consider the
shortest odd cycle in $G$. It must have no chords, hence the graph
$H$ induced by these vertices is an odd cycle, therefore
\[
\beta(H) =
\ceil{\frac{n(H)}{2}} \ne
\floor{\frac{n(H)}{2}} =
\alpha'(H). \qedhere
\]
\end{proof}

\begin{izrek}[Gallai-Roy-Vitaver]
\index{Gallai-Roy-Vitaver's theorem}
Let $G$ be a simple graph. Then
\[
\chi(G) =
\min_{\text{$\vv{D}$ is an orientation}}
\br{\max_{\text{$p$ is a path}} \ell(p)+1}.
\]
\end{izrek}

\begin{proof}
Let $\vv{D}$ be an orientation on $G$. Choose a maximal acyclic
subgraph $\vv{D}'$. Let $d(v)$ be the length of the longest
directed path in $\vv{D}'$ that ends in $v$ and define
$c(v) = d(v) + 1$. We claim that this is a proper colouring. If
$\vv{uv} \in \vv{D}'$, then clearly $c(u) \ne c(v)$. If
$\vv{uv} \in \vv{D} \setminus \vv{D}'$, then by adding $\vv{uv}$ to
$\vv{D}'$ we'd get a cycle, hence there exists a path from
$v$ to $u$. As such, $c(u) \ne c(v)$. This shows that
\[
\chi(G) \leq
\max_{\text{$p$ is a path}} \ell(p)+1.
\]
It remains to prove that the inequality is reversed for a
particular orientation $\vv{D}$. Let $c$ be a colouring of $G$ with
$\chi(G)$ colours and orient edges in $G$ such that
$\vv{uv} \in \vv{D}$ if and only if $c(u) < c(v)$. Then clearly
$\ell(p) + 1 \leq \chi(G)$ for all directed paths $p$.
\end{proof}
