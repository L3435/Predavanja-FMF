\section{Planar graphs}

\subsection{Definition and Euler's formula}

\datum{2024-12-19}

\begin{definicija}
Let $G$ be a graph. The
\emph{drawing of $G$ into a plane}\index{drawing into a plane} is a
function $h$ on $V(G) \cup E(G)$ such that $h(v) \in \R^2$ for all
$v \in V$ and $h(uv)$ is a continuous $h(u) h(v)$-curve for each
$uv \in E$.
\end{definicija}

\begin{definicija}
A
\emph{planar embedding}\index{planar embedding}\footnote{Also \emph{plane graph}.}
of a graph $G$ is a drawing where the curves corresponding to edges
intersect only in the common end vertices.
\end{definicija}

\begin{definicija}
A graph $G$ is \emph{planar}\index{planar graph} if it admits a
planar embedding.
\end{definicija}

\begin{izrek}[Jordan]
\index{Jordan's curve theorem}
Every closed simple curve in the plane divides it into exactly two
regions.
\end{izrek}

\begin{definicija}
Let $G$ be a plane graph. A \emph{face}\index{face} of $G$ is a
maximal region that contains no points from the image of the
embedding function.
\end{definicija}

\begin{definicija}
A \emph{dual graph}\index{dual graph} of a plane graph $G$ is a
graph $G^*$ with faces of $G$ as vertices, in which two vertices
are connected if and only if their corresponding faces have an edge
in common.
\end{definicija}

\begin{opomba}
A dual graph need not be simple, even if $G$ is.
\end{opomba}

\begin{definicija}
The \emph{length}\index{length} $\ell(F)$ of a face $F$ in a plane
graph $G$ is the total length of walks along the boundary of $F$.
\end{definicija}

\begin{izrek}
Let $G$ be a plane graph. The following statements are equivalent:

\begin{enumerate}[i)]
\item The graph $G$ is bipartite.
\item Every face of $G$ has even length.
\item The graph $G^*$ is Eulerian.
\end{enumerate}
\end{izrek}

\begin{proof}
If $G$ is bipartite, then every face must clearly have even length.
If $G$ is not bipartite, let $C$ be an odd cycle. Then
\[
\sum_{\text{$F$ is inside $C$}} \ell(F) =
\ell(C) + 2 \cdot \sum_{\text{$e$ is inside $C$}} 1 \equiv
1 \pmod{2},
\]
therefore at least one face has odd length.

Now we'll prove that the last two statements are equivalent. First
note that $G^*$ is connected. As lengths of faces coincide with
degrees in $G^*$, this is just Euler's theorem.
\end{proof}

\begin{izrek}
Let $G$ is a plane graph and $D \subseteq E(G)$. The set $D$ is a
set of edges of a cycle if and only if the corresponding dual edge
set $D^*$ is a minimal edge cut.
\end{izrek}

\begin{proof}
If $D$ is a cycle, then $D^*$ separates faces inside the cycle from
the ones outside. It is clear that it is a minimal edge cut. If
$E(C)$ is a proper subset of $D$, then $D^*$ is not a minimal cut.
If $D$ does not contain a cycle, then $D^*$ is not even an edge
cut.
\end{proof}

\begin{definicija}
The planar graph $G$ is \emph{outerplanar}\index{outerplanar graph}
if there exists an embedding such that the outer face contains all
vertices.
\end{definicija}

\begin{opomba}
If $G$ is outerplane and $2$-edge-connected, then it is
Hamiltonian.
\end{opomba}

\begin{izrek}
If $G$ is a simple outerplanar graph then $\delta(G) \leq 2$.
\end{izrek}

\begin{proof}
The statement clearly holds for $n \leq 3$. For $n \geq 4$, we
prove a stronger statement with induction -- there exist two
distinct non neighbouring vertices $u$ and $v$ with degrees at most
$2$. Since $K_4$ is not outerplanar, the statement holds for $n=4$.
For general $n$, consider two cases:

\begin{enumerate}[i)]
\item There is a cut vertex $v$. Let $G_i$ be the graphs obtained
by adding back $v$ to the components of $G \setminus v_i$. As every
graph $G_i$ is outerplanar, they all contain at least one vertex of
degree at most $2$ that is distinct from $v$. Taking one from each
from $G_1$ and $G_2$, we satisfy all conditions.

\item There is no cut vertex in $G$. In particular, there is no cut
edge in $G$, hence the outer face is a Hamiltonian cycle. If $G$ is
a cycle, the statement clearly holds. Otherwise, consider a chord
$xy$. It splits the graph into two outerplanar graph, each
containing a vertex of degree at most $2$ that is distinct from $x$
and $y$. Furthermore, there is clearly no edge between such two
vertices. \qedhere
\end{enumerate}
\end{proof}

\begin{izrek}[Euler's formula]
\index{Euler's formula}
Let $G$ be a plane graph with $k$ components. Then
\[
n + f - e = k+1.
\]
\end{izrek}

\begin{opomba}
If $G$ is a simple planar graph, then $e \leq 3n - 6$. Furthermore,
if $G$ is $K_3$-free, then $e \leq 2n - 4$.
\end{opomba}

\datum{2025-1-9}

\begin{definicija}
A \emph{subdivision}\index{subdivision} of $G$ is a graph that is
obtained by replacing some edges of $G$ with internally
vertex-disjoint paths.
\end{definicija}

\begin{opomba}
A subdivision of $G$ is planar if and only if $G$ is planar.
\end{opomba}

\begin{definicija}
A \emph{Kuratowski graph}\index{Kuratowski graph} is a subdivision
of $K_5$ or $K_{3,3}$.
\end{definicija}

\begin{trditev}
If $G$ is a planar graph, it contains no Kuratowski graph.
\end{trditev}

\begin{proof}
Neither $K_5$ or $K_{3,3}$ are planar graphs.
\end{proof}

\begin{lema}
Let $G$ be a planar graph and $e \in E$. Then there exists an
embedding of $G$ such that $e$ is on the boundary of the outer
face.
\end{lema}

\begin{proof}
Apply an inversion with center inside one of the neighbouring
faces.
\end{proof}

\begin{lema}
If $G$ is a minimal non-planar graph, then $G$ is $2$-connected.
\end{lema}

\begin{proof}
Note that $G$ is clearly connected. Suppose that $G$ has a
cut-vertex $v$, and let $G_1, G_2, \dots$ denote the subgraphs
containing $v$ and a connected component of $G \setminus v$. By the
minimality assumption, these are all planar. By the previous lemma,
each has an embedding such that $v$ is on the edge of the outer
face. Using an appropriate transformation, we can connect all these
embeddings into an embedding of $G$.
\end{proof}

%TODO wtf is an S-lobe?

\begin{lema}
If $S = \set{x,y}$ is a minimum vertex-cut in $G$ and $G$ is
non-planar, then $G \setminus S$ contains a component $G_i$ such
that the $S$-lobe $H_i$ with the added edge $xy$ is non-planar.
\end{lema}

\begin{proof}
Otherwise, embed each such $H_i$ into the plane such that $xy$ is
on the boundary of the outer face. Using suitable transformations
we can attach such embeddings using $x$ and $y$. This gives us an
embedding of $G \cup xy$, which induces an embedding for $G$.
\end{proof}

\begin{lema}
If $G$ is a non-planar graph without Kuratowski subgraphs and has
the minimum number of edges among such graphs, then $G$ is
$3$-connected.
\end{lema}

\begin{proof}
As $G$ is minimal, it is $2$-connected. Suppose that
$S = \set{x,y}$ is a vertex cut. Using the notation from the
previous lemma, there exists a graph $H_i$ which is not planar. By
the minimality of $m(G)$, it has a Kuratowski subgraph $F$, which
must contain $xy$. But since there exists a path between $x$ and
$y$ in $G \setminus F$, we can replace $xy$ with this path to
obtain a Kuratowski subgraph in $G$.
\end{proof}

\begin{definicija}
A \emph{contraction} $G \cdot e$ is the graph $\kvoc{G}{xy}$.
\end{definicija}

\begin{izrek}
If $G$ is a $3$-connected graph with $n \geq 5$, then there exists
an edge $e \in E$ such that $G \cdot e$ is $3$-connected.
\end{izrek}

\begin{proof}
Suppose otherwise. Let $S$ be a vertex cut of $G \cdot e$ with $2$
vertices. If $w = [x] \not \in S$, then $S$ remains a a vertex cut
in $G$, which is not possible. Hence the minimum vertex cut
contains $w$. Thus there exists a vertex cut $S' = \set{x,y,z}$ in
$G$.

We consider an edge $f = uv$ such that $G \setminus \set{u,v,z}$
has the largest possible component $G_i$.

Let $z'$ be a vertex adjacent to $z$ which is not in $S \cup G_i$.
Then there exists a vertex $z^*$ such that $\set{z, z', z^*}$ is a
vertex cut.

Denote by $H$ the subgraph induced by $G_i \cup \set{u,v}$. We
consider three cases:

\begin{enumerate}[i)]
\item If $z^* \in V(H)$ and $H \setminus z^*$ is disconnected, then
$\set{z, z^*}$ is a vertex cut in $G$, which is not possible.
\item If $z^* \in V(H)$ and $H \setminus z^*$ is connected, then
$\set{z, z'}$ is a vertex cut in $G$, which is again not possible.
\item If $z^* \not \in V(H)$, we get a contradiction with
maximality. \qedhere
\end{enumerate}
\end{proof}

\begin{lema}
If $G$ contains no Kuratowski subgraph, then $G \cdot e$ contains
no Kuratowski subgraph for any edge $e$.
\end{lema}

\begin{proof}
Suppose otherwise. Then the Kuratowski subgraph $F$ clearly
contains $w$. We consider the following cases:

\begin{enumerate}[i)]
\item If $\deg(w) = 2$, then $G$ clearly already contained a
Kuratowski subgraph.
\item If $\deg(w) \geq 3$ and at most one neighbour of $w$ is not a
neighbour of $x$, we can replace $w$ with $x$ to get a Kuratowski
subgraph in $G$.
\item In all other cases, we find that $\deg_F(w) \geq 4$, but as
$F$ is a Kuratowski subgraph, we deduce that $\deg_F(w) = 4$ and
that $F$ is a subdivision of $K_5$. It is easy to see that $G$ then
contains a subdivision of $K_{3,3}$. \qedhere
\end{enumerate}
\end{proof}

\begin{definicija}
An embedding is \emph{convex}\index{convex embedding} if the
boundary of every face is a convex polygon.
\end{definicija}

\begin{izrek}
If $G$ is a $3$-connected graph without Kuratowski subgraphs, there
exists a convex embedding of $G$ such that no three vertices are
on a line.
\end{izrek}

\begin{proof}
We induct on $n$. The statement clearly holds for $n=4$.
%TODO next time
\end{proof}

\begin{izrek}[Kuratowski]
\index{Kuratowski's theorem}
A graph $G$ is planar if and only if it contains no Kuratowski
subgraph.
\end{izrek}
