\section{Matchings}

\subsection{Independence, matchings and covers}

\datum{2024-10-2}

\begin{definicija}
A set $S \subseteq V$ is \emph{independent}\index{independent set}
if $G[S]$ contains no edges. We denote the maximal cardinality of
an independent set, the independence number, by $\alpha(G)$.
\end{definicija}

\begin{definicija}
A set $T \subseteq V$ is a \emph{vertex cover}\index{vertex cover}
if it contains a vertex of each edge. We denote the minimal
cardinality of a vertex cover, the vertex cover number, by
$\beta(G)$.
\end{definicija}

\begin{trditev}
The equality in $\alpha(G) + \beta(G) = n$ holds.
\end{trditev}

\begin{proof}
The complement of an independent set is a vertex cover and
vice-versa.
\end{proof}

\begin{definicija}
A set $M \subseteq E$ is a \emph{matching}\index{matching} if no
two of its edges contain the same vertex. We denote the maximal
cardinality of a matching, the matching number, by $\alpha'(G)$.
\end{definicija}

\begin{definicija}
A set $C \subseteq E$ is an \emph{edge cover}\index{edge!cover} if
$\bigcup C = V$. If $\delta(G) \geq 1$, we denote the minimal
cardinality of an edge cover, the edge cover number, by
$\beta'(G)$.
\end{definicija}

\begin{trditev}
We have $\alpha'(G) \leq \beta(G)$.
\end{trditev}

\begin{proof}
We must choose a vertex from each edge of a matching to get a
vertex cover.
\end{proof}

\begin{trditev}
We have $\alpha(G) \leq \beta'(G)$.
\end{trditev}

\begin{proof}
Every edge of an edge cover contains at most one vertex of an
independent set.
\end{proof}

\begin{trditev}
We have $\alpha'(G) \leq \frac{n}{2} \leq \beta'(G)$.
\end{trditev}

\obvs

\begin{izrek}[Gallai]\index{Gallai's theorem}
If $\delta(G) \geq 1$, then $\alpha'(G) + \beta'(G) = n$.
\end{izrek}

\begin{proof}
Take a maximum matching $M$ on $G$ and let $S$ be its vertex set.
We can construct an edge cover from $M$ by adding an edge for each
missing vertex, resulting in $x$ new vertices. Then
\[
\alpha'(G) + \beta'(G) \leq
\abs{M} + x \leq
2 \cdot \abs{M} + \abs{S^{\mathsf{c}}} =
n.
\]
Now let $C$ be a minimum edge cover. Note that each of its edges
covers a vertex that is not covered by any other edge in $C$. That
is, the graph $G[S]$ is a forest of $k$ stars. To construct a
matching, we can choose an arbitrary edge of each star, which gives
\[
\alpha'(G) + \beta'(G) \geq k + (n-k) = n. \qedhere
\]
\end{proof}

\begin{definicija}
Let $M$ be a matching. A path is an
\emph{$M$-alternating path}\index{$M$-alternating path} if its
edges alternate between $M$ and $M^{\mathsf{c}}$.
\end{definicija}

\begin{definicija}
An $M$-alternating path is called
\emph{$M$-augmenting}\index{$M$-augmenting path} if its ends are
not covered by $M$.
\end{definicija}

\datum{2024-10-3}

\begin{trditev}
\label{mtch:prop:max_aug}
Maximum matchings do not contain $M$-augmenting paths.
\end{trditev}

\begin{proof}
We can construct a larger matching $M' = M \oplus P$, where $P$ is
an $M$-augmenting path.
\end{proof}

\begin{izrek}[König]\index{König's theorem}
Let $G$ be a bipartite graph. Then $\alpha'(G) = \beta(G)$. If $M$
is a matching in $G$ that contains no $M$-augmenting path, then it
is a maximum matching.
\end{izrek}

\begin{proof}
Let $M$ be a matching such that no $M$-augmenting path exists in
$G$, and let $A$ and $B$ be the parts of $G$. Denote
$X = A \setminus V(M)$ and $Y = B \setminus V(M)$. Now let $A_1$
and $B_1$ be the set of vertices in $A$ and $B$ respectively that
can be reached via an $M$-alternating path from $X$. Furthermore,
let $A_2 = A \setminus (A_1 \cup X)$ and
$B_2 = B \setminus (B_1 \cup Y)$. Then $A_2 \cup B_1$ is a vertex
cover, as there are no edges in the pairs $(X, Y)$, $(X, B_2)$,
$(A_1, B_2)$ and $(A_1, Y)$. We constructed a vertex cover of the
same cardinality as $M$, hence $M$ must be a maximum matching and
$\alpha'(G) = \beta(G)$.
\end{proof}

\begin{posledica}
If $G$ is a bipartite graph, then $\alpha(G) = \beta'(G)$.
\end{posledica}

\begin{proof}
We have
\[
\alpha(G) = n - \beta(G) = n - \alpha'(G) = \beta'(G). \qedhere
\]
\end{proof}

\begin{izrek}[Hall]\index{Hall's theorem}
Let $G$ be a bipartite graph with parts $A$ and $B$. Then the
equality $\alpha'(G) = \abs{A}$ holds if and only if
$\abs{S} \leq \abs{N(S)}$ for all $S \subseteq A$.
\end{izrek}

\begin{proof}
The first implication is evident. Suppose now that
$\alpha'(G) \ne \abs{A}$ and take a maximum matching $M$ in $G$.
Using the notation from König's theorem, let $S = A_1 \cup X$.
Then $N(S) = B_1$, therefore
\[
\abs{N(S)} = \abs{B_1} = \abs{A_1} < \abs{S}. \qedhere
\]
\end{proof}

\begin{definicija}
A matching $M$ is \emph{perfect}\index{perfect matching} if it
covers all vertices.
\end{definicija}

\begin{posledica}
In a bipartite graph $G$ a perfect matching exists if and only if
$\abs{A} = \abs{B}$ and Hall's condition holds.
\end{posledica}

\begin{definicija}
Let $S \subseteq A$ in a bipartite graph. The
\emph{deficiency}\index{deficiency} of $S$ is defined as
\[
\df(S) = \abs{S} - \abs{N(S)}.
\]
\end{definicija}

\begin{izrek}
In a bipartite graph $G$, we have
\[
\alpha'(G) = \abs{A} - \max_{S \subseteq A} \br{\df(S)}.
\]
\end{izrek}

\begin{izrek}
If $G$ is a regular bipartite graph, it has a perfect matching.
\end{izrek}

\obvs

\datum{2024-10-10}

\begin{izrek}
Suppose $M$ is a matching in $G$. Then there exists an
$M$-augmenting path in $G$ if and only if $M$ is not a maximum
matching.
\end{izrek}

\begin{proof}
One implication is precisely proposition~\ref{mtch:prop:max_aug}.
Suppose now that $M$ is a non-maximum matching. That is, there
exists a matching $M'$ with $\abs{M'} > \abs{M}$. Consider
$G' = G[M \oplus M']$. The maximal degree in $G'$ is clearly at
most $2$, hence every component is either a path or a cycle. As we
have no odd cycles, by $\abs{M'} > \abs{M}$ there exists an
odd-length path in $G'$ with both extreme edges are in $M'$, which
that is an $M$-augmenting path.
\end{proof}

\begin{opomba}
Maximum matchings can be found in polynomial time.
\end{opomba}

\begin{izrek}[Tutte]
\index{Tutte's theorem}
Denote by $\sigma(G)$ the number of odd components in $G$. A graph
$G$ has a perfect matching if and only if the inequality
\[
\abs{S} \geq \sigma(G[V \setminus S])
\]
holds for every $S \subseteq V$.
\end{izrek}

\begin{proof}
Suppose $G$ has a perfect matching. Then every odd component of
$G[V \setminus S]$ is matched to a distinct vertex in $S$, hence
Tutte's condition holds.

Now suppose that Tutte's condition holds for $G$. Note that this
implies that $2 \mid n$, as we can take $S = \emptyset$.
Furthermore, suppose that $G$ is a maximal counterexample, that is,
adding any edge to $G$ produces a graph that either breaks Tutte's
condition or contains a perfect matching. We can check that the
former is actually impossible, as adding edges can only decrease
the number $\sigma(G[V \setminus S])$.

Denote $U = \setb{x \in V}{\deg(x) = n-1}$. Clearly, $G[U]$ is a
complete graph, and hence $U \ne V$. We consider two cases:

\begin{enumerate}[i)]
\item Every component $H$ of $G[U^{\mathsf{c}}]$ induces a complete
graph. In this case, just take a maximum matching of each component
and match the last remaining vertex in odd components with vertices
in $U$. This can clearly be done by Tutte's condition.

\item Some component $H$ of $G[U^{\mathsf{c}}]$ is not complete.
Take $x, y \in H$ with $d(x,y) = 2$, and let $xz, yz \in E$. As
$z \not \in U$, there exists some vertex $w \in V$ such that
$zw \not \in E$. Consider the graphs $G_1$ and $G_2$ that we get by
adding edges $xy$ and $zw$ to $G$, respectively. By our assumption
they have perfect matchings $M_1$ and $M_2$. Clearly they contain
$xy$ and $zw$ respectively.

Now consider $M_1 \oplus M_2$. As every vertex has degree $0$ or
$2$, the graph $G' = G[M_1 \oplus M_2]$ splits into isolated
vertices and cycles. Clearly, the cycles have even length. If $xy$
and $zw$ belong to different cycles, we can just switch the edges
of $M_1$ in the cycle containing $xy$, which produces a perfect
matching in $G$.

Now suppose that the same cycle contains both $xy$ and $zw$. We
choose the edge $xz$ or $yz$, such that the cycle splits into two
even components. We can clearly produce a perfect matching in both
components. By adding the edges of $M_1$ from every other
component, we have in fact constructed a perfect matching. \qedhere
\end{enumerate}
\end{proof}

\begin{izrek}[Berge-Tutte formula]
\index{Berge-Tutte formula}
The maximum matching leaves exactly
\[
\max_{S \subseteq V} (\sigma(G[S^{\mathsf{c}}]) - \abs{S})
\]
vertices uncovered.
\end{izrek}

\datum{2024-10-17}

\begin{definicija}
A \emph{factor}\index{factor} of a graph is a spanning subgraph. A
$k$-factor is a $k$-regular spanning subgraph.
\end{definicija}

\begin{opomba}
A $1$-factor is just a perfect matching.
\end{opomba}

\begin{izrek}[Peterson]
\index{Peterson's theorem}
Every bridgeless cubic\footnote{$3$-regular.} graph has a perfect
matching.
\end{izrek}

\begin{proof}
We will prove that Tutte's condition holds for every set
$S \subseteq V$. Denote by $E \br{S, S^{\mathsf{c}}}$ the edges
between $S$ and $S^{\mathsf{c}}$. Clearly,
$\abs{E \br{S, S^{\mathsf{c}}}} \leq 3 \abs{S}$. By the handshake
lemma, we can see that every odd component $H$ of
$G[S^{\mathsf{c}}]$ is connected to $S$ by an odd number of edges.
As the graph is bridgeless, we can infer that
$\abs{E(V(H), S)} \geq 3$. Therefore
\[
3 \abs{S} \geq
\abs{E \br{S, S^{\mathsf{c}}}} \geq
3 \sigma(G[S^{\mathsf{c}}]). \qedhere
\]
\end{proof}

\begin{izrek}
If $G$ is a cubic graph with at most one bridge, then $G$ has a
perfect matching.
\end{izrek}

\begin{proof}
Repeating the proof of Peterson's theorem, we find that
\[
3 \abs{S} \geq
\abs{E \br{S, S^{\mathsf{c}}}} \geq
3 \sigma(G[S^{\mathsf{c}}]) - 2. \qedhere
\]
\end{proof}

\begin{izrek}
If $G$ is a $k$-regular graph and $k$ is even, then $G$ splits into
$2$-factors.
\end{izrek}

\begin{proof}
It suffices to find one $2$-factor and proceed by induction. It is
clearly enough to consider connected graphs. By Euler's theorem
there exists an Eulerian circuit $C$ in $G$, which induces a
directed graph. Define a bipartite graph $F_G$ by taking
$A = \setb{a_i}{i \leq n}$, $B = \setb{b_i}{i \leq n}$, and take
$a_i b_j$ as an edge in $F_G$ if $v_i v_j \in E \br{\vv{G}}$, where
$v_k$ are vertices in $G$. This is a regular bipartite graph. Its
perfect matching coincides with a $2$-factor of $G$.
\end{proof}
