\section{Mnogoterosti}

\subsection{Osnovni pojmi}

\datum{2025-2-17}

\begin{definicija}
Naj bo $X \subseteq \R^N$ vložena podmnogoterost dimenzije $n$ in
$m \in X$. \emph{Tangentni prostor}\index{tangentni prostor}
$T_m X$ podmnogoterosti $X$ v točki $m$ je prostor
\[
T_m X =
\setb{\eval{\frac{d}{dt}}{t=0}{} \gamma(t)}
{\gamma \in \mathcal{C}^\infty((-\varepsilon, \varepsilon), X)
\land
\gamma(0) = m}.
\]
\end{definicija}

\begin{definicija}
Naj bo $X$ gladka mnogoterost in $\varphi_\alpha$ lokalna karta za
okolico točke $m \in X$. Krivulji
$\gamma_1, \gamma_2 \colon (-\varepsilon, \varepsilon) \to X$, za
kateri je $\gamma_1(0) = \gamma_2(0) = m$, sta \emph{ekvivalentni},
če je
\[
\eval{\frac{d}{dt}}{t=0}{} \varphi_\alpha(\gamma_1(t)) =
\eval{\frac{d}{dt}}{t=0}{} \varphi_\alpha(\gamma_2(t)).
\]
\end{definicija}

\begin{opomba}
Relacija je neodvisna od izbire lokalne karte.
\end{opomba}

\begin{definicija}
Naj bo $X$ gladka mnogoterost in $m \in X$.
\emph{Tangentni prostor}\index{tangentni prostor} $T_m X$
mnogoterosti $X$ v točki $m$ je prostor ekvivalenčnih razredov
zgornjih krivulj.
\end{definicija}

\begin{definicija}
\emph{Predstavnik}\index{predstavnik tangentnega vektorja}
tangentnega vektorja $[\gamma]$ je vektor
\[
\dot{\varphi_\alpha(\gamma)} =
\eval{\frac{d}{dt}}{t=0}{} \varphi_\alpha(\gamma(t)).
\]
\end{definicija}

\begin{definicija}
Na $T_m X$ definiramo
\[
[\gamma_1] + [\gamma_2] =
\left[
\varphi_\alpha^{-1} \br{\varphi_\alpha(m) + t \cdot
\br{\dot{\varphi_\alpha(\gamma_1)} +
\dot{\varphi_\alpha(\gamma_2)}}}
\right]
\]
in
\[
a \cdot [\gamma] = [t \mapsto \gamma(at)].
\]
S tema operacijama $T_m X$ postane vektorski prostor.
\end{definicija}

\begin{definicija}
Naj bo $f \colon X \to \R$ gladka funkcija.
\emph{Smerni odvod}\index{smerni odvod} $f$ v točki $m$ v smeri
$[\gamma]$ je
\[
\br{Vf}_{(m)} [\gamma] =
\eval{\frac{d}{dt}}{t=0}{} \br{f \circ \gamma}(t).
\]
\end{definicija}

\begin{opomba}
Smerni odvod je dobro definiran -- ni odvisen od predstavnika.
\end{opomba}
