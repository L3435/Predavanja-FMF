\section{Brauer group}

\epigraph{``The goal of university bureaucrats is for everyone to
have an above-average grade.''}{-- prof.~dr.~Igor Klep}

\subsection{Definition}

\begin{definicija}
Let $S, T$ be finite-dimensional central simple algebras over $k$.
We say that $S$ and $T$ are \emph{similar}\index{similar algebras}
if any of the following equivalent conditions hold:

\begin{enumerate}[i)]
\item If $S \cong M_n(D)$ and $T \cong M_m(E)$ for division rings
$D$ and $E$, then $D \cong E$.
\item There exist numbers $m, n \in \N$ such that
$S \otimes_k M_m(k) \cong T \otimes_k M_n(k)$.
\item There exist numbers $m, n \in \N$ such that
$M_m(S) \cong M_n(T)$.
\item If $M$ and $N$ are the unique $S$ and $T$-modules
respectively, then we have $\End_S(M) \cong \End_T(N)$.
\end{enumerate}

In this case we write $S \sim T$.
\end{definicija}

\begin{opomba}
The tensor product of central simple algebras is again a central
simple algebra. The tensor product of division algebras is not
necessarily a division algebra.
\end{opomba}

\begin{definicija}
Let $k$ be a field. The \emph{Brauer group}\index{Brauer group}
$\Br(k)$ of $k$ is the set of equivalence classes of
finite-dimensional central simple algebras over $k$ with respect to
the similarity relation. The group operation is induced by the
tensor product.
\end{definicija}

\begin{opomba}
The identity element is $[k]$.
\end{opomba}

\begin{opomba}
If $k$ is a finite field or algebraically closed, then the Brauer
group is trivial.
\end{opomba}

\begin{opomba}
We have $[M_n(k)] = [k]$.
\end{opomba}

\begin{opomba}
If $A$ and $B$ are finite-dimensional central simple algebras over
$k$, then $A \cong B$ if and only if $[A] = [B]$ and
$\dim A = \dim B$.
\end{opomba}

\begin{lema}
We have the following:

\begin{enumerate}[i)]
\item For all $k$-algebras $R$, we have
$M_n(R) \cong R \otimes_k M_n(k)$.
\item We have $M_m(k) \otimes M_m(k) \cong M_{mn}(k)$.
\end{enumerate}
\end{lema}

\begin{lema}
If $S_1 \sim S_2$ and $T_1 \sim T_2$, then
$S_1 \otimes T_1 \sim S_2 \otimes T_2$.
\end{lema}

\begin{proof}
Set $S_j \cong M_{n_j}(D)$ and $T_j \cong M_{m_j}(E)$. Then
\begin{align*}
S_j \otimes T_j &\cong
M_{n_j}(D) \otimes M_{m_j}(E)
\\
&\cong
D \otimes M_{n_j}(k) \otimes M_{m_j}(k) \otimes E
\\
&\cong
D \otimes E \otimes M_{n_j m_j}(k)
\\
&\cong
M_{n_j m_j}(D \otimes E). \qedhere
\end{align*}
\end{proof}

\begin{izrek}
The set $\Br(k)$ is an abelian group.
\end{izrek}

\begin{proof}
By the above lemma, the operation is well defined. It is obviously
associative, commutative and has unit $[k]$. Note that every
element has an inverse, as
\[
S \otimes S^{\mathsf{op}} \cong M_n(k) \sim k. \qedhere
\]
\end{proof}

\newpage

\subsection{Relative Brauer group}

\begin{definicija}
Let $\kvoc{K}{k}$ be a field extension and
$\Phi \colon \Br(k) \to \Br(K)$ a homomorphism, given by
$\Phi([S]) = [K \otimes_k S]$. The
\emph{relative Brauer group}\index{relative Brauer group} is the
group
\[
\Br \br{\kvoc{K}{k}} = \ker \br{\Phi}.
\]
\end{definicija}

\begin{opomba}
These are precisely the central division algebras over $k$ that
split over $K$.
\end{opomba}

\begin{definicija}
Let $S$ be a simple $k$-algebra. A
\emph{self-centralizing subfield}\index{self-centralizing subfield}
of $S$ is a field $K \subseteq S$ such that $C(K) = K$.
\end{definicija}

\begin{opomba}
In division rings, maximal subfields coincide with
self-centralizing ones. This is not true in general and in fact
fails for some central simple algebras (even when both exist).
\end{opomba}

\begin{izrek}
The following statements are true:

\begin{enumerate}[i)]
\item Let $S$ be a central simple algebra over $k$ with
$\dim_k S = n^2$. Then any self-centralizing subfield $K$ of $S$ is
a splitting field for $S$ and $[K : k] = [S : k] = n$.
\item Given any field extension $\kvoc{K}{k}$ with $[K : k] = n$,
any element of $\Br \br{\kvoc{K}{k}}$ has a unique representative
$S$ of degree $n^2$ that contains $K$ as a self-centralizing
subfield.
\end{enumerate}
\end{izrek}

\datum{2023-11-30}

\begin{proof}
\phantom{i}
\begin{enumerate}[i)]
\item Using the centralizer theorem, we find that
\[
n^2 = [S : k] = [K : k] \cdot [C(K) : k] = [K : k]^2.
\]
It remains to check that $K$ is a splitting field for $S$. Let
$f \colon S \otimes_k K \to \End_K(S) \cong M_n(K)$ be given by
$f(s \otimes x) = (s' \mapsto s s' x)$. As $S$ is central simple
and $K$ is simple, their tensor product $S \otimes_k K$ is central
simple. As $f$ is not constant, it must therefore be injective. But
as
\[
[S \otimes K : k] = n^3 = [M_n(K) : k],
\]
the map $f$ must be an isomorphism.
\item Let $[D] \in \Br \br{\kvoc{K}{k}}$ for a division algebra
$D$. Equivalently, $K \otimes_k D^{\mathsf{op}} \cong M_m(K)$ for
some positive integer $m$. In particular,
$[D^{\mathsf{op}} : k] = m^2$.

Let $V$ be the unique $K \otimes_k D^{\mathsf{op}}$-module. As
$K \otimes_k D^{\mathsf{op}} \cong V^m$, we then have
\[
[K : k] \cdot [D^{\mathsf{op}} : k] =
[K \otimes_k D^{\mathsf{op}} : k] =
[V^m : k] =
m \cdot [V : k] =
m \cdot [V : D^{\mathsf{op}}] \cdot [D^{\mathsf{op}} : k].
\]
Observe that $K$ acts on $V$ and that the action commutes with
$D^{\mathsf{op}}$. We can therefore embed $K$ into
$\End_{D^{\mathsf{op}}}(V) \cong M_{[V : D^{\mathsf{op}}]}(D) = S$.
Clearly $[S] = [D]$, and
\[
[S : k] =
[V : D^{\mathsf{op}}]^2 \cdot [D : k] =
[V : D^{\mathsf{op}}]^2 \cdot m^2 =
[K : k]^2.
\]
By the double centralizer theorem, we find that
$[K : k]^2 = [S : k] = [K : k] \cdot [C(K) : k]$. It follows that
$[K : k] = [C(K) : k]$, but as $K \subseteq C(K)$, we must have
equality. It follows that $S$ is indeed self-centralizing. It is
unique by the dimension requirement. \qedhere
\end{enumerate}
\end{proof}

\begin{opomba}[Jacobson-Noether theorem]
\index{Jacobson-Noether theorem}
For any division algebra with center $k$ there exists a splitting
field $K \subseteq D$ that is separable over $k$.
\end{opomba}

\begin{posledica}
Let $D$ be a finite-dimensional division algebra with center $k$.
Then there exists a finite Galois extension $\kvoc{K}{k}$ which is
a splitting field for $D$.
\end{posledica}

\begin{proof}
By the Jacobson-Noether theorem there exists a maximal splitting
subfield $L \subseteq D$ that is separable over $k$. Let $K$ be the
normal closure of $L$. Then $\kvoc{K}{k}$ is a Galois extension and
\[
D \otimes_k K \cong
(D \otimes_k L) \otimes_L K \cong
M_n(L) \otimes_L K \cong
M_n(K). \qedhere
\]
\end{proof}

\begin{posledica}
Suppose that $D$ is a central division $k$-algebra with
$[D : k] = n^2$. Then any splitting field $K$ of $D$ satisfies
$n \mid [K : k]$.
\end{posledica}

\begin{proof}
Note that $D \otimes_k K \cong M_n(K)$ by comparing dimensions over
$k$. Let $V$ be the unique simple $M_n(K)$-module. In particular,
$V \cong K^n$. As $V$ is also a $D$-vector space, we can write
$V \cong D^s$. But then
\[
s n^2 =
s \cdot [D : k] =
[V : k] =
n \cdot [K : k]. \qedhere
\]
\end{proof}

\newpage

\subsection{Factor sets and crossed product algebras}

\begin{definicija}
Let $\kvoc{K}{k}$ be a Galois field extension and denote
$G = \Gal \br{\kvoc{K}{k}}$. Let $S$ be a central simple
$k$-algebra with center $K$. For every $\sigma \in G$ choose an
element $x_\sigma \in S^{-1}$ such that
$x_\sigma \cdot a \cdot x_\sigma^{-1} = \sigma(a)$ for all
$a \in K$. Furthermore, let
$a_{\sigma, \tau} =
x_\sigma \cdot x_\tau \cdot x_{\sigma \tau}^{-1}$. The set
$\br{a_{\sigma, \tau}}_{\sigma, \tau \in G}$ is called the
\emph{factor set}\index{factor set} of $S$ relative to $K$.
\end{definicija}

\begin{opomba}
Such elements $x_\sigma$ exist by the Skolem-Noether theorem. A
simple calculation shows that $a_{\sigma, \tau} \in K$ as
$x_\sigma^{-1} \cdot x_\sigma' \in K$.
\end{opomba}

\begin{definicija}
A factor set is \emph{normalized}\index{normalized factor set} if
$x_{\id} = 1$.
\end{definicija}

\begin{opomba}
If the factor set is normalized, we have
$a_{\sigma, \id} = a_{\id, \sigma} = 1$.
\end{opomba}

\begin{opomba}
\label{brg:rmk:eq_cls}
Suppose that $x_\sigma' = f_\sigma \cdot x_\sigma$ gives a factor
set $\br{b_{\sigma, \tau}}_{\sigma, \tau \in G}$. Then
\[
b_{\sigma, \tau} f_{\sigma, \tau} x_{\sigma, \tau} =
b_{\sigma, \tau} x_{\sigma, \tau}' =
x_\sigma' x_\tau' =
f_\sigma x_\sigma f_\tau x_\tau =
f_\sigma \sigma(f_\tau) x_\sigma x_\tau,
\]
therefore
\[
b_{\sigma, \tau} =
\frac{f_\sigma \sigma(f_\tau)}{f_{\sigma, \tau}} a_{\sigma, \tau}.
\]
\end{opomba}

\begin{trditev}
The set $(x_\sigma)_{\sigma \in G}$ is a basis for $S$ over $K$.
\end{trditev}

\begin{proof}
Observe that $\abs{G} = [K : k] = [S : K]$, therefore we only need
to check linear independence. Suppose then that they are linearly
dependent and let $J \subset G$ be a maximal set such that
$(x_\sigma)_{\sigma \in J}$ is linearly independent.

Let $\sigma \in G \setminus J$ and write
\[
x_\sigma = \sum_{\tau \in J} a_\tau x_\tau
\]
for $a_\tau \in K$. For any $r \in K$, we find that
\[
\sum_{\tau \in J} \sigma(r) a_\tau x_\tau =
\sigma(r) x_\sigma =
x_\sigma r =
\sum_{\tau \in J} a_\tau x_\tau r =
\sum_{\tau \in J} a_\tau \tau(r) x_\tau.
\]
By linear independence, we must have
$\sigma(r) a_\tau = a_\tau \tau(r)$ for all $\tau \in J$. But as
at least one of $a_\tau$ is non-zero, in which case we have
$\sigma(r) = \tau(r)$ and therefore $\sigma = \tau \in J$.
\end{proof}

\begin{posledica}
As a $K$-vector space,
\[
S = \bigoplus_{\sigma \in G} K x_{\sigma}
\]
with multiplication $x_\sigma \cdot a = \sigma(a) \cdot x_\sigma$
and $x_\sigma \cdot x_\tau = a_{\sigma, \tau} x_{\sigma \tau}$.
\end{posledica}

\begin{definicija}
Any set $(a_{\sigma, \tau})_{\sigma, \tau} \subseteq K^{-1}$ that
satisfies
\[
\rho(a_{\sigma, \tau}) a_{\rho, \sigma \tau} =
a_{\rho, \sigma} \cdot a_{\rho \sigma, \tau}
\]
is called a \emph{factor set}\index{factor set} relative to $K$.
\end{definicija}

\begin{opomba}
Note that a factor set of $S$ is indeed a factor set by this
definition, as
\[
\rho(a_{\sigma, \tau}) a_{\rho, \sigma \tau} x_{\rho \sigma \tau} =
\rho(a_{\sigma, \tau}) x_\rho x_{\sigma \tau} =
x_\rho x_\sigma x_\tau =
a_{\rho, \sigma} a_{\rho \sigma, \tau} x_{\rho \sigma \tau}.
\]
\end{opomba}

\begin{trditev}
Let $\kvoc{K}{k}$ be a Galois extension with
$G = \Gal \br{\kvoc{K}{k}}$. Then any factor set relative to $K$ is
a factor set of some central simple algebra over $K$. Furthermore,
$A$ contains $K$ as a self-centralizing subfield.
\end{trditev}

\begin{proof}
Let $A$ be the $K$-vector space with basis
$(e_\sigma)_{\sigma \in G}$. We define multiplication on $A$ as
\[
\alpha e_\sigma \cdot \beta e_\tau =
\alpha \sigma(\beta) a_{\sigma, \tau} e_{\sigma \tau}.
\]
Associativity follows from the fact that
$(a_{\sigma, \tau})_{\sigma, \tau \in G}$ is a factor set. Note
that $a_{\id, \id}^{-1} e_{\id}$ is the identity element. Then $K$
is clearly a subfield of $A$ by inclusion
$r \mapsto r a_{\id, \id}^{-1} e_{\id}$.

Next, we prove that $K$ is self-centralizing. Suppose that
\[
x = \sum_{\sigma \in G} a_\sigma e_\sigma \in C(K).
\]
Multiplication by $a \in K$ gives us $a a_\sigma = a_\sigma a$ for
all $\sigma$. If $a_\sigma \ne 0$ for some $\sigma \ne \id$, this
gives a clear contradiction as $a = \sigma(a)$ does not hold for
all $a \in K$. Hence $x \in K$.

We now check that $Z(A) = K$. Clearly, $Z(A) \subseteq K$. Let
$a \cdot e_{\id} \in Z(A)$. In particular,
$a e_{\id} e_\sigma = e_\sigma a e_{\id}$, which implies
\[
a a_{\id, \sigma} e_\sigma = \sigma(a) a_{\sigma, \id} e_\sigma.
\]
But as $a_{\id, \sigma} = a_{\id, \id}$ and
$a_{\sigma, \id} = \sigma(a_{\id, \id})$ by
definition, it follows that
\[
a \cdot a_{\id, \id} = \sigma(a \cdot a_{\id, \id}),
\]
hence $a a_{\id, \id} \in K^G = k$. But then
\[
a \cdot e_{id} =
a a_{\id, \id} \cdot a_{\id, id}^{-1} e_{\id} \in k.
\]
Finally, we prove that $A$ is simple. Suppose that $I \edn A$ is
a proper ideal. Then the map $K \mapsto \kvoc{A}{I}$ is injective.
Let $\oline{e}_\sigma$ be the image of $e_\sigma$ in $\kvoc{A}{I}$.
Then $\oline{e}_\sigma a = a \oline{e}_\sigma$ for all $a \in k$.
Similarly as above, we can see that
$\br{\oline{e}_\sigma}_{\sigma \in G}$ are linearly independent,
hence $\dim \kvoc{A}{I} \geq \abs{G} = \dim A$. It follows that
$I = (0)$.
\end{proof}

\begin{definicija}
The above algebra $A$ is called the
\emph{crossed product}\index{crossed product} of $K$ and $G$
relative to the factor set
$(a_{\sigma, \tau})_{\sigma, \tau \in G}$. We write
$A = (K, G, a)$.
\end{definicija}

\begin{lema}
For factor sets with $x_\sigma' = f_\sigma \cdot x_\sigma$ with
factor sets $b$ and $a$, the map $(K, G, b) \to (K, G, a)$, given
by $x_\sigma' \mapsto f_\sigma x_\sigma$, is a $k$-algebra
isomorphism.
\end{lema}

\begin{izrek}
Let $\kvoc{K}{k}$ be a finite Galois extension with
$G = \Gal \br{\kvoc{K}{k}}$. Then there exists a bijective
correspondence between $\Br \br{\kvoc{K}{k}}$ and equivalence
classes\footnote{Factor sets are equivalent if they satisfy the
equation in remark~\ref{brg:rmk:eq_cls}.} of factor sets.
\end{izrek}

\begin{proof}
Recall that for each $x \in \Br \br{\kvoc{K}{k}}$ there exists a
unique central simple algebra $A$ such that $[A] = x$. This gives
rise to a map $\Br \br{\kvoc{K}{k}}$ and factor sets. Conversely,
the crossed product algebras give us the reverse map. It is clear
that they are inverses.
\end{proof}

\newpage

\subsection{Group cohomology and Brauer group}

\begin{definicija}
Let $G$ be a group that acts on an abelian group $M$. The $n$-th
\emph{cochain group}\index{cochain group} is the group
$C^n(G, M) = \set{f \colon G^n \to M}$ under pointwise addition.
\end{definicija}

\begin{opomba}
The group $G$ acts on $C^n(G, M)$.
\end{opomba}

\begin{definicija}
The $n$-th \emph{coboundary map}\index{coboundary map} is the
homomorphism $\delta_n \colon C^n(G, M) \to C^{n+1}(G, M)$, defined
as
\[
\delta_n f =
g_1 f(g_2, \dots, g_{n+1}) +
\sum_{i=1}^n (-1)^i f(g_1, \dots, g_i g_{i+1}, \dots, g_{n+1}) +
(-1)^{n+1} f(g_1, \dots, g_n).
\]
\end{definicija}

\begin{trditev}
For all $n \in \N_0$ we have $\delta_n \circ \delta_{n+1} = 0$.
\end{trditev}

\begin{proof}
Direct calculation.
\end{proof}

\datum{2023-12-7}

\begin{definicija}
The sequence $(C^n(G, M), \delta_n)$ forms a
\emph{cochain complex}\index{cochain complex}. We define
\emph{$n$-cocylces}\index{cocycle} as $Z^n = \ker(\delta_n)$ and
\emph{$n$-coboundaries}\index{coboundaries} as
x$B^n = \im(\delta_{n-1})$.
\end{definicija}

\begin{definicija}
The \emph{$n$-th cohomology group}\index{cohomology group} of $G$
with coefficients in $M$ is the group
\[
H^n(G, M) = \kvoc{Z^n}{B^n}.
\]
\end{definicija}

\begin{definicija}
Let $\kvoc{K}{k}$ be a finite Galois extension,
$G = \Gal \br{\kvoc{K}{k}}$ and $M = K^{-1}$. The
\emph{$n$-th Galois cohomology group}\index{Galois cohomology group}
of the extension $\kvoc{K}{k}$ with coefficients in $K^{-1}$ is the
group $H^n \br{G, K^{-1}}$.
\end{definicija}

\begin{lema}
Let $K$ be a field and $\sigma_1, \dots, \sigma_n$ be distinct
automorphisms of $K$. Then $\sigma_1, \dots, \sigma_n$ are linearly
independent.
\end{lema}

\begin{proof}
Permuting the automorphisms if needed, suppose that
\[
\sum_{j=1}^r c_j \sigma_j = 0,
\]
where $c_j \ne 0$ and $r$ is minimal. Note that $r > 1$, as
otherwise we'd have $0 = c_1 \sigma_1(1) = c_1$.

There exists some $a \in K$ such that
$\sigma_1(a) \ne \sigma_r(a)$. For all $x \in K$ we have
\[
0 =
\sum_{j=1}^r c_j \sigma_j(ax) =
\sum_{j=1}^r c_j \sigma_j(a) \sigma_j(x).
\]
But then
\[
\sum_{j=1}^{r-1}
\br{c_j \cdot \br{\sigma_j(a) - \sigma_r(a)}} \sigma_j(x) = 0,
\]
which is a contradiction.
\end{proof}

\begin{izrek}[Hilbert's theorem 90]
\index{Hilbert's theorem 90}
The first two Galois cohomology groups are
$H^0(G, K^{-1}) = k^{-1}$ and $H^1(G, K^{-1}) = 1$, respectively.
\end{izrek}

\begin{proof}
Notice that
\[
\ker \delta_0 =
\setb{f \in K^{-1}}{\forall g \in G \colon gf = f} =
K^{\Gal(\kvoc{K}{k})} \setminus \set{0} =
k^{-1},
\]
which proves $H^0(G, K^{-1}) = k^{-1}$.

Now let $f \in Z^1$, that is
\[
1 =
(\delta_1 f)(\sigma, \tau) =
\sigma(f(\tau)) \cdot f(\sigma \tau)^{-1} \cdot f(\sigma).
\]
Equivalently, we have
\[
f(\sigma \tau) = f(\sigma) \cdot \sigma(f(\tau)).
\]
We claim that $f \in B^1$, which is equivalent to $f = \delta_0 g$
for $g \in K^{-1}$, that is
\[
f(\tau) = \tau(g) \cdot g^{-1}.
\]
Consider
\[
\sum_{\tau \in G} f(\tau) \tau.
\]
By the above lemma, this linear combination is non-zero. Choose
$a \in K^{-1}$ such that
\[
b = \sum_{\tau \in G} f(\tau) \tau(a) \ne 0.
\]
We now have
\[
\sigma(b) =
\sum_{\tau \in G} \sigma(f(\tau)) \cdot (\sigma \tau)(a) =
f(\sigma)^{-1} \cdot \sum_{\tau \in G}
f(\sigma \tau) \cdot (\sigma \tau)(a) =
f(\sigma)^{-1} \cdot b.
\]
We therefore have
\[
f(\sigma) = \frac{b}{\sigma(b)} = \frac{\sigma(b^{-1})}{b^{-1}}.
\qedhere
\]
\end{proof}

\begin{opomba}
Observe that $1 = \delta_2(a)(\rho, \sigma, \tau)$ is equivalent to
\[
\rho(a_{\sigma, \tau}) \cdot a_{\rho, \sigma \tau} =
a_{\rho, \sigma} \cdot a_{\rho \sigma, \tau}.
\]
That is, cocycles of $C^2(G, K^{-1})$ are precisely factor sets
relative to $K$. As $B^2(G, K^{-1})$ consists of elements
$\sigma(f_\tau) f_{\sigma, \tau}^{-1} f_\sigma$, the second
cohomology group is in bijective correspondence with equivalence
classes of factor sets.
\end{opomba}

\begin{lema}
The map $\Psi \colon H^2(G, K^{-1}) \to \Br \br{\kvoc{K}{k}}$ with
$\Psi(a) = [(K, G, a)]$ is a group isomorphism.
\end{lema}

\begin{proof}
The map is bijective by the earlier observations. Let
$A = [(K, G, a)]$, $B = [(K, G, b)]$ and $C = [(K, G, ab)]$ and
define $M = A^{\mathsf{op}} \otimes_K B$ as a $K$-module. Then
$M$ is also a right $A \otimes_k B$-module with the natural
multiplication.

We claim that $M$ is a $C$-module as well. Indeed, let
$(u_\sigma)_{\sigma \in G}$, $(v_\sigma)_{\sigma \in G}$ and
$(w_\sigma)_{\sigma \in G}$ be basis for $A$, $B$ and $C$, and
define
\[
(x \cdot w_\sigma) \cdot (a \otimes_K b) =
x u_\sigma a \otimes_K v_\sigma b.
\]
This makes $M$ into a left $C$-module, as
\begin{align*}
(x w_\sigma x' w_\tau) \cdot a \otimes_K b &=
x \sigma(x') a_{\sigma, \tau} b_{\sigma, \tau}
u_{\sigma \tau} a \otimes_K v_{\sigma \tau} b
\\
&=
x \sigma(x') a_{\sigma, \tau} u_{\sigma \tau} a \otimes_K
b_{\sigma, \tau} v_{\sigma \tau} b
\\
&=
x \sigma(x') u_\sigma u_\tau a \otimes_K
v_\sigma v_\tau b
\\
&=
x w_\sigma \cdot (x' w_\tau \cdot a \otimes_K b).
\end{align*}
It follows that there exists a $k$-algebra homomorphism
$\Phi \colon (A \otimes_k B)^{\mathsf{op}} \to \End_C(M)$, given
by $x \mapsto (m \mapsto mx)$. As $A \otimes_k B$ is a central
simple algebra, $\Phi$ is injective.

Now let $n = \abs{G} = [K : k]$. As
$n = [A : K] = [B : K] = [C : K]$, we find that $[M : K] = n^2$.
But then $[M : k] = n^2 \cdot [K : k] = n^3 = n \cdot [C : k]$.
As $C$ is simple, we get $M \cong C^n$ and therefore
\[
\End_C(M) \cong
M_n(\End_C(C)) \cong
M_n \br{C^{\mathsf{op}}} \cong
C^{\mathsf{op}} \otimes_k M_n(k).
\]
But then $\dim_k \End_C(M) = n^2 \cdot [C : k] = n^4 =
\dim_k (A \otimes_k B)$ and $\Phi$ is indeed bijective. It follows
that $(A \otimes_k B)^{\mathsf{op}} \cong
C^{\mathsf{op}} \otimes_k M_n(k)$ and so
$A \otimes_k B \sim C$, as required.
\end{proof}

\begin{izrek}
If $G$ is a finite group, then $\abs{G} \cdot H^2(G, M) = 0$.
\end{izrek}

\begin{proof}
For $f \in Z^2$, we can express
\[
f(g_1, g_2) = g_1 f(g_2, g_3) - f(g_1 g_2, g_3) + f(g_1, g_2 g_3).
\]
Taking a sum over all $g_3 \in G$, we get
\[
\abs{G} \cdot f(g_1, g_2) =
g_1 \cdot h(g_2) - h(g_1 g_2) + h(g_1) = \delta_1(h),
\]
where
\[
h(g) = \sum_{x \in G} f(g, x). \qedhere
\]
\end{proof}

\begin{opomba}
Similarly, $\abs{G} \cdot H^n(G, M) = 0$.
\end{opomba}

\begin{posledica}
For any field $k$, $\Br(k)$ is a torsion abelian group.
\end{posledica}

\begin{proof}
Recall that $\Br(k)$ is the union of $\Br \br{\kvoc{K}{k}}$ over
all Galois extensions $\kvoc{K}{k}$. As each such group is torsion,
so is $\Br(k)$.
\end{proof}

\newpage

\subsection{Primary decomposition for division algebras}

\begin{definicija}
Let $K$ be a splitting field for a central division algebra $D$
over $k$, that is $D \otimes_k K \cong M_n(K)$. The
\emph{degree}\index{degree} of $D$ is defined as
\[
\deg D = n = \sqrt{[D : k]}.
\]
\end{definicija}

\begin{definicija}
Let $A$ be a central simple algebra over $k$ with $A \cong M_m(D)$
for a division algebra $D$. The \emph{index}\index{index} of $A$
is defined as $\ind A = \deg D$. The
\emph{exponent}\index{exponent} of $A$ is defined as
$\exp A = \ord_{\Br(k)} [A]$.
\end{definicija}

\begin{trditev}
For a central simple algebra $A$ over $k$ we have
$[A]^{\ind A} = 1$, that is $\exp A \mid \ind A$.
\end{trditev}

\begin{proof}
Write $[A] = [(K, G, a)]$ for some finite Galois extension
$\kvoc{K}{k}$. As $A \cong M_r(D)$ for some division algebra $D$,
we can write $[D : k] = m^2$ and $\ind A = m$. Furthermore,
$[A : k] = n^2$, where $n = mr$. As $[A]^m = [(K, G, a^m)]$, it
suffices to show that $a^m \in B^2$.

Let $V = (D^{\mathsf{op}})^r$, which is a left
$\End_{D^{\mathsf{op}}}(V)$-module. As
$A \cong M_r(D) \cong \End_{D^{\mathsf{op}}}(V)$, it is also an
$A$-module and hence a $K$-vector space. But then
\[
r m^2 =
[V : D^{\mathsf{op}}] \cdot [D^{\mathsf{op}} : k] =
[V : k] =
[V : K] \cdot n =
[V : K] \cdot rm,
\]
hence $[V : K] = m$.

Fix a basis $\setb{v_i}{i \leq m}$ for $V$ over $K$. For any
$c \in A$, we can write
\[
c v_i = \sum_{j=1}^m c_{i,j} v_j.
\]
This gives us a map $A \to M_m(K)$. Let $X_\sigma$ be the
corresponding matrix for the element $x_\sigma$. Then
\[
a_{\sigma, \tau} X_{\sigma \tau} v =
a_{\sigma, \tau} x_{\sigma \tau} v =
x_\sigma (x_\tau v) =
x_\sigma X_\tau v =
\sigma(X_\tau) X_\sigma v.
\]
Hence
$a_{\sigma, \tau} X_{\sigma \tau} = \sigma(X_{\tau}) X_\sigma$.
Taking the determinant, we find
\[
a_{\sigma, \tau}^m =
\frac{\sigma(\det X_\tau) \det X_\sigma}{\det X_{\sigma \tau}},
\]
which is of course an element of $B^2$.
\end{proof}

\datum{2023-12-14}

\begin{trditev}
Every prime divisor of $\ind A$ divides $\exp A$.
\end{trditev}

\begin{proof}
Take $[A] = [(K, G, a)] = [M_m(D)] = [D]$ as usual and denote
$d = \ind A = \deg D$. Suppose that $p \mid d$. As
$\abs{G}^2 = [(K, G, a) : k] = m^2 d^2$, $p \mid \abs{G}$. Let then
$G_p$ be the $p$-Sylow subgroup of $G$ and let
$K_p = K^{G_p} \subseteq K$. By the fundamental theorem of Galois
theory, $[K : K_p] = p^r$ for some $r \in \N$, and as $G_p$ is
a Sylow subgroup, $p \nmid [K_p : k]$. In particular, $K_p$ cannot
split $A$, hence $\exp \br{A_{K_p}} \ne 1$.

As $A \otimes_k K = (A \otimes_k K_p) \otimes_{K_p} K$ splits and
$[K : K_p] = p^r$, it follows that $\ind \br{A_{K_p}} \mid p^r$.
But then $\exp \br{A_{K_p}} \mid p^r$, hence
$p \mid \exp \br{A_{K_p}}$. As the scalar extension map
$\Br(k) \to \Br(K_p)$, given by $[S] \mapsto [S_{K_p}]$, is a
group homomorphism, we get $\exp \br{A_{K_p}} \mid \exp A$.
\end{proof}

\begin{trditev}
Suppose that $D_1$ and $D_2$ are central division algebras with
coprime degrees. Then $D_1 \otimes D_2$ is a division algebra.
\end{trditev}

\begin{proof}
We can write $D_1 \otimes D_2 \cong M_m(D)$ for some division
algebra $D$. Write $D_1^{\mathsf{op}} \otimes D_1 \cong M_n(k)$.
Then $n = [D_1 : k]$ and
\[
M_n(D_2) \cong
M_n(k) \otimes D_2 \cong
D_1^{\mathsf{op}} \otimes M_m(D) \cong
M_m(D_1^{\mathsf{op}} \otimes D) \cong
M_m(M_r(D')) \cong
M_{mr}(D').
\]
By uniqueness of Wedderburn's decomposition, we get $mr = n$ and
$D' = D_2$, hence $m \mid n = [D_1 : k]$. Similarly,
$m \mid [D_2 : k]$ and so $m = 1$.
\end{proof}

\begin{izrek}
Let $D$ be a finite-dimensional central division algebra over $k$
with
\[
\deg D = \prod_{i=1}^r p_i^{n_i}.
\]
Then there exists a unique decomposition
\[
D = \bigotimes_{i=1}^r D_i,
\]
where $D_j$ are central division algebras with
$\deg D_j = p_j^{n_j}$.
\end{izrek}

\begin{proof}
It suffices to show that we can write $D = D_1 \otimes D_2$ with
$\deg D_j = n_j$ if $n = n_1 n_2$ and $\gcd(n_1, n_2) = 1$. Write
$u n_1 + v n_2 = 1$ and let $D_1$ be the unique central division
algebra with $[D_1] = [D]^{vn_2}$. Define $D_2$ similarly. Then
$[D_1 \otimes D_2] = [D]$ and $[D_1]^{n_1} = [k]$, hence
$\exp D_1 \mid n_1$ and $\exp D_2 \mid n_2$. As $\exp A$ and
$\ind A$ have the same prime factors, $(\ind D_1, \ind D_2) = 1$.
It follows that $D_1 \otimes D_2$ is in fact a division algebra,
therefore $D_1 \otimes D_2 \cong D$. A comparison of dimensions
shows that $\deg D_j = n_j$.
\end{proof}
