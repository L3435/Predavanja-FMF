\section{Primitive rings}

\subsection{Density theorem}

\datum{2023-10-26}

\begin{definicija}
A ring $R$ is \emph{primitive}\index{primitive ring} if it has a
faithful\footnote{$\ann(M) = (0)$.} simple module $M$.
\end{definicija}

\begin{opomba}
Equivalently, the representation $R \to \End(M)$ is injective and
irreducible.
\end{opomba}

\begin{definicija}
Let $V$ be a vector space over a division ring $D$, and let
$R \subseteq \End_D(V)$ be a subring. The ring $R$ is a
\emph{dense ring of linear transformations}\index{dense ring}\footnote{$R$ acts densely on $V$.}
if for every finite set $\set{v_1, \dots, v_n} \subseteq V$ of
linearly independent vectors and any
$\set{w_1, \dots, w_n} \subseteq V$ there exists some $\phi \in R$
such that $\phi(v_i) = w_i$ for all $i \leq n$.
\end{definicija}

\begin{izrek}[Density]
\index{density theorem!semisimple modules}
Let $M$ be a semisimple module over a ring $R$. We denote
$S = \End_R(M)$ and let $\phi \in \End_S(M)$. Then for any finite
set $\set{x_1, \dots, x_n} \subseteq M$ there exists an element
$r \in R$ such that $\phi(x_i) = r x_i$ for all $i \leq n$.
\end{izrek}

\begin{proof}
For $n = 1$ we can write
\[
M = R x_1 \oplus M'
\]
as $M$ is semisimple. Note that $\pi \colon M \to R x_1$ is an
element of $S$. It follows that
\[
\phi(x_1) = \phi(\pi(x_1)) = \pi(\phi(x_1)),
\]
therefore $\phi(x_1) \in R x_1$.

Consider now $M^n$ and $\phi^{(n)} \colon M^n \to M^n$ as the
point-wise application of $\phi$. Observe that
$\phi^{(n)} \in \End_{\End_R(M^n)}(M^n)$. As $M^n$ is a semisimple
$R$-module, we can apply the $n=1$ case.
\end{proof}

\begin{izrek}[Jacobson]
\index{Jacobson theorem}
A ring $R$ is primitive if and only if $R$ is a dense ring of
linear transformations on a vector space over a division ring.
\end{izrek}

\begin{proof}
Assume that $R$ is a primitive ring and let $M$ be a faithful and
simple $R$-module. By Schur's lemma, $D = \End_R(M)$ is a division
ring, therefore, $M$ is a $D$-vector space. Since $M$ is faithful,
we have $R \subseteq \End_D(M)$, therefore $R$ acts as a ring of
linear transformations on $M$. By the density theorem for modules
the ring $R$ is dense.

Assume now that $R$ is a dense ring of linear transformations on a
vector space $V$ over a division ring $D$. In particular, $V$ is
an $R$-module. By definition we have $R \subseteq \End_D(V)$,
therefore $V$ is a faithful $R$-module. It is clear that every
non-zero element generates $V$, therefore $V$ is simple.
\end{proof}

\begin{posledica}
Any simple artinian ring $R$ is isomorphic to $M_n(D)$ for a
division ring $D$.
\end{posledica}

\begin{proof}
As $R$ is simple, it is primitive. Let $M$ be a faithful and simple
$R$-module and denote $D = \End_R(M)$. This is of course a division
ring by Schur's lemma. By Jacobson's theorem, $R$ is a dense
subring of $\End_D(M)$.

Assume that $\dim_D M = \infty$ and let $(v_n)_n$ be an infinite
sequence of linearly independent vectors in $M$. Let
\[
I_n = \setb{r \in R}{\forall i \leq n \colon r v_i = 0}
\]
be a submodule of $R$. Note that these submodules form a strictly
decreasing chain, which is impossible.

As $\dim_D M < \infty$, we know that
$R = \End_D(M) \cong M_{\dim_D M}(D)$.
\end{proof}

\begin{izrek}[Structure]
\index{structure theorem!primitive rings}
Let $R$ be a primitive ring with a faithful simple module $M$ and
denote $D = \End_R(M)$. Then either

\begin{enumerate}[i)]
\item $R \cong M_n(D)$ for some $n \in \N$ or
\item for all $m \in \N$ there exists a subring $R_m \subseteq R$
and an endomorphism $R_m \to M_m(D)$.
\end{enumerate}
\end{izrek}

\begin{proof}
If $\dim_D M < \infty$, we have $R = \End_D(M) = M_n(D)$ for
$n = \dim_D M$. Now assume that $\dim_D M = \infty$. If
$(v_n)_n$ is an infinite sequence of linearly independent vectors
in $M$, form $V_m = \Lin_D \setb{v_i}{i \leq m}$. Now set
\[
R_m = \setb{r \in R}{r \cdot V_m \subseteq V_m}.
\]
Note that
\[
I_m = \setb{r \in R}{r V_m = 0}
\]
is an ideal in $R_m$. By Jacobson's theorem we have
$\kvoc{R_m}{I_m} \cong M_m(D)$.
\end{proof}

\begin{opomba}
In the case of finite-dimensional algebras the notions of primitive
and simple coincide.
\end{opomba}

\begin{opomba}
The free algebra is primitive. Every algebra is the image of some
primitive algebra.
\end{opomba}

\newpage

\subsection{An application of primitive rings}

\begin{trditev}
Suppose that $R$ is a ring in which $x^3 = x$ holds for all
$x \in R$. Then $R$ is commutative.
\end{trditev}

\begin{proof}
Note that if $ab = 0$, we also have $ba = (ba)^3 = 0$. Let
$e \in R$ be an idempotent. Note that for all $x \in R$ we have
$e (x - ex) = 0$, therefore $xe = exe$. Similarly, we have
$(x - xe) e = 0$, therefore $ex = exe$. This implies that
$e \in Z(R)$.

Let $x \in R$ and note that $(x^2)^2 = x^2$. Therefore, $x^2$ is an
idempotent and we have $x^2 \in Z(R)$. Also note that for all
$c \in R$ such that $c^2 = 2c$ we have $c = 2c^2 \in Z(R)$.

Now let $x \in R$. Note that
\[
(x^2 + x)^2 = x^4 + 2x^3 + x^2 = 2(x^2 + x),
\]
therefore, we have both $x^2 + x \in Z(R)$ and
$x = (x^2 + x) - x^2 \in Z(R)$.
\end{proof}

\begin{izrek}[Jacobson]
\index{Jacobson theorem}
Suppose that $R$ is a ring such that for all $x$ there exists an
$n > 1$ such that $x^n = x$. Then $R$ is commutative.
\end{izrek}

\begin{izrek}[Jacobson-Herstein]
\index{Jacobson-Herstein theorem}
A ring $R$ is commutative if and only if for all $x, y \in R$
there exists an $n > 1$ such that
\[
(xy - yx)^n = xy - yx.
\]
\end{izrek}

\begin{trditev}
A ring $R$ is J-semisimple if and only if it has a faithful
semisimple module $M$.
\end{trditev}

\begin{proof}
Suppose that $R$ has a faithful semisimple module $M$. Recall that
the radical is the set of all elements that act trivially on all
simple $R$-modules. It follows that $\rad R \cdot M = 0$, whence
$\rad R = (0)$ as $M$ is faithful.

Suppose now that $R$ is J-semisimple. Let $(M_i)_{i \in I}$ be all
non-isomorphic simple $R$-modules and
\[
M = \bigoplus_{i \in I} M_i.
\]
Note that
\[
\ann(M) = \bigcap_{i \in I} \ann(M_i) = \rad(R) = (0). \qedhere
\]
\end{proof}

\begin{posledica}
Every J-semisimple ring $R$ is a subdirect product of primitive
rings.
\end{posledica}

\begin{proof}
The inclusion
\[
R \hookrightarrow \kvoc{\prod_{i \in I} R}{\ann M_i}
\]
is the desired representation.
\end{proof}
