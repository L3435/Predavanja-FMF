\section{Central simple algebras}

\epigraph{``Coming soon, poker, chess, or three in a
line.''}{-- prof.~dr.~Igor Klep}

\subsection{Cyclic algebras}

\begin{trditev}
Let $k$ be a field and $\sigma \in \Aut(k)$. Let
\[
D =
k((x, \sigma)) =
\setb{\sum_{i=m}^\infty a_i x^i}{m \in \Z \land a_i \in k},
\]
with $x \cdot a = \sigma(a) x$ for all $a \in k$, be a division
ring. Denote $k_0 = \setb{a \in k}{\sigma(a) = a}$. Then
\[
Z(D) =
\begin{cases}
k_0((x^s)), & \ord \sigma = s, \\
k_0, & \text{$\sigma$ has infinite order}.
\end{cases}
\]
Moreover, $\dim_{Z(D)} D < \infty$ is equivalent to $\sigma$ being
of finite order.
\end{trditev}

\begin{proof}
Let
\[
f = \sum_{i=m}^\infty a_i x^i \in Z(D)
\]
with $a_j \ne 0$. Note that, as $af = fa$ for all $a \in k$, we
must have $a a_j = \sigma^j(a) a_j$, therefore $\sigma^j(a) = a$
for all $a \in k$. If $\sigma$ has infinite order, it follows that
$f = a_0$. But as $xf = fx$, we get $a_0 \in k_0$ and therefore
$Z(D) = k_0$.

Suppose now that $s = \ord \sigma$. With same notation and argument
as above, we see that $s \mid j$. As $xf = fx$, we also see that
all coefficients of $f$ are elements of $k_0$. The reverse
inclusion follows from a trivial calculation.
\end{proof}

\begin{definicija}
Let $\kvoc{K}{F}$ be a cyclic Galois extension with
$\Gal \br{\kvoc{K}{F}} = \skl{\sigma}$ and let $s = [K : F]$. Fix a
non-zero $a \in F \setminus \set{0}$. The
\emph{cyclic algebra}\index{cyclic algebra}
$\br{\kvoc{K}{F}, \sigma, a}$
is the vector space
\[
D = \bigoplus_{i=0}^{s-1} K x^i
\]
with multiplication on defined with $x^s = a$ and
$x \cdot b = \sigma(b) x$.
\end{definicija}

\begin{opomba}
We have $\dim_F \br{\kvoc{K}{F}, \sigma, a} = \ord(\sigma)^2$.
\end{opomba}

\begin{opomba}
We have
$\br{\kvoc{K}{F}, \sigma, a} \cong \kvoc{K[t, \sigma]}{(t^s - a)}$,
where $K$ is the skew polynomial ring.
\end{opomba}

\begin{izrek}
Let $D = \br{\kvoc{K}{F}, \sigma, a}$ be a cyclic algebra.

\begin{enumerate}[i)]
\item The algebra $D$ is a simple $F$-algebra.
\item We have
$C_D(K) = \setb{y \in D}{\forall b \in K \colon by = yb} = K$.
\item The field $K$ is a maximal subfield of $D$.
\item We have $Z(D) = F$.
\end{enumerate}
\end{izrek}

\begin{proof}
\phantom{a}
\nopagebreak
\begin{enumerate}[i)]
\item Let $I \edn D$ be a non-trivial ideal in $D$ and let
$z \in I$ be a non-zero element. Let
\[
z = \sum_{k=1}^r b_{i_k} x^{i_k},
\]
where $i_k \leq s-1$ and suppose that $r$ is minimal. Assume that
$r > 1$. Note that $\sigma^{i_1} \ne \sigma^{i_r}$, so suppose they
differ at $b \in K$. But then the element
\[
\sigma^{i_r}(b) z - zb \in I
\]
has smaller $r$, which is a contradiction if $r > 1$. It follows
that $z$ is invertible and therefore $I = D$.
\item Clearly $K \subseteq C_D(K)$. Choose an arbitrary element
$d \in C_D(K)$ and write
\[
z = \sum_{i=0}^{s-1} b_i x^i.
\]
As before, compute both elements $bd$ and $db$ and note that
$b_i = 0$ for all $i > 0$.
\item Suppose that $L \subseteq D$ is a subfield with
$K \subseteq L$. But as $L$ is a field, we have
$L \subseteq C_D(K)$, therefore $L = K$.
\item Note that $F \subseteq Z(D)$. Now let $b \in Z(D)$ be an
arbitrary element. It follows that $b \in C_D(K) = K$. Computing
elements $bx$ and $xb$, we note that $\sigma(b) = b$.\qedhere
\end{enumerate}
\end{proof}

\begin{definicija}
Let $\kvoc{K}{F}$ be a cyclic Galois extension with
$\Gal \br{\kvoc{K}{F}} = \skl{\sigma}$ and let $s = [K : F]$. The
\emph{norm}\index{norm} is the map $N_{\kvoc{K}{F}} \colon K \to F$
given by
\[
N_{\kvoc{K}{F}}(a) = \prod_{k=0}^{s-1} \sigma^k(a).
\]
\end{definicija}

\begin{izrek}
If $a \in N_{\kvoc{K}{F}}(K)$, then
$D = \br{\kvoc{K}{F}, \sigma, a} \cong M_s(F)$.
\end{izrek}

\begin{proof}
Let $d \in K$ be an element with $N_{\kvoc{K}{F}}(d) \cdot a = 1$
and let $y = dx$. A simple calculation shows that
$y^s = N_{\kvoc{K}{F}}(d) x^s = 1$. For all $b \in K$, we also have
$yb = \sigma(b) y$. We can therefore write
$D \cong \br{\kvoc{K}{F}, \sigma, 1}$.

Recall that
$\br{\kvoc{K}{F}, \sigma, 1} \cong \kvoc{B}{(t^s-1)}$, where
$B = K[t, \sigma]$. Of course it holds that
$(t^s-1) \subseteq B(t-1)$, which is a maximal submodule in $B$ as
$\kvoc{B}{B(t-1)} \cong K$. It follows that $M = \kvoc{B}{B(t-1)}$
is a simple module of
$\kvoc{B}{(t^s-1)} \cong  \br{\kvoc{K}{F}, \sigma, 1}$. The module
structure then yields a homomorphism
$\br{\kvoc{K}{F}, \sigma, 1} \to \End_F(M) \cong M_s(F)$. Since
$\br{\kvoc{K}{F}, \sigma, 1}$ is simple, the homomorphism is
injective. Furthermore, as both dimensions are equal to $s^2$, this
is an isomorphism.
\end{proof}

\begin{opomba}
The converse is also true. Furthermore, if $s$ is a prime, then the
algebra $\br{\kvoc{K}{F}, \sigma, a}$ is a division algebra if and
only if $a \not \in N_{\kvoc{K}{F}}(K)$.
\end{opomba}

\newpage

\subsection{Tensor product of algebras}

\begin{definicija}
The \emph{tensor product}\index{tensor product} of $k$-algebras $R$
and $S$ is the algebra $R \otimes_k S$ with multiplication
\[
(r \otimes s) \cdot (r' \otimes s') = rr' \otimes ss'.
\]
\end{definicija}

\begin{opomba}
If $(e_\alpha)_\alpha$ is a basis for $S$ over $k$, each
$x \in R \otimes S$ has a unique expansion
\[
x = \sum_{\alpha} r_\alpha \otimes e_\alpha.
\]
The algebra $R \otimes S$ is a free $R$-module with basis
$(1 \otimes e_\alpha)_\alpha$. In particular, homomorphisms
$i \colon r \mapsto r \otimes 1$ and
$j \colon s \mapsto 1 \otimes s$ are injective.
\end{opomba}

\begin{trditev}[Universal property]
\index{universal property}
Given any $k$-algebras $R$, $S$ and $T$ and $k$-algebra
homomorphism $f \colon R \to T$ and $g \colon S \to T$ such that
$f(R)$ and $g(S)$ commute element-wise and
$\eval{f}{k}{} = \eval{g}{k}{}$. Then there exists a unique
$k$-algebra homomorphism $h \colon R \otimes S \to T$ such that
$hi = h$ and $hj = g$.
\[
\begin{tikzcd}[column sep=large, row sep=large]
& R \arrow[dl, "i"'] \arrow[dr, "f"] \\
R \otimes S \arrow[rr, dashed, "h"] && T \\
& S \arrow[ul, "j"] \arrow[ur, "g"']
\end{tikzcd}
\]
\end{trditev}

\begin{proof}
Define the map $\varphi \colon R \times S \to T$, given by
\[
\varphi(r, s) =  f(r) \cdot g(s).
\]
This is a $k$-bilinear map. Indeed,
\[
\varphi(r_1 + r_2, s) =
f(r_1 + r_2) \cdot g(s) =
f(r_1) \cdot g(s) + f(r_2) \cdot g(s) =
\varphi(r_1, s) + \varphi(r_2, s)
\]
and similarly for $s = s_1 + s_2$, and
\[
\varphi(kr, s) =
f(kr) \cdot g(s) =
k f(r) \cdot g(s) =
k \cdot \varphi(r, s) =
f(r) \cdot g(ks) =
\varphi(r, ks).
\]
By the universal property of tensor products of modules, it follows
that $\varphi$ induces a unique homomorphism
$h \colon R \otimes S \to T$ of $k$-modules with $hi = f$ and
$hj = g$. Also note that $\varphi$ is the unique map such that
$hi = f$ and $hj = g$ hold. The $k$-module homomorphism $h$ is
therefore unique. It follows that if $h$ is a $k$-algebra
homomorphism, it is unique as well.

It remains to check that $h$ is indeed a $k$-algebra homomorphism.
It is clearly enough to check this on the basis, which we do with a
straightforward calculation:
\begin{align*}
h((r_1 \otimes s_1) \cdot (r_2 \otimes s_2)) &=
h(r_1 r_2 \otimes s_1 s_2)
\\
&=
\varphi(r_1 r_2, s_1 s_2)
\\
&=
f(r_1 r_2) \cdot g(s_1 s_2)
\\
&=
f(r_1) g(s_1) \cdot f(r_2) g(s_2)
\\
&=
h(r_1 \otimes s_1) \cdot h(r_2 \otimes s_2). \qedhere
\end{align*}
\end{proof}

\newpage

\subsection{Scalar extensions and semisimplicity}

\begin{definicija}
Let $R$ be a $k$-algebra and $\kvoc{K}{k}$ a field extension. Then
the algebra $R_K = K \otimes_k R$ is the
\emph{extension of scalars}\index{extension of scalars}.
\end{definicija}

\begin{izrek}[Primitive element]
\index{primitive element theorem}
If $\kvoc{K}{k}$ is a finite separable field extension, then
$K = k(c)$ for some $c \in K$.
\end{izrek}

\begin{proof}
Algebra 3, theorem 1.1.8.
\end{proof}

\begin{izrek}
Let $\kvoc{L}{k}$ be a finite field extension. Then
$L_K = K \otimes_k L$ is semisimple for all $\kvoc{K}{k}$ if and
only if $\kvoc{L}{k}$ is separable.
\end{izrek}

\datum{2023-11-7}

\begin{proof}
Suppose that $\kvoc{L}{k}$ is separable. By the primitive element
theorem we can write $L = k(\theta)$. Then $L$ has basis
$\setb{\theta^a}{0 \leq a < n}$ and $\theta$ has minimal polynomial
$f$ of degree $n$, that is $L = \kvoc{k[t]}{(f)}$. Note that $L_K$
has the same basis and $\theta$ satisfies the same polynomial
condition. It follows that $L_K = \kvoc{K[t]}{(f)}$. Since $f$ is
separable, it factors into distinct irreducible polynomials
$f_1, \dots, f_r$. By the Chinese remainder theorem, we have
\[
L_K = \kvoc{K[t]}{(f)} \cong \prod_{j=1}^r \kvoc{K[t]}{(f_j)}.
\]
It follows that $L_K$ is a product of fields and therefore
semisimple.

Now suppose that $\kvoc{L}{k}$ is not separable. Suppose that
$\theta \in L$ is not separable, that is its minimal polynomial $f$
is not separable. Equivalently, $f$ has repeated factors in
$K = k(f)$. Then $k(\theta) = \kvoc{K[t]}{(f)}$ has nilpotent
elements. But as $k(\theta) \subseteq L$, it follows that
$(a) \edn L_K$ is nil, therefore $\rad L_K \ne (0)$.
\end{proof}

\begin{posledica}
The tensor product of two field extensions over $k$ is semisimple,
provided one of the factors is finite and separable over $k$.
\end{posledica}

\newpage

\subsection{Tensor products, simplicity}

\begin{definicija}
Given an algebra $S$ over $k$, its \emph{center}\index{center} is
the set
\[
Z(S) = \setb{y \in S}{\forall x \in S \colon xy = yx}.
\]
We call $S$ \emph{central}\index{central algebra} over $k$ if
$k = Z(S)$. We call $S$ \emph{central simple} if $S$ is both simple
and central.
\end{definicija}

\begin{lema}
Let $S$ and $R$ be algebras over $k$ with $S$ being central simple.
If $J \edn R \otimes S$ is a non-trivial ideal, then
$J \cap R \ne (0)$.
\end{lema}

\begin{proof}
Let $x \in J$ be a minimal\footnote{With respect to the length of
basis expansions.} non-zero element. Write
\[
x = \sum_{i=1}^\ell r_i \otimes s_i.
\]
Then both $(r_i)_i$ and $(s_i)_i$ are $k$-linearly independent. In
particular, $s_1 \ne 0$ and thus $(s_1) = S$. We therefore have
\[
1 = \sum_{i=1}^m x_i s_1 y_i.
\]
Let
\[
x' =
\sum_{j=1}^m (1 \otimes x_j) x (1 \otimes y_j) =
\sum_{j=1}^m \br{\sum_{i=1}^\ell r_i \otimes x_j s_i y_j} =
\sum_{i=1}^\ell r_i \otimes \sum_{j=1}^m x_j s_i y_j =
\sum_{i=1}^\ell r_i \otimes s_i'.
\]
Note that $s_1' = 1$. Obviously, $x' \in J$. As $(r_i)_i$ are
$S$-linearly independent, we get $x' \ne 0$.

For any $s \in S$ observe the element
\[
y =
(1 \otimes s) x' - x' (1 \otimes s) =
\sum_{i=1}^\ell r_i \otimes s s_i' -
\sum_{i=1}^\ell r_i \otimes s_i' s =
\sum_{i=1}^\ell r_i \otimes \br{s s_i' - s_i' s} =
\sum_{i=2}^\ell r_i \otimes \br{s s_i' - s_i' s}.
\]
As $y \in J$, we have $y = 0$ by minimality of $x$. It follows that
$s_i' \in Z(S) = k$ for all $i$. Now rewrite
\[
x' =
\sum_{i=1}^\ell r_i \otimes s_i' =
\br{\sum_{i=1}^\ell r_i s_i'} \otimes 1 \in R. \qedhere
\]
\end{proof}

\begin{izrek}
Let $S$ and $R$ be algebras over $k$ with $S$ being central simple.

\begin{enumerate}[i)]
\item Every two-sided ideal of $R \otimes S$ has the form
$I \otimes S$ for some ideal $I \edn R$. In particular, if $R$ is
simple, then so is $R \otimes S$.
\item We have $Z(R \otimes S) = Z(R)$. In particular, if $R = K$ is
a field, then $S_K = K \otimes S$ is a central simple algebra over
$K$.
\end{enumerate}
\end{izrek}

\pagebreak[3]

\begin{proof}
\phantom{a}
\begin{enumerate}[i)]
\item Let $J \edn R \otimes S$ be an ideal and let $I = J \cap R$.
Consider the map
$\psi \colon R \otimes S \to \br{\kvoc{R}{I}} \otimes S$ with
$\psi(r \otimes s) = (r + I) \otimes s$. We claim that
$\ker \psi = I \otimes S$. Pick a basis $(x_i)_i$ for
$I \subseteq R$ and extend it to a basis $(x_i, y_j)_{i,j}$ of $R$.
Then $(y_j + I)_j$ is a basis for $\kvoc{R}{I}$. Now note that
\[
\sum x_i \oplus a_i + \sum y_j \otimes b_j \in \ker \psi \iff
\forall j \colon b_j = 0.
\]
By the isomorphism theorem, we have
\[
\kvoc{R \otimes S}{I \otimes S} \cong \kvoc{R}{I} \otimes S.
\]
Note that $I \otimes S \subseteq J$. If $I \otimes S \subset J$,
the image of the map
$\Phi \colon J \to \kvoc{R \otimes S}{I \otimes S}$  is nonzero. By
the above lemma, we have $\im(\Phi) \cap \kvoc{R}{I} \ne (0)$,
which is in contradiction with the choice of $I$.
\item Let $x = \sum r_i \otimes s_i \in Z(R \otimes S)$, where
$r_i$ are linearly independent. For any $s \in S$, we can write
\[
0 =
(1 \otimes s) x - x (1 \otimes s) =
\sum r_i \otimes (s s_i - s_i s),
\]
therefore $s_i \in Z(S) = k$ for all $i$. It follows that
\[
\sum r_i \otimes s_i = \sum r_i s_i \otimes 1 = r \otimes 1.
\]
Now, for $y \in R$, we must also have
\[
0 =
(y \otimes 1) x - x (y \otimes 1) =
(yr - ry) \otimes 1,
\]
therefore $r \in Z(R)$, as desired. \qedhere
\end{enumerate}
\end{proof}

\begin{posledica}
If $R$ and $S$ are central simple algebras, then so is
$R \otimes S$.
\end{posledica}

\begin{opomba}
If $R = R_1 \times R_2$, then
$R \otimes S = (R_1 \otimes S) \times (R_2 \otimes S)$.
\end{opomba}

\begin{trditev}
Suppose that $S$ is a simple $k$-algebra with center $C$. Then $S$
is a central simple algebra over $C$ and
$C \cong \End_{S \otimes S^{\mathsf{op}}}(S)$.
\end{trditev}

\begin{proof}
Note that, for $a \in C \setminus \set{0}$, we have $(a) = S$. It
follows that $1 = ab = ba$ for some $b \in S$. In particular, $C$
is a field.

Now let $\Phi \colon C \to \End_{S \otimes S^{\mathsf{op}}}(S)$ be
given by $\Phi(c) = (s \mapsto cs)$. First observe that this is in
fact a well defined map, as
\[
\Phi(c) \br{s_1 \otimes s_2^{\mathsf{op}} \cdot s} =
\Phi(c)(s_1 s s_2) =
c s_1 s s_2 =
s_1 \otimes s_2^{\mathsf{op}} \cdot cs =
s_1 \otimes s_2^{\mathsf{op}} \cdot \Phi(c)(s).
\]
It is clear that the map is a field homomorphism. Suppose that
$\Phi(c) = \Phi(d)$. Then, $\Phi(c)(1) = \Phi(d)(1)$, hence $c = d$
and $\Phi$ is injective. Now choose an arbitrary
$\varphi \in \End_{S \otimes S^{\mathsf{op}}}(S)$ and set
$x = \varphi(1)$. Observe that
\[
sx =
s \otimes 1^{\mathsf{op}} \cdot x =
\varphi \br{s \otimes 1^{\mathsf{op}} \cdot 1} =
\varphi(s) =
\varphi \br{1 \otimes s^{\mathsf{op}} \cdot 1} =
1 \otimes s^{\mathsf{op}} \cdot x =
xs,
\]
therefore $x \in C$. It is evident that $\varphi = \Phi(x)$, hence
$\Phi$ is surjective and therefore an isomorphism.
\end{proof}

\begin{opomba}
We have $R \otimes_k S = (R \otimes_k C) \otimes_C S$.
\end{opomba}

\begin{definicija}
Let $S$ be a finite-dimensional semisimple algebra over $k$. If
$C = Z(S)$, then
\[
C = \prod_{i=1}^m C_i
\]
for some fields $C_i$. The algebra $S$ is
\emph{separable}\index{separable algebra} if each $\kvoc{C_i}{k}$
is separable.
\end{definicija}

\begin{trditev}
If $S$ is separable, then $S_K$ is semisimple for all field
extensions $\kvoc{K}{k}$.
\end{trditev}

\datum{2023-11-9}

\begin{izrek}
If $D$ is a finite-dimensional division algebra over $k = Z(D)$,
then $\dim_k D$ is a perfect square.
\end{izrek}

\begin{proof}
Note that $\dim_k D = \dim_{\oline{k}} D_{\oline{k}}$, where
$\oline{k}$ is the algebraic closure of $k$ and
$D_{\oline{k}} = \oline{k} \otimes D$. This is a simple artinian
algebra, so by Wedderburn's theorem
$D_{\oline{k}} \cong M_n(E)$ for some finite-dimensional division
algebra $E$ over $\oline{k}$. It follows that $E = \oline{k}$ as
it is algebraically closed, therefore
$\dim_{\oline{k}} D_{\oline{k}} = n^2$.
\end{proof}

\begin{posledica}
If $A$ is a finite-dimensional simple algebra over $Z(A)$ then
$\dim_{Z(A)} A$ is a perfect square.
\end{posledica}

\begin{proof}
By Wedderburn we have $A \cong M_n(D)$ for some finite-dimensional
division algebra $D$ over $Z(A)$. But then
\[
[A : Z(A)] = [A : D] \cdot [D : Z(A)] = n^2 \cdot [D : Z(A)],
\]
which is a perfect square.
\end{proof}

\begin{trditev}
For all finite-dimensional central simple algebra $R$ over $k$ we
have $R \otimes R^{\mathsf{op}} \cong M_n(k)$, where
$n = \dim_k R$.
\end{trditev}

\begin{proof}
Denote
\[
A = \setb{L_r \in \End_k(R)}{r \in R}
\quad \text{and} \quad
B = \setb{T_r \in \End_k(R)}{r \in R},
\]
where $L_r(x) = rx$ and $T_r(x) = xr$. Note that $A \cong R$ and
$B \cong R^{\mathsf{op}}$. Elements of $A$ and $B$ obviously
commute.

By the universal property there exists a homomorphism
$\Omega \colon R \otimes R^{\mathsf{op}} \to \End_k(R)$ given by
$\Omega(r \otimes s) = L_r \circ T_s$. As both $R$ and
$R^{\mathsf{op}}$ are central simple algebras,
$R \otimes R^{\mathsf{op}}$ is simple and $\Omega$ is injective.
Since the dimensions are equal, $\Omega$ is an isomorphism.
\end{proof}

\newpage

\subsection{Skolem-Noether theorem}

\begin{lema}
Let $R$ be a finite-dimensional simple $k$-algebra. Suppose $M_1$
and $M_2$ are $R$-modules that are finite-dimensional over $k$. If
$\dim_k M_1 = \dim_k M_2$, then $M_1 \cong M_2$.
\end{lema}

\begin{proof}
Note that $R$ is simple and artinian. By Wedderburn's theorem there
exists a unique simple $R$-module $M$, thus
$M_j \cong M^{\alpha_j}$. As dimensions are equal,
$\alpha_1 = \alpha_2$.
\end{proof}

\begin{izrek}[Skolem-Noether]
\index{Skolem-Noether theorem}
Let $S$ be a finite-dimensional central simple algebra over $k$ and
$R$ be a simple $k$-algebra. If $f, g \colon R \to S$ are
homomorphisms, then there exists an inner automorphism
$\alpha \colon S \to S$ such that
$\alpha \cdot f = g \cdot \alpha$.
\end{izrek}

\begin{proof}
Note that $S$ is artinian, so by Wederburn's theorem
$S \cong \End_D(V)$ for some division algebra $D$ and
finite-dimensional vector space $V$ over $D$.

Homomorphisms $f$ and $g$ induce an $R$-module structure on $V$ by
$r \cdot v = f(r) v$ and $r \cdot g = g(r) v$. These two actions
obviously commute with the actions of $D$. The space $V$ becomes an
$R \otimes D$-module in two different ways. Now as $R \otimes D$ is
artinian and simple, the two modules are isomorphic by the previous
lemma. That is, there exists an abelian group isomorphism
$h \colon V \to V$ such that $h(f(r) \cdot v) = g(r) \cdot h(v)$
and $h(dv) = d h(v)$ holds for all $d \in D$, $r \in R$ and
$v \in V$. But then $h \in \End_D(V) = S$ and
$h \cdot f(r) = g(r) \cdot h$.
\end{proof}

\begin{opomba}
Equivalently, if $R_1$ and $R_2$ are simple subalgebras of $S$,
then for all homomorphisms $f \colon R_1 \to R_2$ there exists an
inner automorphism $\alpha \colon S \to S$ such that
$\eval{\alpha}{R_1}{} = f$. In particular, every automorphism of
$S$ is inner.
\end{opomba}

\begin{posledica}
If $\alpha \colon M_n(k) \to M_n(k)$ is an automorphism, then there
exists a matrix $P \in \GL_n(k)$ such that $\alpha(x) = P^{-1} x P$
for all $x \in M_n(k)$.
\end{posledica}

\newpage

\subsection{The (double) centralizer theorem}

\begin{definicija}
Let $R$ be an algebra and $S \subseteq R$. The
\emph{centralizer}\index{centralizer} of $S$ in $R$ is the
subalgebra
\[
C_R(S) = C(S) = \setb{r \in R}{\forall s \in S \colon rs = sr}.
\]
\end{definicija}

\begin{opomba}
If $S$ is a central simple algebra, then $C(C(S)) = C(k) = S$.
\end{opomba}

\begin{opomba}
For arbitrary $R$, we have $R \subseteq C(C(R))$.
\end{opomba}

\begin{definicija}
Let $S$ be a finite-dimensional simple algebra. By Wedderburn's
theorem, $S \cong M_n(D)$ for a division ring $D$. We write
$S \sim D$ and say that $S$ and $D$ are \emph{equivalent}.
\end{definicija}

\begin{opomba}
Note that the division ring $D$ is unique.
\end{opomba}

\begin{izrek}[Centralizer]
\index{centralizer!theorem}
Let $S$ be a finite-dimensional central simple algebra over $k$ and
$R$ be a simple subalgebra of $S$.

\begin{enumerate}[i)]
\item The subalgebra $C(R)$ is simple.
\item If $S \sim D_1$ and $R \otimes D_1^{\mathsf{op}} \sim D_2$,
then $C(R) \sim D_2^{\mathsf{op}}$.
\item We have $[S : k] = [R : k] \cdot [C(R) : k]$.
\item We have $C(C(R)) = R$.\footnote{Also known as the
\emph{double centralizer theorem}.}
\end{enumerate}
\end{izrek}

\begin{proof}
Applying Wedderburn's theorem to $S$, we get
$S \cong \End_D(V) \cong M_n(D^{\mathsf{op}})$, where $D$ is a
division algebra and $V$ a $n$-dimensional $D$-vector space. Note
that $V$ is an $R \otimes D$-module and
$C(R) = \End_{R \otimes D}(V)$.

\begin{enumerate}[i)]
\item The algebra $R \otimes D$ is simple, therefore it is
isomorphic to $\End_E(W)$ where $W$ is the unique simple
$R \otimes D$-module and $E = \End_{R \otimes D}(W)$. By
Wedderburn, we have $V \cong W^m$ as an $R \otimes D$-module. Then
\[
C(R) =
\End_{R \otimes D}(V) \cong
\End_{R \otimes D}(W^m) \cong
M_n(\End_{R \otimes D}(W)) =
M_n(E),
\]
which is simple.
\item We have $S \sim D^{\mathsf{op}} = D_1$ and
$R \otimes D_1^{\mathsf{op}} =
R \otimes D \sim E^{\mathsf{op}} = D_2$. It follows that
$C(R) \sim E = D_2^{\mathsf{op}}$.
\item We have $C(R) \cong M_m(E)$, therefore
$[C(R) : k] = m^2 \cdot [E : k]$. We also have
\[
[V : k] = m \cdot [W : k] = m \cdot [W : E] \cdot [E : k],
\]
hence
\[
[C(R) : k] =
m^2 \cdot \frac{[V : k]^2}{m^2 \cdot [W : E]^2 \cdot [E : k]} =
\frac{[V : k]^2}{[W : E]^2 \cdot [E : k]}.
\]
Note that $W$ is a vector space over $D$, therefore $W \cong E^d$
and $R \otimes D \cong M_d(E^{\mathsf{op}})$. It follows that
\[
[R : k] \cdot [D  : k] =
[R \otimes D : k] =
d^2 \cdot [E : k] =
[W : E]^2 \cdot [E : k].
\]
Returning to the centralizer, we get
\[
[C(R) : k] =
\frac{[V : k]^2}{[W : E]^2 \cdot [E : k]} =
\frac{[V : k]^2}{[R : k] \cdot [D : k]} =
\frac{[V : D]^2 \cdot [D : k]}{[R : k]} =
\frac{[S : k]}{[R : k]}.
\]
\item Note that
\[
[R : k] \cdot [C(R) : k] =
[S : k] =
[C(R) : k] \cdot [C(C(R)) : k]. \qedhere
\]
\end{enumerate}
\end{proof}

\begin{posledica}
If $R$ is a central simple algebra contained in a
finite-dimensional central simple algebra $S$, then
$S \cong R \otimes C(R)$.
\end{posledica}

\begin{proof}
Consider the homomorphism $\Psi \colon R \otimes C(R) \to S$ with
$\Psi(r \otimes r') = rr'$. Since $R \otimes C(R)$ is simple,
$\Psi$ is injective, but given
\[
[S : k] = [R : k] \cdot [C(R) : k] = [R \otimes C(R) : k],
\]
we conclude $\Psi$ is an isomorphism.
\end{proof}

\begin{definicija}
Let $D$ be a division algebra over $k$. A field $K \supseteq k$
such that $D_K = D \otimes D \cong M_n(K)$ is called a
\emph{splitting filed}\index{splitting field} for $D$.

A central simple algebra of the form $M_n(k)$ is a
\emph{split}\index{split central simple algebra} central simple
algebra. We call $n$ the \emph{degree}\index{degree} of $D$ and the
number $n^2$ the \emph{rank}\index{rank} of $D$.
\end{definicija}

\begin{opomba}
The algebraic closure of $k$ is a splitting field for any
finite-dimensional division algebra $D$ over $k$.
\end{opomba}

\begin{opomba}
If $K$ splits $D$ and $\kvoc{K'}{K}$ is an extension of $K$, then
$K'$ also splits $D$, as
\[
D_{K'} \cong K' \otimes_K D_K \cong
K' \otimes_K M_n(K) \cong M_n(K').
\]
\end{opomba}

\begin{trditev}\label{thm_divr:prop:fie_dim}
Let $D$ be a division algebra with center $k$ and $[D : k] = n^2$.
If $K$ is a maximal subfield of $D$, then $[K : k] = n$.
Furthermore, any such $K$ is a splitting field for $D$.
\end{trditev}

\begin{proof}
For $\alpha \in C(K) \setminus K$ the field $K(\alpha)$ is a proper
field extension of $K$. As $K$ is maximal, this is not possible,
so we have $C(K) \subseteq K$ and therefore $C(K) = K$. We can then
write
\[
n^2 = [D : k] = [K : k] \cdot [C(K) : k] = [K : k]^2.
\]
Note that $D$ is a simple $D \otimes K$-module, therefore
\[
\End_{D \otimes K}(D) \cong
C_D(K)^{\mathsf{op}} =
K^{\mathsf{op}} =
K.
\]
Since $D \otimes K$ is simple, it is isomorphic to matrices over
$\End_{D \otimes K}(D) \cong K$.
\end{proof}

\begin{izrek}[Wedderburn-Koethe]
\index{Wedderburn-Koethe thorem}
In any finite-dimensional division algebra there is a separable
maximal subfield.
\end{izrek}

\newpage

\subsection{Theorems about division rings}

\begin{izrek}[Little Wedderburn]
\index{little Wedderburn theorem}
Every finite division ring is a field.
\end{izrek}

\begin{proof}
Let $D$ be a finite division ring and denote $k = Z(D)$. Let $K$ be
a maximal subfield of $D$ and assume $K \ne D$. Since
$[D : k] = n^2$ for some $n \in \N$, we have $[K : k] = n$ by
\ref{thm_divr:prop:fie_dim}. If $\abs{k} = q = p^m$, then
$\abs{K} = q^n$. Any two subfields of $D$ of order $q^n$ are
isomorphic. By the Skolem-Noether theorem, they are conjugate.

Every element of $D$ is contained in a maximal subfield, so we have
\[
D = \bigcup_{x \in D^{-1}} x K x^{-1},
\]
so
\[
D^{-1} = \bigcup_{x \in D^{-1}} x K^{-1} x^{-1}.
\]
This is not possible, as a finite group cannot be a union of
conjugates of a proper subgroup.
\end{proof}

\begin{izrek}[Frobenius]
\index{Frobenius theorem}
Let $D$ be a division algebra with $\R \subseteq Z(D)$ and suppose
$[D : \R] < \infty$. Then $D$ is either $\R$, $\C$ or $\HH$.
\end{izrek}

\begin{proof}
Without loss of generality assume $[D : \R] > 1$. For any
$\alpha \in D \setminus \R$, the field $\R(\alpha)$ is a proper
algebraic field extension of $\R$, so $\R(\alpha) = \C$. Fix a copy
of $\C$ in $D$ and define
\[
D^+ = \setb{d \in D}{di = id}
\quad \text{and} \quad
D^- = \setb{d \in D}{di = -id}.
\]
Clearly, $D^+ \oplus D^- = D$. Note that $\kvoc{\C(d)}{\C}$ is an
algebraic field extension for every $d \in D^+$. As $\C$ is
algebraically closed, $D^+ = \C$.

If $D^- = (0)$, then $D = D^+ = \C$. Otherwise, take
$z \in D^- \setminus \set{0}$ and consider the map
$\mu \colon D^- \to D^+$ with $\mu(x) = xz$. This is a well defined
injective map. As it is linear over $\C$, we have
$\dim_\C D^- \leq \dim_\C D^+ = 1$ and $D^- \cong \C$.

Observe that $z$ is algebraic over $\R$, hence $z^2 \in \R + \R z$.
As $z^2 = \mu(z) \in \C$, it follows that $z^2 \in \R$. Of course
$z^2 \ne r^2$, as then $z = \pm r \in \R$. Hence $z^2 = -r^2$ for
some $r > 0$. Taking $j = \frac{z}{r}$, we find that
$j^2 = i^2 = -1$ and $ij = -ji$.
\end{proof}

\newpage

\subsection{Jacobson-Herstein theorem}

\datum{2023-11-23}

\begin{trditev}
Let $D$ be a division ring. If $y \in D$ commutes with all
commutators, then $y \in Z(D)$.
\end{trditev}

\begin{proof}
Assume that $y \not \in Z(D)$. Equivalently, there exists some
$x \in D$ such that $[x,y] \ne 0$. Note that
$[x,xy] = x \cdot [x,y]$, but as $y$ commutes with both
$[x,y]$ and $[x,xy]$, it also commutes with $x$.
\end{proof}

\begin{posledica}
\label{cen_alg:cor:com_field}
If all commutators in a division ring $D$ are central, then $D$ is
a field.
\end{posledica}

\begin{trditev}
Let $D$ be a division ring and $K \subseteq D$ a finite subring.
Then $K$ is a field.
\end{trditev}

\begin{proof}
For every $a \in K \setminus \set{0}$ consider $L_a \colon K \to K$
with $L_a(x) = ax$. This is an injective map, but as $K$ is finite,
it is bijective. It follows that $K$ is a division ring. By the
little Wedderburn theorem, $K$ is a field.
\end{proof}

\begin{lema}
If $F$ is a field and $G \leq F^{-1}$ is a finite subgroup, then
$G$ is cyclic.
\end{lema}

\begin{proof}
Note that, as $G$ is a finite abelian group, we have
\[
G \cong \bigoplus_{i=1}^r \Z_{m_i}.
\]
Denote $n = \lcm(m_i)$. As $x^n - 1$ has at most $n$ solutions in
$F$, we have $n \leq \abs{G} \leq n$ and therefore $G \cong \Z_n$.
\end{proof}

\begin{posledica}
Let $D$ be a division ring with $\chr D = p > 0$. Then any finite
subgroup $G \leq D^{-1}$ is also cyclic.
\end{posledica}

\begin{proof}
Define
\[
K =
\setb{\sum_{i=1}^r \alpha_i g_i}
{r \in \N_0 \land \alpha_i \in \F_p \land g_i \in G}.
\]
As $K$ is a finite subgroup of $D$, it is a field. By construction
$G \leq K^{-1}$, therefore it is cyclic.
\end{proof}

\begin{lema}
Let $D$ be a division ring with $\chr D = p > 0$. Suppose that
$a \in D$ is non-central and torsion. Then there exists some
$y \in D^{-1}$ such that $y a y^{-1} = a^i \ne a$ for some
$i \in \N$. Furthermore, we can choose $y$ to be a commutator.
\end{lema}

\begin{proof}
Let $K = \F_p[a]$. Since $a$ is torsion, $K$ is a finite field,
thus $\abs{K} = p^n$, in particular, $a^{p^n} = a$. Now define
$\delta_a \colon D \to D$ as $\delta_a(r) = [a, r]$. Since $a$ is
non-central, we have $\delta_a \ne 0$, but
$\eval{\delta_a}{K}{} = 0$. It follows that
$\delta_a \in \End_K(D)$.

Denote $\delta_a = L_a - R_a$ where $L_a(x) = ax$ and
$R_a(x) = xa$. We again have $L_a, R_a \in \End_K(D)$. Now compute
\[
\br{\delta_a}^{p^n} =
\br{L_a - R_a}^{p^n} =
L_a^{p^n} + (-R_a)^{p^n} =
L_a^{p^n} - R_a^{p^n} =
\delta_a.
\]
It follows that
\[
0 = \delta_a^{p^n} - \delta_a =
\prod_{b \in K} (\delta_a - b) =
\prod_{b \in K^{-1}} (\delta_a - b) \cdot \delta_a.
\]
Since $\delta_a \ne 0$, there exists a map $\delta_a - b$ that is
not injective. Such $b$ is an eigenvalue for $\delta_a$ -- that is,
there exists an element $x \in D^{-1}$ such that
$\delta_a(x) = bx$.

We can now write $ax - xa = bx$, which is equivalent to
\[
x a x^{-1} = b - a \in K \setminus{0}.
\]
As $a$ and $x a x^{-1}$ have the same order in the cyclic group
$K^{-1}$, they generate the same subgroup. In particular,
$x a x^{-1} = a^i \ne a$ for some $i \in \N$.

Instead of $x$, we can also use $y = [a, x]$. Indeed,
\[
ya =
(ax - xa) \cdot a =
a^i \cdot ax - a^i \cdot xa =
a^i y. \qedhere
\]
\end{proof}

\begin{izrek}[Jacobson-Herstein]
Let $D$ be a division ring. If for all $a, b \in D$ there exists
some $n > 1$ such that $(ab - ba)^n = ab - ba$, then $D$ is a
field.
\end{izrek}

\begin{proof}
Suppose that $D \ne Z(D)$. By lemma~\ref{cen_alg:cor:com_field}
there exists elements $b_1, b_2 \in D$ such that
$a = [b_1, b_2] \not \in Z(D)$. For each
$c \in Z(D) \setminus \set{0}$, we have
\[
ca = [c b_1, b_2].
\]
By assumption, there exists some $k \geq 1$ such that
$1 = a^k = (ca)^k$, therefore $c^k = 1$ as well. In particular,
$\chr(D) = p > 0$. By the previous lemma there exists a commutator
$y \in D$ such that $y a y^{-1} = a^i \ne a$. Now observe the
group $\skl{a} \cdot \skl{y}$. This is again a finite group since
$y$ normalizes $\skl{a}$. It follows that it is cyclic, thus
abelian, but this contradicts $y a y^{-1} \ne a$.
\end{proof}
