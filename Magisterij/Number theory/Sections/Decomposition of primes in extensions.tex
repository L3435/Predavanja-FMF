\section{Decomposition of primes in extensions}

\epigraph{The even case is a bit more odd.}
{-- gost.~izr.~prof.~dr.~rer.~nat.~Daniel Smertnig}

\subsection{Prime ideals in extensions}

\datum{2024-5-10}

\begin{lema}
Let $K \subseteq L$ be number fields,
$\mathfrak{p} \in \mathcal{P}(\mathcal{O}_K)$ and
$\mathfrak{P} \in \mathcal{P}(\mathcal{O}_L)$. Then
\[
\mathfrak{P} \mid \mathfrak{p} \mathcal{O}_L \iff
\mathfrak{p} \subseteq \mathfrak{P} \iff
\mathfrak{P} \cap \mathcal{O}_K = \mathfrak{p}.
\]
\end{lema}

\begin{proof}
Suppose first that $\mathfrak{P} \mid \mathfrak{p} \mathcal{O}_L$.
Then we have
$\mathfrak{p} \subseteq
\mathfrak{p} \mathcal{O}_L \subseteq
\mathfrak{P}$.

Now suppose that $\mathfrak{p} \subseteq \mathfrak{P}$. Then as
$\mathfrak{p} \subseteq \mathfrak{P} \cap \mathcal{O}_K$ is a
maximal ideal, it must be equal to this intersection.

Finally, suppose that the last condition holds. Then
\[
\mathfrak{p} \mathcal{O}_L =
\br{\mathfrak{P} \cap \mathcal{O}_K} \mathcal{O}_L \subseteq
\mathfrak{P} \mathcal{O}_L =
\mathfrak{P}. \qedhere
\]
\end{proof}

\begin{definicija}
If any of the above conditions hold, we say that
$\mathfrak{P}$ \emph{lies over}\index{lies over, under}
$\mathfrak{p}$. Similarly, $\mathfrak{p}$ \emph{lies under}
$\mathfrak{P}$.
\end{definicija}

\begin{lema}
Every $\mathfrak{P} \in \mathcal{P}(\mathcal{O}_L)$ lies over a
unique $\mathfrak{p} \in \mathcal{P}(\mathcal{O}_K)$.
\end{lema}

\begin{proof}
Uniqueness follows from the previous lemma, therefore we only need
to show that $\mathfrak{p} = \mathfrak{P} \cap \mathcal{O}_K$ is a
prime ideal. To see that it is non-empty, apply
lemma~\ref{dedek:lm:pr_id_intsc}. Now it is clear that it is a
prime ideal by definition.
\end{proof}

\begin{opomba}
By lemma~\ref{dedek:lm:pr_id_intsc}, the quotient
$\kvoc{\mathcal{O}_L}{\mathfrak{p} \mathcal{O}_L}$ is finite,
therefore each prime ideals lies under at most finitely many prime
ideals.
\end{opomba}

\begin{opomba}
The ring homomorphism
$\mathcal{O}_K \hookrightarrow
\mathcal{O}_L \to
\kvoc{\mathcal{O}_L}{\mathfrak{P}}$
induces a field embedding
$\kvoc{\mathcal{O}_K}{\mathfrak{p}} \to
\kvoc{\mathcal{O}_L}{\mathfrak{P}}$.
\end{opomba}

\begin{definicija}
Let $\mathfrak{P} \in \mathcal{P}(\mathcal{O}_L)$ and
$\mathfrak{p} = \mathfrak{P} \cap \mathcal{O}_K$.

\begin{enumerate}[i)]
\item The quotient $\kvoc{\mathcal{O}_L}{\mathfrak{P}}$ is the
\emph{residue field}\index{residue field} of $\mathfrak{P}$.
\item The number
$\left[\kvoc{\mathcal{O}_L}{\mathfrak{P}} :
\kvoc{\mathcal{O}_K}{\mathfrak{p}}\right]$
is the \emph{inertia degree}\index{inertia degree}, which we denote
by $f = f(\mathfrak{P} \mid \mathfrak{p})$.
\item The multiplicity
$\nu_{\mathfrak{P}}(\mathfrak{p} \mathcal{O}_L)$ of $\mathfrak{P}$
in $\mathfrak{p} \mathcal{O}_L$ is the
\emph{ramification index}\index{ramification index} of
$\mathfrak{P}$, which we denote by
$e = e(\mathfrak{P} \mid \mathfrak{p})$.
\end{enumerate}
\end{definicija}

\begin{izrek}
Let $n = [L : K]$, where $K \subseteq L$ are number fields. Let
$\mathfrak{p} \in \mathcal{P}(\mathcal{O}_K)$ and
$\mathfrak{P}_1, \dots, \mathfrak{P}_r \in
\mathcal{P}(\mathcal{O}_L)$ be the distinct prime ideals over
$\mathfrak{p}$. Denote by
$e_i = e(\mathfrak{P}_i \mid \mathfrak{p})$ and
$f_i = f(\mathfrak{P}_i \mid \mathfrak{p})$. Then
\[
\sum_{i=1}^r e_i f_i = n.
\]
\end{izrek}

\begin{proof}
Let $\kappa = \kvoc{\mathcal{O}_K}{\mathfrak{p}}$. Let
$\alpha_1, \dots, \alpha_m \in \mathcal{O}_L$ be such that
$\oline{\alpha}_1, \dots, \oline{\alpha}_m \in
\kvoc{\mathcal{O}_L}{\mathfrak{p} \mathcal{O}_L}$ is a
$\kappa$-basis.

Suppose that
\[
\sum_{i=1}^m c_i \alpha_i = 0
\]
where $c_i \in K$ are not all equal to $0$. By clearing
denominators, we can take $c_i \in \mathcal{O}_K$. Denote
\[
0 \ne \mathfrak{c} =
\skl{c_1, \dots, c_m}_{\mathcal{O}_K} \edn \mathcal{O}_K
\]
and let
$d \in \mathfrak{c}^{-1} \setminus \mathfrak{c}^{-1} \mathfrak{p}$.
Then
\[
\sum_{i=1}^m d c_i \alpha_i = 0
\]
and all $d c_i$ are elements of $\mathcal{O}_K$, but
$d c_i \not \in \mathfrak{p}$ for some index $i$. It follows that
\[
\sum_{i=1}^r \oline{d c_i} \oline{\alpha}_i = 0,
\]
which is a contradiction.

Now let $M = \skl{\alpha_1, \dots, \alpha_m}_{\mathcal{O}_K}$ and
write $N = \kvoc{\mathcal{O}_L}{M}$ as a $\mathcal{O}_K$-module.
Note that, by the choice of $\alpha_i$,
$\mathcal{O}_L = M + \mathfrak{p} \mathcal{O}_L$ holds. We can
check that $N = \mathfrak{p} N$.

As $\mathcal{O}_L$ is a finitely generated $\Z$-module, it is
finitely generated as a $\mathcal{O}_K$-module. Denote
$N = \skl{\beta_1, \dots, \beta_s}_{\mathcal{O}_K}$. Note that we
can write
\[
\beta_i = \sum_{j=1}^s c_{i,j} \beta_j
\]
for $c_{i,j} \in \mathfrak{p}$. Let
$C = \begin{bmatrix} c_{i,j} \end{bmatrix}_{i,j}$. It follows that
$(C-I) \beta = 0$. By construction we have
$d = \det(C-I) = (-1)^s \pmod{\mathfrak{p}}$, hence
\[
d \beta = \adj(C-I) \cdot (C-I) \beta = 0
\]
and so $d \beta_i = 0$ for all $i$. By definition, it follows
that $dN = 0$, therefore $d \mathcal{O}_L \subseteq M$. But then
\[
L =
dL =
d \skl{\mathcal{O}_L}_K \subseteq
\skl{M}_K =
\skl{\alpha_1, \dots, \alpha_m}_K,
\]
therefore $\setb{\alpha_i}{i \leq m}$ is a $K$-basis of $L$.

In particular,
$\dim_\kappa \kvoc{\mathcal{O}_L}{\mathfrak{p}\mathcal{O}_L} =
\dim_K L = n$. But then
\[
N(\mathfrak{p} \mathcal{O}_L) =
\abs{\kvoc{\mathcal{O}_L}{\mathfrak{p} \mathcal{O}_L}} =
\abs{\kappa}^n =
N(\mathfrak{p})^n,
\]
and
\[
N(\mathfrak{p} \mathcal{O}_L) =
\prod_{i=1}^r N(\mathfrak{P}_i)^{e_i} =
\prod_{i=1}^r N(\mathfrak{p})^{e_i f_i}. \qedhere
\]
\end{proof}

\begin{definicija}
The \emph{conductor}\index{conductor} of $\mathcal{O}_K[\alpha]$ in
$\mathcal{O}_L$ is the set
\[
\mathfrak{f} =
\setb{\beta \in \mathcal{O}_L}
{\beta \mathcal{O}_L \subseteq \mathcal{O}_K[\alpha]}.
\]
\end{definicija}

\begin{opomba}
The conductor is the largest common ideal of $\mathcal{O}_L$ and
$\mathcal{O}_K[\alpha]$.
\end{opomba}

\begin{lema}
If $\alpha \in \mathcal{O}_L$, then the minimal polynomial $g$ of
$\alpha$ over $K$ has coefficients in $\mathcal{O}_K$.
\end{lema}

\begin{proof}
Denote $n = [K(\alpha) : K]$ and let
$\Hom_K(K(\alpha), \C) = \setb{\sigma_i}{i \leq n}$. Then
$\sigma_i(\alpha)$ are algebraic conjugates of $\alpha$ and
therefore algebraic integers. It follows that the coefficients of
$g$ are algebraic integers as well by Vieta's formulae. As they are
contained in $K$ by definition, the coefficients are elements of
$\mathcal{O}_K$.
\end{proof}

\begin{izrek}[Dedekind-Kummer]
\index{Dedekind-Kummer theorem}
\label{decomp:thm:dedekum}
Let $\alpha \in \mathcal{O}_L$ be an element such that
$L = K(\alpha)$ and let $\mathfrak{f}$ be the conductor of
$\mathcal{O} = \mathcal{O}_K[\alpha]$ in $\mathcal{O}_L$. Let
$g \in \mathcal{O}_K[x]$ be the minimal polynomial of $\alpha$ over
$K$ and let $\mathfrak{p} \in \mathcal{P}(\mathcal{O}_K)$ be
coprime to $\mathfrak{f} \cap \mathcal{O}_K$. Suppose that
monic polynomials $g_1, \dots, g_r \in \mathcal{O}_K[x]$ and
integers $e_1, \dots, e_r \in \N$ are such that
\[
\oline{g} = \prod_{i=1}^r \oline{g}_i^{e_i}
\]
is the prime factorisation of $\oline{g}$ in
$\kvoc{\mathcal{O}_K}{\mathfrak{p}}[x]$. Finally, for
$i \leq r$, let
$\mathfrak{P}_i =
\mathfrak{p} \mathcal{O}_L + g_i(\alpha) \mathcal{O}_L$. Then
$\mathfrak{P}_i$ are the prime ideals of $\mathcal{O}_L$ lying over
$\mathfrak{p}$, $e_i = e(\mathfrak{P}_i \mid \mathfrak{p})$ and
$\deg g_i = f(\mathfrak{P}_i \mid \mathfrak{p})$.
\end{izrek}

\begin{proof}
Denote $\kappa = \kvoc{\mathcal{O}_K}{\mathfrak{p}}$. Consider the
homomorphism
$\varphi \colon \mathcal{O} \hookrightarrow
\mathcal{O}_L \to
\kvoc{\mathcal{O}_L}{\mathfrak{p} \mathcal{O}_L}$. By assumption,
we have
$\mathfrak{p} + \br{\mathfrak{f} \cap \mathcal{O}_K} =
\mathcal{O}_K$, therefore
$\mathfrak{p} \mathcal{O}_L + \mathfrak{f} = \mathcal{O}_L$.
Since $\mathfrak{f} \subseteq \mathcal{O}$, $\varphi$ is
surjective. Note that
$\ker \varphi = \mathcal{O} \cap \mathfrak{p} \mathcal{O}_L$. As
$\mathfrak{p} \mathcal{O} + \mathfrak{f} = \mathcal{O}$, we have
\[
\ker \varphi =
\mathfrak{p} \mathcal{O}_L \cap \mathcal{O} =
\br{\mathfrak{p} \mathcal{O} + \mathfrak{f}}
\br{\mathfrak{p} \mathcal{O}_L \cap \mathcal{O}} \subseteq
\mathfrak{p} \mathcal{O},
\]
therefore $\ker \varphi = \mathfrak{p} \mathcal{O}$ and so
\[
\kvoc{\mathcal{O}_L}{\mathfrak{p} \mathcal{O}_L} \cong
\kvoc{\mathcal{O}}{\mathfrak{p} \mathcal{O}}.
\]
But as
\[
\kvoc{\mathcal{O}}{\mathfrak{p} \mathcal{O}} \cong
\kvoc{\mathcal{O}_K[x]}{(\mathfrak{p}, g)} \cong
\kvoc{\kvoc{\mathcal{O}_K}{\mathfrak{p}}[x]}{(g)},
\]
we in fact have
\[
\kvoc{\mathcal{O}_L}{\mathfrak{p} \mathcal{O}_L} \cong
\kvoc{\kappa[x]}{\br{\oline{g}}}.
\]
By the Chinese remainder theorem, we can further write
\[
R = \kvoc{\kappa[x]}{\br{\oline{g}}} \cong
\prod_{i=1}^r \kvoc{\kappa[x]}{\br{\oline{g}_i^{e_i}}}
\]
The ideals of each component are precisely $\br{\oline{g}_i^j}$ for
some $j \leq e_i$, therefore $R$ has precisely $r$ maximal ideals.
Denote them by $\mathfrak{m}_i$. Note that
\[
\dim_\kappa \kvoc{R}{\mathfrak{m}_i} =
\dim_\kappa \kvoc{\kappa[x]}{\br{\oline{g}_i}} =
\deg(g_i)
\]
and
\[
\bigcap_{i=1}^r \mathfrak{m}_i^{e_i} = \set{0}.
\]
Let now $\oline{\mathfrak{P}}_i$ be the preimages of
$\mathfrak{m}_i$ under the above ring isomorphism -- it therefore
has the same properties as described above. Furthermore,
$\mathfrak{P}_i$ are precisely the preimages of
$\oline{\mathfrak{P}}_i$ under the homomorphism
$\mathcal{O}_L \to
\kvoc{\mathcal{O}_L}{\mathfrak{p} \mathcal{O}_L}$. They are the
maximal ideals containing $\mathfrak{p} \mathcal{O}_L$ and
\[
f(\mathfrak{P}_i \mid \mathfrak{p}) =
\left[\kvoc{\mathcal{O}_L}{\mathfrak{P}_i} : \kappa \right] =
\deg \br{\oline{g}_i} =
f_i.
\]
We can easily check that $\mathfrak{P}_i^{e_i}$ are the preimages
of $\oline{\mathfrak{P}}_i$, therefore
\[
\prod_{i=1}^r \mathfrak{P}_i^{e_i} =
\cap_{i=1}^r \mathfrak{P}_i^{e_i} \subseteq
\mathfrak{p} \mathcal{O}_L.
\]
We can therefore write
\[
\mathfrak{p} \mathcal{O}_L =
\prod_{i=1}^r \mathfrak{P}_i^{m_i},
\]
but as
\[
n = \sum_{i=1}^r m_i f_i \leq \sum_{i=1}^r e_i f_i = \deg g = n,
\]
we in fact have $m_i = e_i$.
\end{proof}

\begin{definicija}
Let $K \subseteq L$ be number fields with $n = [L : K]$. Let
$\mathfrak{p} \in \mathcal{P}(\mathcal{O}_K)$ be such that
\[
\mathfrak{p} \mathcal{O}_L =
\prod_{i=1}^r \mathfrak{P}^{e_i}
\]
is a factorisation in $\mathcal{O}_L$. Denote
$f_i = f(\mathfrak{P} \mid \mathfrak{p})$ and
$e_i = e(\mathfrak{P}_i \mid \mathfrak{p})$.

\begin{enumerate}[i)]
\item The ideal $\mathfrak{p}$ is
\emph{completely split}\index{completely split}\footnote{Also
\emph{totally split}.} if $r=n$.
\item The ideal $\mathfrak{p}$ is \emph{non-split}\index{non-split}
if $r=1$.
\item The ideal $\mathfrak{p}$ is \emph{inert}\index{inert} if
$\mathfrak{p} \mathcal{O}_L$ is a prime ideal (equivalently,
$r=e_1=1$ and $f_1=n$).
\item The ideal $\mathfrak{P}_i$ is
\emph{unramified}\index{ramified, unramified} over $K$ if $e_i=1$
and \emph{ramified} if $e_i>1$.
\item The ideal $\mathfrak{P}_i$ is \emph{totally ramified} over
$K$ if $e_i>1$ and $f_i=1$.
\item The ideal $\mathfrak{p}$ is \emph{unramified} in $L$ if all
$\mathfrak{P}_i$ are unramified and \emph{ramified} otherwise.
\end{enumerate}
\end{definicija}

\begin{opomba}
The elements of $\Gal \br{\kvoc{L}{K}}$ permute prime ideals lying
over $\mathfrak{p}$.
\end{opomba}

\begin{izrek}
Let $\mathfrak{p} \in \mathcal{P}(\mathcal{O}_K)$ and let
$p \in \P$ be such that $\mathfrak{p} \cap \Z = p\Z$. If
$\mathfrak{p}$ is ramified in $L$, then $p \mid \disc(L)$. In
particular, only finitely many primes of $\mathcal{O}_K$ ramify in
$L$.
\end{izrek}

\begin{proof}
Note that if $\mathfrak{p}$ is ramified in $L$, then $p\Z$ is also
ramified in $L$. Since the set
$\setb{\mathfrak{p} \in \mathcal{P}(\mathcal{O}_K)}
{\mathfrak{p} \cap \Z = p\Z}$
is finite for a fixed prime $p$, it suffices to consider $K = \Q$.

Let now $p \in \P$ be a prime number and
$\mathfrak{p} \in \mathcal{P}(\mathcal{O}_L)$ be a prime ideal with
$p\Z \subseteq \mathfrak{p}$. Set
$e = e(\mathfrak{p} \mid p\Z) > 1$. Write
$p \mathcal{O}_L = \mathfrak{pa}$ for an ideal
$\mathfrak{a} \edn \mathcal{O}_L$ and let
$\mathfrak{p}_1, \dots, \mathfrak{p}_r \in
\mathcal{P}(\mathcal{O}_L)$ be the prime ideals lying over
$p \mathcal{O}_L$. Since $e>1$, we have
\[
\mathfrak{a} \subseteq \bigcap_{i=1}^r \mathfrak{p}_i.
\]
Let $\alpha_1, \dots, \alpha_n$ be an integral basis of
$\mathcal{O}_L$ and choose an element
$\alpha \in \mathfrak{a} \setminus p \mathcal{O}_L$. We can write
\[
\alpha = \sum_{i=1}^n c_i \alpha_i,
\]
where $p \nmid c_i$ for some $i$. Without loss of generality let
$i = 1$. Consider now
\[
A =
\skl{\alpha, \alpha_2, \dots, \alpha_n}_\Z =
\skl{c_1 \alpha_1, \alpha_2, \dots, \alpha_n}_\Z \subseteq
\mathcal{O}_L.
\]
As
\[
\disc(\alpha, \alpha_2, \dots, \alpha_n) =
\abs{\mathcal{O}_L : A}^2 \cdot \disc(\mathcal{O}_L) =
c_1^2 \cdot \disc(\mathcal{O}_L),
\]
it suffices to show that
$p \mid d = \disc(\alpha, \alpha_2, \dots, \alpha_n)$.

Let $\kvoc{N}{L}$ be a finite extension such that $\kvoc{N}{\Q}$ is
Galois. Now we can extend the $n = [L : \Q]$ embeddings of $L$ into
$\C$ to automorphisms $\sigma_i \in \Gal \br{\kvoc{N}{\Q}}$. For
any $\mathfrak{P} \in \mathcal{P}(\mathcal{O}_N)$ lying over $p\Z$,
the ideal $\mathfrak{P} \cap \mathcal{O}_L$ is a prime ideal of
$\mathcal{O}_L$ lying over $p\Z$, hence
$\alpha \in \mathfrak{P} \cap \mathcal{O}_L$. In particular,
$\alpha$ is contained in every prime ideal of $\mathcal{O}_N$ lying
over $p\Z$.

Fix a prime ideal $\mathfrak{P} \in \mathcal{P}(\mathcal{O}_N)$
lying over $p\Z$. For any $\sigma \in \Gal \br{\kvoc{N}{\Q}}$, the
set $\sigma^{-1}(\mathfrak{P})$ is another such prime ideal,
meaning $\alpha \in \sigma(\mathfrak{P})$. By the definition of the
discriminant, we get $d \in \mathfrak{P} \cap \Z = p\Z$, hence
$p \mid d$.
\end{proof}

\newpage

\subsection{Quadratic fields, quadratic reciprocity and cyclotomic fields}

\datum{2024-5-17}

\begin{izrek}
Let $K = \Q \br{\sqrt{d}}$, where $d \ne 1$ is a square-free
integer.

\begin{enumerate}[i)]
\item Let $p$ be an odd prime. The prime factorisation of
$p \mathcal{O}_K$ is of the following form:

\begin{enumerate}
\item If $p \nmid d$ and $d \equiv b^2 \pmod{p}$, then
$p \mathcal{O}_K = \br{p, \sqrt{d}+b} \br{p, \sqrt{d}-b}$.
\item If $d$ is a non-square modulo $p$, then $p \mathcal{O}_K$ is
a prime ideal.
\item If $p \mid d$, then
$p \mathcal{O}_K = \br{p, \sqrt{d}}^2$.
\end{enumerate}

\item The prime factorisation of $2 \mathcal{O}_K$ is of the
following form:

\begin{enumerate}
\item If $2 \mid d$, then $2 \mathcal{O}_K = \br{2, \sqrt{d}}^2$.
\item If $d \equiv 3 \pmod{4}$, then
$2 \mathcal{O}_K = \br{2, 1+\sqrt{d}}^2$.
\item If $d \equiv 1 \pmod{8}$, then
$2 \mathcal{O}_K =
\br{2, \frac{1+\sqrt{d}}{2}} \br{2, \frac{1-\sqrt{d}}{2}}$.
\item If $d \equiv 5 \pmod{8}$, then $2 \mathcal{O}_K$ is a prime
ideal.
\end{enumerate}
\end{enumerate}
\end{izrek}

\begin{proof}
\phantom{i}
\begin{enumerate}[i)]
\item Note that $\mathfrak{f} \cap \Z \in \set{\Z, 2\Z}$, which is
coprime to $p\Z$.

\begin{enumerate}
\item We can factor
\[
x^2 - \oline{d} = \br{x - \oline{b}} \br{x + \oline{b}} \in
\F_p[x].
\]
As $p$ is odd, the factors are distinct. The conclusion follows
from theorem~\ref{decomp:thm:dedekum}.
\item Note that $x^2 - \oline{d}$ is irreducible in $\F_p[x]$ and
apply theorem~\ref{decomp:thm:dedekum}.
\item The polynomial $x^2$ factors trivially, so we can again apply
theorem~\ref{decomp:thm:dedekum}.
\end{enumerate}

\item \phantom{i}
\begin{enumerate}
\item Note that $\mathcal{O}_K = \Z \left[\sqrt{d}\right]$ and
$\mathfrak{f} = \mathcal{O}_K$. We can therefore again apply
theorem~\ref{decomp:thm:dedekum} with the trivial factorisation.
\item Same as the previous case.
\item Now $\mathcal{O}_K = \Z \left[\frac{1+\sqrt{d}}{2}\right]$.
The minimal polynomial is therefore given by
\[
g(x) = x^2 - x + \frac{1-d}{4}.
\]
Since $d \equiv 1 \pmod{8}$, we have
$\oline{g}(x) = x(x-1) \in \F_2[x]$. Now apply
theorem~\ref{decomp:thm:dedekum}.
\item Same as the previous case, but now
$\oline{g}(x) = x^2 + x + \oline{1}$ is irreducible. \qedhere
\end{enumerate}
\end{enumerate}
\end{proof}

\begin{definicija}
Let $p$ be a prime number. An integer $a$ is a
\emph{quadratic residue}\index{quadratic residue} modulo $p$ if
$a \equiv b^2 \pmod{p}$ for some integer $b$. We define the
\emph{Legendre symbol}\index{Legendre symbol} as
\[
\br{\frac{a}{p}} =
\begin{cases}
1, & \text{$p \nmid a$ and $a$ is a quadratic residue modulo $p$,}
\\
-1, & \text{$p \nmid a$ and $a$ is a quadratic non-residue modulo
$p$,}
\\
0, & p \mid a.
\end{cases}
\]
\end{definicija}

\begin{opomba}
For $p \ne 2$, then $\br{\F_p^*}^2$ is the unique subgroup of index
$2$ of $\F_p^*$. From this we deduce that
\[
\br{\frac{ab}{p}} = \br{\frac{a}{p}} \cdot \br{\frac{b}{p}}.
\]
In particular, $\br{\frac{\cdot}{p}} \colon \F_p^* \to S^0$ is a
group homomorphism with kernel $\br{\F_p^*}^2$.
\end{opomba}

\begin{lema}
Let $p$ be an odd prime and $a \in \Z$. Then
\[
\br{\frac{a}{p}} \equiv a^{\frac{p-1}{2}} \pmod{p}.
\]
\end{lema}

\begin{proof}
The group $\F_p^*$ is cyclic with order $p-1$, and the generator
maps to $-1$ under both homomorphisms.
\end{proof}

\begin{izrek}[Quadratic reciprocity law]
\index{Quadratic reciprocity law}
Let $p$ and $q$ be distinct odd primes. Then
\[
\br{\frac{p}{q}} \cdot \br{\frac{q}{p}} =
(-1)^{\frac{p-1}{2} \cdot \frac{q-1}{2}}.
\]
\end{izrek}

\begin{proof}
Let $\zeta \in \mu_p^*(\C)$. The following calculations are all
done in $\Z[\zeta]$.

Define the Gauss sum
\[
\tau =
\sum_{a \in \F_p^*} \br{\frac{a}{p}} \zeta^a =
\sum_{j=1}^{n-1} \br{\frac{j}{p}} \zeta^j.
\]
Let $c$ be a quadratic non-residue modulo $p$. Then
\[
-\sum_{a \in \F_p^*} \br{\frac{a}{p}} =
\br{\frac{c}{p}} \cdot \sum_{a \in \F_p^*} \br{\frac{a}{p}} =
\sum_{a \in \F_p^*} \br{\frac{ac}{p}} =
\sum_{a \in \F_p^*} \br{\frac{a}{p}},
\]
therefore
\[
\sum_{a \in \F_p^*} \br{\frac{a}{p}} = 0.
\]
Also, recall that
\[
\sum_{a \in \F_p^*} \zeta^{ab} = -1
\]
for all $b \in \F_p^*$, as $\zeta^b$ is also a primitive root of
unity. As $\br{\frac{a}{p}} = \br{\frac{a^{-1}}{p}}$, we find that
\begin{align*}
\tau^2 &=
\sum_{a, b \in \F_p^*}
\br{\frac{a}{p}} \br{\frac{b}{p}} \zeta^{a+b}
\\
&=
\br{\frac{-1}{p}} \cdot \sum_{a, b \in \F_p^*}
\br{\frac{ab^{-1}}{p}} \zeta^{a-b}
\\
&=
\br{\frac{-1}{p}} \cdot \sum_{b, c \in \F_p^*}
\br{\frac{c}{p}} \zeta^{cb - b}
\\
&=
\br{\frac{-1}{p}} \cdot
\br{\sum_{b \in \F_p^*} 1 +
\sum_{\substack{c \in \F_p^* \\ c \ne 1}}
\br{\frac{c}{p}} \cdot \sum_{b \in \F_p^*} \zeta^{b(c-1)}}
\intertext{As $c-1 \ne 0$ in the innermost sum, we can further
compute}
\tau^2 &=
\br{\frac{-1}{p}} \cdot \br{p-1 -
\sum_{\substack{c \in \F_p^* \\ c \ne 1}} \br{\frac{c}{p}}}
\\
&=
p \cdot \br{\frac{-1}{p}}.
\end{align*}
In $\Z_q[\zeta]$ we can now compute
\[
\tau^q =
\tau \cdot \br{(-1)^{\frac{p-1}{2}} \cdot p}^{\frac{q-1}{2}} =
\tau \cdot (-1)^{\frac{p-1}{2} \cdot \frac{q-1}{2}} \cdot
\br{\frac{p}{q}}
\]
and
\[
\tau^q =
\sum_{a \in \F_p^*} \br{\frac{a}{p}} \zeta^{aq} =
\br{\frac{q}{p}} \cdot \sum_{a \in \F_p^*}
\br{\frac{aq}{p}} \zeta^{aq} =
\br{\frac{q}{p}} \tau.
\]
Equating and multiplying by $\tau$, we get
\[
\br{\frac{-1}{p}} p \cdot
(-1)^{\frac{p-1}{2} \cdot \frac{q-1}{2}} \cdot
\br{\frac{p}{q}} =
\tau^2 \cdot (-1)^{\frac{p-1}{2} \cdot \frac{q-1}{2}} \cdot
\br{\frac{p}{q}} =
\tau^2 \cdot \br{\frac{q}{p}} =
\br{\frac{-1}{p}} \cdot p \cdot \br{\frac{q}{p}}.
\]
As $p$ is invertible in $\Z_q[\zeta]$, we get the sought equality.
\end{proof}

\begin{trditev}
If $p$ is an odd prime, then
\[
\br{\frac{2}{p}} = (-1)^{\frac{p^2-1}{8}}.
\]
\end{trditev}

\begin{proof}
Note that, in $\Z_p[i]$, we have
\[
1 + i \cdot (-1)^{\frac{p-1}{2}} =
1 + i^p =
(1+i)^p =
(1+i) \cdot 2^{\frac{p-1}{2}} \cdot i^{\frac{p-1}{2}} =
\br{\frac{2}{p}} \cdot (1+i) \cdot i^{\frac{p-1}{2}}.
\]
If $p \equiv 1 \pmod{4}$, we multiply the above equation by
$\frac{1-i}{2}$ to get
\[
1 = \br{\frac{2}{p}} \cdot (-1)^{\frac{p-1}{4}}
\]
in $\Z_p[i]$. Similarly, if $p \equiv 3 \pmod{4}$, multiply the
equation by $\frac{1+i}{2}$ instead to get
\[
1 =
\br{\frac{2}{p}} \cdot i \cdot i^{\frac{p-1}{2}} =
\br{\frac{2}{p}} \cdot (-1)^{\frac{p+1}{4}}. \qedhere
\]
\end{proof}

\begin{trditev}
Let $p$ be a prime number and $k, m \in \N$ be integers such that
$p \nmid m$. Let
\[
f =
\ord_{\Z_m^*} \br{\oline{p}} =
\min \setb{\ell \in \N}{p^\ell \equiv 1 \pmod{m}}.
\]

\begin{enumerate}[i)]
\item If $\zeta \in \F_{p^k}$ is a primitive $m$-th root of unity
and $g \in \F_p[x]$ is the minimal polynomial of $\zeta$, then
\[
\F_p(\zeta) \cong \kvoc{\F_p[x]}{(g)} \cong \F_{p^f}.
\]
In particular, $\deg g = f$.
\item If $\Phi_m \in \Z[x]$ is the $m$-th cyclotomic polynomial,
then
\[
\oline{\Phi} = \prod_{i=1}^r \oline{g}_i \in \F_p[x]
\]
for pairwise distinct monic irreducible polynomials
$\oline{g}_i \in \F_p[x]$ with $\deg \br{\oline{g}_i} = f$ for all
$i$.
\end{enumerate}
\end{trditev}

\begin{proof}
\phantom{i}
\begin{enumerate}[i)]
\item Note that
\[
\F_p(\zeta) \cong \kvoc{\F_p[x]}{(g)}
\]
is a finite field, therefore $\F_p(\zeta) \cong \F_{p^k}$ for some
$k \geq 1$. Note that $\F_{p^k}^*$ contains a primitive $m$-th
root of unity if and only if $p^k \equiv 1 \pmod{m}$. By choice
of $f$, we have $f \mid k$ and therefore
\[
\F_{p^f} = \setb{x \in \F_{p^k}}{x^{p^f} = x}.
\]
By definition of $f$, it contains all $m$-th roots of unity of
$\F_{p^k}$, hence $\F_p(\zeta) \subseteq \F_{p^f}$. It follows that
$k=f$.
\item Recall that
\[
x^m-1 = \prod_{\ell \mid m} \Phi_\ell.
\]
In particular, every $m$-th root of unity of $\F_{p^f}$ is a root
of some cyclotomic polynomial $\oline{\Phi}_\ell \in \F_p[x]$ with
$\ell \mid m$. As $\F_{p^f}$ contains precisely $\varphi(\ell)$
primitive $\ell$-th roots of unity, they are exactly the roots of
$\oline{\Phi}_\ell$. In particular, $\oline{\Phi}_m$ has no
repeated roots in $\F_{p^f}$. We can therefore factor
\[
\oline{\Phi}_m = \prod_{i=1}^r \oline{g}_i,
\]
where each $\oline{g}_i$ is a minimal polynomial of some primitive
$m$-th root of unity. In particular, $\deg \br{\oline{g}_i} = f$.
\qedhere
\end{enumerate}
\end{proof}

\datum{2024-5-24}

\begin{izrek}
Let $n$ be a natural number and $\zeta \in \mu_n^*(\C)$. Denote
$K = \Q(\zeta)$ and let $p \in \P$. Let $v = \nu_p(n)$ and denote
$m = \frac{n}{p^v}$ and
\[
f =
\ord_{\Z_m^*} \br{\oline{p}} =
\min \setb{\ell \in \N}{p^\ell \equiv 1 \pmod{m}}.
\]
Then
$p \mathcal{O}_K =
(\mathfrak{p}_1 \cdots \mathfrak{p}_r)^{\varphi(p^v)}$ with
distinct $\mathfrak{p}_i$ and $f(\mathfrak{p}_i \mid p) = f$.
\end{izrek}

\begin{proof}
As the conductor is trivial, we can apply the Dedekind-Kummer
theorem to all $p \in \P$. The minimal polynomial of $\zeta$ is of
course $\Phi_n$. Recall that
\[
\mu_n^*(\C) =
\setb{\xi \cdot \omega}
{\xi \in \mu_{p^v}^*(\C) \land \omega \in \mu_m^*(\C)}.
\]
For such $\xi$ we have
\[
(\xi - 1)^{p^v} \equiv \xi^{p^v} - 1 \equiv 0 \pmod{\mathfrak{p}}
\]
for all $\mathfrak{p} \mid p \mathcal{O}_K$, therefore
$\xi \equiv 1 \pmod{\mathfrak{p}}$. We can therefore factor
\[
\Phi_n =
\prod_{\substack{\xi \in \mu_{p^v}^*(\C) \\ \omega \in \mu_m^*(\C)}}
\br{x - \xi \omega} \equiv
\prod_{\omega \in \mu_m^*(\C)} (x - \omega)^{\varphi(p^v)} \equiv
\Phi_m^{\varphi(p^v)} \pmod{\mathfrak{p}}.
\]
But then $\Phi_n = \Phi_m^{\varphi(p^v)}$ in $\F_p[x]$, hence
\[
\oline{\Phi}_n = \prod_{i=1}^r \oline{g}_i^{\varphi(p^v)}
\]
by the previous proposition. Furthermore, $\oline{g}_i$ are monic,
irreducible and distinct with $\deg \br{\oline{g}_i} = f$.
\end{proof}

\begin{posledica}
A prime $p \ne 2$ is completely split if and only if
$p \equiv 1 \pmod{n}$.
\end{posledica}

\begin{posledica}
A prime number $p \in \P$ is ramified if and only if $p \mid n$,
except if $p = 2 = \gcd(n, 4)$.
\end{posledica}
