\section{Dedekind domains}

\epigraph{Sorry if you're \TeX{}ing this now.}
{-- gost.~izr.~prof.~dr.~rer.~nat.~Daniel Smertnig}

\subsection{Prime ideal factorisation}

\datum{2024-4-5}

\begin{definicija}
Let $D$ and $D'$ be domains with $D \subseteq D'$ and let $K$ be
the quotient field of $D$. An element $\alpha' \in D'$ is
\emph{integral}\index{integral element} over $D$ if there exists a
monic polynomial $f \in D[x]$ such that $f(\alpha) = 0$. The domain
$D$ is \emph{integrally closed}\index{integrally closed domain} if
\[
D = \setb{\alpha \in K}{\text{$\alpha$ is integral over $D$}}.
\]
\end{definicija}

\begin{lema}
\label{dedek:lm:pr_id_intsc}
Let $K$ be a number field and $\mathfrak{a} \edn \mathcal{O}_K$ a
non-zero ideal. Then
$\abs{\kvoc{\mathcal{O}_K}{\mathfrak{a}}} < \infty$. Furthermore,
if $\mathfrak{p} \edn \mathcal{O}_K$ is a non-zero prime ideal,
then $\mathfrak{p} \cap \Z = p \Z$ for some prime number $p$. The
ring $\kvoc{\mathcal{O}_K}{\mathfrak{p}}$ is a finite field
extension of $\Z_p$.
\end{lema}

\begin{proof}
Let $\alpha \in \mathfrak{a}$ be a non-zero element. As
$\alpha$ is an algebraic integer, we can write
\[
\alpha^m + \sum_{j=0}^{m-1} a_j \alpha^j = 0
\]
for integers $a_j$, where we assume $a_0 \ne 0$. But then
we must have $a_0 \in \mathfrak{a}$, therefore
$a_0 \mathcal{O}_K \subseteq \mathfrak{a}$. Hence
$\kvoc{\mathcal{O}_K}{\mathfrak{a}}$ is a quotient of
$\kvoc{\mathcal{O}_K}{a_0 \mathcal{O}_K}$. By the structure theorem
the quotient is finite, as the above free abelian groups both have
the same rank.

Now let $\mathfrak{p}$ be a prime ideal. Note that
$\kvoc{\mathcal{O}_K}{\mathfrak{p}}$ is a finite domain and
therefore a field. Note that
$a_0 \in \mathfrak{p} \cap \Z \setminus \set{0}$, therefore the
intersection $\mathfrak{p} \cap \Z$ is non-trivial. In particular,
it is a prime ideal of $\Z$ and hence
$\mathfrak{p} \cap \Z = p \Z$ for some prime number $p$. As the
kernel of the map $\Z \to \kvoc{\mathcal{O}_K}{\mathfrak{p}}$ is
$p \Z$, it induces an injective map
$\Z_p \to \kvoc{\mathcal{O}_K}{\mathfrak{p}}$.
\end{proof}

\begin{izrek}
Let $K$ be a number field. Then $\mathcal{O}_K$ is a noetherian
integrally closed domain and every non-zero prime ideal of
$\mathcal{O}_K$ is maximal.
\end{izrek}

\begin{proof}
We already know that $\mathcal{O}_K$ is noetherian. Let
$\alpha \in K$ be integral over $\mathcal{O}_K$. It follows that
$\mathcal{O}_K[\alpha]$ is a finitely-generated
$\mathcal{O}_K$-module and hence a finitely-generated $\Z$-module.
This implies that $\alpha$ is an algebraic integer.

If $\mathfrak{p}$ is a non-zero prime ideal, then
$\kvoc{\mathcal{O}_K}{\mathfrak{p}}$ is a field and hence
$\mathfrak{p}$ is maximal.
\end{proof}

\begin{definicija}
A \emph{Dedekind domain}\index{Dedekind domain} is a noetherian
integrally closed domain in which every non-zero prime ideal is
maximal.
\end{definicija}

\begin{definicija}
Let $D$ be a domain and $K$ its quotient field.

\begin{enumerate}[i)]
\item A \emph{fractional ideal}\index{fractional ideal} of $D$ is a
$D$-submodule of $K$ that is of the form $c^{-1} I$ for some
$c \in D \setminus \set{0}$ and $0 \ne I \edn D$.
\item A fractional ideal $I$ is \emph{invertible} if there exists
a fractional ideal $J$ such that $IJ = D$.
\end{enumerate}
\end{definicija}

\begin{opomba}
For a fractional ideal $I$, we write
\[
I^{-1} = \setb{x \in K}{xI \subseteq D}.
\]
If $I$ is invertible, then $I^{-1}$ is its unique inverse.
\end{opomba}

\begin{lema}
Let $D$ be a Dedekind domain that is not a field. For every
non-zero ideal $I \edn D$ there exists an integer $r \geq 0$ and
non-zero prime ideals $P_i \edn D$ such that
\[
\prod_{i=1}^r P_i \subseteq I.
\]
\end{lema}

\begin{proof}
Let $\Omega$ be the set of ideals $I$ for which the above does not
hold. Suppose that $\Omega \ne \emptyset$. As $D$ is noetherian,
there exists a maximal ideal $I \in \Omega$, which clearly cannot
be a prime ideal. Also note that $I \ne D$. It follows that there
exist $a, b \in D \setminus I$ such that $ab \in I$. But then both
$aD + I$ and $bD + I$ are not in $\Omega$ by maximality of $I$. Now
we can just take the product of their respective prime ideals,
which gives a contradiction.
\end{proof}

\begin{lema}
Let $D$ be a Dedekind domain that is not a field and $P \edn D$
be a non-zero prime ideal. For every non-zero ideal $I \edn D$ we
have $I \subset I P^{-1}$.
\end{lema}

\begin{proof}
Consider first the case $I = D$. Let $a \in P \setminus \set{0}$
and write
\[
\prod_{i=1}^r P_i \subseteq aD \subseteq P,
\]
where $r$ is minimal. As $P$ is a prime ideal, we must have
$P_i \subseteq P$ for some $i$ -- without loss of generality let
this be $P_1$. As prime ideals are maximal, we must hence have
$P_1 = P$. By minimality of $r$, we must have
\[
\prod_{i=2}^r P_i \not \subseteq aD,
\]
hence it has an element $b$ such that $b \not \in aD$ but
$bP \subseteq aD$. But then $\frac{b}{a} \in P^{-1} \setminus D$,
as required.

Now consider the general case. Note that, as $D$ is noetherian, the
ideal $I$ is finitely generated -- write
$I = \skl{a_i \mid i \leq m}_D$. Suppose that $I = I P^{-1}$ and
let $x \in P^{-1}$.

Choose $c_{i,j}$ such that
\[
xa =
x \cdot
\begin{bmatrix}
a_1 \\ a_2 \\ \vdots \\ a_m
\end{bmatrix}
=
\underbrace{\begin{bmatrix}
c_{1,1} & c_{1,2} &  \dots & c_{1,m} \\
c_{2,1} & c_{2,2} &  \dots & c_{2,m} \\
\vdots  & \vdots  & \ddots & \vdots  \\
c_{m,1} & c_{m,2} &  \dots & c_{m,m} \\
\end{bmatrix}}_C
\cdot
\begin{bmatrix}
a_1 \\ a_2 \\ \vdots \\ a_m
\end{bmatrix}.
\]
But then $xa = Ca$, therefore $\det(xI_m - C) = 0$. Expanding the
determinant, we get a monic polynomial with $x$ as a root,
therefore $x$ is integral over $D$ and hence $x \in D$. It follows
that $P^{-1} \subseteq D$, which we have already shown cannot
happen.
\end{proof}

\begin{izrek}
If $D$ is a Dedekind domain, then every non-zero ideal is a product
of prime ideals. Such a representation is unique up to the order
of factors.
\end{izrek}

\begin{proof}
If $D$ is a field, its only ideals are $0$ and $D$ itself, which
clearly factors.

Let $\Omega$ be the set of all non-zero ideals of $D$ that cannot
be factored as a product of prime ideals. As $D$ is noetherian,
there exists a maximal element $I \in \Omega$. Note that $I \ne D$.
Let $P$ be a maximal ideal of $D$ containing $I$. Then
$I \subset I P^{-1}$ and $P \subset P P^{-1} subseteq D$, but as
$P$ is maximal, we actually have $P P^{-1} = D$. By maximality of
$P$, we can factor
\[
I P^{-1} = \prod_{r=2}^m P_i,
\]
but then
\[
I = I P P^{-1} = P \cdot \prod_{r=2}^m P_i.
\]
Next, we show that this factorisation is unique. Suppose otherwise
that
\[
\prod_{i=1}^r P_i = \prod_{i=1}^s Q_i
\]
for prime ideals $P_i$ and $Q_i$. But this implies that
$Q_i \subseteq P_1$ for some $i$, as $P_1$ is prime. Without loss
of generality let $Q_1 \subseteq P_1$. As $Q_1$ is maximal, we must
have $Q_1 = P_1$. Multiplying by $P_1^{-1}$ and using the fact that
$P_1 P_1^{-1} = D$, we get uniqueness by induction.
\end{proof}

\begin{posledica}
If $D$ is a Dedekind domain, then every fractional ideal is
invertible.
\end{posledica}

\begin{proof}
Let $I$ be a fractional ideal. Let $c \in D^*$ be an element such
that $cI \edn D$. We can therefore factor $cI$ as a product of
prime ideals $P_i$. But all of there are invertible and so
\[
I \cdot c \prod_{i=1}^r P_i^{-1} = D. \qedhere
\]
\end{proof}

\newpage

\subsection{Fractional ideals and the class group}

\begin{definicija}
Let $D$ be a Dedekind domain and $P \edn D$ be a non-zero prime
ideal. The
\emph{$P$-adic valuation}\index{P-adic valuation@$P$-adic valuation}
$\nu_P(I)$ of a non-zero ideal $I \edn D$ is the exponent of $P$ in
the factorization of $I$.
\end{definicija}

\begin{opomba}
We denote the prime ideals of $D$ by $\mathcal{P}(D)$. The monoid
of of all non-zero ideals is denoted by $\mathcal{I}(D)^\bullet$,
while the monoid of fractional ideals is denoted by
$\mathcal{F}(D)$.
\end{opomba}

\begin{izrek}
There is a group isomorphism
$\mathcal{F}(D) \to \Z^{\mathcal{P}(D)}$,
$I \mapsto \br{\nu_P(I)}_{P \in \mathcal{P}(D)}$, that restricts to
a monoid isomorphism
$\mathcal{I}(D)^\bullet \to \N_0^{\mathcal{P}(D)}$.
\end{izrek}

\begin{definicija}
Let $D$ be a Dedekind domain. Let $\mathcal{H}(D)$ be all the
non-zero principal ideals of $D$. The abelian group
$\mathcal{C}(D) = \kvoc{\mathcal{F}(D)}{\mathcal{H}(D)}$ is the
\emph{class group}\index{class group} of $D$.
\end{definicija}

\begin{opomba}
The sequence
\[
\begin{tikzcd}
1 \arrow[r] &
D^* \arrow[r] &
K^* \arrow[r] &
\mathcal{F}(D) \arrow[r] &
\mathcal{C}(D) \arrow[r] &
1
\end{tikzcd}
\]
is exact.
\end{opomba}

\datum{2024-4-12}

\begin{izrek}
Let $D$ be a Dedekind domain. The following statements are
equivalent.

\begin{enumerate}[i)]
\item The domain $D$ is a unique factorisation domain.
\item The class group $\mathcal{C}(D)$ is trivial.
\item The domain $D$ is a principal ideal domain.
\end{enumerate}
\end{izrek}

\begin{proof}
Note that we only need to prove that the class group of a unique
factorisation domain is trivial. It therefore suffices to show that
every prime ideal $P \subseteq D$ is principal. Let
$a \in P \setminus \set{0}$ and write
\[
a = \prod_{i=1}^r p_i
\]
for prime elements $p_i$ of $D$. It follows that $p_i \in P$ for
some $i$. But then $p_i D \subseteq P$ is also a prime ideal, which
must be equal to $P$ by maximality.
\end{proof}

\begin{trditev}
Every principal ideal domain is a Dedekind domain.
\end{trditev}

\begin{proof}
As every ideal of $D$ is generated by one element, it is a
noetherian ring.

Let $K$ be the quotient field of $D$. Suppose that
$f \br{\frac{a}{b}} = 0$ for a monic polynomial $f$ and
$\frac{a}{b} \in K$. Since $D$ is a unique factorisation domain, we
can further assume that $a$ and $b$ have no non-trivial common
factor. As
\[
0 = b^m f \br{\frac{a}{b}},
\]
we can deduce that $b \mid a^m$ in $D$. This immediately shows that
$b$ is a unit and therefore $\frac{a}{b} \in D$, which means that
$D$ is integrally closed.

Now let $P \edn D$ be a non-zero prime ideal, contained in a
maximal ideal $M$. It is clear that $P = (p)$ and $M = (q)$ for
some prime elements $p, q \in D$. But this implies $q \mid p$ and
hence $(p) = (q)$, therefore $P = M$ is maximal.
\end{proof}

\newpage

\subsection{Chinese remainder theorem}

\begin{izrek}[Chinese remainder theorem]
\index{Chinese remainder theorem}
Let $R$ be a ring and let $I_1, \dots, I_m \edn R$ be ideals that
are pairwise comaximal.\footnote{That is, $I_i + I_j = R$ for all
$i \ne j$.} Then the map
\[
\kvoc{R}{\bigcap_{i=1}^m I_i} \to \prod_{i=1}^m \kvoc{R}{I_i},
\]
given by
\[
r + \bigcap_{i=1}^m I_i \mapsto \br{r+I_1, \dots, r+I_m},
\]
is an isomorphism of $R$-algebras.
\end{izrek}

\begin{proof}
It suffices to show that the above homomorphism is surjective. Let
$a_1, \dots, a_m \in R$. For all $i, j$ there exist elements
$x_{i,j} \in I_i$ and $y_{i,j} \in I_j$ such that
$x_{i,j} + y_{i,j} = 1$. Setting
\[
z_i = \prod_{j \ne i} y_{i,j},
\]
it is clear that $z_i \equiv \delta_{i,j} \pmod{I_j}$. But then
\[
\varphi \br{\sum_{i=1}^m z_i a_i} =
\br{a_1 + I_1, \dots, a_m + I_m}. \qedhere
\]
\end{proof}

\begin{posledica}
Let $D$ be a Dedekind domain, $P_1, \dots, P_m \edn D$ be
pairwise distinct prime ideals, and $e_1, \dots, e_m \in \N_0$. If
$a_1, \dots, a_m \in D$, then there exists an element $a \in D$
such that for all $i \leq m$ we have
\[
a \equiv a_i \pmod{P_i^{e_i}}.
\]
\end{posledica}

\obvs

\begin{posledica}
Let $D$ be a Dedekind domain, $P_1, \dots, P_m \edn D$ be
pairwise distinct prime ideals, and $e_1, \dots, e_m \in \Z$. Then
there exists an element $x \in K^*$ with $v_{P_i}(x) = e_i$ for all
$i \leq m$ and $v_P(x) \geq 0$ for all non-zero primes $P \ne P_i$.
\end{posledica}

\begin{proof}
The case where $e_i \geq 0$ for all $i$ follows from the previous
corollary. Construct an element $b \in D$ such that
$\nu_{P_i}(b) = \max(0, -e_i)$ for all $i$. Then, construct an
element $a \in D$ such that $\nu_{P_i}(a) = \max(0, e_i)$ for all
$i$ and $\nu_Q(a) \geq \nu_Q(b)$ for all other prime ideals $Q$.
Then $\frac{a}{b}$ is one such element.
\end{proof}

\begin{izrek}
Let $D$ be a Dedekind domain and let $I \edn D$ be a non-zero
ideal. If $a \in I$ is a non-zero element, then there exists some
$b \in I$ such that $I = (a, b)$.
\end{izrek}

\begin{proof}
Consider all prime ideals $P_i$ with $\nu_{P_i}(aD) > 0$. Note that
$\nu_{P_i}(aD) \geq \nu_{P_i}(I)$. Choose an element
$b$ such that $\nu_{P_i}(b) = \nu_{P_i}(I)$ for all $I$. It is
clear that
\[
\nu_P(I) = \min(\nu_P(aD), \nu_P(bD)) = \nu_P(aD + bD)
\]
holds, hence $I = (a,b)$.
\end{proof}
