\section{Minkowski theory}

\epigraph{We'll skip this so we don't have to do any actual
integrals, so if you're bored ...}
{-- gost.~izr.~prof.~dr.~rer.~nat.~Daniel Smertnig}

\subsection{Lattices}

\begin{definicija}
Let $V$ be an $\R$-vector space of dimension $n$. A
\emph{lattice}\index{lattice} is a subgroup
\[
\Gamma = \sum_{i=1}^m \Z v_i \subseteq V,
\]
where $v_i$ are $\R$-linearly independent vectors. The tuple
$(v_1, \dots, v_m)$ is called the \emph{basis}\index{basis} of the
lattice. The lattice is \emph{complete}\index{complete lattice} if
$m = n$. The set
\[
F =
\setb{\sum_{i=1}^m x_i v_i}{\forall i \leq m \colon x_i \in [0, 1)}
\]
is the \emph{fundamental domain}\index{fundamental domain} of the
basis $(v_1, \dots, v_m)$.
\end{definicija}

\begin{trditev}
Let $V$ be an $n$-dimensional $\R$-vector space and
$\Gamma \subseteq V$ be a subgroup. Then the following statements
are equivalent:

\begin{enumerate}[i)]
\item The set $\Gamma$ is a lattice.
\item The point $0$ is not an accumulation point of $\Gamma$.
\item The set $\Gamma$ is discrete.
\end{enumerate}
\end{trditev}

\begin{proof}
Suppose that $\Gamma$ is a lattice. Extend its basis
$(v_1, \dots, v_m)$ to a basis of $\R^n$. Then
\[
\setb{\sum_{i=1}^n x_i v_i}{\forall i \colon x_i \in (-1,1)}
\]
contains no points of $\Gamma$ other than $0$.

If $0$ is not an accumulation point of $\Gamma$, the set is clearly
discrete, as accumulation points are translation invariant.

Now suppose that $\Gamma$ is discrete and let $W = \R \Gamma$.
Choose a basis $(w_1, \dots, w_m) \subseteq \Gamma$ for $W$. The
set
\[
\Gamma_0 = \bigoplus_{i=1}^m w_i \Z \subseteq \Gamma
\]
is therefore a complete lattice in $W$. The fundamental domain
$F_0$ of $\Gamma_0$ is a set of representatives for
$\kvoc{W}{\Gamma_0}$. But then there exists a set $R \subseteq F_0$
of representatives of $\kvoc{\Gamma}{\Gamma_0}$, which is both
bounded and discrete, and therefore finite. For
$d = [\Gamma : \Gamma_0]$ we then have
$\Gamma \subseteq \frac{1}{d} \Gamma_0$, which must then be a free
abelian group of rank $m$ by the structure theorem. Since it spans
$W$, its generators must be $\R$-linearly independent.
\end{proof}

\begin{lema}
A lattice $\Gamma \subseteq V$ is complete if and only if
$\kvoc{V}{\Gamma}$ has a bounded system of representatives.
\end{lema}

\begin{proof}
If $\Gamma$ is complete, then any fundamental domain gives us a
bounded system of representatives.

Suppose now that $\Gamma \subseteq V$ is a lattice and $B$ a
bounded set with
\[
V = \bigcup_{\gamma \in \Gamma} (\gamma + B).
\]
Let $W = \Lin(\Gamma)$. As it is a finite-dimensional subspace in
$V$, it is a closed subspace. Take an arbitrary $v \in V$. We can
write $n \cdot v = \gamma_n + \beta_n$ for some
$\gamma_n \in \Gamma$ and $\beta_n \in B$. It follows that
\[
v =
\lim_{n \to \infty} \frac{1}{n} \cdot \br{\gamma_n + \beta_n} =
\lim_{n \to \infty} \frac{1}{n} \cdot \gamma_n \in W. \qedhere
\]
\end{proof}

\datum{2024-4-19}

\begin{definicija}
Let $\Gamma \subseteq \R^n$ be a complete lattice with fundamental
domain $F$. We define its \emph{volume}\index{volume} as
\[
\vol(\Gamma) = \vol(F).
\]
\end{definicija}

\begin{izrek}[Minkowski]
\index{Minkowski's theorem}
Let $\Gamma \subseteq \R^n$ be a complete lattice and
$X \subseteq \R^n$ a set with the following properties:

\begin{enumerate}[i)]
\item It is symmetric around $0$.
\item It is convex.
\item We have $\vol(X) > 2^n \vol(\Gamma)$.
\end{enumerate}

Then $X$ contains a non-zero point of $\Gamma$.
\end{izrek}

\begin{proof}
Suppose that the family
$\set{\frac{1}{2} X + \gamma}_{\gamma \in \Gamma}$ is pairwise
disjoint. We can write
\[
\R^n = \bigcup_{\gamma \in \Gamma} (\gamma + F),
\]
where $F$ is the fundamental domain of $\Gamma$. It follows that
\[
\frac{1}{2} X =
\bigcup_{\gamma \in \Gamma} \br{\frac{1}{2} X \cap (\gamma + F)},
\]
therefore
\[
\frac{1}{2^n} \vol(X) =
\sum_{\gamma \in \Gamma}
\vol \br{\frac{1}{2} X \cap (\gamma + F)} =
\sum_{\gamma \in \Gamma}
\vol \br{\br{\frac{1}{2} X - \gamma} \cap F} \leq
\vol(F),
\]
which is a contradiction.

We can now write
\[
\gamma_1 + \frac{1}{2} x_1 = \gamma_2 + \frac{1}{2} x_2
\]
for some distinct $\gamma_i \in \Gamma$ and $x_i \in X$. It is
clear that the point $\frac{1}{2} (x_1 - x_2) \ne 0$ is in both $X$
and $\Gamma$.
\end{proof}

\newpage

\subsection{From ideals to lattices}

\begin{definicija}
Let $K$ be a number field of degree $n$. An embedding
$\sigma \in \Hom_\Q(K, \C)$ is called a
\emph{real embedding}\index{real, complex embedding} if
$\sigma(K) \subseteq \R$. Otherwise, it is called a
\emph{complex embedding}.
\end{definicija}

\begin{opomba}
A conjugate of a complex embedding is again a complex embedding.
We denote by $r$ the number of real embeddings and by
$s = \frac{n-r}{2}$ the number of pairs of conjugated complex
embeddings.
\end{opomba}

\begin{opomba}
Henceforth we assume the notation that $\sigma_1, \dots, \sigma_r$
are real embeddings and $\sigma_{r+i} = \oline{\sigma}_{r+i+s}$.
\end{opomba}

\begin{opomba}
We can embed $j \colon K \to \R^n$ as
\[
j(\alpha) =
\br{\sigma_1(\alpha), \dots, \sigma_r(\alpha),
\Re \sigma_{r+1}(\alpha), \dots, \Re \sigma_{r+s}(\alpha),
\Im \sigma_{r+1}(\alpha), \dots, \Im \sigma_{r+s}(\alpha)}.
\]
\end{opomba}

\begin{trditev}
Let $\mathfrak{a} \subseteq K$ be a fractional ideal. Then
$j(\mathfrak{a})$ is a complete lattice with
\[
\vol(j(\mathfrak{a})) = 2^{-s} \sqrt{\abs{\disc(\mathfrak{a})}}.
\]
\end{trditev}

\begin{proof}
Let $\alpha_1, \dots, \alpha_n$ be a $\Z$-basis of $\mathfrak{a}$.
Then
\[
\disc(\mathfrak{a}) =
\det \begin{bmatrix}
\sigma_k(\alpha_\ell)
\end{bmatrix}_{k, \ell \leq n}^2.
\]
Note that
\[
\begin{bmatrix}
\Re \sigma_{r+\ell}(\alpha) \\
\Im \sigma_{r+\ell}(\alpha)
\end{bmatrix}
=
\begin{bmatrix}
 \frac{1}{2} &  \frac{1}{2}  \\
\frac{1}{2i} & -\frac{1}{2i}
\end{bmatrix}
\cdot
\begin{bmatrix}
\sigma_{r+\ell}(\alpha)   \\
\sigma_{r+\ell+s}(\alpha)
\end{bmatrix}.
\]
It follows that
\[
j(\alpha) =
\underbrace{\begin{bmatrix}
I_r &        0         &        0          \\
 0  &  \frac{1}{2} I_s &  \frac{1}{2} I_s  \\
 0  & \frac{1}{2i} I_s & -\frac{1}{2i} I_s
\end{bmatrix}}_C
\cdot
\begin{bmatrix}
\sigma_1(\alpha) \\
\sigma_2(\alpha) \\
     \vdots      \\
\sigma_n(\alpha)
\end{bmatrix}.
\]
As $\det(C) = \frac{1}{2^s} \cdot \br{-\frac{1}{i}}^s$, we get
$\abs{\det(C)} = \frac{1}{2^s}$. Finally, we get
\[
\vol(j(\mathfrak{a})) =
\abs{\det \br{j(\alpha_1), j(\alpha_2), \dots, j(\alpha_n)}} =
\abs{\det C} \cdot
\abs{\det \begin{bmatrix}
\sigma_k(\alpha_\ell)
\end{bmatrix}_{k, \ell \leq n}} =
2^{-s} \cdot \sqrt{\abs{\disc(\mathfrak{a})}}. \qedhere
\]
\end{proof}

\begin{izrek}
\label{mink:thm:bdb_disc_exist}
Let $\mathfrak{a}$ be a fractional ideal of $\mathcal{O}_K$. For
$i \leq r+s$ let $c_i > 0$ be real numbers such that
\[
\prod_{i=1}^r c_i \prod_{i=1}^s c_{i+r}^2 >
\br{\frac{2}{\pi}}^s \sqrt{\abs{\disc(\mathfrak{a})}}.
\]
Then there exists a non-zero $\alpha \in \mathfrak{a}$ such that
$\abs{\sigma_i(\alpha)} < c_i$ for all $i \leq n$.
\end{izrek}

\begin{proof}
Let
\[
X = \setb{x \in \R^n}
{\forall i \leq r \colon \abs{x_i} < c_i \land
\forall i \leq s \colon x_{r+i}^2 + x_{r+s+i}^2 < c_{r+i}^2}.
\]
We can then calculate
\[
\vol(X) =
\prod_{i=1}^r (2 c_i) \cdot
\prod_{i=1}^s \br{c_{r+i}^2 \cdot \pi} =
2^r \cdot \pi^s \cdot \prod_{i=1}^r c_i \prod_{i=1}^s c_{i+r}^2 >
2^{r+s} \cdot \sqrt{\abs{\disc(\mathfrak{a})}} =
2^n \cdot \vol(j(\mathfrak{a})).
\]
The set $j(\mathfrak{a}) \cap X$ therefore contains a non-zero
element. Its preimage is the sought element.
\end{proof}

\begin{izrek}[Minkowski]
\index{Minkowski's theorem}
Let $\mathfrak{a}$ be a fractional ideal of $\mathcal{O}_K$. Then
there exists a non-zero $\alpha \in \mathfrak{a}$ such that
\[
\abs{N^K(\alpha)} \leq
\frac{n!}{n^n} \cdot \br{\frac{4}{\pi}}^s
\sqrt{\abs{\disc(\mathfrak{a})}}.
\]
\end{izrek}

\begin{proof}
Choose a real $c > 0$ such that
\[
c^n > n! \cdot \br{\frac{4}{\pi}}^s
\sqrt{\abs{\disc(\mathfrak{a})}}
\]
and let
\[
Y = \setb{x \in \R^n}{\sum_{i=1}^r \abs{x_i} +
2 \sum_{i=1}^s \sqrt{x_{r+i}^2 + x_{r+s+i}^2} < c}.
\]
Someone who actually knows how to integrate can show that
\[
\vol(Y) =
2^r \cdot \br{\frac{\pi}{2}}^s \cdot \frac{c^n}{n!} =
2{r+s} \cdot \br{\frac{\pi}{4}}^s \cdot \frac{c^n}{n!} >
2^{r+s} \cdot \sqrt{\abs{\disc(\mathfrak{a})}} =
2^n \vol(j(\mathfrak{a})).
\]
It follows that $Y \cap j(\mathfrak{a})$ contains a non-zero
element. Equivalently, there exists some non-zero
$\alpha \in \mathfrak{a}$ such that $j(\alpha) \in Y$.

Now note that
\begin{align*}
\sqrt[n]{N^K(\alpha)} &=
\prod_{i=1}^r \abs{\sigma_i(\alpha)}^{\frac{1}{n}} \cdot
\prod_{i=1}^s \sqrt{\br{\Re \sigma_{r+i}(\alpha)}^2 +
\br{\Im \sigma_{r+i}(\alpha)}^2}^{\frac{2}{n}}
\\
&\leq
\frac{1}{n} \cdot \br{
\sum_{i=1}^r \abs{\sigma_i(\alpha)} + 2 \sum_{i=1}^s
\sqrt{\br{\Re \sigma_{r+i}(\alpha)}^2 +
\br{\Im \sigma_{r+i}(\alpha)}^2}}
\\
&< \frac{c}{n}.
\end{align*}
But then
\[
\abs{N^K(\alpha)} <
\frac{c^n}{n^n} \leq
\frac{n!}{n^n} \br{\frac{4}{\pi}}^s
\sqrt{\abs{\disc(\mathfrak{a})}} + \varepsilon
\]
for some $\varepsilon > 0$. Note that the set
$\abs{N^K(\mathfrak{a})}$ is discrete -- taking $c$ small enough
we therefore get
\[
\abs{N^K(\alpha)} \leq
\frac{n!}{n^n} \br{\frac{4}{\pi}}^s
\sqrt{\abs{\disc(\mathfrak{a})}}. \qedhere
\]
\end{proof}

\newpage

\subsection{Finiteness of the class group}

\begin{definicija}
Let $\mathfrak{a} \edn \mathcal{O}_K$ be a non-zero ideal. We
define the \emph{norm}\index{norm} of $\mathfrak{a}$ as
\[
N(\mathfrak{a}) = \abs{\kvoc{\mathcal{O}_K}{\mathfrak{a}}}.
\]
\end{definicija}

\begin{trditev}
Let $K$ be a number field.

\begin{enumerate}[i)]
\item If $\mathfrak{a}, \mathfrak{b} \edn \mathcal{O}_K$ are
non-zero ideals, then
$N(\mathfrak{ab}) = N(\mathfrak{a}) \cdot N(\mathfrak{b})$.
\item If $\mathfrak{a} = (\alpha)$ for a non-zero
$\alpha \in \mathcal{O}_K$, then
$N(\mathfrak{a}) = \abs{N^K(\alpha)}$.
\end{enumerate}
\end{trditev}

\begin{proof}
\phantom{i}
\begin{enumerate}[i)]
\item If $\mathfrak{a}$ and $\mathfrak{b}$ are coprime, the
conclusion follows from the Chinese remainder theorem. It therefore
suffices to consider the case where $\mathfrak{a}$ and
$\mathfrak{b}$ are both powers of the same non-zero prime ideal
$\mathfrak{p}$, that is,
$N(\mathfrak{p}^{e+1}) = N(\mathfrak{p}^e) \cdot N(\mathfrak{p})$
for $e \geq 0$.

We will show that
$\kvoc{\mathcal{O}_K}{\mathfrak{p}} \cong
\kvoc{\mathfrak{p}^e}{\mathfrak{p}^{e+1}}$. Take
$a \in \mathfrak{p}^e \setminus \mathfrak{p}^{e+1}$ and consider
the homomorphism $\varphi \colon \mathcal{O}_K \to
\kvoc{\mathfrak{p}^e}{\mathfrak{p}^{e+1}}$, given by
$\varphi(x) = ax + p^{e+1}$. This induces a homomorphism
$\kvoc{\mathcal{O}_K}{\mathfrak{p}} \to
\kvoc{\mathfrak{p}^e}{\mathfrak{p}^{e+1}}$, which means that
$\kvoc{\mathfrak{p}^e}{\mathfrak{p}^{e+1}}$ is a
$\kvoc{\mathcal{O}_K}{\mathfrak{p}}$-vector space. If the above
rings were not isomorphic, its dimension would be at least $2$,
therefore it would have a non-trivial subspace of the form
$\kvoc{\mathfrak{b}}{\mathfrak{p}^{e+1}}$ for an ideal
$\mathfrak{b} \edn \mathcal{O}_K$. But then
$\mathfrak{p}^{e+1} \subset \mathfrak{b} \subset \mathfrak{p}^e$,
which implies
$\mathfrak{p} \subset
\mathfrak{p}^{-e} \mathfrak{b} \subset
\mathcal{O}_K$,
which contradicts $\mathfrak{p}$ being a maximal ideal.
\item Let $\beta_1, \dots, \beta_n$ be a $\Z$-basis of
$\mathcal{O}_K$. Then $\alpha \beta_1, \dots, \alpha \beta_n$ is a
$\Z$-basis of $(\alpha)$. We therefore have
\[
\disc(\mathfrak{a}) =
[\mathcal{O}_K : \mathfrak{a}]^2 \cdot \disc(\mathcal{O}_K).
\]
It therefore suffices to show that
\[
\disc(\mathfrak{a}) =
N^K(\alpha)^2 \cdot \disc(\mathcal{O}_K).
\]
Indeed, we have
\begin{align*}
\disc(\mathfrak{a}) &=
\det \begin{bmatrix}
\sigma_k(\alpha \beta_\ell)
\end{bmatrix}_{k, \ell \leq n}^2
\\
&=
\det \begin{bmatrix}
\sigma_k(\alpha) \sigma_k(\beta_\ell)
\end{bmatrix}_{k, \ell \leq n}^2
\\
&=
\prod_{k=1}^n \sigma_k(\alpha) \cdot
\det \begin{bmatrix}
\sigma_k(\beta_\ell)
\end{bmatrix}_{k, \ell \leq n}^2
\\
&=
N^K(\alpha)^2 \cdot \disc(\mathcal{O}_K). \qedhere
\end{align*}
\end{enumerate}
\end{proof}

\begin{opomba}
The norm multiplicatively extends to a map
$\mathcal{F}(\mathcal{O}_K) \to \Q^*$.
\end{opomba}

\begin{izrek}
The class group of $\mathcal{O}_K$ is finite. Furthermore, every
ideal class contains a representative $\mathfrak{a}$ with
\[
N(\mathfrak{a}) \leq
\frac{n!}{n^n} \br{\frac{4}{\pi}}^s \sqrt{\abs{\disc(K)}}.
\]
\end{izrek}

\begin{proof}
We claim that for every $M > 0$ there exist only finitely many
non-zero ideals $\mathfrak{a} \edn \mathcal{O}_K$ with
$N(\mathfrak{a}) \leq M$. Indeed, suppose that
$\abs{\kvoc{\mathcal{O}_K}{\mathfrak{a}}} \leq M$. Then
$M! \cdot \kvoc{\mathcal{O}_K}{\mathfrak{a}} = 0$, therefore
\[
M! \cdot \mathcal{O}_K \subseteq
\mathfrak{a} \subseteq
\mathcal{O}_K.
\]
But as $\kvoc{\mathcal{O}_K}{M! \mathcal{O}_K}$ is finite, there
are only finitely many possible $\mathfrak{a}$ satisfying the above
condition.

It now suffices to show the above bound. Let
$\mathfrak{a}_0 \edn \mathcal{O}_K$ be a representative of an ideal
class and let $\mathfrak{b} = \alpha \mathfrak{a}_0^{-1}$ be an
ideal. By Minkowski's theorem, there exists an element
$\beta \in \mathfrak{b}$ with
\[
\abs{N^K(\beta)} \leq
\frac{n!}{n^n} \cdot \br{\frac{4}{\pi}}^s \cdot
\sqrt{\abs{\disc(\mathfrak{b})}}.
\]
As $\disc(\mathfrak{b}) = \disc(K) \cdot N(\mathfrak{b})^2$, we get
\[
N\br{\beta \mathfrak{b}^{-1}} =
\abs{N^K(\beta)} \cdot N(\mathfrak{b})^{-1} \leq
\frac{n!}{n^n} \cdot \br{\frac{4}{\pi}}^2 \cdot
\sqrt{\abs{\disc(K)}}.
\]
But since $[\beta \mathfrak{b}^{-1}] = [\mathfrak{a}_0]$, this
ideal satisfies our conditions.
\end{proof}

\begin{definicija}
The \emph{class number}\index{class number} of $\mathcal{O}_K$ is
defined as the size of its class group, that is
$h_K = \abs{\mathcal{C}(\mathcal{O}_K)}$.
\end{definicija}

\begin{izrek}[Minkowski]
\index{Minkowski's theorem}
If $n = [K : \Q] \geq 2$, then
\[
\abs{\disc(K)} \geq
\br{\frac{\pi^s n^n}{4^s n!}}^2 >
1.
\]
Furthermore, the lower bound diverges as $n \to \infty$.
\end{izrek}

\begin{proof}
By Minkowski's theorem, there exists an element
$\alpha \in \mathcal{O}_K$ with
\[
\abs{N^K(\alpha)} \leq
\frac{n!}{n^n} \cdot \br{\frac{4}{\pi}}^s \cdot
\sqrt{\abs{\disc(K)}}.
\]
Since $\abs{N^K(\alpha)} \geq 1$, we get
\[
\abs{\disc(K)} \geq
\br{\frac{n^n}{n!}}^2 \cdot \br{\frac{\pi}{4}}^{2s} \geq
\br{\frac{n^n}{n!}}^2 \cdot \br{\frac{\pi}{4}}^n =
f(n).
\]
Since $f(2) = \frac{\pi^2}{4} > 2$ and
\[
\frac{f(n+1)}{f(n)} =
\frac{\pi}{4} \cdot \br{\frac{n+1}{n}}^{2n} \geq
\frac{3 \pi}{4} >
1
\]
by Bernoulli's inequality, the lower bound indeed diverges and is
greater than $1$.
\end{proof}

\datum{2024-4-26}

\begin{izrek}[Hermite]
\index{Hermite's theorem}
For all $D \geq 0$ there exist only finitely many number fields $K$
with $\abs{\disc(K)} \leq D$.
\end{izrek}

\begin{proof}
By Minkowski's theorem, it suffices to show that there exist only
finitely many number fields $K$ with $\disc(K) = d$ and
$[K : \Q] = n$. This is clear for $n=1$, hence assume $n>1$.

First note that there exists some
$\alpha \in \mathcal{O}_K \setminus \set{0}$ such that
$\abs{\sigma_1(\alpha)} < \sqrt{d}+1$ and
$\abs{\sigma_i(\alpha)} < 1$ for $i \geq 2$ by
theorem~\ref{mink:thm:bdb_disc_exist}. But then all conjugates of
$\alpha$ are bounded in terms of $d$, hence so are the coefficients
of its minimal polynomial. Therefore there are only finitely many
such $\alpha$ for fixed $n$.

Next, we show that $K = \Q(\alpha)$, which shows that there are
only finitely many such number fields. We split two cases.

\begin{enumerate}[i)]
\item Suppose that $r > 0$. Then
\[
\abs{\sigma_1(\alpha)} =
\abs{N^K(\alpha)} \cdot \prod_{i=2}^n \abs{\sigma_i(\alpha)}^{-1} >
\abs{N^K(\alpha)} \geq 1.
\]
Now consider
$\eval{\sigma_1}{\Q(\alpha)}{} \in \Hom_\Q(\Q(\alpha), \C)$. It has
exactly $[K : \Q(\alpha)]$ extensions to an element of
$\Hom_\Q(K, \C)$. Since
$\abs{\tilde{\sigma}_1(\alpha)} = \abs{\sigma_1(\alpha)} > 1$,
we must have $\tilde{\sigma}_1 = \sigma_1$ and so
$[K : \Q(\alpha)] = 1$.

\item Now suppose that $r=0$. Modifying the proof of
theorem~\ref{mink:thm:bdb_disc_exist}, we can further take
$\abs{\Re \sigma_1(\alpha)} < 1$ and
$\abs{\Im \sigma_1(\alpha)} < C \sqrt{d}$ for some constant $C$.
Then
\[
\abs{\sigma_1(\alpha)}^2 =
\abs{N^K(\alpha)} \cdot \prod_{i=2}^n \abs{\sigma_i(\alpha)}^{-2} >
\abs{N^K(\alpha)} \geq 1.
\]
Again consider
$\eval{\sigma_1}{\Q(\alpha)}{} \in \Hom_\Q(\Q(\alpha), \C)$. As
above, we see that every extension satisfies
$\abs{\tilde{\sigma}_1(\alpha)} = \abs{\sigma_1(\alpha)} > 1$,
therefore $\tilde{\sigma}_1 \in \set{\sigma_1, \oline{\sigma}_1}$.
Since they differ in $\alpha$ by our modified assumptions, only one
extends $\eval{\sigma_1}{\Q(\alpha)}{}$ and so
$[K : \Q(\alpha)] = 1$.
\qedhere
\end{enumerate}
\end{proof}

\begin{opomba}
A \emph{Pisot number}\index{Pisot number} is a real algebraic
integer $\alpha > 1$ whose all conjugates have absolute value less
than $1$.
\end{opomba}

\newpage

\subsection{Dirichlet's unit theorem}

\begin{definicija}
Let $K$ be a number field. We denote the set of all roots of unity
in $K$ by $\mu(K)$.
\end{definicija}

\begin{definicija}
We define a map $\lambda \colon \mathcal{O}_K^* \to \R^{r+s}$ as
\[
\lambda(\alpha) =
\br{\log \abs{\sigma_1(\alpha)}, \dots,
\log \abs{\sigma_r(\alpha)},
2 \log \abs{\sigma_{r+1}(\alpha)}, \dots,
2 \log \abs{\sigma_{r+s}(\alpha)}}.
\]
\end{definicija}

\begin{opomba}
Note that $\lambda \colon (\mathcal{O}_K^*, \cdot) \to (\R, +)$ is
a group homomorphism.
\end{opomba}

\begin{lema}
The set $\lambda \br{\mathcal{O}_K^*}$ is a lattice in the
hyperplane
\[
H = \setb{x \in \R^{r+s}}{\sum_{i=1}^{r+s} x_i = 0}.
\]
\end{lema}

\begin{proof}
It clearly suffices to show that $\lambda \br{\mathcal{O}_K^*}$ is
discrete. That is, there exists a neighbourhood of $0$ containing
only finitely many points in this set.

Let $B = [-C, C]^{r+s}$. Clearly, $j \br{\lambda^{-1}(B)}$ is
bounded. Since $j(\mathcal{O}_K)$ is a lattice, $\lambda^{-1}(B)$
is finite and hence so is $B \cap \lambda(\mathcal{O}_K^*)$.
\end{proof}

\begin{lema}
We have $\ker \lambda = \mu(K)$, which is a finite cyclic group.
\end{lema}

\begin{proof}
First note that if $\zeta \in \mu(K)$, then clearly
$\abs{\sigma_i(\zeta)} = 1$ and so $\lambda(\zeta) = 0$. As
$\lambda(\ker(\lambda))$ is trivially bounded, the proof of the
previous lemma shows that $\ker \lambda$ is finite. This means
that every element of $\ker \lambda$ has finite order and is
therefore a root of unity. As every finite multiplicative subgroup
of a field is cyclic, the conclusion follows.
\end{proof}

\begin{trditev}
We have $\mathcal{O}_K^* \cong \mu(K) \times \Z^t$ for some
$t \leq r+s-1$.
\end{trditev}

\begin{proof}
The short exact sequence
\[
\begin{tikzcd}
1 \arrow[r] &
\mu(K) \arrow[r, hook] &
\mathcal{O}_K^* \arrow[r, "\lambda"] &
\Z^t \arrow[r] &
0
\end{tikzcd}
\]
is exact and therefore splits.
\end{proof}

\begin{lema}
Let $M \geq 0$. Up to associativity, there exist only finitely many
elements $\alpha \in \mathcal{O}_K$ with $\abs{N^K(\alpha)} < M$.
\end{lema}

\begin{proof}
The condition is equivalent to $N((\alpha)) < M$, but there are
only finitely many such ideals.
\end{proof}

\begin{izrek}[Dirichlet's unit theorem]
\index{Dirichlet's unit theorem}
Let $K$ be a number field. Then $\mu(K)$ is a finite cyclic group
and $\mathcal{O}_K^* \cong \mu(K) \times \Z^{r+s-1}$.
\end{izrek}

\begin{proof}
We already know that $\mathcal{O}_K^* \cong \mu(K) \times \Z^t$ and
that $\mu(K)$ is cyclic. It is therefore enough to show that
$t = r+s-1$. To do so, we will show that
$\lambda \br{\mathcal{O}_K^*} \subseteq H$ is a complete lattice.
Equivalently, we need to show that
$\kvoc{H}{\lambda \br{\mathcal{O}_K^*}}$ has a bounded system of
representatives.

Set $g(x) = \br{\abs{\sigma_1(x)}, \dots, \abs{\sigma_{r+s}(x)}}$
and
$l(x) =
\br{\log x_1, \dots, \log x_r,
2 \log x_{r+1}, \dots, 2 \log x_{r+s}}$, so that
$\lambda = l \circ g$. Furthermore, let
\[
\norm{x} = \prod_{i=1}^r x_i \cdot \prod_{i=1}^s x_{r+i}^2.
\]
Then $\norm{g(x)} = 1$ for all $x \in \mathcal{O}_K^*$. Finally,
set $S = l^{-1}(H)$.

We claim that there exists a bounded set $T \subseteq S$ such that
\[
S = \bigcup_{\varepsilon \in \mathcal{O}_K^*} g(\varepsilon) T.
\]
To see this, choose $c \in \br{\R^+}^{r+s}$ such that
\[
\norm{c} > \br{\frac{2}{\pi}}^s \cdot \sqrt{\abs{\disc(K)}}
\]
and set
\[
X = \setb{x \in \br{\R^+}^{r+s}}{\forall i \colon x_i < c_i}.
\]
Note that, for any $y \in S$, we have $\norm{cy^{-1}} = \norm{c}$.
By theorem~\ref{mink:thm:bdb_disc_exist} there exists a non-zero
element $\alpha \in \mathcal{O}_K$ with $g(\alpha) \in yX$. This
element also satisfies $\abs{N^K(\alpha)} \leq \norm{c}$. There are
only finitely many such elements up to associativity -- denote them
by $\alpha_1, \dots, \alpha_m$.

We claim that
\[
T = S \cap \bigcup_{i=1}^m g(\alpha_i)^{-1} X
\]
satisfies the conditions. Indeed, it is clearly bounded. For any
$y \in S$, and set $g(\alpha) = y^{-1} x$ for some
$\alpha \in \mathcal{O}_K$ and $x \in X$, where
$\abs{N^K(\alpha)} \leq \norm{c}$. This means that
$\varepsilon = \alpha^{-1} \cdot \alpha_i \in \mathcal{O}_K^*$ for
some $i \leq m$. Hence
\[
y =
g(\alpha)^{-1} x =
g \br{\alpha_i \cdot \varepsilon^{-1}}^{-1} x =
g(\varepsilon) \cdot g(\alpha_i)^{-1} x \in
g(\varepsilon) T,
\]
as required.

For each $x \in T$, we now have that $x_i$ are bounded from above.
But as $\norm{x} = 1$, they are also bounded from below. The set
$l(T)$ is therefore bounded, but as
\[
H =
l(S) =
\bigcup_{\varepsilon \in \mathcal{O}_K^*} l(g(\varepsilon) T) =
\bigcup_{\varepsilon \in \mathcal{O}_K^*}
\br{\lambda(\varepsilon) + l(T)},
\]
the set $l(T)$ is a bounded set of representatives for
$\kvoc{H}{\lambda(\mathcal{O}_K^*)}$.
\end{proof}
