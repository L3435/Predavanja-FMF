\section{Algebraic integers}

\subsection{Gaussian integers}

\datum{2024-3-8}

\begin{definicija}
A domain $R$ is \emph{Euclidean}\index{Euclidean domain} if there
is a $\delta \colon \R \setminus \set{0} \to \N_0$, such that for
all $a \in R$ and $b \in R \setminus \set{0}$ we can write
$a = bq + r$ for $q, r \in R$, such that $r = 0$ or
$\delta(r) < \delta(b)$.
\end{definicija}

\begin{trditev}
The Gaussian integers are an Euclidean domain.
\end{trditev}

\begin{proof}
Algebra 2, theorem 6.3.4.
\end{proof}

\begin{definicija}
A domain $R$ is a
\emph{unique factorisation domain}\index{unique factorisation domain}
if every $\alpha \in R$ is of the form
\[
\alpha = \prod_{i=1}^n p_i
\]
for irreducible elements in a unique way up to permutation and
multiplication of factors by a unit element.
\end{definicija}

\begin{opomba}
Principal ideal domains (and therefore $\Z[i]$) are unique
factorisation domains.
\end{opomba}

\begin{lema}
The function $N \colon \Z[i] \to \N_0$, given by
$N(a + bi) = a^2 + b^2$, has the following properties:

\begin{enumerate}[i)]
\item The equality $N(\alpha) = 0$ is equivalent to $\alpha = 0$.
\item For all $\alpha, \beta \in \Z[i]$ we have
$N(\alpha \beta) = N(\alpha) \cdot N(\beta)$.
\item An element $\alpha \in \Z[i]$ is invertible if and only if
$N(\alpha) = 1$.
\end{enumerate}
\end{lema}

\obvs

\begin{lema}
Let $p \in \P$ be a prime. Then $-1$ is a quadratic residue modulo
$p$ if and only if $p = 2$ or $p \equiv 1 \pmod{4}$.
\end{lema}

\begin{proof}
If $p \equiv 1 \pmod{4}$, we can write
$-1 \equiv \br{e^{\frac{p-1}{4}}}^2 \pmod{p}$.
If $p \equiv 3 \pmod{4}$ and $p \mid c^2 + 1$, then
\[
1 \equiv \br{c^2}^{\frac{p-1}{2}} \equiv (-1)^{\frac{p-1}{2}} = -1,
\]
a clear contradiction.
\end{proof}

\begin{izrek}[Fermat]
\index{Fermatov izrek}
Let $p$ be an odd prime. Then $p$ can be written as a sum of two
squares if and only if $p \equiv 1 \pmod{4}$.
\end{izrek}

\begin{proof}
It is clear that primes $p \equiv 3 \pmod{4}$ cannot be written in
such a way. Now suppose that $p \equiv 1 \pmod{4}$ and take
$b \in \N$ such that $b^2 \equiv -1 \pmod{p}$. Now note that
$p \mid (b-i)(b+i)$, but $p$ clearly can't divide either factor. It
follows that $p$ is not a prime element, hence we can factor it as
$p = \alpha \beta$.

Now, write $p^2 = N(p) = N(\alpha) \cdot N(\beta)$, therefore
$N(\alpha) = p$, which gives us a representation of $p$ as a sum of
two squares.
\end{proof}

\begin{trditev}
Up to associativity, the prime elements of $\Z[i]$ are the
following:

\begin{enumerate}[i)]
\item $1 + i$,
\item $a + bi$ with $a^2 + b^2 = p \in \P$ with
$p \equiv 1 \pmod{4}$ and $0 < \abs{b} < a$,
\item $p \in \P$ with $p \equiv 3 \pmod{4}$.
\end{enumerate}
\end{trditev}

\begin{proof}
It is clear that $1+i$ is a prime element. Elements of the second
form are prime by the proof of the previous theorem. For the last
one, if $p = \alpha \beta$ for non-invertible $\alpha$ and $\beta$,
then $N(\alpha) = N(\beta) = p$, which is of course impossible.
Clearly, they are not associated.

Suppose now that $p \in \Z[i]$ is a prime element. Then
$N(p) = p \oline{p}$, which can be factored in integers. But then
$p$ divides some prime element $q \in \P$. It follows that
$N(p) \mid q^2$, but as $q^2$ can be factored by the above prime
elements, $p$ is of such form.
\end{proof}

\begin{izrek}
Let $n \in \N$. Then there exist integers $a$ and $b$ such that
$n = a^2 + b^2$ if and only if $2 \mid \nu_p(n)$ for all prime
numbers $p \equiv 3 \pmod{4}$.
\end{izrek}

\begin{proof}
It is clear that all such numbers can be written as a sum of two
squares, as the property is multiplicative. For the converse,
suppose that $n = a^2 + b^2$ and take any prime number $p \in \P$
such that $p \equiv 3 \pmod{4}$ and $p \mid n$. Then, if $b$ is
invertible in $\Z_p$, we can write
\[
\left.p~\middle\vert~\br{\frac{a}{b}}^2 + 1\right.,
\]
which is impossible. It follows that $p \mid a, b$. The theorem is
now proven by infinite descent.
\end{proof}

\begin{opomba}
This theorem can also be proven by factoring $\alpha = a + bi$.
\end{opomba}

\begin{opomba}
A positive integer $n$ can be written as a sum of $3$ squares if
and only if it is not of the form $n = 4^a \cdot (8k+7)$.
\end{opomba}

\begin{trditev}
Let $\alpha \in \Q(i)$. Then $\alpha \in \Z[i]$ if and only if
there exist some $c, d \in \Z$ such that $\alpha$ is a root of the
polynomial $P(x) = x^2 + cx + d$.
\end{trditev}

\begin{proof}
We see that $P(\alpha) = 0$ and $\alpha \not \in \Q$ is equivalent
to $P(x) = (x - \alpha) \br{x - \oline{\alpha}}$. Of course, if
$\alpha \in \Q$, we must have $\alpha \in \Z$ by 
characterisation of rational roots of integer polynomials.
Otherwise, for $\alpha = a + bi$, the condition is equivalent to
$2a \in \Z$ and $a^2 + b^2 \in \Z$, which is only possible if both
$a \in \Z$ and $b \in \Z$.
\end{proof}
