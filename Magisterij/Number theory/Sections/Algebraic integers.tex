\section{Algebraic integers}

\epigraph{This is usually attributed to Fermat, but it's not quite
correct.}
{-- gost.~izr.~prof.~dr.~rer.~nat.~Daniel Smertnig}

\subsection{Gaussian integers}

\datum{2024-3-8}

\begin{definicija}
A domain $R$ is \emph{Euclidean}\index{Euclidean domain} if there
is a function $\delta \colon R \setminus \set{0} \to \N_0$, such
that for all $a \in R$ and $b \in R \setminus \set{0}$ we can write
$a = bq + r$ for $q, r \in R$, such that $r = 0$ or
$\delta(r) < \delta(b)$.
\end{definicija}

\begin{trditev}
The Gaussian integers are an Euclidean domain.
\end{trditev}

\begin{proof}
Algebra 2, theorem 6.3.4.
\end{proof}

\begin{definicija}
A domain $R$ is a
\emph{unique factorisation domain}\index{unique factorisation domain}
if every $\alpha \in R$ is of the form
\[
\alpha = \prod_{i=1}^n p_i
\]
for irreducible elements in a unique way up to permutation and
multiplication of factors by a unit element.
\end{definicija}

\begin{opomba}
Principal ideal domains (and therefore $\Z[i]$) are unique
factorisation domains.
\end{opomba}

\begin{lema}
The function $N \colon \Z[i] \to \N_0$, given by
$N(a + bi) = a^2 + b^2$, has the following properties:

\begin{enumerate}[i)]
\item The equality $N(\alpha) = 0$ is equivalent to $\alpha = 0$.
\item For all $\alpha, \beta \in \Z[i]$ we have
$N(\alpha \beta) = N(\alpha) \cdot N(\beta)$.
\item An element $\alpha \in \Z[i]$ is invertible if and only if
$N(\alpha) = 1$.
\end{enumerate}
\end{lema}

\obvs

\begin{lema}
Let $p \in \P$ be a prime. Then $-1$ is a quadratic residue modulo
$p$ if and only if $p = 2$ or $p \equiv 1 \pmod{4}$.
\end{lema}

\begin{proof}
If $p \equiv 1 \pmod{4}$, we can write
$-1 \equiv \br{e^{\frac{p-1}{4}}}^2 \pmod{p}$, where $e$ is a
primitive root modulo $p$. If $p \equiv 3 \pmod{4}$ and
$p \mid c^2 + 1$, then
\[
1 \equiv \br{c^2}^{\frac{p-1}{2}} \equiv (-1)^{\frac{p-1}{2}} = -1,
\]
a clear contradiction.
\end{proof}

\begin{izrek}[Fermat]
\index{Fermat's theorem}
Let $p$ be an odd prime. Then $p$ can be written as a sum of two
squares if and only if $p \equiv 1 \pmod{4}$.
\end{izrek}

\begin{proof}
It is clear that primes $p \equiv 3 \pmod{4}$ cannot be written in
such a way. Now suppose that $p \equiv 1 \pmod{4}$ and take
$b \in \N$ such that $b^2 \equiv -1 \pmod{p}$. Now note that
$p \mid (b-i)(b+i)$, but $p$ clearly can't divide either factor. It
follows that $p$ is not a prime element, hence we can factor it as
$p = \alpha \beta$.

Now, note that $p^2 = N(p) = N(\alpha) \cdot N(\beta)$, but as
$\alpha$ and $\beta$ are not invertible, we have $N(\alpha) = p$,
which gives us a representation of $p$ as a sum of two squares.
\end{proof}

\begin{trditev}
Up to associativity, the prime elements of $\Z[i]$ are the
following:

\begin{enumerate}[i)]
\item $1 + i$,
\item $a + bi$, where $a^2 + b^2 = p \in \P$ with
$p \equiv 1 \pmod{4}$ and $0 < \abs{b} < a$,
\item $p \in \P$ with $p \equiv 3 \pmod{4}$.
\end{enumerate}
\end{trditev}

\begin{proof}
It is clear that $1+i$ is a prime element. Elements of the second
form are prime since their norm is a prime number. For the last
one, if $p = \alpha \beta$ for non-invertible $\alpha$ and $\beta$,
then $N(\alpha) = N(\beta) = p$, which is of course impossible.
Clearly, they are not associated.

Suppose now that $p \in \Z[i]$ is a prime element. Then
$N(p) = p \oline{p}$, which can be factored in integers. But then
$p$ divides some prime number $q \in \P$. It follows that
$N(p) \mid q^2$, but as $q^2$ can be factored by the above prime
elements, $p$ is of such form.
\end{proof}

\begin{izrek}
Let $n \in \N$. Then there exist integers $a$ and $b$ such that
$n = a^2 + b^2$ if and only if $2 \mid \nu_p(n)$ for all prime
numbers $p \equiv 3 \pmod{4}$.
\end{izrek}

\begin{proof}
It is clear that all such numbers can be written as a sum of two
squares, as the property is
multiplicative.\footnote{$(a^2+b^2)(c^2+d^2) =
(ac+bd)^2 + (ad-bc)^2$.} For the converse, suppose that
$n = a^2 + b^2$ and take any prime number $p \in \P$ such that
$p \equiv 3 \pmod{4}$ and $p \mid n$. Then, if $b$ is invertible in
$\Z_p$, we can write
\[
\left.p~\middle\vert~\br{\frac{a}{b}}^2 + 1\right.,
\]
which is impossible. It follows that $p \mid a, b$. The theorem is
now proven by infinite descent.
\end{proof}

\begin{opomba}
This theorem can also be proven by factoring $\alpha = a + bi$.
\end{opomba}

\begin{opomba}
A positive integer $n$ can be written as a sum of $3$ squares if
and only if it is not of the form $n = 4^a \cdot (8k+7)$.
\end{opomba}

\begin{trditev}
Let $\alpha \in \Q(i)$. Then $\alpha \in \Z[i]$ if and only if
there exist some $c, d \in \Z$ such that $\alpha$ is a root of the
polynomial $P(x) = x^2 + cx + d$.
\end{trditev}

\begin{proof}
We see that $P(\alpha) = 0$ and $\alpha \not \in \Q$ is equivalent
to $P(x) = (x - \alpha) \br{x - \oline{\alpha}}$. Of course, if
$\alpha \in \Q$, we must have $\alpha \in \Z$ by the properties of
rational roots of integer polynomials. Otherwise, for
$\alpha = a + bi$, the condition is equivalent to $2a \in \Z$ and
$a^2 + b^2 \in \Z$, which is only possible if both $a \in \Z$ and
$b \in \Z$.
\end{proof}

\newpage

\subsection{Number fields and their rings of integers}

\datum{2024-3-15}

\begin{definicija}
A \emph{number field}\index{number field} is a subfield of $\C$
such that $[K : \Q] < \infty$. Elements of $K$ are called
\emph{algebraic numbers}\index{algebraic!number}.
\end{definicija}

\begin{definicija}
A field extension $\kvoc{K}{\Q}$ is
\emph{algebraic}\index{algebraic!field extension} if every element
$\alpha \in K$ is a root of a polynomial $f \in \Q[x]$. We denote
the minimal polynomial of $\alpha$ by $m_\alpha$. Furthermore, set
$\deg(\alpha) = \deg(m_\alpha)$.\index{degree}
\end{definicija}

\begin{izrek}[Primitive element theorem]
\index{primitive element theorem}
Let $K$ be a number field. Then there exists some element
$\alpha \in K$ such that $K = \Q(\alpha)$.\footnote{In other words,
$\kvoc{K}{\Q}$ is simple.}
\end{izrek}

\begin{proof}
Algebra 3, theorem~1.1.7.
\end{proof}

\begin{trditev}
Let $K$ be a number field. Then $\kvoc{K}{\Q}$ is a separable
extension.
\end{trditev}

\begin{proof}
Suppose otherwise. Then $\gcd(m_\alpha, m_\alpha')$ is a polynomial
of lower degree with $\alpha$ as a root.
\end{proof}

\begin{opomba}
The roots of $m_\alpha$ are called the
\emph{algebraic conjugates}\index{algebraic!conjugate} of $\alpha$.
\end{opomba}

\begin{posledica}
There are exactly $\deg(\alpha)$ embeddings
$\Q(\alpha) \hookrightarrow \C$.
\end{posledica}

\begin{definicija}
A complex number $\alpha$ is an
\emph{algebraic integer}\index{algebraic!integer} if there exists a
monic polynomial $f \in \Z[x]$ such that $f(\alpha) = 0$.
\end{definicija}

\begin{lema}
Let $f \in \Z[x]$ be monic and suppose that $f = gh$ for monic
polynomials $g, h \in \Q[x]$. Then $g, h \in \Z[x]$.
\end{lema}

\begin{proof}
Let $d, e \in \N$ be minimal integers such that $dg, eh \in \Z[x]$.
Note that the coefficients of $dg$ (and similarly $eh$) are
coprime. Suppose that $p \mid de$ for some $p \in \P$. It follows
that $p \mid def = dg eh$. In particular, the ring
$\kvoc{\Z[x]}{p \Z[x]}$ has a zero divisor, which is impossible, as
$\kvoc{\Z[x]}{p \Z[x]} \cong Z_p[x]$ is an integral domain.
\end{proof}

\begin{lema}
A complex number $\alpha$ is an algebraic integer if and only if
$m_\alpha$ has integer coefficients.
\end{lema}

\obvs

\begin{trditev}
Let $K$ be a number field and $\alpha \in K$. Then the following
statements are equivalent:

\begin{enumerate}[i)]
\item The number $\alpha$ is an algebraic integer.
\item The group $(\Z[\alpha], +)$ is finitely generated.
\item There exists a subring $R \subseteq K$ such that
$\alpha \in R$ and the group $(R, +)$ is finitely generated.
\item There exists a finitely generated subgroup
$(A, +) \subseteq (K, +)$ such that $A \ne 0$ and
$\alpha A \subseteq A$.
\end{enumerate}
\end{trditev}

\begin{proof}
Note that we only need to prove that the last statement implies the
first one. Write $A = \skl{\beta_i \mid i \leq n}$. We can
therefore write
\[
\alpha \beta = C \beta,
\]
where
\[
\beta =
\begin{bmatrix}
\beta_1 \\ \vdots \\ \beta_n
\end{bmatrix}
\]
and $C$ is some matrix with integer coefficients. In particular,
$\alpha$ is an eigenvalue of $C$, which means it is a root of
$\det(C - I \alpha)$, which is a polynomial with integer
coefficients.
\end{proof}

\begin{posledica}
Let $K$ be a number field. Then
\[
\mathcal{O}_K =
\setb{\alpha \in K}{\text{$\alpha$ is an algebraic integer}}
\]
is a subring of $K$.
\end{posledica}

\begin{proof}
Suppose that $\alpha, \beta \in \mathcal{O}_K$, that is,
$\Z[\alpha]$ and $\Z[\beta]$ are finitely generated. Then
$\Z[\alpha, \beta]$ is finitely generated as well. As both
$\alpha + \beta$ and $\alpha \cdot \beta$ are elements of this
subring, both are elements of $\mathcal{O}_K$.
\end{proof}

\begin{definicija}
With the notation of the above corollary, we call $\mathcal{O}_K$
the \emph{ring of integers}\index{ring of integers} in $K$.
\end{definicija}

\begin{trditev}
Let $K = \Q \br{\sqrt{d}}$, where $d \in \Z$ is a square-free
integer.

\begin{enumerate}[i)]
\item If $d \equiv 2 \pmod{4}$ or $d \equiv 3 \pmod{4}$, then
$\mathcal{O}_K = \Z \left[\sqrt{d}\right]$.
\item If $d \equiv 1 \pmod{4}$, then
$\mathcal{O}_K = \Z \left[\frac{1+\sqrt{d}}{2}\right]$.
\end{enumerate}
\end{trditev}

\begin{proof}
Let $\alpha = \frac{a + b \sqrt{d}}{2}$ for $a, b \in \Q$. Clearly,
$\Q \cap \mathcal{O}_K = \Z$. Suppose therefore that $b \ne 0$ and
set $\alpha' = \frac{a - b \sqrt{d}}{2}$ and note that
\[
m_\alpha =
(x - \alpha) \cdot (x + \alpha) =
x^2 - ax + \frac{a^2 - db^2}{4}.
\]
It follows that $\alpha \in \mathcal{O}_K$ if and only if
$a \in \Z$ and $a^2 - db^2 \in 4\Z$. in particular, $db^2 \in \Z$
and hence $b \in \Z$, as $d$ is square-free.

\begin{enumerate}[i)]
\item Considering $a^2 - d b^2 \bmod 4$, we see that both $a$ and
$b$ must be even, which gives
$\alpha \in \Z \left[\sqrt{d}\right]$.
\item The same equation modulo $4$ now gives us
$a \equiv b \pmod{2}$. A direct calculation now shows that
$\mathcal{O}_K = \Z \left[\frac{1+\sqrt{d}}{2}\right]$. \qedhere
\end{enumerate}
\end{proof}

\begin{opomba}
All quadratic number fields are of this form.
\end{opomba}

\begin{definicija}
Let $\omega_n$ be a primitive $n$-th root of
unity\index{root of unity}. The $n$-th
\emph{cyclotomic field}\index{cyclitomic field} is the field
$\Q(\omega_n)$. We denote by $\mu_n(\C)$ the $n$-th roots of unity
and by $\mu_n^*(\C)$ the primitive ones.
\end{definicija}

\begin{opomba}
For odd $n$, we have $\Q(\omega_n) = \Q(\omega_{2n})$.
\end{opomba}

\begin{trditev}
Let $\omega \in \mu_n^*(\C)$. If $k \in \N$ is coprime with $n$,
then $\omega$ and $\omega^k$ are algebraic conjugates.
\end{trditev}

\begin{proof}
As algebraic conjugation is an equivalence relation, it suffices to
prove the proposition for $k = p \in \P$. Let $f = x^n - 1$ and
write $f = g m_\omega$. Suppose that $g(\omega^p) = 0$. Then
$\omega$ is a root of $g(x^p)$, therefore it is divisible by
$m_\omega$ in $\Z[x]$. Let $\oline{g}$ be the projection of $g$ in
$\kvoc{\Z[x]}{p \Z[x]} \cong \Z_p[x]$. As
$\oline{g}(x^p) = \oline{g}(x)^p$, we find that
$\oline{m}_\alpha \mid \oline{g}(x)^p$. In particular,
$\oline{m}_\alpha$ and $\oline{g}$ share a common factor
$\oline{h} \in \Z_p[x]$. But then
$\oline{f} = \oline{g} \cdot \oline{m}_\alpha$ is divisible by
$\oline{h}^2$, therefore $\oline{f}$ and $\oline{f}'$ share a
common factor. As $p \nmid n$,
$\oline{f}' = n \cdot X^{n-1} \ne 0$, which is clearly coprime to
$\oline{f}$.
\end{proof}

\begin{definicija}
The $n$-th
\emph{cyclotomic polynomial}\index{cyclotomic polynomial} is the
polynomial
\[
\Phi_n = \prod_{\omega \in \mu_n^*(\C)} (x - \omega).
\]
\end{definicija}

\begin{opomba}
The polynomial $\Phi_n$ is irreducible by the previous proposition.
We have $\deg \Phi_n = \varphi(n)$.
\end{opomba}

\begin{trditev}
Let $\omega \in \mu_n^*(\C)$. Then
$[\Q(\omega) : \Q] = \varphi(n)$. Furthermore, the map
$\br{\kvoc{\Z}{n \Z}}^\times \to \Gal \br{\kvoc{\Q(\omega)}{\Q}}$
given by $i \mapsto (\omega \mapsto \omega^i)$ is an isomorphism.
In particular, $\kvoc{\Q(\omega)}{\Q}$ is Galois.
\end{trditev}

\begin{proof}
Note that $[\Q(\omega) : \Q] = \deg \Phi_n = \varphi(n)$. The
described map is obviously a bijective homomorphism.
\end{proof}

\begin{posledica}
\label{alg_int:cor:root_mult}
Let $\omega \in \mu_n^*(\C)$ Then the roots of unity in
$\Q(\omega)$ are precisely $\mu_n(\C)$ is $n$ is even and
$\mu_{2n}(\C)$ if $n$ is odd.
\end{posledica}

\begin{proof}
It is enough to consider even $n$. Suppose that
$\lambda \in \Q(\omega)$ is a primitive $k$-th root of unity for
$k \nmid n$. We can assume that $\gcd(k, n) = 1$ by replacing
$\lambda$ with $\lambda^{\gcd(k,n)}$. We now claim that
$\lambda \omega$ is a primitive $kn$-th root of unity. Indeed, if
$(\lambda \omega)^m = 1$, then $\omega^{km} = 1$ and
$\lambda^{nm} = 1$, hence $n \mid km$ and $k \mid nm$. As $k$ and
$n$ were chosen to be coprime, we find that $nk \mid m$. It follows
that $\Q \subseteq \Q(\omega_{kn}) \subseteq \Q(\omega)$, which is
impossible by considering the degrees over $\Q$, as
$\varphi(kn) \mid \varphi(n)$ implies $k \in \set{1, 2}$.
\end{proof}

\begin{posledica}
There is a bijection between $2 \N$ and cyclotomic fields, given by
$m \mapsto \Q \br{e^{\frac{2 \pi i}{m}}}$.
\end{posledica}

\newpage

\subsection{Trace, norm and discriminant}

\begin{definicija}
Let $\Q \subseteq K \subseteq L$ be number fields. We define
\[
\Hom_K(L, \C) =
\setb{\sigma \colon L \to \C}{\eval{\sigma}{K}{} = \id}.
\]
\end{definicija}

\begin{opomba}
Every $\varphi \in \Hom_\Q(K, \C)$ has precisely $[L : K]$ distinct
extensions in $\Hom_\Q(L, \C)$.
\end{opomba}

\begin{definicija}
Let $K \subseteq L$ be number fields,
$\Hom_K(L, \C) = \setb{\sigma_i}{i \leq n}$ and $\alpha \in L$. The
\emph{relative trace}\index{trace} and
\emph{relative norm}\index{norm} of $\alpha$ are defined as
\[
T_K^L(\alpha) = \sum_{i=1}^n \sigma_i(\alpha)
\quad \text{and} \quad
N_K^L(\alpha) = \prod_{i=1}^n \sigma_i(\alpha).
\]
If $K = \Q$, we omit the subscript.
\end{definicija}

\begin{trditev}
The trace is a linear map and the norm is multiplicative.
\end{trditev}

\obvs

\datum{2024-3-22}

\begin{trditev}
Let $K \subseteq L$ be a number field with $[L : K] = n$. Let
$\alpha \in L$ and set
\[
f = x^d + \sum_{k=0}^{d-1} a_k x^k
\]
to be the minimal polynomial of $\alpha$. Then
$T(\alpha) = -\frac{n}{d} a_{d-1}$ and
$N(\alpha) = (-1)^n a_0^{\frac{n}{d}}$. In particular,
$N(\alpha), T(\alpha) \in K$.
\end{trditev}

\begin{proof}
Let $K' = K(\alpha) \subseteq L$. Then $[K' : K] = d$ and
$n = d \cdot [L : K']$.  We can factor $f$ as
\[
f = \prod_{\sigma \in \Hom_K(K', \C)} (x - \sigma(a)).
\]
As each $\sigma \in \Hom_K(K', \C)$ extends to exactly
$\frac{n}{d}$ elements of $\Hom_K(L, \C)$, the proposition follows
from Vieta's formulae.
\end{proof}

\begin{opomba}
If $\alpha \in \mathcal{O}_L$, then
$N(\alpha), T(\alpha) \in \mathcal{O}_K$.
\end{opomba}

\begin{lema}
Let $K \subseteq L \subseteq M$ be number fields. Then
\[
N_K^M = N_K^L \circ N_L^M
\quad \text{and} \quad
T_K^M = T_K^L \circ T_L^M.
\]
\end{lema}

\begin{proof}
Take an element $\alpha \in M$. We now define an equivalence
relation on $\Hom_K(M, \C)$ as
$\sigma \sim \sigma' \iff
\eval{\sigma}{L}{} = \eval{\sigma'}{L}{}$. Note that there are
precisely $m = [L : K]$ equivalence classes. Let
$\sigma_i \in \Hom_K(M, \C)$ be the representatives of the
equivalence classes. Now denote
$G_i = \Hom_{\sigma_i(L)}(\sigma_i(M), \C)$ and compute
\[
T_K^M(\alpha) =
\sum_{i=1}^m \br{\sum_{\sigma \sim \sigma_i} \sigma(\alpha)} =
\sum_{i=1}^m \br{\sum_{\sigma \in G_i} \sigma(\sigma_i(\alpha))} =
\sum_{i=1}^m T_{\sigma_i(L)}^{\sigma_i(M)} \br{\sigma_i(\alpha)}.
\]
Now note that
$\sigma_i \br{T_L^M(\alpha)} =
T_{\sigma_i(L)}^{\sigma_i(M)} \br{\sigma_i(\alpha)}$, hence
\[
T_K^M(\alpha) =
\sum_{i=1}^m \sigma_i \br{T_L^M(\alpha)} =
T_K^L \circ T_L^M(\alpha).
\]
The proof for the norm is analogous.
\end{proof}

\begin{opomba}
For $K \subseteq L$ and $\alpha \in L$, the map
$\varphi_a \colon L \to L$ given by $x \mapsto \alpha x$ is
$K$-linear. The norm and trace of $\alpha$ coincide with the
determinant and trace of this map.
\end{opomba}

\begin{lema}
Let $\alpha \in \mathcal{O}_K$. Then $\alpha$ is invertible
if and only if $N_\Q^K(\alpha) = \pm 1$.
\end{lema}

\begin{proof}
If $\alpha$ is invertible, then clearly $N_\Q^K(\alpha) = \pm 1$,
as the norm is multiplicative. Now suppose that
$N_\Q^K(\alpha) = \pm 1$ and let $d = \deg m_\alpha$ be the degree
of the minimal polynomial
\[
m_\alpha = x^d + \sum_{k=0}^{d-1} a_k x^k
\]
of $\alpha$. By our assumption, $a_0 = \pm 1$, therefore
\[
1 = \pm \alpha \cdot \sum_{k=1}^d a_k x^{k-1}. \qedhere
\]
\end{proof}

\begin{opomba}
If $N_\Q^K(\alpha) \in \P$, then $\alpha$ is irreducible.
\end{opomba}

\begin{definicija}
Let $K$ be a number field. Suppose that $[K : \Q] = n$ and denote
$\Hom_\Q(K, \C) = \setb{\sigma_i}{i \leq n}$. The
\emph{discriminant}\index{discriminant} of
$(\alpha_1, \dots, \alpha_n)$ is defined as
\[
\disc(\alpha_1, \dots, \alpha_n) =
\det
\begin{bmatrix}
\sigma_i(\alpha_j)
\end{bmatrix}_{i, j \leq n}^2.
\]
\end{definicija}

\begin{trditev}
The following statements hold:

\begin{enumerate}[i)]
\item For any $\alpha_i \in K$ we have
$\disc(\alpha_1, \dots, \alpha_n) =
\det
\begin{bmatrix}
T_\Q^K(\alpha_i \alpha_j)
\end{bmatrix}_{i,j}$.
\item For any $\alpha_i \in K$ we have
$\disc(\alpha_1, \dots, \alpha_n) \in \Q$. If
$\alpha_i \in \mathcal{O}_K$, then the discriminant is an integer.
\item If $\beta = A \alpha$ for some matrix $A \in M_n(\Q)$, then
\[
\disc(\beta_1, \dots, \beta_n) =
\det(A)^2 \cdot \disc(\alpha_1, \dots, \alpha_n).
\]
\end{enumerate}
\end{trditev}

\pagebreak[3]

\begin{proof}
\phantom{i}
\begin{enumerate}[i)]
\item Let
$C = \begin{bmatrix} \sigma_i(\alpha_j) \end{bmatrix}_{i,j}$. Then
\[
\disc(\alpha_1, \dots, \alpha_n) =
\det(C)^2 =
\det(C^\top C).
\]
Now note that
\[
(C^\top C)_{i,j} =
\sum_{k=1}^n \sigma_k(\alpha_i) \sigma_k(\alpha_j) =
T_\Q^K(\alpha_i \alpha_j).
\]
\item Follows from the previous statement.
\item Let $A = \begin{bmatrix} a_{i,j} \end{bmatrix}_{i,j}$. Then
\[
\sigma_i(\beta_j) = \sum_{k=1}^n a_{j,k} \sigma_i(a_k),
\]
hence
\[
\begin{bmatrix}
\sigma_i(\beta_j)
\end{bmatrix}_{i,j} =
\begin{bmatrix}
\sigma_i(\alpha_j)
\end{bmatrix}_{i,j} \cdot
A^\top. \qedhere
\]
\end{enumerate}
\end{proof}

\begin{trditev}
\label{alg_int:prop:disc_norm}
Let $K = \Q(\alpha)$ and $n = [K : \Q]$. Denote by
$\alpha_1, \dots, \alpha_n$ the algebraic conjugates of $\alpha$.
Then
\[
\disc \br{1, \alpha, \dots, \alpha^{n-1}} =
\prod_{i \ne j} (\alpha_j - \alpha_i)^2 =
(-1)^{\frac{n(n-1)}{2}} \cdot N_\Q^K(f'(\alpha)),
\]
where $f$ is the minimal polynomial of $\alpha$ over $\Q$.
\end{trditev}

\begin{proof}
Order $\alpha_i$ such that $\alpha = \alpha_1$ and
$\sigma_i(\alpha) = \alpha_i$. The first equality is now clear from
the Vandermonde determinant. Now note that
\[
f' =
\sum_{i=1}^n \prod_{j \ne i} (x - \alpha_j),
\]
therefore
\[
f'(\alpha_i) = \prod_{j \ne i} (\alpha_i - \alpha_j).
\]
A straightforward calculation now shows that
\[
N_\Q^K(f'(\alpha)) =
\prod_{i=1}^n \sigma_i(f'(\alpha)) =
\prod_{i=1}^n f'(\alpha_i) =
\prod_{i=1}^n \prod_{j \ne i} (\alpha_i - \alpha_j) =
(-1)^{\frac{n(n-1)}{2}} \cdot
\prod_{i \ne j} (\alpha_j - \alpha_i)^2. \qedhere
\]
\end{proof}

\begin{izrek}
Let $K$ be a number field with $n = [K : \Q]$. Elements
$\alpha_1, \dots, \alpha_n \in K$ form a $\Q$-basis of $K$ if and
only if
\[
\disc(\alpha_1, \dots, \alpha_n) \ne 0.
\]
\end{izrek}

\begin{proof}
Let $K = \Q(\beta)$. Then $(1, \beta, \dots, \beta^{n-1})$ form a
basis of $K$, so we can write $\alpha = A \beta$ for some matrix
$A \in M_n(\Q)$. As we have
\[
\disc(\alpha_1, \dots, \alpha_n) =
\det(A)^2 \cdot \disc(\beta_1, \dots, \beta_n)
\]
and $\disc(\beta_1, \dots, \beta_n) \ne 0$, the conclusion follows.
\end{proof}

\newpage

\subsection{Integral bases}

\begin{trditev}
Let $K$ be a number field. Then
\[
K =
\setb{\frac{\alpha}{d}}{d \in \N \land \alpha \in \mathcal{O}_K}.
\]
\end{trditev}

\begin{proof}
Take $\beta \in K$ and let
\[
f = \sum_{k=0}^n a_k x^k \in \Z[x]
\]
be a polynomial with $f(\beta) = 0$. Then, multiplying by
$a_n^{n-1}$, we find a monic polynomial with $a_n \beta$ as a root,
hence $a_n \beta \in \mathcal{O}_K$.
\end{proof}

\begin{definicija}
Let $K$ be a number field. An
\emph{integral basis}\index{integral basis} of $\mathcal{O}_K$ is
a $\Z$-module basis of $\mathcal{O}_K$.
\end{definicija}

\begin{izrek}[Structure]
\index{structure theorem}
\phantom{i}
\begin{enumerate}[i)]
\item If $M$ is a finitely generated $\Z$-module, then
$M = F \oplus T$ where $F$ is a finitely generated free $\Z$-module
and $T$ is finite.
\item Let $F$ be a finitely generated free $\Z$-module of rank $n$.
If $G \subseteq F$ is a submodule, then $G$ is also finitely
generated and free as a $\Z$-module with rank at most $n$.
Furthermore, there exists a basis $(b_1, \dots, b_n)$ of $F$ and
$d_1, \dots, d_m \in \N$ with $d_i \mid d_{i+1}$ such that
$(d_1 b_1, \dots, d_m b_m)$ is a basis of $G$.
\item Let $T$ be a finite abelian group. Then
\[
T = \bigoplus_{i=1}^r \Z_{n_i}.
\]
Furthermore, we can choose $n_i$ such that $n_i \mid n_{i+1}$ --
such choice of $n_i$ is unique.
\end{enumerate}
\end{izrek}

\begin{lema}
\label{alg_int:lm:disc_base}
Suppose that $(\alpha_1, \dots, \alpha_n)$ is a $\Q$-basis of $K$,
contained in $\mathcal{O}_K$, and denote
$d = \disc(\alpha_1, \dots, \alpha_n)$. Then
\[
\mathcal{O}_K \subseteq
\frac{1}{d} \bigoplus_{i=1}^n \Z \alpha_i.
\]
\end{lema}

\begin{proof}
Let $\beta \in \mathcal{O}_K$ and write
\[
\beta = \sum_{i=1}^n x_i \alpha_i.
\]
Now compute
\[
T_\Q^K(\alpha_i \beta) =
T_\Q^K \br{\sum_{j=1}^n x_j \alpha_i \alpha_j} =
\sum_{j=1}^n x_j T_\Q^K(\alpha_i \alpha_j),
\]
hence
\[
b =
\begin{bmatrix}
T(\alpha_1 \beta) \\
\vdots \\
T(\alpha_n \beta)
\end{bmatrix}
=
\underbrace{
\begin{bmatrix}
T_\Q^K(\alpha_i \alpha_j)
\end{bmatrix}_{i,j}}_C
\cdot
\begin{bmatrix}
x_1 \\
\vdots \\
x_n
\end{bmatrix}.
\]
As $\det C = d \ne 0$, we can write $x = C^{-1} b$. As
$\det C \cdot C^{-1} \in M_n(\Z)$, the conclusion follows.
\end{proof}

\begin{izrek}
The set $\mathcal{O}_K$ is a free $\Z$-module of rank
$n = [K : \Q]$. If $I \edn \mathcal{O}_K$ is a non-zero ideal, then
$I$ is a finitely generated free $\Z$-module of rank $n$. In
particular, $\mathcal{O}_K$ is a noetherian ring.
\end{izrek}

\begin{proof}
Let $(\alpha_1, \dots, \alpha_n)$ be a $\Q$-basis of $K$ contained
in $\mathcal{O}_K$ and set $d = \disc(\alpha_1, \dots, \alpha_n)$.
Then
\[
\bigoplus_{i=1}^n \Z \alpha_i \subseteq
\mathcal{O}_K \subseteq
\bigoplus_{i=1}^n \Z \frac{\alpha_i}{d}.
\]
By the structure theorem, $\mathcal{O}$ is finitely generated.
As it contains a submodule of rank $n$, it itself has rank $n$.

Let $I \edn \mathcal{O}_K$ be a non-zero ideal and
$\gamma \in I \setminus \set{0}$. As
$\gamma \mathcal{O}_K \subseteq I \subseteq \mathcal{O}_K$, we can
apply the same argument as above.
\end{proof}

\begin{opomba}
If $(\alpha_1, \dots, \alpha_n)$ and $(\beta_1, \dots, \beta_n)$
are two $\Z$-basis of $I$, then clearly
$\disc(\alpha_1, \dots, \alpha_n) =
\disc(\beta_1, \dots, \beta_n)$. We can therefore define
$\disc(I) = \disc(\alpha_1, \dots, \alpha_n)$.
\end{opomba}

\begin{opomba}
If $J \subseteq I$ are both finitely generated free $\Z$-modules,
each containing a $\Q$-basis of $K$, then
\[
\disc(J) = \abs{\kvoc{I}{J}}^2 \cdot \disc(I)
\]
by the structure theorem.
\end{opomba}

\begin{izrek}
Let $K$ be a number field and let $I \subseteq \mathcal{O}_K$ be a
finitely generated free $\Z$-module containing a $\Q$-basis
$(\alpha_1, \dots, \alpha_n)$ of $K$. Set
$d = \abs{\disc(\alpha_1, \dots, \alpha_n)}$ and write
$d = d_0^2 d_1$ with $d_1$ being square-free. For
$1 \leq i \leq n$, choose $c_{i,j} \in \Z$ and $c_{i,i} \in \N$
such that
\[
\beta_i = \frac{1}{d_0} \sum_{j=1}^i c_{i,j} \alpha_j \in I
\]
and $c_{i,i}$ are minimal. Then $(\beta_1, \dots, \beta_n)$ is a
$\Z$-basis of $I$.
\end{izrek}

\datum{2024-3-29}

\begin{proof}
Write
\[
J = \bigoplus_{i=1}^n \Z \alpha_i \subseteq
I \subseteq
\mathcal{O}_K.
\]
Note that $\disc(I)$ and $\disc(J)$ are both integers and
\[
d_0^2 \cdot d_1 = d = \disc(J) = [I : J]^2 \cdot \disc(I),
\]
and as $d_1$ is square-free, it follows that $[I : J] \mid d_0$,
therefore $d_0 I \subseteq J$. Note that
$(\beta_1, \dots, \beta_n)$ are $\Q$-linearly independent and
$\skl{\beta_i~\middle\vert~i \leq n}_\Z \subseteq I$. It therefore
suffices to show that
$I \subseteq \skl{\beta_i~\middle\vert i \leq n}_\Z$. Suppose
otherwise, and let
$\gamma \in I \setminus \skl{\beta_i~\middle\vert i \leq n}_\Z$. As
$\gamma \in \frac{1}{d_0} J$, we can write
\[
\gamma = \frac{1}{d_0} \sum_{i=1}^s x_i \alpha_i
\]
with $x_i \in \Z$ and $x_s \ne 0$. Choose $\gamma$ such that $s$ is
minimal, and among those, the one with minimal $\abs{x_s}$. Assume
further that $x_s > 0$. But then, as $x_s \geq c_{s,s}$ by choice
of $\beta_s$, we find that
$x_s - \beta_s \in \skl{\beta_i~\middle\vert i \leq n}_\Z$ by
minimality, which is a contradiction.
\end{proof}

\begin{posledica}
The ring $\mathcal{O}_K$ has an integral basis of the form
$\setb{\alpha_i}{i \leq n}$ with $\alpha_1 = 1$.
\end{posledica}

\begin{proof}
Apply the previous theorem to a $\Q$-basis of $\mathcal{O}_K$ of
the form $(1, \alpha_2', \dots, \alpha_n')$.
\end{proof}

\begin{opomba}
If $\alpha_1, \dots, \alpha_n \in \mathcal{O}_K$ are elements such
that $\disc(\alpha_1, \dots, \alpha_n)$ is square-free, they form
an integral basis.
\end{opomba}

\begin{definicija}
Let $K$ be a number field and $(\alpha_1, \dots, \alpha_n)$ an
integral basis of $\mathcal{O}_K$. We then define
\[
\disc(K) =
\disc(\mathcal{O}_K) =
\disc(\alpha_1, \dots, \alpha_n) \in \Z.
\]
\end{definicija}

\begin{opomba}
If $d$ is square-free and $K = \Q \br{\sqrt{d}}$, then
\[
\disc(K) =
\begin{cases}
 d, & d \equiv 1 \pmod{4},   \\
4d, & d \equiv 2, 3 \pmod{4}.
\end{cases}
\]
\end{opomba}

\newpage

\subsection{Integral bases of Cyclotomic fields}

\begin{lema}
\label{alg_int:lm:pp}
Suppose that $n = p^e$ with $p \in \P$ and $e \geq 1$. Choose
$\zeta \in \mu_n^*(\C)$ and set $K = \Q(\zeta)$.

\begin{enumerate}[i)]
\item We have
\[
N^K(1-\zeta) = \prod_{p \nmid j} (1-\zeta^j) = p.
\]
If $n \ne 2$, then $N^K(1-\zeta) = N^K(\zeta-1)$.
\item We have
\[
\left.(1-\zeta)^{\varphi(n)}~\middle\vert~p\right.
\]
in $\Z[\zeta]$.
\end{enumerate}
\end{lema}

\begin{proof}
\phantom{i}
\begin{enumerate}[i)]
\item Recall that
\[
\Hom_\Q(K, \C) =
\setb{\zeta \mapsto \zeta^j}{p \nmid j}.
\]
It follows that
\[
N^K(1-\zeta) =
\prod_{\sigma \in \Hom_\Q(K, \C)} (1 - \sigma(\zeta)) =
\prod_{p \nmid j} (1-\zeta^j).
\]
If $n \ne 2$, then $\varphi(n)$ is even and
$N^K(1-\zeta) = N^K(\zeta-1)$ follows. Now note that
\[
\Phi_{p^e}(x) =
\frac{x^{p^e}-1}{x^{p^{e-1}}-1} =
\sum_{j=0}^{p-1} x^{j p^{e-1}} =
\prod_{p \nmid j} (x - \zeta^j).
\]
Evaluating the expression at $x = 1$, we get $N^K(1-\zeta) = p$.
\item Note first that $1-\zeta \mid 1-\zeta^j$ for all $j \in \N$.
But then
\[
\left.(1-\zeta)^{\varphi(n)}~\middle\vert~
\prod_{p \nmid j} (1-\zeta^j)\right. =
p. \qedhere
\]
\end{enumerate}
\end{proof}

\begin{lema}
If $\zeta \in \mu_p^*(\C)$ for $p \in \P$, then
\[
\disc(1, \zeta, \dots, \zeta^{p-2}) =
\begin{cases}
 p^{p-2}, & p \equiv 1, 2 \pmod{4}, \\
-p^{p-2}, & p \equiv 3 \pmod{4}.
\end{cases}
\]
\end{lema}

\begin{proof}
Without loss of generality assume $p \ne 2$. Then
\[
m_\zeta = \Phi_p = \frac{x^p-1}{x-1} = \sum_{j=0}^{p-1} x^j.
\]
By proposition~\ref{alg_int:prop:disc_norm}, it holds that
\[
\disc(1, \zeta, \dots, \zeta^{p-2}) =
(-1)^{\frac{(p-1)(p-2)}{2}} \cdot N^K(\Phi_p'(\zeta)).
\]
As
\[
\Phi_p + (x-1) \Phi_p' = p \cdot x^{p-1},
\]
we get
\[
\Phi_p'(\zeta) =
\frac{p \cdot \zeta^{p-1}}{\zeta - 1},
\]
therefore
\[
N(\Phi_p'(\zeta)) =
\frac{N(p) \cdot N(\zeta^{-1})}{N(\zeta-1)} =
\frac{p^{p-1} \cdot 1}{p} =
p^{p-2}. \qedhere
\]
\end{proof}

\begin{lema}
Let $n \in \N$ and $\zeta \in \mu_n^*(\C)$. Then
\[
\left.\disc \br{1, \zeta, \dots, \zeta^{\varphi(n)-1}}
~\middle\vert~n^{\varphi(n)}\right..
\]
\end{lema}

\begin{proof}
Write
\[
x^n-1 = \Phi_n(x) \cdot g(x)
\]
for $g \in \Z[x]$. Then
$n x^{n-1} = \Phi_n'(x) \cdot g(x) + \Phi_n(x) \cdot g'(x)$,
therefore
\[
n \zeta^{n-1} = \Phi_n'(\zeta) \cdot g(\zeta).
\]
Taking the norm, we get
\[
n^{\varphi(n)} \cdot N(\zeta^{n-1}) =
N(\Phi_n'(\zeta)) \cdot  N(g(\zeta)),
\]
but as $N(g(\zeta)) \in \Z$ and $N(\zeta^{n-1}) = \pm 1$, the
conclusion follows.
\end{proof}

\begin{izrek}
\label{alg_int:thm:pp_basis}
Let $n = p^e$ for $p \in \P$ and $e \geq 1$. Choose
$\zeta \in \mu_n^*(\C)$ and set $K = \Q(\zeta)$. Then
\[
\mathcal{O}_K =
\Z[\zeta] =
\bigoplus_{j=0}^{\varphi(n)-1} \Z \zeta^j.
\]
\end{izrek}

\begin{proof}
Let $m = [K : \Q] = \varphi(n)$. By the previous lemma, we have
\[
\disc(1, \zeta, \dots, \zeta^{m-1}) = \pm p^t
\]
for some $t \geq 0$. By lemma~\ref{alg_int:lm:disc_base}, we see
that
\[
\mathcal{O}_K \subseteq
\frac{1}{p^t} \cdot \skl{(1-\zeta)^j~\middle\vert~j \leq m-1}_\Z,
\]
as $\Z[\zeta] = \Z[1-\zeta]$. Suppose that
$\Z[1-\zeta] \subset \mathcal{O}_K$. Then there exists some
\[
\alpha =
\frac{1}{p} \cdot \sum_{j=i}^{m-1} a_j (1-\zeta)^j \in
\mathcal{O}_K \setminus \Z[1-\zeta]
\]
with $0 \leq i \leq m-1$ and $a_j \in \Z$ with $p \nmid a_i$. By
lemma~\ref{alg_int:lm:pp}, we get $(1-\zeta)^{i+1} \mid p$,
therefore
\[
\frac{p \alpha}{(1-\zeta)^{i+1}} =
\frac{a_i}{1-\zeta} + \sum_{j=i+1}^{m-1} a_j (1-\zeta)^{j-i-1},
\]
and so $1-\zeta \mid a_i$. But then
$\pm p = N(1-\zeta) \mid N(a_i)$, which is impossible as we have
$N(a_i) = a_i^m$.
\end{proof}

\begin{lema}
Let $K$ and $L$ be number fields with $m = [K : \Q]$ and
$n = [L : \Q]$. Assume that $[KL : \Q] = mn$. Then for every pair
$\sigma \in \Hom_\Q(K, \C)$ and $\varphi \in \Hom_\Q(L, \C)$ there
exists a unique $\psi \in \Hom_\Q(KL, \C)$ such that
$\eval{\psi}{K}{} = \sigma$ and $\eval{\psi}{L}{} = \varphi$.
\end{lema}

\begin{proof}
Note that every $\sigma \in \Hom_\Q(K, \C)$ extends to
$\psi \in \Hom_\Q(KL, \C)$ in $[KL : K] = n$ distinct ways. The $n$
maps $\eval{\psi}{L}{}$ are then clearly distinct, hence one of
them is equal to $\varphi$. Uniqueness is obvious.
\end{proof}

\begin{izrek}
Let $K$ and $L$ be number fields with $m = [K : \Q]$ and
$n = [L : \Q]$. Suppose that $(\alpha_1, \dots, \alpha_m)$ and
$(\beta_1, \dots, \beta_n)$ are integral basis of $\mathcal{O}_K$
and $\mathcal{O}_L$ respectively. If $[KL : \Q] = mn$ and
$\gcd(\disc(K), \disc(L)) = 1$, then
\[
(\alpha_i \beta_j \mid i \leq m \land j \leq n)
\]
is an integral basis for $\mathcal{O}_{KL}$. Furthermore,
\[
\disc(KL) = \disc(K)^n \cdot \disc(L)^m.
\]
\end{izrek}

\begin{proof}
Let $\gamma \in \mathcal{O}_{KL}$ and write
\[
\gamma = \sum_{i,j} c_{i,j} \alpha_i \beta_j
\]
with $c_{i,j} \in \Q$. This representation is unique, as
$\setb{\alpha_i \beta_j}{i \leq m \land j \leq n}$ is a $\Q$-basis
of $\mathcal{O}_{KL}$. Now write
\[
\xi_j = \sum_{i=1}^m c_{i,j} \alpha_i \in K.
\]
That gives us
\[
\gamma = \sum_{j=1}^n \beta_j \xi_j.
\]
Let $\Hom_K(KL, \C) = \setb{\varphi_i}{i \leq n}$. Applying
$\varphi_i$ to the above equation, we get
\[
b =
\begin{bmatrix}
\varphi_1(\gamma) \\
\varphi_2(\gamma) \\
\vdots \\
\varphi_n(\gamma)
\end{bmatrix}
=
\underbrace{\begin{bmatrix}
\varphi_1(\beta_1) & \varphi_1(\beta_2) &
\dots & \varphi_1(\beta_n) \\
\varphi_2(\beta_1) & \varphi_2(\beta_2) &
\dots & \varphi_2(\beta_n) \\
\vdots & \vdots & \ddots & \vdots \\
\varphi_n(\beta_1) & \varphi_n(\beta_2) &
\dots & \varphi_n(\beta_n)
\end{bmatrix}}_B
\cdot
\begin{bmatrix}
\xi_1 \\ \xi_2 \\ \vdots \\ \xi_n
\end{bmatrix}.
\]
Let $d = \disc(L) = \det(B)^2$. Then
\[
d \xi =
d B^{-1} b =
d \cdot \frac{\adj(B)}{\det(B)} \cdot b =
\det(B) \cdot \adj(B) \cdot b.
\]
It follows that $d \xi_j$ are algebraic integers, therefore
$d \cdot c_{i,j} \in \Z$ for all $i$ and $j$. By symmetry, the same
holds for $d' = \disc(K)$. As $\gcd(d, d') = 1$, we get
$c_{i,j} \in \Z$.

Let now $\Hom_L(KL, \C) = \setb{\sigma_j}{j \leq m}$ and denote
by $\psi_{i,j}$ the element of $\Hom_Q(KL, \C)$ with
$\eval{\psi_{i,j}}{K}{} = \sigma_i$ and
$\eval{\psi_{i,j}}{L}{} = \varphi_j$. Denote by $A$ the
$(mn) \times (mn)$ matrix with
\[
A =
\begin{bmatrix}
\psi_{i,j}(\alpha_s \beta_t)
\end{bmatrix}_{\substack{i,s \leq m \\ j,t \leq n}}
=
\begin{bmatrix}
B & 0 & \dots & 0 \\
0 & B & \dots & 0 \\
\vdots & \vdots & \ddots & \vdots \\
0 & 0 & \dots & B
\end{bmatrix}
\cdot
\begin{bmatrix}
\sigma_1(\alpha_1) I_n & \sigma_1(\alpha_2) I_n &
\dots & \sigma_1(\alpha_m) I_n \\
\sigma_2(\alpha_1) I_n & \sigma_2(\alpha_2) I_n &
\dots & \sigma_2(\alpha_m) I_n \\
\vdots & \vdots & \ddots & \vdots \\
\sigma_m(\alpha_1) I_n & \sigma_m(\alpha_2) I_n &
\dots & \sigma_m(\alpha_m) I_n
\end{bmatrix}.
\]
Reindexing, we find that
\[
\det \begin{bmatrix}
\sigma_1(\alpha_1) I_n & \sigma_1(\alpha_2) I_n &
\dots & \sigma_1(\alpha_m) I_n \\
\sigma_2(\alpha_1) I_n & \sigma_2(\alpha_2) I_n &
\dots & \sigma_2(\alpha_m) I_n \\
\vdots & \vdots & \ddots & \vdots \\
\sigma_m(\alpha_1) I_n & \sigma_m(\alpha_2) I_n &
\dots & \sigma_m(\alpha_m) I_n
\end{bmatrix}
=
\det \begin{bmatrix}
C & 0 & \dots & 0 \\
0 & C & \dots & 0 \\
\vdots & \vdots & \ddots & \vdots \\
0 & 0 & \dots & C
\end{bmatrix},
\]
where
\[
C =
\begin{bmatrix}
\sigma_1(\alpha_1) & \sigma_1(\alpha_2) &
\dots & \sigma_1(\alpha_m) \\
\sigma_2(\alpha_1) & \sigma_2(\alpha_2) &
\dots & \sigma_2(\alpha_m) \\
\vdots & \vdots & \ddots & \vdots \\
\sigma_1(\alpha_1) & \sigma_1(\alpha_2) &
\dots & \sigma_1(\alpha_m)
\end{bmatrix}.
\]
It follows that
\[
\disc(KL) =
\det(A)^2 =
\br{\det(B)^m \cdot \det(C)^n}^2 =
\disc(L)^m \cdot \disc(K)^n. \qedhere
\]
\end{proof}

\begin{izrek}
Let $n \geq 1$ and $\zeta \in \mu_n^*(\C)$. Denote $K = \Q(\zeta)$.
Then $(1, \zeta, \dots, \zeta^{\varphi(n)-1})$ is an integral
basis of $\mathcal{O}_K$.
\end{izrek}

\begin{proof}
We prove the theorem by induction on the number of distinct prime
factors of $n$. The claim clearly holds for $n=1$ and prime powers
by theorem~\ref{alg_int:thm:pp_basis}. Now write $n=st$ for
$s, t < n$ with $\gcd(s,t) = 1$. Choose $\zeta_s \in \mu_s^*(\C)$
and $\zeta_t \in \mu_t^*(\C)$. By the proof of
corollary~\ref{alg_int:cor:root_mult}, $\zeta_s \cdot \zeta_t$ is
a primitive $st$-th root of unity, therefore
$\Q(\zeta_n) = \Q(\zeta_s) \Q(\zeta_t)$. We therefore get
\[
[\Q(\zeta_n) : \Q] =
\varphi(n) =
\varphi(s) \cdot \varphi(t) =
[\Q(\zeta_s) : \Q] \cdot [\Q(\zeta_t) : \Q].
\]
By the induction hypothesis,
\[
\disc(\Q(\zeta_s)) = \disc(\Z[\zeta_s]) \mid s^{\varphi(s)}
\]
and similarly for $t$. In particular,
$\gcd(\disc(\Q(\zeta_s), \Q(\zeta_t)) = 1$, therefore
\[
\mathcal{O}_K = \Z[\zeta_s, \zeta_t] = \Z[\zeta_n]
\]
by the previous theorem.
\end{proof}

\begin{opomba}
We can in fact show that
\[
\disc(K) =
(-1)^{\frac{\varphi(n)}{2}} \cdot n^{\varphi(n)} \cdot
\prod_{\substack{p \in \P \\ p \mid n}}
p^{-\frac{\varphi(n)}{p-1}}.
\]
\end{opomba}

\begin{izrek}[Stickelberger]
\index{Stickelberger's discriminant theorem}
Let $K$ be a number field. Then $\disc(K) \equiv 0, 1 \pmod{4}$.
\end{izrek}

\begin{proof}
Let $L$ be the Galois closure of $K$, that is the smallest field
$L$ containing $K$ such that
$\Hom_\Q(L, \C) = \Gal \br{\kvoc{L}{\Q}}$. Denote $n = [K : \Q]$
and choose
$\setb{\sigma_i}{i \leq n} \subseteq \Gal \br{\kvoc{L}{\Q}}$ to be
extensions of elements of $\Hom_\Q(K, \C)$. Furthermore, let
$(\alpha_1, \dots, \alpha_n)$ be an integral basis of
$\mathcal{O}_K$. Denote
\[
C =
\begin{bmatrix}
\sigma_1(\alpha_1) & \sigma_1(\alpha_2) &
\dots & \sigma_1(\alpha_n) \\
\sigma_2(\alpha_1) & \sigma_2(\alpha_2) &
\dots & \sigma_2(\alpha_n) \\
\vdots & \vdots & \ddots & \vdots \\
\sigma_n(\alpha_1) & \sigma_n(\alpha_2) &
\dots & \sigma_n(\alpha_n)
\end{bmatrix}.
\]
Then $\disc(K) = \det(C)^2$. Write
\[
P =
\sum_{\substack{\pi \in S_n \\ \sgn(\pi) = 1}}
\prod_{i=1}^n \sigma_{\pi(i)}(\alpha_i)
\quad \text{and} \quad
N =
\sum_{\substack{\pi \in S_n \\ \sgn(\pi) = -1}}
\prod_{i=1}^n \sigma_{\pi(i)}(\alpha_i).
\]
As $\det(C) = P-N$, we get
\[
\disc(K) = (P-N)^2 = (P+N)^2 - 4 PN.
\]
It is clear that both $P+N$ and $PN$ are elements of
$\mathcal{O}_L$. For all $\varphi \in \Gal \br{\kvoc{L}{\Q}}$ we
have $\eval{\varphi \circ \sigma_i}{K}{} \in \Hom_\Q(K, \C)$,
therefore there exists a permutation $\tau \in S_n$ such that
$\eval{\varphi \circ \sigma_i}{K}{} =
\eval{\sigma_{\tau(i)}}{K}{}$ for all $i$. As
$\sgn(\tau \circ \pi) = \sgn(\tau) \cdot \sgn(\pi)$, we get
\[
\varphi(P) =
\sum_{\substack{\pi \in S_n \\ \sgn(\pi)=1}}
\prod_{i=1}^n \varphi(\sigma_{\pi(i)}(\alpha_i)) =
\sum_{\substack{\pi \in S_n \\ \sgn(\pi) = \sgn(\tau)}}
\prod_{i=1}^n \sigma_{\pi(i)}(\alpha_i) =
\begin{cases}
P, & \sgn(\tau) = 1,  \\
N, & \sgn(\tau) = -1.
\end{cases}
\]
We get a similar condition on $\varphi(N)$. It follows that
$\varphi(P+N) = P+N$ and $\varphi(P \cdot N) = P \cdot N$.
Therefore $P+N$ and $P \cdot N$ are both integers and hence
\[
\disc(K) \equiv (P+N)^2 \equiv 0, 1 \pmod{4}. \qedhere
\]
\end{proof}

\begin{opomba}
The Galois closure $L$ of $K$ if given by
\[
L = \prod_{\sigma \in \Hom_\Q(K, \C)} \sigma(K).
\]
\end{opomba}
