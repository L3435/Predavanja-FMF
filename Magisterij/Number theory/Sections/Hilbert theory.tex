\section{Hilbert theory}

\epigraph{The rest of this theorem becomes a sudoku with these
numbers.}
{-- gost.~izr.~prof.~dr.~rer.~nat.~Daniel Smertnig}

\subsection{Decomposition of primes in Galois extensions}

\begin{trditev}
Let $p \in \P$ and $n \geq 1$.

\begin{enumerate}[i)]
\item The map $\varphi \colon \F_{p^n} \to \F_{p^n}$, given by
$x \mapsto x^p$, is a field automorphism.\footnote{This is the
Frobenius automorphism\index{Frobenius automorphism}.}
\item The group $\Gal \br{\kvoc{\F_{p^n}}{\F_p}}$ is generated by
$\varphi$, which is of order $n$.
\item We have $m \mid n$ if and only if we can embed $\F_{p^m}$
into $\F_{p^n}$.
\item Every extension $\kvoc{\F_{p^n}}{\F_{p^m}}$ is a cyclic
Galois group generated by $\varphi^m$.
\end{enumerate}
\end{trditev}

\begin{proof}
\phantom{i}
\begin{enumerate}[i)]
\item Note that $(x+y)^p = x^p + y^p$, therefore $\varphi$ is
additive and injective.
\item Recall that $\abs{\Gal \br{\kvoc{\F_{p^n}}{\F_p}}} \leq n$,
hence we only need to show that $\varphi$ is of order $n$, which is
clear by considering the generator of $\F_{p^n}^*$.
\item By the Galois correspondence, the subfields of $\F_{p^n}$ are
precisely $\F_{p^n}^{\skl{\varphi^d}}$ for $d \mid n$.
\item Note that $\F_{p^m} = \F_{p^n}^{\skl{\varphi^m}}$, hence
$\kvoc{\F_{p^n}}{\F_{p^m}}$ is Galois with
$\Gal \br{\kvoc{\F_{p^n}}{\F_{p^m}}} = \skl{\varphi^m}$. \qedhere
\end{enumerate}
\end{proof}

\begin{lema}
Let $K \subseteq L$ be number fields and suppose that $\kvoc{L}{K}$
is Galois. Let $\mathfrak{p} \in \mathcal{P}(\mathcal{O}_K)$. Then
$G = \Gal \br{\kvoc{L}{K}}$ acts transitively on
$\setb{\mathfrak{P} \in \mathcal{P}(\mathcal{O}_L}
{\mathfrak{P} \mid \mathfrak{p}}$.
\end{lema}

\begin{proof}
Suppose that
$\mathfrak{P} \cap \mathcal{O}_K =
\mathfrak{P}' \cap \mathcal{O}_K =
\mathfrak{p}$ and that $\mathfrak{P}'$ is not in the orbit of
$\mathfrak{P}$. In particular, $\mathfrak{P}'$ is comaximal to each
$\sigma(\mathfrak{P})$. By the Chinese remainder theorem, there
exists some $\alpha \in \mathcal{O}_L$ such that
$\alpha \equiv 0 \pmod{\mathfrak{P}'}$ and
$\alpha \equiv 1 \pmod{\sigma(\mathfrak{P})}$ for all
$\sigma \in G$. But then
\[
N_K^L(\alpha) = \prod_{\sigma \in G} \sigma(\alpha) \in
\mathfrak{P}' \cap \mathcal{O}_K =
\mathfrak{p}
\]
and $\sigma(\alpha) \not \in \mathfrak{P}$ for all $\sigma$. As
$\mathfrak{P}$ is prime, it follows that
$N_K^L(\alpha) \not \in \mathfrak{P}$, which is a contradiction.
\end{proof}

\begin{trditev}
Let $\mathfrak{p} \in \mathcal{P}(\mathcal{O}_K)$. Furthermore, let
$\mathfrak{P}$ and $\mathfrak{P'}$ be prime ideals of
$\mathcal{O}_L$ with
$\mathfrak{P}, \mathfrak{P}' \mid \mathfrak{p}$.

\begin{enumerate}[i)]
\item We have
$e(\mathfrak{P} \mid \mathfrak{p}) =
e(\mathfrak{P}' \mid \mathfrak{p})$.
\item We have
$\kvoc{\mathcal{O}_L}{\mathfrak{P}} \cong
\kvoc{\mathcal{O}_L}{\mathfrak{P}'}$ as
$\kvoc{\mathcal{O}_K}{\mathfrak{p}}$-algebras. In particular,
$f(\mathfrak{P} \mid \mathfrak{p}) =
f(\mathfrak{P}' \mid \mathfrak{p})$.
\end{enumerate}
\end{trditev}

\pagebreak

\begin{proof}
\phantom{i}
\begin{enumerate}[i)]
\item Let
\[
\mathfrak{p} \mathcal{O}_L = \prod_{i=1}^r \mathfrak{P}_i^{e_i}
\]
and denote $\mathfrak{P} = \mathfrak{P}_1$. Let $\sigma$ be an
automorphism such that $\sigma(\mathfrak{P}) = \mathfrak{P}'$. Then
\[
\mathfrak{p} \mathcal{O}_L =
\sigma(\mathfrak{p} \mathcal{O}_L) =
\mathfrak{P}_i^{e_1} \prod_{i=2}^r \sigma(\mathfrak{P}_i)^{e_i}.
\]
\item Note that $\sigma$ induces a homomorphism
$\mathcal{O}_L \to \kvoc{\mathcal{O}_L}{\sigma(\mathfrak{P}}$ by
$\alpha \mapsto \sigma(\alpha) + \sigma(\mathfrak{P})$. As its
kernel is $\mathfrak{P}$, we get
$\kvoc{\mathcal{O}_L}{\mathfrak{P}} \cong
\kvoc{\mathcal{O}_L}{\mathfrak{P}'}$.  \qedhere
\end{enumerate}
\end{proof}

\begin{definicija}
Let $\mathfrak{p} \in \mathcal{P}(\mathcal{O}_K)$ and
$\mathfrak{P} \in \mathcal{P}(\mathcal{O}_L)$ be such that
$\mathfrak{P} \mid \mathfrak{p}$. Then
\[
D(\mathfrak{P}) =
\setb{\sigma \in G}{\sigma(\mathfrak{P}) = \mathfrak{P}}
\]
is the \emph{decomposition group}\index{decomposition group, field}
of $\mathfrak{P}$. The fixed field $L^{D(\mathfrak{P})}$ is the
\emph{decomposition field} of $\mathfrak{P}$.
\end{definicija}

\begin{opomba}
As $G$ acts transitively, we have
$[G : D(\mathfrak{P})] = r = [L^{D(\mathfrak{P})} : K]$. In
particular, $\mathfrak{p}$ is non-split if and only if
$L^{D(\mathfrak{P})} = K$ and is completely split if and only if
$L^{D(\mathfrak{P})} = L$.
\end{opomba}

\begin{opomba}
Every $\sigma \in D(\mathfrak{P})$ induces an automorphism
$\oline{\sigma}$ of $\kvoc{\mathcal{O}_L}{\mathfrak{P}}$ by
$\alpha + \mathfrak{P} \mapsto \sigma(\alpha) + \mathfrak{P}$.
\end{opomba}

\begin{opomba}
Denote $\kappa(\mathfrak{P}) = \kvoc{\mathcal{O}_L}{\mathfrak{P}}$
and $\kappa(\mathfrak{p}) = \kvoc{\mathcal{O}_K}{\mathfrak{p}}$.
Then $\oline{\sigma} \in
\Gal \br{\kvoc{\kappa(\mathfrak{P})}{\kappa(\mathfrak{p})}}$.
Furthermore, $\sigma \mapsto \oline{\sigma}$ is a group
homomorphism.
\end{opomba}

\begin{trditev}
Let $\mathfrak{p} \in \mathcal{P}(\mathcal{O}_K)$ and
$\mathfrak{P} \in \mathcal{P}(\mathcal{O}_L)$ be such that
$\mathfrak{P} \mid \mathfrak{p}$. Then the monomorphism
$D(\mathfrak{P}) \to
\Gal \br{\kvoc{\kappa(\mathfrak{P})}{\kappa(\mathfrak{p})}}$ is
surjective.
\end{trditev}

\begin{proof}
Let $\alpha \in \mathcal{O}_L$ be such that
$\oline{\alpha} \in \kappa(\mathcal{P})$ is a primitive element of
the field extension
$\kvoc{\kappa(\mathfrak{P})}{\kappa(\mathfrak{p})}$. Let
$\oline{g} \in \kappa(\mathfrak{p})[x]$ and
$h \in \mathcal{O}_K[x]$ be the minimal polynomials of
$\oline{\alpha}$. It follows that $\oline{g} \mid \oline{h}$.

As $\kvoc{L}{K}$ is Galois, the polynomial $h$ splits into linear
factors, that is
\[
h = \prod_{\tau \in \Hom_K(K(\alpha), \C)} (x - \tau(\alpha)),
\]
and each $\tau$ extends to some $\sigma_i \in \Hom_K(L, \C) = G$.


Let $\tau \in
\Gal \br{\kvoc{\kappa(\mathfrak{P})}{\kappa(\mathfrak{p})}}$. Then
$\oline{g} \br{\tau {r{\oline{\alpha}}}} =
\tau \br{\oline{g} \br{\oline{\alpha}}} = 0$, hence
$\tau \br{\oline{\alpha}}$ is a root of $\oline{g}$ and
$\oline{h}$. Hence
$\tau \br{\oline{\alpha}} = \oline{\sigma_i(\alpha)}$ for some $i$
and therefore $\tau = \oline{\sigma}_i$.
\end{proof}

\begin{definicija}
Let $\mathfrak{p} \in \mathcal{P}(\mathcal{O}_K)$ and
$\mathfrak{P} \in \mathcal{P}(\mathcal{O}_L)$ be such that
$\mathfrak{P} \mid \mathfrak{p}$. The group
\[
I(\mathfrak{P}) =
\ker \br{D(\mathfrak{P}) \to
\Gal \br{\kvoc{\kappa(\mathfrak{P})}{\kappa(\mathfrak{p})}}} =
\setb{\sigma \in G}
{\forall \alpha \in \mathcal{O}_L \colon
\sigma(\alpha) - \alpha \in \mathfrak{P}}
\]
is the \emph{inertia group}\index{inertia group, field} of
$\mathfrak{P}$ and the fixed field $L^{I(\mathfrak{P})}$ is the
\emph{inertia field} of $\mathfrak{P}$.
\end{definicija}

\begin{izrek}
\label{hilb:thm:fer}
Let $\mathfrak{p} \in \mathcal{P}(\mathcal{O}_K)$ and
$\mathfrak{P} \in \mathcal{P}(\mathcal{O}_L)$ be such that
$\mathfrak{P} \mid \mathfrak{p}$. Denote as usual
$f = f(\mathfrak{P} \mid \mathfrak{p})$ and
$e = e(\mathfrak{P} \mid \mathfrak{p})$ and let
$r =
\abs{\setb{\mathfrak{P}' \in \mathcal{P}(\mathcal{O}_L)}
{\mathfrak{P}' \mid \mathfrak{p}}}$.
Finally, let
\[
\mathfrak{P}_I = \mathfrak{P} \cap L^{I(\mathfrak{P})}
\quad \text{and} \quad
\mathfrak{P}_D = \mathfrak{P} \cap L^{D(\mathfrak{P})}.
\]

\begin{enumerate}[i)]
\item The extension
$\kvoc{L^{I(\mathfrak{P})}}{L^{D(\mathfrak{P})}}$ is Galois with
\[
\Gal \br{\kvoc{L^{I(\mathfrak{P})}}{L^{D(\mathfrak{P})}}} \cong
\Gal \br{\kvoc{\kappa(\mathfrak{P})}{\kappa(\mathfrak{p})}}.
\]
Furthermore,
\[
\abs{I(\mathfrak{P})} =
\left[L : L^{I(\mathfrak{P})}\right] =
e
\quad \text{and} \quad
\abs{D(\mathfrak{P}) : I(\mathfrak{P})} =
\left[L^{I(\mathfrak{P})} : L^{D(\mathfrak{P})}\right] =
f.
\]
\item We have
$e(\mathfrak{P}_D \mid \mathfrak{p}) =
f(\mathfrak{P}_D \mid \mathfrak{p}) = 1$.
\item We have $e(\mathfrak{P}_I \mid \mathfrak{P}_D) = 1$ and
$f(\mathfrak{P}_I \mid \mathfrak{P}_D) = f$.
\item We have $e(\mathfrak{P} \mid \mathfrak{P}_I) = e$ and
$f(\mathfrak{P} \mid \mathfrak{P}_I) = 1$.
\end{enumerate}
\end{izrek}

\begin{proof}
\phantom{i}
\begin{enumerate}[i)]
\item By construction, $\kvoc{L}{L^{D(\mathfrak{P})}}$ is Galois
with $\Gal \br{\kvoc{L}{L^{D(\mathfrak{P})}}} = D(\mathfrak{P})$.
As $I(\mathfrak{P}) \edn D(\mathfrak{P})$, the extension in
question is indeed Galois and
\[
\Gal \br{\kvoc{L^{I(\mathfrak{P})}}{L^{D(\mathfrak{P})}}} \cong
\kvoc{D(\mathfrak{P})}{I(\mathfrak{P})} \cong
\Gal \br{\kvoc{\kappa(\mathfrak{P})}{\kappa(\mathfrak{p})}}.
\]
Recall that $[L : K] = n = ref$. As
$\abs{G : D(\mathfrak{P})} = r$, we conclude
$\left[L : L^{D(\mathfrak{P})}\right] = ef$. But then
\[
\abs{D(\mathfrak{P}) : I(\mathfrak{P})} =
\abs{\Gal \br{\kvoc{\kappa(\mathfrak{P})}{\kappa(\mathfrak{p})}}} =
f
\]
and so $\abs{I(\mathfrak{P})} = e$.
\item First note that
\[
e =
e(\mathfrak{P} \mid \mathfrak{P}_I) \cdot
e(\mathfrak{P}_I \mid \mathfrak{P}_D) \cdot
e(\mathfrak{P}_D \mid \mathfrak{p})
\quad \text{and} \quad
f =
f(\mathfrak{P} \mid \mathfrak{P}_I) \cdot
f(\mathfrak{P}_I \mid \mathfrak{P}_D) \cdot
f(\mathfrak{P}_D \mid \mathfrak{p}).
\]
By construction, $\Gal \br{\kvoc{L}{L^{D(\mathfrak{P})}}}$ fixes
$\mathcal{P}$, but also acts transitively on prime ideals lying
over $\mathfrak{P}$. It follows that $\mathfrak{P}_D$ is non-split
in $L$. We deduce that
\[
ef =
\left[L : L^{D(\mathfrak{P})}\right] =
e(\mathfrak{P} \mid \mathfrak{P}_D) \cdot
f(\mathfrak{P} \mid \mathfrak{P}_D),
\]
therefore
$e(\mathfrak{P}_D \mid \mathfrak{p}) =
f(\mathfrak{P}_D \mid \mathfrak{p}) = 1$.
\item The inertia group of $\mathfrak{P}$ in
$\kvoc{L}{L^{D(\mathfrak{P})}}$ is $I(\mathfrak{P})$. But then
\[
f(\mathfrak{P}_I \mid \mathfrak{P}_D) =
\abs{\Gal \br{\kvoc{\kappa(\mathfrak{P})}{\kappa(\mathfrak{p})}}} =
\abs{D(\mathfrak{P}) : I(\mathfrak{P})} =
f.
\]
This also shows that $e(\mathfrak{P}_I \mid \mathfrak{P}_D) = 1$.
\item Evident from the previous two statements. \qedhere
\end{enumerate}
\end{proof}

\datum{2024-5-31}

\begin{lema}
Let $p$ be an odd prime and $\zeta \in \mu_p^*(\C)$. Then the
unique quadratic subfield of $\Q(\zeta)$ is
$\Q \br{\sqrt{p^*}}$, where $p^* = (-1)^{\frac{p-1}{2}} p$.
\end{lema}

\begin{proof}
The extension $\kvoc{\Q(\zeta)}{\Q}$ is Galois with cyclic Galois
group isomorphic to $\Z_{p-1}$. It therefore has a unique subgroup
of index $2$, which gives us the sought after field. Denote it by
$K = \Q \br{\sqrt{d}}$. As $p$ is the only ramified prime in
$\kvoc{\Q(\zeta)}{\Q}$, it is also ramified in $K$. It follows that
$p$ is the only prime number dividing $d$, but also note that
$2 \nmid p$, as $2$ is unramified. That also implies
$d \equiv 1 \pmod{4}$, therefore $d = (-1)^{\frac{p-1}{2}} p$, as
required.
\end{proof}

\begin{izrek}
Let $p$ be an odd prime, $\zeta \in \mu_p^*(\C)$ and
$p^* = (-1)^{\frac{p-1}{2}} p$. Then $q \in \P$ splits in
$\Q \br{\sqrt{p^*}}$ if and only if $q$ lies under an even number
of prime ideals in $\Q(\zeta)$.
\end{izrek}

\begin{proof}
Let $K = \Q \br{\sqrt{p^*}}$ and $L = \Q(\zeta)$. Suppose first
that $q$ splits, that is
$q \mathcal{O}_K = \mathfrak{q}_1 \mathfrak{q}_2$, where
$\mathfrak{q}_1 \ne \mathfrak{q}_2 \in \mathcal{P}(\mathcal{O}_K)$.
Choose an automorphism $\sigma \in \Gal \br{\kvoc{L}{\Q}}$ such
that $\sigma(\mathfrak{q}_1) = \mathfrak{q}_2$. Then $\sigma$
induces a bijection
\[
\setb{\mathfrak{Q} \in \mathcal{P}(\mathcal{O}_L)}
{\mathfrak{Q} \mid \mathfrak{q}_1} \to
\setb{\mathfrak{Q} \in \mathcal{P}(\mathcal{O}_L)}
{\mathfrak{Q} \mid \mathfrak{q}_2},
\]
therefore the cardinality of the set
$\setb{\mathfrak{Q} \in \mathcal{P}(\mathcal{O}_L)}
{\mathfrak{Q} \mid q}$
is even.

Let $\mathfrak{Q} \in \mathcal{P}(\mathcal{O}_L)$ be such that
$\mathfrak{Q} \mid q$ and suppose that the cardinality $r$ of the
set
$\setb{\mathfrak{Q}' \in \mathcal{P}(\mathcal{O}_L)}
{\mathfrak{Q}' \mid q}$
is even. Then
\[
r = \abs{\Gal \br{\kvoc{L}{\Q}} : D(\mathfrak{Q})}
\]
is even, therefore
\[
\left[L^{D(\mathfrak{Q})} : \Q\right]
\]
is even and therefore contains the unique quadratic subfield $K$ of
$L$. By theorem~\ref{hilb:thm:fer}, we have
\[
e \br{\mathfrak{Q} \cap L^{D(\mathfrak{Q})}~\middle\vert~q} =
f \br{\mathfrak{Q} \cap L^{D(\mathfrak{Q})}~\middle\vert~q} =
1
\]
and therefore
$e(\mathfrak{q}_i \mid \mathfrak{q}) =
f(\mathfrak{q}_i \mid \mathfrak{q}) =
1$, hence $q$ splits.
\end{proof}

\begin{izrek}[Quadratic reciprocity law]
\index{Quadratic reciprocity law}
Let $p$ and $q$ be distinct odd primes. Then
\[
\br{\frac{p}{q}} \cdot \br{\frac{q}{p}} =
(-1)^{\frac{p-1}{2} \cdot \frac{q-1}{2}}.
\]
\end{izrek}

\begin{proof}
As before, let $p^* = (-1)^{\frac{p-1}{2}} p$ and
$K = \Q \br{\sqrt{p^*}}$. Then
\[
\br{\frac{p^*}{q}} =
\br{\frac{-1}{q}}^{\frac{p-1}{2}} \cdot \br{\frac{p}{q}} =
(-1)^{\frac{p-1}{2} \cdot \frac{q-1}{2}} \cdot \br{\frac{p}{q}}.
\]
Note that $\br{\frac{p^*}{q}} = 1$ is equivalent to $q$ splitting
in $K$, which is in turn equivalent to $q$ lying under an even
number of prime ideals in $\Q(\zeta)$.

Denote $f = \ord_{\Z_p^*} \br{\oline{q}}$. Then $q$ lies under
precisely
\[
\frac{[\Q(\zeta) : \Q]}{f} =
\frac{\varphi(p)}{f} =
\frac{p-1}{f}
\]
prime ideals. The number $\frac{p-1}{f}$ is even if and only if
$\left.f~\middle\vert~\frac{p-1}{2}\right.$, which is equivalent to
$\br{\frac{q}{p}} \equiv q^{\frac{p-1}{2}} \equiv 1 \pmod{p}$.
\end{proof}

\newpage

\subsection{Frobenius elements}

\begin{definicija}
Let $\mathfrak{P} \in \mathcal{P}(\mathcal{O}_L)$ be unramified.
The \emph{Frobenius element}\index{Frobenius element} of
$\mathfrak{P}$, denoted by
\[
\br{\frac{\kvoc{L}{K}}{\mathfrak{P}}} \in \Gal \br{\kvoc{L}{K}}
\]
is the unique automorphism of $\kvoc{L}{K}$ that maps to the
Frobenius automorphism of
$\kvoc{\kappa(\mathfrak{P})}{\kappa(\mathfrak{p})}$. In other
words, $\sigma = \br{\frac{\kvoc{L}{K}}{\mathfrak{P}}}$ is the
unique automorphism such that
$\sigma(\alpha) - \alpha^q \in \mathfrak{P}$ for
$q = N(\mathfrak{p})$.
\end{definicija}

\begin{lema}
Let $\mathfrak{P} \in \mathcal{P}(\mathcal{O}_L)$ be unramified.
Then
$\ord \br{\br{\frac{\kvoc{L}{K}}{\mathfrak{P}}}} =
f(\mathfrak{P} \mid \mathfrak{p})$.
\end{lema}

\obvs

\begin{lema}
Let $\tau \in \Gal \br{\kvoc{L}{K}}$,
$\mathfrak{P} \in \mathcal{P}(\mathcal{O}_L)$ and
$\mathfrak{P}' = \tau(\mathfrak{P})$. Then
\[
\br{\frac{\kvoc{L}{K}}{\mathfrak{P}'}} =
\tau \br{\frac{\kvoc{L}{K}}{\mathfrak{P}}} \tau^{-1}.
\]
\end{lema}

\begin{proof}
By definition, we have $\sigma \in D(\mathfrak{P})$, therefore
$\tau \sigma \tau^{-1} \in D(\mathfrak{P}')$. But then
\[
\sigma \br{\tau^{-1}(\alpha)} - \tau^{-1}(\alpha)^q \in
\mathfrak{P}
\]
for all $\alpha \in \mathcal{O}_L$, which implies
\[
\tau \br{\sigma \br{\tau^{-1}(\alpha)}} - \alpha^q \in
\mathfrak{P}'. \qedhere
\]
\end{proof}

\begin{opomba}
If the Galois group is abelian, this defines a unique Frobenius
element for each $\mathfrak{p} \in \mathcal{P}(\mathcal{O}_K)$.
\end{opomba}

\begin{lema}
Let $K \subseteq M \subseteq L$ be number fields such that
$\kvoc{L}{K}$ is abelian.\footnote{That is, it is Galois with
abelian Galois group.} For unramified (in $L$)
$\mathfrak{p} \in \mathcal{P}(\mathcal{O}_K)$ we have
\[
\eval{\br{\frac{\kvoc{L}{K}}{\mathfrak{p}}}}{M}{} =
\br{\frac{\kvoc{M}{K}}{\mathfrak{p}}}.
\]
\end{lema}

\begin{proof}
Let $\mathfrak{P} \in \mathcal{P}(\mathcal{O}_L)$ be a prime ideal
such that $\mathfrak{p} = \mathfrak{P} \cap \mathcal{O}_K$. Denote
$q = \abs{\kvoc{\mathcal{O}_K}{\mathfrak{p}}}$ and
$\sigma = \br{\frac{\kvoc{L}{K}}{\mathfrak{p}}}$. Since
$\kvoc{M}{K}$ is Galois, we have that
$\eval{\sigma}{M}{} \in \Gal \br{\kvoc{M}{K}}$. It follows that
\[
\sigma(\alpha) - \alpha^q \in \mathcal{O}_M \cap \mathfrak{P}.
\qedhere
\]
\end{proof}

\begin{izrek}[Quadratic reciprocity law]
\index{Quadratic reciprocity law}
Let $p$ and $q$ be distinct odd primes. Then
\[
\br{\frac{p}{q}} \cdot \br{\frac{q}{p}} =
(-1)^{\frac{p-1}{2} \cdot \frac{q-1}{2}}.
\]
\end{izrek}

\begin{proof}
As before, we will prove that
$\br{\frac{p^*}{q}} = \br{\frac{q}{p}}$. Let
$\zeta \in \mu_p^*(\C)$ and denote $L = \Q(\zeta)$ and
$K = \Q \br{\sqrt{p^*}}$. Since $\kvoc{L}{\Q}$ is abelian, we have
\[
\eval{\br{\frac{\kvoc{L}{\Q}}{q}}}{K}{} =
\br{\frac{\kvoc{K}{\Q}}{q}} =
\br{\frac{p^*}{q}}
\]
as an element of $\Gal \br{\kvoc{K}{\Q}} \cong S^0$. But by
definition, $\br{\frac{\kvoc{L}{\Q}}{q}}(\zeta) = \zeta^q$. The map
\[
\Z_p^* \cong \Gal \br{\kvoc{L}{\Q}} \to
\Gal \br{\kvoc{K}{\Q}} \cong S^0
\]
induced by the restriction has kernel $\br{\Z_p^*}^2$, as it is the
only subgroup of index $2$. Thus the element
$\eval{\br{\frac{\kvoc{L}{\Q}}{q}}}{K}{}$ is trivial if and only if
$q$ is a square modulo $p$, hence
\[
\eval{\br{\frac{\kvoc{L}{\Q}}{q}}}{K}{} = \br{\frac{q}{p}}.
\qedhere
\]
\end{proof}

\newpage

\subsection{Chebotarev's density theorem}

\begin{definicija}
Let $K$ be a number field and
$S \subseteq \mathcal{P}(\mathcal{O}_K)$. We say that $S$ has
\emph{natural density}\index{natural density} $\delta \in [0, 1]$
if
\[
\lim_{M \to \infty}
\frac{\abs{\setb{\mathfrak{p} \in S}{N(\mathfrak{p}) \leq M}}}
{\abs{\setb{\mathfrak{p} \in \mathcal{P}(\mathcal{O}_K}
{N(\mathfrak{p}) \leq M}}} =
\delta.
\]
\end{definicija}

\begin{izrek}[Chebotarev]
\index{Chebotarev's theorem}
Let $K$ and $L$ be number fields with $\kvoc{L}{K}$ being Galois
and denote $G = \Gal \br{\kvoc{L}{K}}$. Furthermore, let
$C \subseteq G$ be a conjugacy class. Then
\[
\setb{\mathfrak{p} \in \mathcal{P}(\mathcal{O}_K)}
{\br{\frac{\kvoc{L}{K}}{\mathfrak{p}}} = C}
\]
has density $\frac{\abs{C}}{\abs{G}}$.
\end{izrek}

\begin{posledica}
In a quadratic number field, half of the prime numbers split and
half are inert (asymptotically).
\end{posledica}

\begin{posledica}
The completely split primes have density $\frac{1}{[L : K]}$.
\end{posledica}

\begin{posledica}
Every class in $\mathcal{C}(\mathcal{O}_K)$ contains infinitely
many prime ideals.
\end{posledica}

\begin{izrek}[Dirichlet]
\index{Dirichlet's theorem}
Let $a, b \in \N$ be coprime. Then there are infinitely many prime
numbers of the form $a + bn$ for $n \in \N$. Furthermore, their
density is equal to $\frac{1}{\varphi(b)}$.
\end{izrek}

\begin{proof}
Let $K = \Q(\zeta)$, where $\zeta \in \mu_b^*(\C)$. Then
$\Gal \br{\kvoc{K}{\Q}} \cong \Z_b^*$. Note that
$\br{\frac{\kvoc{K}{\Q}}{p}} = p + b\Z$ for $p \nmid b$. Such
primes have density
$\frac{1}{\abs{\Gal \br{\kvoc{K}{\Q}}}} = \frac{1}{\varphi(b)}$.
\end{proof}
