\section{Distribution of prime numbers}

\subsection{Riemann zeta function}

\datum{2024-2-23}

\begin{definicija}
The \emph{prime counting function}\index{prime counting function}
is defined as
\[
\pi(x) = \abs{\setb{p \in \P}{p \leq x}}.
\]
\end{definicija}

\begin{definicija}
Let $(a_n)_n \subseteq \C$ be a sequence. The infinite product
\[
\prod_{n=1}^\infty a_n
\]
converges \emph{absolutely}\index{absolute convergence} if it
converges normally as a product of constant functions.
\end{definicija}

\begin{izrek}
Let $\sigma > 1$ be a real number. For $s \in \C$ with
$\Re(s) \geq \sigma$, we have
\[
\sum_{n=1}^\infty = \prod_{p \in \P} (1 - p^{-s})^{-1},
\]
with both the product and sum converging uniformly and
absolutely.\footnote{See Complex analysis, section 3 for definition
and properties of convergence for products.}
\end{izrek}

\begin{proof}
Note that
\[
\sum_{n=1}^\infty \abs{\frac{1}{n^s}} =
\sum_{n=1}^\infty \frac{1}{n^{\Re(s)}} \leq
\sum_{n=1}^\infty \frac{1}{n^\sigma}
\]
is convergent, hence the given series converges as well. To prove
the convergence of the product, first note that
\[
\prod_{p \in \P} (1-p^{-s})^{-1} =
\prod_{p \in \P} \br{\sum_{k=0}^\infty p^{-sk}}.
\]
As
\[
\sum_{p \in \P} \abs{\sum_{k=1}^\infty p^{-sk}} \leq
\sum_{p \in \P} \br{\sum_{k=1}^\infty (p^k)^{-\sigma}} \leq
\sum_{n=1}^\infty n^{-\sigma}
\]
converges normally, so does the product. To prove equality, we can
bound
\[
\abs{\prod_{\substack{p \in \P \\ p \leq x}}
\frac{1}{1 - p^{-s}} - \sum_{n=1}^x \frac{1}{n^s}} \leq
\sum_{n=x+1}^\infty \abs{\frac{1}{n^s}} \leq
\sum_{n=x+1}^\infty \frac{1}{n^\sigma},
\]
which converges to $0$ as $x \to \infty$.
\end{proof}

\begin{definicija}
The \emph{Riemann zeta function}\index{Riemann zeta function} is
defined as
\[
\zeta(s) = \sum_{n=1}^\infty \frac{1}{n^s}
\]
for $\Re(s) > 1$.
\end{definicija}

\begin{lema}
If $\Re(s) > 1$, then $\zeta(s) \ne 0$.
\end{lema}

\begin{proof}
No term in the infinite product is equal to $0$.
\end{proof}

\begin{trditev}
The function $\zeta(s) - \frac{1}{s-1}$ has a holomorphic
continuation to $\Re(s) > 0$.
\end{trditev}

\begin{proof}
We can write
\begin{align*}
\zeta(s) - \frac{1}{s-1} &=
\sum_{n=1}^\infty n^{-s} - \int_1^\infty x^{-s}\,dx
\\
&=
\sum_{n=1}^\infty \br{n^{-s} - \int_n^{n+1} x^{-s}\,dx}
\\
&=
\sum_{n=1}^\infty \int_n^{n+1} \br{n^{-s} - x^{-s}}\,dx
\end{align*}
as long as $\Re(s) > 1$. Now, for $n \leq x \leq n+1$, we can bound
\[
\abs{n^{-s} - x^{-s}} =
\abs{\int_n^x s u^{-s-1}\,du} \leq
\frac{\abs{s}}{n^{\Re(s)+1}}.
\]
Let $L \subseteq \setb{z \in \C}{\Re(z) > 0}$ be a compact set. As
\[
\abs{\sum_{n=1}^\infty \int_n^{n+1} \br{n^{-s} - x^{-s}}\,dx} \leq
\sum_{n=1}^\infty \frac{\abs{s}}{n^{\Re(s)+1}} \leq
\norm{\id}_L \cdot \sum_{n=1}^\infty \frac{1}{n^{\sigma+1}}
\]
for all $s \in L$, where $\sigma = \min_L \abs{z}$, the series
converges uniformly on compact sets.
\end{proof}

\begin{opomba}
The $\zeta$ function can be analytically extended to
$\C \setminus \set{1}$ by
\[
\zeta(1-s) =
2 \cdot (2 \pi)^{-s} \cos \br{\frac{\pi}{2} s} \Gamma(s) \zeta(s).
\]
It has a simple pole with residue $1$ at $1$.
\end{opomba}

\begin{lema}
The equation $\oline{\zeta(\oline{s})} = \zeta(s)$ holds for all
$s \in \C \setminus \set{1}$.
\end{lema}

\begin{proof}
The function $\oline{\zeta(\oline{s})}$ is holomorphic. As it
coincides with $\zeta(s)$ for $s \geq 1$, the functions are equal.
\end{proof}

\newpage

\subsection{Prime number theorem}

\begin{trditev}
\label{prime:prop:phi}
The series
\[
\sum_{p \in \P} \frac{\log(p)}{p^s}
\]
converges uniformly and absolutely for $\Re(s) \geq \sigma > 1$.
\end{trditev}

\begin{proof}
We can bound
\[
\sum_{p \in \P} \abs{\frac{\log(p)}{p^s}} \leq
\sum_{p \in \P} \frac{\log(p)}{p^\sigma} \leq
\sum_{n=1}^\infty \frac{\log(p)}{n^\varepsilon} \cdot
\frac{1}{n^{\sigma-\varepsilon}},
\]
which clearly converges for $0 < \varepsilon < \sigma-1$.
\end{proof}

\begin{definicija}
We define functions
\[
\theta(x) =
\sum_{\substack{p \in \P \\ p \leq x}} \log(p)
\]
and
\[
\phi(s) = \sum_{p \in \P} \frac{\log(p)}{p^s}.
\]
\end{definicija}

\begin{opomba}
The function $\phi$ is holomorphic for $\Re(s) > 1$.
\end{opomba}

\begin{trditev}
The function $\phi$ has a meromorphic continuation to
$\Re(s) > \frac{1}{2}$. It has simple poles at points $s=1$ and
zeros of $\zeta(s)$.
\end{trditev}

\begin{proof}
Calculate the logarithmic derivative of $\zeta$ as
\begin{align*}
- \frac{\zeta'(s)}{\zeta(s)} &=
-\sum_{p \in \P} \frac{\br{(1-p^{-s})^{-1}}'}{(1-p^{-s})^{-1}}
\\
&=
-\sum_{p \in \P}
\frac{-(1-p^{-s})^{-2} \cdot p^{-s} \log(p)}{(1-p^{-s})^{-1}}
\\
&=
\sum_{p \in \P} \frac{\log(p)}{p^s-1}
\\
&=
\phi(s) + \sum_{p \in \P} \frac{\log(p)}{p^s(p^s-1)}.
\end{align*}
Similarly as in the proof of proposition~\ref{prime:prop:phi}, we
can show that the above series converges uniformly and absolutely
for $\Re(s) > \frac{1}{2}$.
\end{proof}

\begin{izrek}
If $\Re(s) = 1$, then $\zeta(s) \ne 0$.
\end{izrek}

\begin{proof}
Let $\mu = \ord_{1 + ib} \zeta \geq 0$. As
$\zeta(\oline{z}) = \oline{\zeta(z)}$, we also have
$\mu = \ord_{1 - ib} \zeta$.
$\theta = \ord_{1 + 2ib} \zeta = \ord_{1 - 2ib} \zeta$. As $\phi$
has a simple pole at $1$, we have
\[
\lim_{\varepsilon \to 0} \varepsilon \phi(1+\varepsilon) = 1.
\]
Similarly,
\[
\lim_{\varepsilon \to 0} \varepsilon \phi(1 + \varepsilon \pm ib) =
-\mu,
\]
as the logarithmic derivative of $\zeta$ at $b$ has residue $-\mu$,
and
\[
\lim_{\varepsilon \to 0}
\varepsilon \phi(1 + \varepsilon \pm 2ib) =
-\theta.
\]
Now compute
\[
f(\varepsilon) =
\sum_{r=-2}^2 \binom{4}{2+r} \phi(1 + \varepsilon + rib) =
\sum_{p \in \P} \frac{\log(p)}{p^{1+\varepsilon}} \cdot
\br{p^{\frac{ib}{2}} - p^{-\frac{ib}{2}}}^4 =
\sum_{p \in \P} \frac{\log(p)}{p^{1+\varepsilon}} \cdot
\br{2 \Re \br{p^{\frac{ib}{2}}}}^4.
\]
It follows that
\[
0 \leq
\lim_{\varepsilon \to 0} \varepsilon \cdot f(\varepsilon) =
6 - 8 \mu - 2 \theta.
\]
As $\theta \geq 0$, we have $\mu = 0$.
\end{proof}

\begin{posledica}
The function $\phi$ is holomorphic for $\Re(s) = 1$, except for a
simple pole with residue $1$ at $1$. In particular, the function
\[
g(z) = \frac{\phi(z+1)}{z+1} - \frac{1}{z}
\]
is holomorphic for $\Re(z) \geq 0$.
\end{posledica}
