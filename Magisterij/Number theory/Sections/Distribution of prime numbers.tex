\section{Distribution of prime numbers}

\subsection{Riemann zeta function}

\datum{2024-2-23}

\begin{definicija}
The \emph{prime counting function}\index{prime counting function}
is defined as
\[
\pi(x) = \abs{\setb{p \in \P}{p \leq x}}.
\]
\end{definicija}

\begin{definicija}
Let $(a_n)_n \subseteq \C$ be a sequence. The infinite product
\[
\prod_{n=1}^\infty a_n
\]
converges \emph{absolutely}\index{absolute convergence} if it
converges normally as a product of constant functions.
\end{definicija}

\begin{izrek}
Let $\sigma > 1$ be a real number. For $s \in \C$ with
$\Re(s) \geq \sigma$, we have
\[
\sum_{n=1}^\infty \frac{1}{n^s} =
\prod_{p \in \P} (1 - p^{-s})^{-1},
\]
with both the product and sum converging uniformly and
absolutely.\footnote{See Complex analysis, section 3 for definition
and properties of convergence for products.}
\end{izrek}

\begin{proof}
Note that
\[
\sum_{n=1}^\infty \abs{\frac{1}{n^s}} =
\sum_{n=1}^\infty \frac{1}{n^{\Re(s)}} \leq
\sum_{n=1}^\infty \frac{1}{n^\sigma}
\]
is convergent, hence the given series converges as well. To prove
the convergence of the product, first note that
\[
\prod_{p \in \P} (1-p^{-s})^{-1} =
\prod_{p \in \P} \br{\sum_{k=0}^\infty p^{-sk}}.
\]
As
\[
\sum_{p \in \P} \abs{\sum_{k=1}^\infty p^{-sk}} \leq
\sum_{p \in \P} \br{\sum_{k=1}^\infty (p^k)^{-\sigma}} \leq
\sum_{n=1}^\infty n^{-\sigma}
\]
converges normally, so does the product. To prove equality, we can
bound
\[
\abs{\prod_{\substack{p \in \P \\ p \leq x}}
\frac{1}{1 - p^{-s}} - \sum_{n=1}^x \frac{1}{n^s}} \leq
\sum_{n=x+1}^\infty \abs{\frac{1}{n^s}} \leq
\sum_{n=x+1}^\infty \frac{1}{n^\sigma},
\]
which converges to $0$ as $x \to \infty$.
\end{proof}

\begin{definicija}
The \emph{Riemann zeta function}\index{Riemann zeta function} is
defined as
\[
\zeta(s) = \sum_{n=1}^\infty \frac{1}{n^s}
\]
for $\Re(s) > 1$.
\end{definicija}

\begin{lema}
If $\Re(s) > 1$, then $\zeta(s) \ne 0$.
\end{lema}

\begin{proof}
No term in the infinite product is equal to $0$.
\end{proof}

\begin{trditev}
The function $\zeta(s) - \frac{1}{s-1}$ has a holomorphic
continuation to $\Re(s) > 0$.
\end{trditev}

\begin{proof}
We can write
\begin{align*}
\zeta(s) - \frac{1}{s-1} &=
\sum_{n=1}^\infty n^{-s} - \int_1^\infty x^{-s}\,dx
\\
&=
\sum_{n=1}^\infty \br{n^{-s} - \int_n^{n+1} x^{-s}\,dx}
\\
&=
\sum_{n=1}^\infty \int_n^{n+1} \br{n^{-s} - x^{-s}}\,dx
\end{align*}
as long as $\Re(s) > 1$. Now, for $n \leq x \leq n+1$, we can bound
\[
\abs{n^{-s} - x^{-s}} =
\abs{\int_n^x s u^{-s-1}\,du} \leq
\frac{\abs{s}}{n^{\Re(s)+1}}.
\]
Let $L \subseteq \setb{z \in \C}{\Re(z) > 0}$ be a compact set. As
\[
\abs{\sum_{n=1}^\infty \int_n^{n+1} \br{n^{-s} - x^{-s}}\,dx} \leq
\sum_{n=1}^\infty \frac{\abs{s}}{n^{\Re(s)+1}} \leq
\norm{\id}_L \cdot \sum_{n=1}^\infty \frac{1}{n^{\sigma+1}}
\]
for all $s \in L$, where $\sigma = \min_L \abs{z}$, the series
converges uniformly on compact sets.
\end{proof}

\begin{opomba}
The $\zeta$ function can be analytically extended to
$\C \setminus \set{1}$ by
\[
\zeta(1-s) =
2 \cdot (2 \pi)^{-s} \cos \br{\frac{\pi}{2} s} \Gamma(s) \zeta(s).
\]
It has a simple pole with residue $1$ at $1$.
\end{opomba}

\begin{lema}
The equation $\oline{\zeta(\oline{s})} = \zeta(s)$ holds for all
$s \in \C \setminus \set{1}$.
\end{lema}

\begin{proof}
The function $\oline{\zeta(\oline{s})}$ is holomorphic. As it
coincides with $\zeta(s)$ for $s \geq 1$, the functions are equal.
\end{proof}

\newpage

\subsection{Prime number theorem}

\begin{trditev}
\label{prime:prop:phi}
The series
\[
\sum_{p \in \P} \frac{\log(p)}{p^s}
\]
converges uniformly and absolutely for $\Re(s) \geq \sigma > 1$.
\end{trditev}

\begin{proof}
We can bound
\[
\sum_{p \in \P} \abs{\frac{\log(p)}{p^s}} \leq
\sum_{p \in \P} \frac{\log(p)}{p^\sigma} \leq
\sum_{n=1}^\infty \frac{\log(p)}{n^\varepsilon} \cdot
\frac{1}{n^{\sigma-\varepsilon}},
\]
which clearly converges for $0 < \varepsilon < \sigma-1$.
\end{proof}

\begin{definicija}
We define functions
\[
\theta(x) =
\sum_{\substack{p \in \P \\ p \leq x}} \log(p)
\]
and
\[
\phi(s) = \sum_{p \in \P} \frac{\log(p)}{p^s}.
\]
\end{definicija}

\begin{opomba}
The function $\phi$ is holomorphic for $\Re(s) > 1$.
\end{opomba}

\begin{trditev}
The function $\phi$ has a meromorphic continuation to
$\Re(s) > \frac{1}{2}$. It has simple poles at points $s=1$ and
zeros of $\zeta(s)$.
\end{trditev}

\begin{proof}
Calculate the logarithmic derivative of $\zeta$ as
\begin{align*}
- \frac{\zeta'(s)}{\zeta(s)} &=
-\sum_{p \in \P} \frac{\br{(1-p^{-s})^{-1}}'}{(1-p^{-s})^{-1}}
\\
&=
-\sum_{p \in \P}
\frac{-(1-p^{-s})^{-2} \cdot p^{-s} \log(p)}{(1-p^{-s})^{-1}}
\\
&=
\sum_{p \in \P} \frac{\log(p)}{p^s-1}
\\
&=
\phi(s) + \sum_{p \in \P} \frac{\log(p)}{p^s(p^s-1)}.
\end{align*}
Similarly as in the proof of proposition~\ref{prime:prop:phi}, we
can show that the above series converges locally uniformly and
absolutely for $\Re(s) > \frac{1}{2}$.
\end{proof}

\begin{izrek}
If $\Re(s) = 1$, then $\zeta(s) \ne 0$.
\end{izrek}

\begin{proof}
Let $\mu = \ord_{1 + ib} \zeta \geq 0$. As
$\zeta(\oline{z}) = \oline{\zeta(z)}$, we also have
$\mu = \ord_{1 - ib} \zeta$.
$\theta = \ord_{1 + 2ib} \zeta = \ord_{1 - 2ib} \zeta$. As $\phi$
has a simple pole at $1$, we have
\[
\lim_{\varepsilon \to 0} \varepsilon \phi(1+\varepsilon) = 1.
\]
Similarly,
\[
\lim_{\varepsilon \to 0} \varepsilon \phi(1 + \varepsilon \pm ib) =
-\mu,
\]
as the logarithmic derivative of $\zeta$ at $b$ has residue $-\mu$,
and
\[
\lim_{\varepsilon \to 0}
\varepsilon \phi(1 + \varepsilon \pm 2ib) =
-\theta.
\]
Now compute
\[
f(\varepsilon) =
\sum_{r=-2}^2 \binom{4}{2+r} \phi(1 + \varepsilon + rib) =
\sum_{p \in \P} \frac{\log(p)}{p^{1+\varepsilon}} \cdot
\br{p^{\frac{ib}{2}} - p^{-\frac{ib}{2}}}^4 =
\sum_{p \in \P} \frac{\log(p)}{p^{1+\varepsilon}} \cdot
\br{2 \Re \br{p^{\frac{ib}{2}}}}^4.
\]
It follows that
\[
0 \leq
\lim_{\varepsilon \to 0} \varepsilon \cdot f(\varepsilon) =
6 - 8 \mu - 2 \theta.
\]
As $\theta \geq 0$, we have $\mu = 0$.
\end{proof}

\begin{posledica}
\label{prime:cor:g_def}
The function $\phi$ is holomorphic for $\Re(s) = 1$, except for a
simple pole with residue $1$ at $1$. In particular, the function
\[
g(z) = \frac{\phi(z+1)}{z+1} - \frac{1}{z}
\]
is holomorphic for $\Re(z) \geq 0$.
\end{posledica}

\obvs

\datum{2024-3-1}

\begin{lema}
Let $x \geq 0$. Then $\theta(x) \leq 4x$.
\end{lema}

\begin{proof}
First let $n \in \N$ be an integer. Then
\[
e^{\theta(2n) - \theta(n)} =
\prod_{n < p \leq 2n} p \leq
\binom{2n}{n} \leq
2^{2n},
\]
therefore $\theta(2n) - \theta(n) \leq 2n \log(2)$. Now let
$n = \ceil{\frac{x}{2}}$. Then
\[
\theta(x) - \theta \br{\frac{x}{2}} \leq
\theta(2n) - \theta(n-1) \leq
\log(n) + 2n \log(2) \leq
3n \leq
2x
\]
for all $x \geq 6$, but we can manually check that it holds for
$x < 6$ as well. But then
\[
\theta(x) =
\sum_{n=0}^\infty
\br{\theta \br{\frac{x}{2^n}} - \theta \br{\frac{x}{2^{n+1}}}} \leq
\sum_{n=0}^\infty \frac{2x}{2^n} =
4x. \qedhere
\]
\end{proof}

\begin{lema}
Let $h \colon \R_{\geq 0} \to \C$ be bounded and locally
integrable. Then the following statements are true:

\begin{enumerate}[i)]
\item The Laplace transform
\[
H(z) = \int_0^\infty h(t) e^{-zt}\,dt
\]
of $h$ is holomorphic for $\Re(z) > 0$.
\item The function
\[
\int_0^T h(t) e^{-zt}\,dt
\]
is holomorphic for all $z \in \C$.
\end{enumerate}
\end{lema}

\begin{proof}
\phantom{i}
\begin{enumerate}[i)]
\item Analysis 2b, proposition 4.1.4.
\item Evident. \qedhere % Morera če si fancy
\end{enumerate}
\end{proof}

\begin{izrek}
\label{prime:thm:ana}
Let $h \colon \R_{\geq 0} \to \C$ be bounded and locally
integrable. Suppose that its Laplace transform
\[
H(z) = \int_0^\infty h(t) e^{-zt}\,dt
\]
extends to a holomorphic function on $\Re(z) \geq 0$. Then
\[
H(0) = \int_0^\infty h(t)\,dt.
\]
\end{izrek}

\begin{proof}
Define
\[
H_T(z) = \int_0^T h(t) e^{-zt}\,dt
\]
for $T > 0$. Fix some $R > 0$ and consider the region
\[
\Omega = \setb{z \in \dsk(R)}{\Re(z) \geq -\delta}.
\]
By compactness of $i[-R, R]$, we can pick a $\delta$ such that
$H$ is holomorphic on $\Omega$. Now partition $\partial \Omega$
into sets $C_1 = \setb{z \in \partial \Omega}{\Re(z) \geq 0}$,
$C_2 = \setb{z \in \partial \Omega}{-\delta < \Re(z) < 0}$ and
$C_3 = \setb{z \in \partial \Omega}{\Re(z) = -\delta}$. Taking
\[
I(z) = \frac{H(z) - H_T(z)}{z} e^{zT} \br{1 + \frac{z^2}{R^2}},
\]
we can write
\[
H(0) - H_T(0) =
\frac{1}{2 \pi i} \olint_{\partial \Omega} I(z)\,dz
\]
using the Cauchy integral formula. Setting
$B = \max \setb{\abs{h(t)}}{t \in \R_{\geq 0}}$, we can bound
\[
\abs{H(z) - H_T(z)} \leq
\int_T^\infty \abs{h(t)} \cdot \abs{e^{-zt}}\,dt \leq
B \frac{e^{- \Re(z) T}}{\Re(z)},
\]
hence
\[
\abs{I(z)} \leq
\frac{B}{\Re(z)} \cdot \abs{1 + \frac{z^2}{R^2}} \cdot
\abs{\frac{1}{z}} =
\frac{B}{R \Re(z)} \cdot \abs{\frac{z}{R} + \frac{R}{z}} =
\frac{B}{R \Re(z)} \cdot 2 \Re \br{\frac{z}{R}} =
\frac{2B}{R^2}
\]
for $z \in C_1$. Integrating, we find that
\[
\frac{1}{2 \pi} \cdot \lint_{C_1} \abs{I(z)}\,dz \leq
\frac{B}{R}.
\]

Next, we bound the integral of $H_T$ over $C_2 \cup C_3$. As
$H_T$ is holomorphic, we can write
\[
\lint_{C_2 \cup C_3} H_T(z)\,dz =
\lint_{-C_1} H_T(z)\,dz,
\]
but as
\[
\abs{H_T(z)} \leq
\int_0^T \abs{h(z) e^{-zt}}\,dt \leq
B \int_0^T e^{-\Re(z) t}\,dt =
\frac{B}{\Re(z)} \cdot \br{1 - e^{-\Re(z) T}} \leq
B \frac{e^{-\Re(z)T}}{\abs{\Re(z)}},
\]
which is the same bound as above. 
As
\[
\abs{H(z) \cdot \br{1 + \frac{z^2}{R^2}} \cdot \frac{1}{z}} \leq M
\]
on $C_2 \cup C_3$ for some $M > 0$, we see that
\[
\abs{H(z) \cdot \br{1 + \frac{z^2}{R^2}} \cdot \frac{1}{z}} \cdot
\abs{e^{zT}}
\]
converges to $0$ as $T \to \infty$. By the dominated convergence
theorem, the integral
\[
\frac{1}{2 \pi} \cdot \lint_{C_2 \cup C_3}
\abs{H(z) \br{1 + \frac{z^2}{R^2}} \cdot \frac{1}{z}} \cdot
\abs{e^{zT}}\,dz
\]
converges to $0$ as well. Then
\[
\limsup_{T \to \infty} \abs{H(0) - H_T(0)} \leq \frac{2B}{R},
\]
which, by taking $R \to \infty$, implies
\[
\lim_{T \to \infty} H_T(0) = H(0). \qedhere
\]
\end{proof}

\begin{lema}
For $\Re(z) > 0$, we have
\[
g(z) = \int_0^\infty \br{\theta \br{e^t} e^{-t} - 1} e^{-zt}\,dt,
\]
where $g$ is defined as in corollary~\ref{prime:cor:g_def}.
\end{lema}

\begin{proof}
Note that $\theta \br{e^t} e^{-t} - 1$ is bounded, hence the
given Laplace transform exists. Let $(p_n)_n$ be the ascending
sequence of prime numbers. Setting $p_0 = 1$, we have
\begin{align*}
\phi(s) &=
\sum_{p \in \P} \frac{\log(p)}{p^s}
\\
&=
\sum_{j=1}^\infty \frac{\theta(p_j) - \theta(p_{j-1})}{p_j^s}
\\
&=
\sum_{j=0}^\infty \theta(p_j) \cdot
\br{\frac{1}{p_j^s} - \frac{1}{p_{j+1}^s}}
\\
&=
\sum_{j=0}^\infty \theta(p_j) s
\int_{p_j}^{p_{j+1}} \frac{1}{x^{s+1}}\,dx
\\
&=
\sum_{j=0}^\infty s
\int_{p_j}^{p_{j+1}} \frac{\theta(x)}{x^{s+1}}\,dx
\\
&=
s \int_1^\infty \frac{\theta(x)}{x^{s+1}}\,dx
\\
&=
s \int_0^\infty \theta(e^t) e^{-st}\,dt
\end{align*}
for all $\Re(s) > 1$. Hence
\[
g(z) =
\int_0^\infty \theta(e^t) e^{-(z+1)t}\,dt -
\int_0^\infty e^{-zt}\,dt =
\int_0^\infty \br{\theta \br{e^t} e^{-t} - 1} e^{-zt}\,dt. \qedhere
\]
\end{proof}

\begin{izrek}
The integral
\[
\int_1^\infty \frac{\theta(x) - x}{x^2}\,dx
\]
exists.
\end{izrek}

\begin{proof}
We compute
\[
\int_1^{e^T} \frac{\theta(x) - x}{x^2}\,dx =
\int_0^T \br{\theta \br{e^t} e^{-t} - 1}\,dt.
\]
Applying theorem~\ref{prime:thm:ana}, the claim follows.
\end{proof}

\begin{izrek}
We have $\theta(x) \sim x$, that is
\[
\lim_{x \to \infty} \frac{\theta(x)}{x} = 1.
\]
\end{izrek}

\begin{proof}
Suppose otherwise. We split two cases:

\begin{enumerate}[i)]
\item For some $\lambda > 1$, there exist arbitrarily large $x$
such that $\theta(x) \geq \lambda x$. We can compute
\[
\int_x^{\lambda x} \frac{\theta(t) - t}{t^2}\,dt \geq
\int_x^{\lambda x} \frac{\lambda x - t}{t^2}\,dt =
\int_1^\lambda \frac{\lambda x - xy}{x^2 y^2} x\,dy =
\int_1^\lambda \frac{\lambda - y}{y^2}\,dy =
c >
0.
\]
This contradicts the previous theorem.
\item For some $\lambda < 1$, there exist arbitrarily large $x$
such that $\theta(x) \leq \lambda x$. As above, we can compute
\[
\int_{\lambda x}^x \frac{\theta(t) - t}{t^2}\,dt \leq
\int_{\lambda x}^x \frac{\lambda x - t}{t^2}\,dt =
\int_\lambda^1 \frac{\lambda - y}{y^2}\,dy =
c <
0.
\]
This again contradicts the previous theorem. \qedhere
\end{enumerate}
\end{proof}

\begin{izrek}[Prime number theorem]
\index{prime number theorem}
The prime counting function is asymptotically equivalent to
$\frac{x}{\log(x)}$.
\end{izrek}

\begin{proof}
Note that
\[
\theta(x) \leq \log(x) \cdot \pi(x)
\]
and
\[
\theta(x) \geq
\sum_{\substack{p \in \P \\ x^{1-\varepsilon} \leq p \leq x}}
\log(p) \geq
(1 - \varepsilon) \log(x) \cdot \br{\pi(x) - x^{1-\varepsilon}},
\]
therefore
\[
\frac{\theta(x)}{x} \leq
\frac{\pi(x) \log(x)}{x} \leq
\frac{\theta(x)}{(1-\varepsilon)x} + \frac{\log(x)}{x^\varepsilon}.
\]
This implies
\[
1 \leq
\limsup_{x \to \infty} \frac{\pi(x) \log(x)}{x} \leq
\frac{1}{1-\varepsilon}
\]
and
\[
1 \leq
\liminf_{x \to \infty} \frac{\pi(x) \log(x)}{x} \leq
\frac{1}{1-\varepsilon}. \qedhere
\]
\end{proof}
