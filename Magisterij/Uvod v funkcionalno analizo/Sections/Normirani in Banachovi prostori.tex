\section{Normirani in Banachovi prostori}

\epigraph{">Odpreš posnetek predavanj na Youtube pa Call of Duty
zraven."<}{-- prof.~dr.~Igor Klep}

\subsection{Definicije in osnovni zgledi}

\datum{2022-10-6}

\begin{definicija}
Naj bo $X$ vektorski prostor nad $\K \in \set{\R, \C}$. Preslikava
$\norm{.} \colon X \to \R$ je \emph{norma}\index{Norma}, če
velja\footnote{Preslikavi, ki ne izpolnjuje pogoja
$\norm{x} = 0 \iff x = 0$ pravimo \emph{polnorma}.}

\begin{enumerate}[i)]
\item za vse $x \in X$ velja $\norm{x} \geq 0$ z enakostjo natanko
tedaj, ko je $x=0$,
\item za vse $\lambda \in \K$ in $x \in X$ velja
$\norm{\lambda \cdot x} = \abs{\lambda} \cdot \norm{x}$ in
\item za vse $x,y \in X$ velja $\norm{x+y} \leq \norm{x} + \norm{y}$.
\end{enumerate}

Pravimo, da je $(X, \norm{.})$
\emph{normiran vektorski prostor}\index{Vektorski prostor!Normiran}.
\end{definicija}

\begin{definicija}
Na normiranem vektorskem prostoru vpeljemo metriko
$d \colon X \times X \to \R$ kot $d(x,y) = \norm{x-y}$.
\end{definicija}

\begin{opomba}
Tako definirana metrika je translacijsko invariantna in homogena --
za vse $x,y,a \in X$ in $\lambda \in \K$ velja
\[
d(x+a, y+a) = d(x, y)
\quad \text{in} \quad
d(\lambda x, \lambda y) = \abs{\lambda} \cdot d(x,y).
\]
\end{opomba}

\begin{definicija}
Polnim normiranim vektorskim prostorom pravimo
\emph{Banachovi prostori}\index{Vektorski prostor!Banachov}.
\end{definicija}

\begin{definicija}
Algebra $A$ nad $\K$ je \emph{unitalna}\index{Algebra!Unitalna},
če ima enoto $e$.
\end{definicija}

\begin{definicija}
Algebra $A$ nad $\K$ je \emph{normirana}\index{Algebra!Normirana},
če je normiran prostor, v katerem za vse $x,y \in A$
velja\footnote{Pri unitalnih algebrah zahtevamo še $\norm{e} = 1$.}
\[
\norm{xy} \leq \norm{x} \cdot \norm{y}.
\]
Če je $A$ tudi poln vektorski prostor, ji pravimo
\emph{Banachova algebra}\index{Algebra!Banachova}.
\end{definicija}

\begin{opomba}
V normirani algebri sta operaciji seštevanja in množenja zvezni.
\end{opomba}

\begin{definicija}
Za Hausdorffov topološki prostor $X$ naj $\mathcal{C}_b(X)$
označuje množico zveznih omejenih funkcij $f \colon X \to \K$ z
normo $\norm{f}_\infty = \sup_{x \in X} \abs{f(x)}$.
\end{definicija}

\begin{trditev}
Algebra $\mathcal{C}_b(X)$ je Banachova.\footnote{Operacije
definiramo po točkah.}
\end{trditev}

\begin{proof}
Ni težko preveriti, da je $\norm{.}_\infty$ res norma. Algebra je
normirana, saj velja $\norm{1}_\infty = 1$ in
\[
\abs{(fg)(x)} = \abs{f(x)} \cdot \abs{g(x)} \leq
\norm{f}_\infty \cdot \norm{g}_\infty.
\]
Naj bo $(f_n)_{n=1}^\infty$ Cauchyjevo zaporedje v
$\mathcal{C}_b(X)$. Za vsak $\varepsilon > 0$ tako obstaja tak
$N \in \N$, da za vse $m, n > N$ velja
$\norm{f_m - f_n} < \varepsilon$. V tem primeru za vsak $x \in X$
velja $\abs{f_m(x) - f_n(x)} < \varepsilon$, zato je zaporedje
$(f_n(x))_{n=1}^\infty$ Cauchyjevo in ima limito $f(x)$.

Pokažimo, da je $f \in \mathcal{C}_b(X)$. Ker velja
\[
\varepsilon \geq \lim_{m \to \infty} \abs{f_n(x) - f_m(x)} =
\abs{f_n(x) - f(x)},
\]
je zaradi omejenosti $f_n$ omejena tudi $f$.

Naj bo $\varepsilon > 0$. Zaradi Cauchyjeve lastnosti obstaja tak
$N \in \N$, da za vse $n > N$ velja
\[
\norm{f - f_n}_\infty < \frac{\varepsilon}{3}.
\]
Ker je $f_n$ zvezna, za vsak $x \in X$ obstaja taka okolica $U$,
da za vse $y \in U$ velja
\[
\abs{f_n(x) - f_n(y)} < \frac{\varepsilon}{3}.
\]
Sledi torej
\[
\abs{f(x) - f(y)} \leq
\abs{f_n(x) - f(x)} + \abs{f_n(x) - f_n(y)} +\abs{f_n(y) - f(y)} <
\varepsilon. \qedhere
\]
\end{proof}

\begin{opomba}
Če je $X$ kompakten prostor, je algebra zveznih funkcij Banachova.
\end{opomba}

\begin{trditev}
Naj bo $Y$ vektorski podprostor Banachovega prostora $X$. Tedaj je
$Y$ Banachov natanko tedaj, ko je $Y$ zaprt.
\end{trditev}

\begin{proof}
Prostor $Y$ je zaprt natanko tedaj, ko so limite zaporedij iz $Y$
elementi $Y$.
\end{proof}

\begin{zgled}
Naj bo $X$ lokalno kompakten Hausdorffov prostor. Naj bo
\[
\mathcal{C}_0(X) =
\setb{f \in \mathcal{C}(X)}{\forall \varepsilon > 0\,
\exists K \subseteq X \colon (\text{$K$ je kompakt} \land
\forall x \in X \setminus K \colon \abs{f(x)} < \varepsilon)}.
\]
Množica $\mathcal{C}_0(X)$ je zaprt podprostor v $\mathcal{C}_b(X)$
in je ideal v $\mathcal{C}_b(X)$.
\end{zgled}

\begin{proof}
Prostor $\mathcal{C}_0(X)$ je očitno vektorski podprostor. Naj bo
$(f_n)_{n=1}^\infty$ zaporedje v $\mathcal{C}_0(X)$ z limito v
$\mathcal{C}_b(X)$. Za poljuben $\varepsilon > 0$ obstaja tak
$N \in \N$, da za vse $n > \N$ velja
\[
\norm{f - f_n}_\infty < \frac{\varepsilon}{2}.
\]
Po definiciji $\mathcal{C}_0(X)$ obstaja tak kompakt $K$, da za
$x \in X \setminus K$ velja $\abs{f_n(x)} < \frac{\varepsilon}{2}$.
Za $x \in X \setminus K$ torej sledi
\[
\abs{f(x)} < \varepsilon,
\]
zato je $f \in \mathcal{C}_0(X)$. $\mathcal{C}_0(X)$ je torej zaprt
podprostor.

Naj bosta sedaj $f \in \mathcal{C}_0(X)$ in
$g \in \mathcal{C}_b(X) \setminus \set{0}$ funkciji. Za vsak
$\varepsilon > 0$ obstaja tak kompakt $K \subseteq X$, da za vse
$x \in X \setminus K$ velja
\[
\abs{f(x)} < \frac{\varepsilon}{\norm{g}_\infty}.
\]
Za vse $x \in X \setminus K$ torej velja
\[
\abs{(fg)(x)} \leq \abs{f(x)} \cdot \norm{g}_\infty < \varepsilon,
\]
zato je res $\mathcal{C}_0(X) \edn \mathcal{C}_b(X)$.
\end{proof}

\begin{definicija}
Prostor $c$ je podprostor $\ell^\infty = \mathcal{C}_b(\N)$, ki je
sestavljen iz konvergentnih zaporedij.
\end{definicija}

\begin{opomba}
Prostor $c$ je Banachova algebra.
\end{opomba}

\begin{zgled}
Za $p \in [1, \infty)$ na $\K^n$ definiramo
\[
\norm{x}_p = \sqrt[p]{\sum_{i=1}^n \abs{x_i}^p}.
\]
S tem postane $\K^n$ Banachov prostor. Tudi
\[
\ell^p = \setb{x \in \K^\N}{\sum_{i=1}^\infty \abs{x_i}^p < \infty}
\]
je Banachov.
\end{zgled}

\newpage

\subsection{Napolnitve}

\begin{definicija}
Naj bo $X$ normiran prostor in $\widetilde{X}$ vektorski prostor
vseh Cauchyjevih zaporedij v $X$. Na $\widetilde{X}$ vpeljemo
ekvivalenčno relacijo
\[
(x_n)_{n=1}^\infty \sim (y_n)_{n=1}^\infty \iff
\lim_{n \to \infty} (x_n - y_n) = 0.
\]
Z $\widehat{X} = \kvoc{\widetilde{X}}{\sim}$ označimo vektorski
prostor ekvivalenčnih razredov.
\end{definicija}

\begin{trditev}
Preslikava
\[
\norm{[(x_n)_{n=1}^\infty]} = \lim_{n \to \infty} \norm{x_n}
\]
je norma na prostoru $\widehat{X}$.
\end{trditev}

\begin{proof}
Ker so elementi prostora $\widetilde{X}$ Cauchyjeva zaporedja, je
$(\norm{x_n})_{n=1}^\infty$ Cauchyjevo in ima limito. Zgornja
operacija je tako polnorma na $\widetilde{X}$, ki inducira polnormo
na $\widehat{X}$. Ni težko videti, da je na tem prostoru to tudi
norma.
\end{proof}

\begin{opomba}
Obstaja izometrija\footnote{Preslikava, ki ohranja normo.}
$j \colon X \to \widehat{X}$ s predpisom
\[
j(x) = [(x,x,\dots)].
\]
\end{opomba}

\begin{izrek}
Prostor $\widehat{X}$ je Banachov prostor, v katerem je $j(X)$ gost
podprostor.
\end{izrek}

\begin{opomba}
Prostoru $\widehat{X}$ pravimo
\emph{napolnitev}\index{Vektorski prostor!Napolnitev} prostora $X$.
\end{opomba}

\begin{zgled}
Napolnitev prostora $\mathcal{C}([a,b])$ z normo
\[
\norm{f}_2 = \sqrt{\int_a^b \abs{f(x)}^2\,dx}
\]
so kvadratno integrabilne funkcije.
\end{zgled}

\newpage

\subsection{Osnove konstrukcije z Banachovimi prostori}

\datum{2022-10-12}

\begin{definicija}
Naj bo $E$ vektorski prostor nad $\K$. Normi $\norm{.}_1$ in
$\norm{.}_2$ na $E$ sta
\emph{ekvivalentni}\index{Norma!Ekvivalentna}, če obstajata taka
$\alpha, \beta > 0$, da za vse $x \in E$ velja
\[
\alpha \cdot \norm{x}_1 \leq \norm{x}_2 \leq
\beta \cdot \norm{x}_1.
\]
\end{definicija}

\begin{opomba}
Če sta normi $\norm{.}_1$ in $\norm{.}_2$ ekvivalentni, je
$(E, \norm{.}_1)$ poln natanko tedaj, ko je poln $(E, \norm{.}_2)$.
\end{opomba}

\begin{opomba}
Norma je Lipschitzevo zvezna -- velja
\[
\abs{\norm{x} - \norm{y}} \leq \norm{x-y}.
\]
\end{opomba}

\begin{opomba}
Množenje s skalarjem in seštevanje sta zvezni operaciji.
\end{opomba}

\begin{opomba}
Če je $F \leq E$, je tudi $\overline{F} \leq E$.
\end{opomba}

\begin{definicija}
Naj bo $E$ vektorski prostor in $F \leq E$. Na $E$ vpeljemo
ekvivalenčno relacijo $x \sim y \iff x-y \in F$.
\emph{Kvocientni prostor}\index{Vektorski prostor!Kvocientni} je
množica odsekov
\[
\kvoc{E}{F} = \setb{x + F}{x \in E}
\]
s standardnimi operacijami.
\end{definicija}

\begin{trditev}
Naj bo $(E, \norm{.})$ normiran prostor in $F \leq E$ zaprt
podprostor. Tedaj je
\[
\norm{x + F} = \inf \setb{\norm{x+y}}{y \in F}
\]
norma na $\kvoc{E}{F}$. Če je $E$ Banachov, je tudi $\kvoc{E}{F}$
Banachov.
\end{trditev}

\begin{proof}
Opazimo, da je $\norm{0 + F} = 0$. Če je $\norm{x + F} = 0$,
obstaja zaporedje $(y_n)_{n=1}^\infty$ elementov $F$, za katere je
\[
\lim_{n \to \infty} \norm{x + y_n} = 0.
\]
Sledi, da zaporedje konvergira k $-x$. Ker je $F$ zaprt, sledi
$-x \in F$ in zato $x \in F$. Ni težko videti, da velja
\[
\norm{\lambda(x + F)} =
\inf \setb{\norm{\lambda x + y}}{y \in F} =
\abs{\lambda} \cdot \inf \setb{\norm{x + y}}{y \in F}.
\]
Preverimo še trikotniško neenakost. Naj bosta $x, y \in E$ in
$\varepsilon > 0$. Obstajata taka $z_1, z_2 \in F$, da velja
\[
\norm{x + z_1} < \norm{x + F} + \frac{\varepsilon}{2}
\quad \text{in} \quad
\norm{y + z_2} < \norm{y + F} + \frac{\varepsilon}{2}.
\]
Sledi, da je
\[
\norm{(x+F) + (y+F)} =
\norm{(x+y) + F} \leq
\norm{(x+y) + (z_1 + z_2)} <
\norm{x+F} + \norm{y+F} + \varepsilon.
\]
Naj bo sedaj $E$ Banachov prostor in $(x_n + F)_{n=1}^\infty$
Cauchyjevo zaporedje. Za vsak $i$ obstaja tak $n_i \in \N$, da je
\[
\norm{(x_{n_i+1} - x_{n_i}) + F} < 2^{-i},
\]
Brez škode za splošnost je zaporedje $n_i$ naraščajoče. Obstaja
torej tak $y_i \in F$, da je
\[
\norm{(x_{n_{i+1}} - x_{n_i}) + y_i} < 2^{-i}.
\]
Naj bo
$z_1 = 0$ in
\[
z_{i+1} = z_i + y_i.
\]
Sledi, da je
\[
\norm{(x_{n_{i+1}} + z_{i+1}) - (x_{n_i} + z_i)} < 2^{-i}.
\]
Očitno je
\[
w_i = x_{n_i} + z_i
\]
Cauchyjevo, zato je konvergentno z limito $x \in E$. Velja
\[
\norm{(x_{n_i} + F) - (x + F)} \leq
\norm{x_{n_i} - x + z_i} =
\norm{w_i - x},
\]
kar konvergira proti $0$. Sledi, da podzaporedje $(x_{n_i} + F)$
konvergira, zato zaporedje $(x_i + F)$ konvergira.
\end{proof}

\begin{trditev}
Naj bo $E$ normiran prostor in $F \leq E$ zaprt podprostor. Če sta
$F$ in $\kvoc{E}{F}$ Banachova prostora, je tudi $E$ Banachov.
\end{trditev}

\begin{proof}
Naj bo $(x_n)_{n=1}^\infty$ Cauchyjevo zaporedje v $E$. Očitno je
tedaj tudi zaporedje odsekov Cauchyjevo, zato ima limito $x + F$.
Ker je
\[
\lim_{n \to \infty} \norm{(x_n - x) + F} = 0,
\]
obstaja zaporedje $(y_n)_{n=1}^\infty$ v $F$, za katero je
\[
\lim_{n \to \infty} \norm{x_n - x + y_n} = 0.
\]
Tudi zaporedje $(y_n)_{n=1}^\infty$ je Cauchyjevo -- res, velja
\[
\norm{y_n - y_m} \leq \norm{y_n + x_n - x} +
\norm{x - x_m - y_m} + \norm{x_m - x_n},
\]
vsi členi pa konvergirajo proti $0$. Ker je tudi $F$ poln, ima
zaporedje limito $y$. Ker je
\[
\lim_{n \to \infty} \norm{x_n - x + y} \leq
\lim_{n \to \infty} \norm{x_n - x + y_n} + \norm{y - y_n},
\]
je $y$ limita zaporedja $(x_n)_{n=1}^\infty$.
\end{proof}

\begin{posledica}
Vsak končnorazsežen normiran vektorski prostor je Banachov.
\end{posledica}

\begin{proof}
Predpostavimo, da je $\dim E = 1$. Naj bo $x \in E$ tak, da je
$\norm{x} = 1$. Ker je preslikava $q \colon \K \to E$ s predpisom
$q(\lambda) = \lambda x$ izometrija, je bijektivna. Ker je $\K$
poln, je tak tudi $E$.

Za prostore višjih dimenzij uporabimo indukcijo -- za $x \ne 0$ naj
bo $F = \K \cdot x$. To je očitno podprostor dimenzije $1$. Sledi,
da je poln, torej je zaprt. Velja
\[
\dim \left(\kvoc{E}{F}\right) = \dim E - 1,
\]
zato je $\kvoc{E}{F}$ po indukcijski predpostavki poln.
\end{proof}

\begin{zgled}
Če sta $E$ in $F$ normirana prostora, je $E \times F$ normiran z
normo
\[
\norm{(x,y)} = \max(\norm{x}_E, \norm{y}_E).
\]
Če sta $E$ in $F$ Banachova, je tak tudi $E \times F$.
\end{zgled}

\begin{zgled}
Naj bodo $E_1, \dots, E_r$ normirani prostori. Tedaj sta
\[
\norm{(x_1, \dots, x_r)}_\infty =
\max \setb{\norm{x_i}_{E_i}}{i \leq r}
\quad \text{in} \quad
\norm{(x_1, \dots, x_r)}_p =
\sqrt[p]{\sum_{i=1}^r \norm{x_i}_{E_i}^p}
\]
normi na kartezičnem produktu.
\end{zgled}
