\section{Osnovni izreki o operatorjih med Banachovimi prostori}

\epigraph{">Vsi nekako kimate, samo žal ne vsi v isto smer."<}{
-- prof.~dr.~Igor Klep}

\subsection{Bairov izrek}

\begin{izrek}[Baire]
Naj bo $(X,d)$ poln metrični prostor, $D_n$ pa odprte goste
množice. Tedaj je tudi
\[
\bigcap_{n=1}^\infty D_n
\]
gosta.
\end{izrek}

\begin{proof}
Naj bo $x \in X$ in $r > 0$. Induktivno definiramo zaporedji
$(x_n)_{n=1}^\infty$ in $(r_n)_{n=1}^\infty$, kjer so $x_i \in X$
in $r_i > 0$, in sicer s pogojema
\[
\oline{\mathcal{B}}(x_{n+1}, r_{n+1}) \subseteq D_n \cap
\spr(x_n, r_n)
\quad \text{in} \quad
r_n \leq \frac{1}{n}.
\]
Ni težko videti, da je dobljeno zaporedje središč Cauchyjevo, zato
ima limito $x_0$. Velja pa $d(x_0, x_m) \leq r_m$, zato je
\[
x_0 \subseteq
\spr(x_1, r_1) \cap \bigcap_{m=1}^\infty D_m \subseteq
\spr(x, r) \cap \bigcap_{m=1}^\infty D_m. \qedhere
\]
\end{proof}

\begin{posledica}
Naj bo $X \ne \emptyset$ poln metrični prostor, $A_n \subseteq X$
pa zaprte množice. Če velja
\[
X = \bigcup_{n=1}^\infty A_n,
\]
ima vsaj ena neprazno notranjost.
\end{posledica}

\begin{proof}
Uporabimo zgornji izrek za množice $A_n^{\mathsf{c}}$.
\end{proof}

\begin{opomba}
Izrek ne velja v splošnih metričnih prostorih - protiprimer je že
$X = \Q$ in $D_n = \Q \setminus \set{q_n}$.
\end{opomba}

\newpage

\subsection{Izrek o odprti preslikavi}

\datum{2022-10-26}

\begin{izrek}[O odprti preslikavi]\index{Izrek!O odprti preslikavi}
Naj bosta $X$ in $Y$ Banachova prostora, $T \colon X \to Y$ pa
linearna omejena surjekcija. Tedaj je $T$ odprta preslikava.
\end{izrek}

\begin{proof}
Naj bo $U \subseteq X$ odprta množica in $0 \in U$. Obstaja torej
tak $\varepsilon > 0$, da je
$\oline{\mathcal{B}}(0, \varepsilon) \subseteq U$. Sledi, da za
$x \in X \setminus \set{0}$ velja
$x \in \frac{\norm{x}}{\varepsilon} \in U$, zato je
\[
X = \bigcup_{n=1}^\infty nU.
\]
Dobimo
\[
Y = T(X) = \bigcup_{n=1}^\infty T(nU) =
\bigcup_{n=1}^\infty n T(U) = \bigcup_{n=1}^\infty n \oline{T(U)}.
\]
Po Baireovem izreku ima ena izmed množic $m \oline{T(U)}$ neprazno
notranjost, zato ima $\oline{T(U)}$ notranjo točko.

Naj bo $V = \spr\left(0, \frac{\varepsilon}{2}\right)$. Očitno je
\[
V-V = \setb{v-w}{v, w \in V} \subseteq
\spr(0, \varepsilon) \subseteq U.
\]
Po enakem argumentu kot zgoraj ima $\oline{T(V)}$ notranjo točko,
torej obstaja odprta množica $W \subseteq \oline{T(V)}$. Velja
\begin{align*}
W-W &\subseteq \oline{T(V)} - \oline{T(V)}
\\
&= s\left(\oline{T(V)} \times \oline{T(V)}\right)
\\
&= s\left(\oline{T(V) \times T(V)}\right)
\\
&\subseteq \oline{T(V) - T(V)}
\\
&= \oline{T(V-V)} \subseteq \oline{T(U)},
\end{align*}
kjer $s$ označuje razliko. Vidimo, da je
\[
W-W = \bigcup_{w \in W}(W-w)
\]
odprta množica in velja $0 \in (W-W)$. Sledi, da je $0$ notranja v
$\oline{T(U)}$.

Naj bo sedaj $\varepsilon > 0$,
$\varepsilon_0 = \frac{\varepsilon}{2}$ in
\[
\varepsilon_0 = \sum_{i=1}^\infty \varepsilon_i,
\]
kjer so $\varepsilon_i > 0$. Po prejšnjem delu dokaza za vsak
$i \in \N_0$ obstaja tak $\eta_i > 0$, da je
\[
\spr(0, \eta_i) \subseteq \oline{T\br{\spr(0, \varepsilon_i)}}.
\]
Če je $\norm{x} < \varepsilon_i$, je
$\norm{Tx} < \varepsilon_i \norm{T}$. Velja torej
\[
T(\spr(0, \varepsilon_i)) \subseteq
\spr(0, \varepsilon_i \norm{T}).
\]
Sledi, da je $0 < \eta_i \leq \varepsilon_i \norm{T}$, zato
$\eta_i$ konvergirajo k $0$.

Naj bo $y \in \spr(0, \eta_0) \subseteq
\oline{T\br{\spr(0, \varepsilon_0)}}$. Oglejmo si
$\spr(y, \eta_1)$. Ta seka $T(\spr(0, \varepsilon_0))$ -- obstaja
$x_0$, za katerega je $\norm{x_0} < \varepsilon_0$ in
$\norm{y - Tx_0} < \eta_1$. Velja torej, da je
$y-Tx_0 \in \spr(0, \eta_1) \subseteq
\oline{T\br{\spr(0, \varepsilon_1)}}$. Postopek lahko nadaljujemo
induktivno -- dobimo zaporedje $(x_n)_{n=1}^\infty$, za katere je
$\norm{x_n} < \varepsilon_i$ in
\[
\norm{y - \sum_{i=0}^n Tx_i} < \eta_{n+1}.
\]
Oglejmo si zaporedje delnih vsot
\[
s_n = \sum_{i=0}^n x_i.
\]
Ker je
\[
\norm{s_n - s_m} = \norm{\sum_{i=m+1}^n x_i} \leq
\sum_{i=m+1}^n \norm{x_i} < \sum_{i=m+1}^n \varepsilon_i,
\]
je to zaporedje Cauchyjevo in vrsta
\[
\sum_{i=0}^\infty x_i
\]
konvergira k $x$. Vidimo še, da velja
\[
\norm{x} \leq \sum_{i=0}^\infty \norm{x_i} < \varepsilon
\]
in
\[
y = \sum_{i=0}^\infty Tx_i = T\br{\sum_{i=0}^\infty x_i} = Tx.
\]
Sledi, da je $T\br{\spr(0,\varepsilon)}$ okolica $0$, zato je
$0$ notranja točka $T(U)$.

Sedaj lahko splošen izrek dobimo s preprosto translacijo.
\end{proof}

\begin{posledica}\label{ps:od:1}
Naj bosta $X$ in $Y$ Banachova prostora, $T \colon X \to Y$ pa
omejena linearna bijekcija. Tedaj je tudi $T^{-1}$ omejen.
\end{posledica}

\begin{proof}
Po izreku o odprti preslikavi je $T$ odprta, zato je $T^{-1}$
zvezna.
\end{proof}

\begin{posledica}
Naj bo $X$ vektorski prostor, ki je v normah $\norm{.}_1$ in
$\norm{.}_2$. Če obstaja tak $c > 0$, da za vse $x \in X$ velja
$\norm{x}_1 \leq c \norm{x}_2$, sta normi ekvivalentni.
\end{posledica}

\begin{proof}
Preslikava $\id \colon (X, \norm{.}_1) \to (X, \norm{.}_2)$ ustreza
pogojem posledice~\ref{ps:od:1}.
\end{proof}

\newpage

\subsection{Princip enakomerne omejenosti}

\begin{izrek}[Banach-Steinhaus]\index{Izrek!Banach-Steinhaus}
\label{iz:bs:1}
Naj bo $X$ Banachov, $Y$ pa normiran prostor. Naj bo
$\mathcal{A} \subseteq \mathcal{B}(X,Y)$ taka množica, da je za
vsak $x \in X$ množica $\setb{\norm{Tx}}{T \in \mathcal{A}}$
omejena.\footnote{Temu pogoju pravimo tudi \emph{omejenost po
točkah}.} Tedaj je $\mathcal{A}$ omejena.
\end{izrek}

\begin{proof}
Za $n \in N$ naj bo
\[
A_n = 
\setb{x \in X}{\forall T \in \mathcal{A} \colon \norm{Tx} \leq n} =
\bigcap_{T \in \mathcal{A}} \setb{x \in X}{\norm{Tx} \leq n}.
\]
Očitno so množice $A_n$ zaprte, velja pa
\[
\bigcup_{n=1}^\infty A_n = X.
\]
Po Baireovem izreku obstajajo taki $n_0 \in N$, $x_0 \in X$ in
$\varepsilon > 0$, da velja
\[
\spr(x_0, 2\varepsilon) \subseteq A_{n_0}.
\]
Vzemimo $x \in X$, za katerega velja $\norm{x} = 1$. Za vsak
$T \in \mathcal{A}$ velja
\begin{align*}
\norm{Tx} &= \frac{1}{\varepsilon} \norm{T(\varepsilon x)}
\\
&= \frac{1}{\varepsilon} \norm{Tx_0 - T(x_0 - \varepsilon x)}
\\
&\leq \frac{1}{\varepsilon} \norm{Tx_0} +
\frac{1}{\varepsilon} \norm{T(x_0 - \varepsilon x)}
\\
&\leq \frac{2n_0}{\varepsilon}. \qedhere
\end{align*}
\end{proof}

\begin{trditev}
Naj bo $X$ normiran prostor. Tedaj lahko $X$ vložimo v $X^{**}$.
\end{trditev}

\begin{proof}
Vsakemu $x$ priredimo operator $F_x \colon X^* \to \K$, ki slika po
predpisu $f \mapsto f(x)$. Očitno je $F_x$ linearen, velja pa
\[
\abs{F_x(f)} = \abs{f(x)} \leq \norm{x} \cdot \norm{f},
\]
zato je $F_x$ tudi omejen -- velja $\norm{F_x} \leq \norm{x}$. Po
posledici~\ref{ps:hb:1} obstaja tak $f \in X^*$, da je $\norm{f}=1$
in $f(x) = \norm{x}$. Sledi, da je
\[
F_x(f) = \norm{x},
\]
zato je $\norm{F_x} = \norm{x}$.
\end{proof}

\begin{trditev}
Zgornja vložitev $\iota$ je linearna izometrija.
\end{trditev}

\begin{proof}
Vložitev je izometrija po prejšnji trditvi. Velja pa
\[
F_{x+y}(f) = f(x+y) = (F_x + F_y)(f)
\]
in
\[
F_{\lambda x}(f) = f(\lambda x) = (\lambda F_x)(f). \qedhere
\]
\end{proof}

\begin{definicija}
Normiran prostor je
\emph{refleksiven}\index{Vektorski prostor!Refleksiven}, če je
$\iota \colon X \to X^{**}$ surjektivna.
\end{definicija}

\begin{opomba}
Vsak refleksiven prostor je Banachov.
\end{opomba}

\begin{zgled}
Prostori $\ell^p$ za $p>1$ so refleksivni. Refleksivni so tudi vsi
končnorazsežni vektorski prostori.
\end{zgled}

\begin{zgled}
Prostor $c_0$ ni refleksiven, čeprav je Banachov.
\end{zgled}

\begin{opomba}
Obstaja Banachov prostor $X$, za katerega je $X \cong X^{**}$, a ni
refleksiven.
\end{opomba}

\begin{izrek}
Naj bo $X$ normiran prostor in $A \subseteq X$. Denimo, da za vsak
$f \in X^*$ obstaja tak $k_f \in \R^+$, da za vse $x \in A$
velja\footnote{Pravimo, da je $A$ \emph{šibko omejena}.}
\[
\abs{f(x)} \leq k_f.
\]
Tedaj je $A$ omejena.
\end{izrek}

\begin{proof}
Za vse $x \in A$ velja
\[
\abs{F_x(f)} = \abs{f(x)} \leq k_f.
\]
Množica
\[
\mathcal{A} = \setb{F_x}{x \in A} = \iota(A)
\]
je podmnožica Banachovega prostora $X^{**}$. Po predpostavki je
$\mathcal{A}$ omejena po točkah, zato je po izreku~\ref{iz:bs:1}
omejena.
\end{proof}

\begin{posledica}
Naj bo $X$ Banachov, $Y$ pa normiran prostor. Naj bo
$\mathcal{A} \subseteq \mathcal{B}(X,Y)$ taka množica, da za vse
$f \in Y^*$ in $x \in X$ obstaja tak $k(f,x) \in \R^+$, da za vse
$T \in \mathcal{A}$ velja
\[
\abs{f(Tx)} \leq k(f,x).
\]
Tedaj je $\mathcal{A}$ omejena.
\end{posledica}

\begin{proof}
Za poljuben $x \in X$ je množica $A_x=\setb{Tx}{T \in \mathcal{A}}$
šibko omejena, zato je po zgornjem izreku omejena. Sledi, da je
$\mathcal{A}$ omejena po točkah, zato je po izreku~\ref{iz:bs:1}
omejena.
\end{proof}

\newpage

\subsection{Izrek o zaprtem grafu}

\datum{2023-10-27}

\begin{definicija}
\emph{Graf}\index{Funkcija!Graf} funkcije $f \colon X \to Y$ je
množica
\[
\Gamma(f) = \setb{\br{x, f(x)}}{x \in X}.
\]
\end{definicija}

\begin{izrek}[O zaprtem grafu]\index{Izrek!O zaprtem grafu}
Naj bosta $X$ in $Y$ Banachova prostora in $T \colon X \to Y$
linearen operator. Potem je $T$ omejen natanko tedaj, ko je
$\Gamma(T)$ zaprta podmnožica $X \times Y$.
\end{izrek}

\begin{proof}
Ker je graf zvezne funkcije vedno zaprt, je dovolj pokazati obratno
implikacijo. Denimo torej, da je $\Gamma(T)$ zaprta. Ni težko
videti, da je $\Gamma(T)$ vektorski podprostor v $X \times Y$. Ker
je $X \times Y$ Banachov, je tak tudi $\Gamma(T)$.

Projekcija $\pi_1 \colon X \times Y \to X$ je omejena linearna
surjekcija. Ni težko videti, da je $P = \eval{\pi_1}{\Gamma(T)}{}$
omejena, linearna in bijektivna. Po izreku o odprti preslikavi
sledi, da obstaja omejena preslikava
$P^{-1} \colon X \to \Gamma(T)$. Sledi, da je
$T = \pi_2 \circ P^{-1}$ kompozitum omejenih preslikav, zato je
omejen.
\end{proof}
