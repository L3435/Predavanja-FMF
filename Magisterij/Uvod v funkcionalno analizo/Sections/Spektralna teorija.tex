\section{Spektralna teorija}

\epigraph{">Če kdo to reši, bo dobil poleg časti in slave verjetno
še intervju na RTV."<}{-- prof.~dr.~Igor Klep}

\subsection{Splošna teorija}

\begin{definicija}
Naj bo $X$ kompleksen Banachov prostor in $A \in \mathcal{B}(X)$.
\emph{Resolventna množica}\index{Operator!Resolventna množica} je
\[
\rho(A) = \setb{\lambda \in \C}
{\text{$\lambda \cdot I - A$ ima inverz v $\mathcal{B}(X)$}}.
\]
Označimo
\[
R(\lambda, A) = (\lambda I - A)^{-1}.
\]
Funkciji $\lambda \mapsto R(\lambda,A)$ pravimo \emph{resolventa},
množici
\[
\sigma(A) = \operatorname{sp}(A) = \C \setminus \rho(A)
\]
pa \emph{spekter}\index{Operator!Spekter} operatorja $A$.
\end{definicija}

\begin{opomba}
Za $\lambda \in \C$ velja $\lambda \in \rho(A)$ natanko tedaj, ko
je $\lambda I - A$ bijektiven.
\end{opomba}

\begin{definicija}
Število $\lambda \in \C$ je
\emph{lastna vrednost}\index{Operator!Lastna vrednost} operatorja
$A$, če je $\ker(\lambda I - A) \ne \set{0}$.
\emph{Točkasti spekter} je množica
\[
\sigma_p (A) =
\setb{\lambda \in \C}{\ker(\lambda I - A) \ne \set{0}}.
\]
\end{definicija}

\begin{opomba}
Če je $X$ končnorazsežen, velja $\sigma(A) = \sigma_p(A)$.
\end{opomba}

\begin{zgled}
Naj bo $S \colon \ell^2 \to \ell^2$ premik v desno. Tedaj je očitno
$0 \in \sigma(S)$ in $0 \not \in \sigma_p(S)$.
\end{zgled}

\begin{trditev}
Naj bo $H$ Hilbertov prostor in $A \in \mathcal{B}(H)$. Tedaj je
\[
\sigma(A^*) = \oline{\sigma(A)}.
\]
\end{trditev}

\obvs

\begin{trditev}
Naj bo $X$ Banachov prostor in $T \in \mathcal{B}(X)$. Naj bo $M$
množica lastnih vektorjev za $T$, ki pripadajo paroma različnim
lastnim vrednostim. Tedaj je $M$ linearno neodvisna.
\end{trditev}

\begin{proof}
Denimo, da je
\[
\sum_{k=1}^n \alpha_k x_k = 0,
\]
za neničelne skalarje $\alpha_k$, kjer je $Tx_k = \lambda_k x_k$.
Pri tem vzamemo najmanjšo tako kombinacijo. Sledi, da je tudi
\[
\sum_{k=1}^n \alpha_k \lambda_k x_k = 0,
\]
zato je
\[
\sum_{k=1}^{n-1} \alpha_k (\lambda_k - \lambda_n) x_k = 0,
\]
kar je protislovje.
\end{proof}

\begin{trditev}
Naj bo $A \in \mathcal{B}(H)$ sebiadjungiran operator. Tedaj so
njegove lastne vrednosti realne.
\end{trditev}

\begin{proof}
Velja
\[
\lambda \skl{x,x} = \skl{Ax,x} =
\skl{x,Ax} = \oline{\lambda} \skl{x,x}. \qedhere
\]
\end{proof}

\begin{trditev}
Naj bo $A \in \mathcal{B}(H)$ sebiadjungiran operator. Tedaj so
lastni vektorji različnih lastnih vrednosti ortogonalni.
\end{trditev}

\begin{proof}
Naj bo $Ax = \lambda x$ in $Ay = \mu y$ za $\lambda, \mu \in \C$ in
$x,y \ne 0$. Tedaj velja
\[
\lambda \skl{x,y} = \skl{Ax,y} = \skl{x,Ay} = \mu \skl{x,y}.
\qedhere
\]
\end{proof}

\begin{definicija}
Naj bo $X$ Banachov prostor nad $\C$, $A \in \mathcal{B}(X)$
operator in $\lambda \in \sigma(A)$.

\begin{enumerate}[i)]
\item $\lambda$ je v \emph{zveznem spektru} operatorja $A$, če je
$\lambda I - A$ injektiven in je $\im(\lambda I - A)$ gosta v $X$.
Zvezni spekter označimo s $\sigma_c(A)$.
\item $\lambda$ je v \emph{residualnem spektru} operatorja $A$, če
je $\lambda I - A$ injektiven in je $\im(\lambda I - A)$ ni gosta v
$X$. Residualni spekter označimo s $\sigma_r(A)$.
\end{enumerate}
\end{definicija}

\datum{2022-12-1}

\begin{trditev}
Naj bo $X$ Banachov prostor, $A \in \mathcal{B}(X)$ operator in
$\lambda \in \sigma_c(A)$. Tedaj obstaja
zaporedje\footnote{Zaporedje aproksimativnih lastnih vektorjev.}
$(x_n)_n \subseteq X$, za katerega je $\norm{x_n} = 1$ za vsak $n$
in je
\[
\lim_{n \to \infty} \norm{Ax_n - \lambda x_n} = 0.
\]
\end{trditev}

\begin{proof}
Ker je $\lambda I - A$ injektiven, ima inverz
$(\lambda I - A)^{-1} \colon \im (\lambda I - A) \to X$.
Predpostavimo, da je ta operator omejen. Sledi, da ga lahko
razširimo do operatorja $B \colon X \to X$, za katerega je
$\eval{B}{\im (\lambda I - A)}{} = (\lambda I - A)^{-1}$. Pri tem
upoštevamo, da je $X = \oline{\im (\lambda I - A)}$. Ker je
operator $(\lambda I - A)B$ omejen in se z $I$ ujema na
$\im (\lambda I - A)$, sledi, da je $(\lambda I - A)B = I$. Podobno
sklepamo $B(\lambda I - A) = I$, kar pa je protislovje, saj
$\lambda I - A$ ni obrnljiv.

Sledi, da obstaja tako zaporedje $(w_n)_n \in \im (\lambda I - A)$,
da za vsak $n$ velja $\norm{w_n} = 1$ in
\[
\norm{(\lambda I - A)^{-1}w_n} \geq n.
\]
Naj bo
\[
x_n =
\frac{(\lambda I - A)^{-1}w_n}{\norm{(\lambda I - A)^{-1}w_n}}.
\]
Dobimo
\[
\norm{(\lambda I - A) x_n} =
\frac{\norm{w_n}}{\norm{(\lambda I - A)^{-1}w_n}} \leq
\frac{1}{n}. \qedhere
\]
\end{proof}

\begin{trditev}
Naj bo $H$ Hilbertov prostor in $A \in \mathcal{B}(H)$.

\begin{enumerate}[i)]
\item Če je $\lambda \in \sigma_r(A)$, je
$\oline{\lambda} \in \sigma_p(A^*)$.
\item Če je $\lambda \in \sigma_p(A)$, je
$\oline{\lambda} \in \sigma_p(A^*) \cup \sigma_r(A^*)$.
\end{enumerate}
\end{trditev}

\begin{proof}
Naj bo $\lambda \in \sigma_r(A)$ in
$Y = \oline{\im (\lambda I - A)}$. Sledi, da je
\[
\set{0} \ne
Y^\bot =
\oline{\im(\lambda I - A)}^\bot =
\ker ((\lambda I - A)^*) =
\ker ((\oline{\lambda} I - A^*)).
\]
Naj bo sedaj $\lambda \in \sigma_p(A)$. Tedaj je
\[
\set{0} \ne \ker (\lambda I - A) =
\im (\oline{\lambda} I - A^*)^\bot,
\]
zato je $\oline{\lambda} \in \sigma(A^*)$ in
$\oline{\lambda} \not \in \sigma_c(A^*)$.
\end{proof}

\begin{posledica}
Če je $A = A^*$, je $\sigma_r(A) = \emptyset$.
\end{posledica}

\begin{proof}
Za $\lambda \in \sigma_r(A)$ velja
$\oline{\lambda} \in \sigma_p(A^*) = \sigma_p(A) \subseteq \R$,
zato je $\lambda = \oline{\lambda}$, kar je protislovje.
\end{proof}

\begin{izrek}
Za $A = A^* \in \mathcal{B}(H)$ velja $\sigma(A) \subseteq \R$.
\end{izrek}

\begin{proof}
Vemo že, da je $\sigma_r(A) = \emptyset$ in
$\sigma_p(A) \subseteq \R$. Naj bo
$\lambda_1 + i \lambda_2 \in \sigma_c(A)$. Obstaja torej zaporedje
normiranih vektorjev $(x_n)_n$, za katerega je
\[
\lim_{n \to \infty} (\lambda I - A) x_n = 0.
\]
Velja pa
\[
\norm{(\lambda I - A) x_n} \geq
\abs{\skl{(\lambda I - A) x_n, x_n}} =
\abs{\lambda_1 + i \lambda_2 - \skl{Ax_n, x_n}} \geq
\abs{\lambda_2},
\]
zato je $\lambda_2 = 0$.
\end{proof}

\begin{izrek}[Von Neumannova vsota]
\index{Izrek!Von!Neumannova vsota}
Naj bo $X$ Banachov prostor in $A \in \mathcal{B}(X)$ operator, za
katerega je $\norm{A} < 1$. Tedaj je $(I-A)$ obrnljiv operator in
velja
\[
(I-A)^{-1} = \sum_{n=0}^\infty A^n.
\]
\end{izrek}

\begin{proof}
Vrsta je absolutno konvergentna, zato je zgornji izraz dobro
definiran. Ni težko preveriti, da je res inverz.
\end{proof}

\begin{trditev}
Naj bo $X$ Banachov prostor in $A \in \mathcal{B}(X)$. Tedaj je
$\rho(A)$ odprta v $\C$.
\end{trditev}

\begin{proof}
Naj bo $\lambda_0 \in \rho(A)$. Tedaj velja
\[
\lambda I - A =
(\lambda_0 I - A)(I - (\lambda_0 - \lambda) \cdot R(\lambda_0, A)).
\]
Če je $\norm{(\lambda_0 - \lambda) \cdot R(\lambda_0, A)} < 1$,
je drugi faktor obrnljiv. Ker je $(\lambda_0 I - A)$ obrnljiv po
predpostavki, je $\lambda \in \rho(A)$.
\end{proof}

\begin{posledica}
Za vsak $A \in \mathcal{B}(X)$ je $\sigma(A)$ zaprta.
\end{posledica}

\begin{trditev}
Za vsaka $A \in \mathcal{B}(X)$ in $\lambda \in \sigma(A)$ velja
$\abs{\lambda} \leq \norm{A}$.
\end{trditev}

\begin{proof}
Naj bo $\lambda \in \C$ število, za katerega je
$\abs{\lambda} > \norm{A}$. Tedaj je
\[
\lambda I - A = \lambda \br{I - \frac{A}{\lambda}},
\]
ki je očitno obrnljiv, zato je $\lambda \in \rho(A)$.
\end{proof}

\begin{posledica}
Za $A \in \mathcal{B}(X)$ je $\sigma(A)$ kompaktna podmnožica $\C$.
\end{posledica}

\begin{trditev}
Resolventa operatorja $A$ je zvezna funkcija.
\end{trditev}

\begin{proof}
Za vse $\lambda$ blizu $\lambda_0$ velja
\[
R(\lambda, A) =
(I - (\lambda_0 - \lambda) \cdot R(\lambda_0, A))^{-1} \cdot
R(\lambda_0, A) =
\sum_{n=0}^\infty (\lambda_0 - \lambda)^n \cdot
R(\lambda_0, A)^{n+1},
\]
zato je
\begin{align*}
\norm{R(\lambda, A) - R(\lambda_0, A)} &=
\norm{\sum_{n=1}^\infty (\lambda_0 - \lambda)^n \cdot
R(\lambda_0, A)^{n+1}}
\\
&\leq
\sum_{n=1}^\infty \abs{\lambda_0 - \lambda}^n \cdot
\norm{R(\lambda_0, A)}^{n+1}
\\
&=
\abs{\lambda_0 - \lambda} \cdot \norm{R(\lambda_0, A)}^2 \cdot
\frac{1}{1 - \abs{\lambda_0 - \lambda} \norm{R(\lambda_0, A)}}.
\qedhere
\end{align*}
\end{proof}

\begin{izrek}
Za vsak $A \in \mathcal{B}(X)$ je $\sigma(A)$ neprazna kompaktna
podmnožica $\C$.
\end{izrek}

\begin{proof}
Za $\abs{\lambda} > \norm{A}$ velja
\[
\norm{R(\lambda, A)} =
\norm{\sum_{n=0}^\infty \frac{A^n}{\lambda^{n+1}}} \leq
\frac{1}{\abs{\lambda}} \cdot
\frac{1}{1 - \frac{\norm{A}}{\abs{\lambda}}}.
\]
Sledi, da je
\[
\lim_{\abs{\lambda} \to \infty} \norm{R(\lambda, A)} = 0.
\]
Za poljuben $f^* \in \mathcal{B}(X)^*$ je
$g(\lambda) = f(R(\lambda, A))$ holomorfna, saj velja
\[
R(\lambda, A) - R(\mu, A) =
(\mu - \lambda) R(\mu, A) \cdot R(\lambda, A)
\]
in zato
\[
\frac{f(R(\lambda, A)) - f(R(\mu, A))}{\lambda - \mu} =
- f(R(\mu, A) \cdot R(\lambda, A)).
\]
Sledi, da je $g$ omejena holomorfna funkcija. Če je
$\sigma(A) = \emptyset$, je $g$ tako cela funkcija. Po
Liouvilleovem izreku je omejena in zato konstantno enaka $0$. Ker
pa po Hahn-Banachovem izreku obstaja $f \in \mathcal{B}(X)^*$, za
katerega je $\norm{f} = 1$ in
$g(\lambda) = f(R(\lambda, A)) = \norm{R(\lambda, A)} \ne 0$,
to ni mogoče.
\end{proof}

\datum{2022-12-7}

\begin{opomba}
Za vsak neprazen kompakt $K \subseteq \C$ obstaja operator
$A \in \mathcal{B}(H)$, za katerega je $\sigma(A) = K$.
\end{opomba}

\begin{definicija}
\emph{Spektralni polmer}\index{Operator!Spekter!Polmer} operatorja
$A \in \mathcal{B}(X)$ je
\[
r(A) = \max \setb{\abs{\lambda}}{\lambda \in \sigma(A)}.
\]
\end{definicija}

\begin{zgled}
Velja $r(I) = 1$.
\end{zgled}

\begin{lema}
Velja
\[
\limsup_n \sqrt[n]{\norm{A^n}} \leq r(A).
\]
\end{lema}

\begin{proof}
Za $\lambda > \norm{A}$ velja
\[
R(\lambda,A) = (\lambda I - A)^{-1} =
\sum_{n=0}^\infty \frac{A^n}{\lambda^{n+1}}.
\]
Naj bo $f \in \mathcal{B}(X)^*$ in $g(\lambda) = f(R(\lambda, A))$.
Za $\abs{\lambda} > \norm{A}$ tako sledi
\[
g(\lambda) = \sum_{n=0}^\infty \frac{f(A^n)}{\lambda^{n+1}},
\]
kar je Laurentova vrsta. Ker je $g$ definirana za vse
$\abs{\lambda} > r(A)$, vrsta konvergira za te $\lambda$.

Naj bo $\varepsilon > 0$. Tedaj vrsta
\[
\sum_{n=0}^\infty \frac{f(A^n)}{(r+\varepsilon)^{n+1}}
\]
konvergira in je
\[
\lim_{n \to \infty} \frac{f(A^n)}{(r+\varepsilon)^{n+1}} = 0.
\]
Množica
\[
\setb{\frac{A^n}{(r+\varepsilon)^{n+1}}}{n \in \N}
\]
je zato šibko omejena, zato je omejena. Sledi, da je
\[
\norm{A^n} < c \cdot (r+\varepsilon)^{n+1},
\]
oziroma
\[
\sqrt[n]{\norm{A^n}} <
\sqrt[n]{c} \cdot (r+\varepsilon)^{\frac{n+1}{n}}. \qedhere
\]
\end{proof}

\begin{trditev}
Če je $\lambda \in \sigma(A)$, je $\lambda^n \in \sigma(A^n)$.
\end{trditev}

\begin{proof}
Denimo, da je $\lambda^n I - A^n$ obrnljiv. Velja
\[
\lambda^n I - A^n =
(\lambda I - A)\br{\sum_{k=0}^{n-1} \lambda^k A^{n-1-k}} =
\br{\sum_{k=0}^{n-1} \lambda^k A^{n-1-k}}(\lambda I - A).
\]
Sledi, da je tudi $(\lambda I - A)$ obrnljiv.
\end{proof}

\begin{lema}
Za vsak $n \in \N$ velja
\[
\sqrt[n]{\norm{A^n}} \geq r(A).
\]
\end{lema}

\obvs

\begin{izrek}[Gelfandova formula]\index{Izrek!Gelfandova formula}
Naj bo $A \in \mathcal{B}(X)$. Tedaj velja
\[
r(A) = \lim_{n \to \infty} \sqrt[n]{\norm{A^n}}.
\]
\end{izrek}

\begin{proof}
Velja
\[
\limsup_n \sqrt[n]{\norm{A^n}} \leq
r(A) \leq
\liminf_n \sqrt[n]{\norm{A^n}}. \qedhere
\]
\end{proof}

\begin{posledica}
Naj bo $A \in \mathcal{B}(H)$ sebiadjungiran operator. Tedaj je
$r(A) = \norm{A}$.
\end{posledica}

\begin{proof}
Induktivno pokažemo
\[
\norm{A^{2^n}} = \norm{A^{2^{n-1}} \cdot \br{A^{2^{n-1}}}^*} =
\norm{A^{2^{n-1}}}^2 = \norm{A}^{2^n}.
\]
Sledi, da je
\[
r(A) = \lim_{n \to \infty} \sqrt[n]{\norm{A^n}} =
\lim_{n \to \infty} \sqrt[2^n]{\norm{A^{2^n}}} =
\norm{A}. \qedhere
\]
\end{proof}

\newpage

\subsection{Spekter kompaktnega operatorja}

\begin{trditev}
Naj bo $H$ Hilbertov prostor in $K \in \mathcal{K}(H)$ kompakten
operator. Za $\lambda \ne 0$ ima $T = \lambda I - K$ naslednje
lastnosti:

\begin{enumerate}[i)]
\item Velja $\dim \ker T < \infty$.
\item $\im T$ je zaprt podprostor.
\end{enumerate}
\end{trditev}

\begin{proof}
Ker velja
\[
T = \br{I - \frac{1}{\lambda} K},
\]
imata $T$ in $I - \frac{1}{\lambda} K$ enako jedro in sliko. Brez
škode za splošnost naj bo tako $\lambda = 1$.

Naj bo $Y = \ker T \leq H$ zaprt podprostor v $H$. Sledi, da za vse
$y \in Y$ velja $Ky = y$. Sledi, da je
$\eval{K}{Y}{} = \eval{\id}{Y}{}$ kompakten, zato je
$\dim Y < \infty$.

Naj bo $Z = Y^\bot$. Velja, da je $Z$ zaprt podprostor in je
$H = Y \oplus Z$. Operator $\eval{T}{Z}{}$ je omejen injektiven
operator -- če velja $\br{\eval{T}{Z}{}}z = 0$, je namreč
$z \in \ker T \cap Z$, torej je $z = 0$. Očitno velja tudi
$\im \eval{T}{Z}{} = \im T$. Preslikava
$\eval{T}{Z}{} \colon Z \to \im T$ je zato bijektivna in ima inverz
$S$.

Denimo, da operator $S$ ni omejen. Sledi, da za vsak $n \in \N$
obstaja tak $w_n = Tz_n$, da je $\norm{w_n} = 1$ in
$\norm{z_n} \geq n$.

Naj bo $u_n = \frac{z_n}{\norm{z_n}}$. Velja
\[
\norm{Tu_n} = \frac{1}{\norm{z_n}} \cdot \norm{Tz_n} \leq
\frac{1}{n}.
\]
Zaporedje $(Tu_n)_n$ tako konvergira k $0$. Zaporedje $(u_n)_n$ je
omejeno, ima $(Ku_n)_n$ konvergentno podzaporedje $\br{Ku_{n_k}}_k$
z limito $y$. Sledi, da je
\[
u_{n_k} = T u_{n_k} + K u_{n_k},
\]
kar konvergira k $y$. Ker je $Z$ zaprt, je $y \in Z$. Ker so $u_n$
normirani, je $\norm{y} = 1$. Zaporedje $\br{T u_{n_k}}_k$
tako konvergira k $Ty = 0$, zato je $y \in Y \cap Z = \set{0}$,
kar je protislovje.
\end{proof}

\begin{trditev}
Naj bo $H$ Hilberetov prostor, $K \in \mathcal{K}(H)$ pa kompakten
operator. Potem je
\[
\sigma(K) \setminus \set{0} \subseteq \sigma_p(K).
\]
\end{trditev}

\begin{proof}
Denimo, da je $\lambda \in \C \setminus \set{0}$ in
$\lambda \not \in \sigma_p(K)$. Naj bo $S = \lambda I - K$. Sledi,
da je $S$ injektiven, po prejšnji trditvi pa je $\im S \leq H$
zaprt podprostor.

Denimo, da je $H_1 = SH \subset H = H_0$. Za $H_n = SH_{n-1}$ tako
velja
\[
H_0 \supset H_1 \supset H_2 \supset \dots,
\]
prostori $H_j$ pa so zaprti. Za vsak $n$ tako obstaja tak
$Z_{n+1}$, da je
\[
H_n = H_{n+1} \oplus Z_{n+1}.
\]
Naj bo $x_n \in Z_{n+1}$ poljuben enotski vektor. Ti tvorijo
omejeno zaporedje, zato ima $(Kx_n)_n$ konvergentno podzaporedje
$\br{Kx_{n_k}}_k$. Za $a < b$ velja
\[
\frac{1}{\lambda} \br{Kx_{n_a} - Kx_{n_b}} =
\br{I - \frac{1}{\lambda} S} x_{n_a} -
\br{I - \frac{1}{\lambda} S} x_{n_b} =
x_{n_a} + \br{-x_{n_b} +
\frac{1}{\lambda} S \br{x_{n_a} - x_{n_b}}}.
\]
Ker je $x_{n_a} \in Z_{n_a+1}$,
$x_{n_k} \in H_{n_b} \subseteq H_{n_a+1}$ in
$S \br{x_{n_a} - x_{n_b}} \in H_{n_a+1}$ in je
$x_{n_a} \perp H_{n_a + 1}$, po Pitagorovem izreku sledi
\[
\norm{\frac{1}{\lambda} \br{Kx_{n_a} - Kx_{n_b}}}^2 =
\norm{x_{n_a}}^2 +
\norm{-x_{n_b} + \frac{1}{\lambda} S \br{x_{n_a} - x_{n_b}}}^2 \geq
1,
\]
kar je v protislovju s predpostavko, da je $(K x_{n_k})_k$
konvergentno.
\end{proof}

\datum{2022-12-8}

\begin{izrek}
Naj bo $H$ kompleksen Hilbertov prostor in $K \in \mathcal{K}(H)$.

\begin{enumerate}[i)]
\item Vsako od $0$ različno število v $\sigma(K)$ je lastna
vrednost za $K$, pripadajoč lastni podprostor pa je končnorazsežen.
\item Če je $\dim H = \infty$, je $0 \in \sigma(K)$.
\item $\sigma(K)$ je števna množica brez neničelnih stekališč.
\end{enumerate}
\end{izrek}

\begin{proof}
\phantom{a}
\begin{enumerate}[i)]
\item Posledica prejšnjih dveh trditev.
\item Če je $0 \in \rho(K)$, je $K$ obrnljiv. Ker so kompaktni
operatorji ideal v omejenih operatorjih, je tudi
$I = K \cdot K^{-1}$ kompakten operator, zato je $\dim H < \infty$.
\item Denimo, da ima $\sigma(K)$ neničelno stekališče $\lambda$.
Naj bo $(\lambda_n)_n$ zaporedje paroma različnih neničelnih
elementov $\sigma$, ki konvergirajo k $\lambda$. Po prvi točki so
$\lambda_n$ lastne vrednosti operatorja $K$, zato lahko tvorimo
zaporedje lastnih vektorjev $(x_n)_n$. To zaporedje je linearno
neodvisno.

Naj bo $X_n = \Lin \setb{x_i}{1 \leq i \leq n}$ zaprt podprostor v
$H$. Ker ti tvorijo strogo naraščajočo verigo, za vsak $n$ obstaja
enotski vektor $y_{n+1} \in X_{n+1}$, za katerega je
$y_{n+1} \perp X_n$. Za poljuben $\mu \in \C$ naj bo
$S_\mu = \mu I - K$.

Velja
\[
S_{\lambda_n} x_k =
\lambda_n x_k - Kx_k =
(\lambda_n - \lambda_k) x_k,
\]
zato za vse $n, k \in \N$ velja $S_{\lambda_n} X_k \subseteq X_k$
in $S_{\lambda_n} X_n \subseteq X_{n-1}$. Tako lahko za $n > m$
izrazimo
\[
\frac{1}{\lambda_n} K y_n - \frac{1}{\lambda_m} K y_m =
\br{I - \frac{1}{\lambda_n} S_{\lambda_n}} y_n -
\br{I - \frac{1}{\lambda_m} S_{\lambda_m}} y_m =
y_n - z,
\]
kjer je $z \in X_{n-1}$. Po Pitagorovem izreku tako dobimo
\[
\norm{\frac{1}{\lambda_n} K y_n - \frac{1}{\lambda_m} K y_m}^2 =
\norm{y_n}^2 + \norm{z}^2 \geq
1.
\]
Ker je $K$ kompakten operator, ima zaporedje $(K y_n)_n$
konvergentno podzaporedje. To torej velja tudi za
$\br{\frac{1}{\lambda_n} K y_n}_n$, kar je seveda protislovje.

Za vsak $n$ je
\[
A_n = \setb{z \in \sigma{K}}{\abs{z} \geq \frac{1}{n}}
\]
kompaktna množica brez stekališč, zato je končna. Množico
$\sigma(K)$ lahko torej zapišemo kot števno unijo končnih množic,
zato je kvečjemu števna. \qedhere
\end{enumerate}
\end{proof}

\newpage

\subsection{Diagonalizacija kompaktnega sebiadjungiranega operatorja}

\begin{izrek}
Naj bo $H$ Hilbertov prostor in $0 \ne K \in \mathcal{K}(H)$
sebiadjungiran operator. Tedaj obstajata zaporedji\footnote{Lahko
tudi končni.} $(\lambda_n)_n \subseteq \R \setminus \set{0}$ in
$(x_n)_n \subseteq H$, za kateri velja naslednje:

\begin{enumerate}[i)]
\item $\abs{\lambda_1} \geq \abs{\lambda_2} \geq \dots$, pri
neskončnih zaporedjih pa velja še, da $\lambda_n$ konvergirajo k
$0$.
\item Vektorji $(x_n)_n$ tvorijo ortonormiran sistem, za vsak $n$
pa velja $K x_n = \lambda x_n$.
\item Če je $\lambda \in \sigma_p(K) \setminus \set{0}$, se
$\lambda$ v zaporedju $(\lambda_n)_n$ pojavi natanko
$\dim (\lambda I - K)$-krat.
\item Za vsak $x \in H$ velja
\[
K x = \sum_{n=1}^\infty \lambda_n \skl{x, x_n} \cdot x_n.
\]
\end{enumerate}
\end{izrek}

\begin{proof}
Operator $K$ ima lastno vrednost $\lambda_1$, za katero je
$\abs{\lambda_1} = \norm{K}$. Naj bo $x_1$ pripadajoč enotski
lastni vektor in $H_1 = \Lin \set{x_1}$. Opazimo, da je tudi
$\eval{K}{H_1^\bot}{} \colon H_1^\bot \to H_1^\bot$ sebiadjungiran
operator. Tudi za ta operator lahko skonstruiramo lastni par.

Denimo, da velja $\eval{K}{H_n^\bot}{} = 0$ za nek $n \in \N$. Za
vsak $x \in H$ lahko zapišemo $x = y + z$, pri čemer je $y \in H_n$
in $z \in H_n^\bot$. Tako dobimo
\[
K x =
K y + K z =
K \br{\sum_{i=1}^n \skl{y, x_i} x_i} =
\sum_{i=1}^n \lambda_i \skl{x, x_i} x_i.
\]
Naj bo $\lambda \in \sigma_p(K) \setminus \set{0}$. Ker za vsak
$x \in \ker (\lambda I - K)$ velja
\[
x =
\frac{1}{\lambda} Kx =
\sum_{i=1}^n \frac{\lambda_i}{\lambda} \skl{x, x_i} x_i
\]
in
\[
x = \sum_{i=1}^n \skl{x, x_i} x_i,
\]
saj je $x \in K$, morata ti linearni kombinaciji imeti enake
koeficiente. Tako sledi $\skl{x, x_i} = 0$ za vse $i$, za katere
velja $\lambda_i \ne \lambda$. Tako velja
$\ker (\lambda I - K) = \Lin \setb{x_i}{\lambda_i = \lambda}$. Za
ti zaporedji so torej izpolnjeni vsi pogoji izreka.

Denimo sedaj, da velja $\eval{K}{H_n^\bot}{} \ne 0$ za vsak
$n \in \N$. Pokažimo, da dobljeni zaporedji ustrezata zgornjim
pogojem.

Denimo, da za vsak $n \in \N$ velja
$\abs{\lambda_n} \geq \delta > 0$. Sledi
\[
\norm{K x_n - K x_m}^2 =
\norm{\lambda_n x_n - \lambda_m x_m}^2 \geq
2 \delta^2,
\]
zato zaporedje $(K x_n)_n$ ne vsebuje konvergentnega podzaporedja,
kar je v protislovji s kompaktnostjo $K$. Prva dva pogoja sta tako
izpolnjena.

Naj bo $L = [\setb{x_n}{n \in \N}]$. Ker je $K(L) \subseteq L$, je
tudi $K \br{L^\bot} \subseteq L^\bot$. Za vsak $x \in L^\bot$ velja
$x \in H_n^\bot$ za vsak $n \in \N$, zato velja
\[
\abs{\skl{Kx, x}} \leq
\norm{Kx} \cdot \norm{x} \leq
\norm{\eval{K}{H_n^\bot}{}} \cdot \norm{x}^2 =
\abs{\lambda_{n+1}} \cdot \norm{x}^2,
\]
kar konvergira proti $0$. Sledi, da je $\skl{Kx, x} = 0$ za vsak
$x \in L^\bot$, oziroma $\eval{K}{L^\bot}{} = 0$. Preostanek dokaza
je enak kot v prejšnjem primeru.
\end{proof}

\begin{opomba}
Zaporedje $(x_n)_n$ lahko s kompletnim ortonormiranim sistemom za
$\ker K$ dopolnimo do kompletnega ortonormitanega sistema za $H$.
V tej bazi je $K$ diagonalni operator.
\end{opomba}

\begin{opomba}
Prostor $H$ je direktna vsota lastnih podprostorov operatorja $K$.
\end{opomba}

\begin{opomba}
Naj $P_\lambda$ označuje ortogonalni projektor na
$\ker (\lambda I - K)$. Tedaj velja
\[
K =
\sum_{\lambda \in \sigma_p(K) \setminus \set{0}}
\lambda \cdot P_\lambda.
\]
\end{opomba}

\datum{2022-12-14}

\begin{trditev}
Naj bo $H$ Hilbertov prostor, $K \in \mathcal{K}(H)$ pa operator,
za katerega je $K \geq 0$. Tedaj obstaja enolično določen
$S \in \mathcal{K}(H)$, za katerega velja $S \geq 0$ in $S^2 = K$.
\end{trditev}

\begin{proof}
Po spektralnem izreku zapišemo
\[
K x = \sum_{n=1}^\infty \lambda_n \skl{x, x_n} \cdot x_n.
\]
Ker je $K \geq 0$, velja $\lambda_n \geq 0$. Tako lahko definiramo
operator
\[
S x = \sum_{n=1}^\infty \sqrt{\lambda_n} \skl{x, x_n} x_n.
\]
Ker je $S$ glede na ortonormiran sistem $(x_n)_n$ diagonalen
operator, je prav tako sebiadjungiran. Za poljuben $x \in H$ lahko
zapišemo
\[
x = \sum_{n=1}^\infty \skl{x, x_n} x_n + e,
\]
kjer je $e \in \ker K$. Tako dobimo
\begin{align*}
\skl{Sx, x} &=
\skl{\sum_{n=1}^\infty \skl{x, x_n} x_n + Se,
\sum_{m=1}^\infty \skl{x, x_m} x_m + e}
\\
&=
\sum_{m, n \in \N} \skl{x, x_n} \skl{x_m, x} \skl{x_n, x_m}
\\
&=
\sum_{n=1}^\infty \abs{\skl{x, x_n}}^2
\\
&\geq
0,
\end{align*}
zato je $S \geq 0$. Očitno je $S^2 = K$.

Naj bo
\[
T_N x = \sum_{k=1}^N \sqrt{\lambda_k} \skl{x, x_k} x_k.
\]
Ker velja
\[
\norm{Sx - T_N x}^2 =
\norm{\sum_{k=N+1}^\infty \sqrt{\lambda_k} \skl{x, x_k} x_k}^2 =
\sum_{k=N+1}^\infty \lambda_k \cdot \abs{\skl{x, x_l}}^2 \leq
\lambda_{N+1} \norm{x}^2,
\]
velja
\[
\norm{S - T_N} \leq \sqrt{\lambda_{N+1}}.
\]
Ker imajo operatorji $T_N$ končen rang, je $S$ limita zaporedja
operatorjev končnega ranga in zato kompakten.

Denimo sedaj, da zgornjim pogojem zadošča tudi
$R \in \mathcal{K}(H)$. Zapišemo lahko
\[
R = \sum_{k} \nu_k \cdot Q_k,
\]
zato je
\[
\sum_{k} \lambda_k P_k =
K =
R^2 =
\sum_k \nu_k^2 Q_k.
\]
Sledi. da je $\nu_k^2 = \lambda_k$ in $Q_k = P_k$ za vsak $k$, zato
je $R = S$.
\end{proof}

\begin{opomba}
Označimo $S = \sqrt{K}$.
\end{opomba}

\begin{definicija}
\emph{Absolutna vrednost}\index{Operator!Absolutna vrednost}
operatorja $T \in \mathcal{K}(H_1, H_2)$ je operator
\[
\abs{T} = \sqrt{T^* T}.
\]
\end{definicija}

\begin{definicija}
Naj bosta $H_1$ in $H_2$ Hilbertova prostora. Operator
$U \in \mathcal{B}(H_1, H_2)$ je
\emph{parcialna izometrija}\index{Operator!Parcialna izometrija},
če je $\eval{U}{(\ker U)^\bot}{}$ izometrija.
\end{definicija}

\begin{izrek}[Polarni razcep]
\index{Izrek!Polarni razcep}
Za vsak $K \in \mathcal{K}(H_1, H_2)$ obstaja enolična parcialna
izometrija $U \in \mathcal{B}(H_1, H_2)$, za katero je
$\ker U = \ker K$ in $K = U \cdot \abs{K}$.
\end{izrek}

\begin{proof}
Naj bo $\widetilde{U}(\abs{K} x) = Kx$. Zlahka preverimo, da je to
izometrija na $\im \abs{K}$, ki jo po zveznosti lahko razširimo do
izometrija ne $\oline{\im \abs{K}}$. To lahko razširimo do
operatorja $U$, za katerega je $Ux = 0$ za vse
$x \in \br{\im \abs{K}}^\bot = \ker K$.
\end{proof}

\begin{izrek}[Singularni razcep]
\index{Izrek!Singularni razcep}
Naj bosta $H_1$ in $H_2$ Hilbertova prostora,
$K \in \mathcal{K}(H_1, H_2)$ pa operator. Tedaj obstajata taka
ortonormirana sistema $(e_n)_n \subseteq H_1$ in
$(f_n)_n \subseteq H_2$ ter singularne vrednosti
$\sigma_1 \geq \sigma_2 \geq \dots \geq 0$ operatorja $K$, da za
vse $x \in H_1$ velja
\[
Kx = \sum_{n=1}^\infty \sigma_n \skl{x, e_n} f_n.
\]
Če je $\dim H_1 = \infty$, $\sigma_n$ konvergirajo k $0$.
\end{izrek}

\begin{proof}
S spektralnim razcepom za $\abs{K}$ lahko zapišemo
\[
Kx =
U \abs{K}x =
U \br{\sum_{n=1}^\infty \sigma_n \skl{x, e_n} e_n} =
\sum_{n=1}^\infty \sigma_n \skl{x, e_n} U e_n.
\]
Ker velja $e_n \in \im \abs{K} \subseteq (\ker K)^\bot$, za
$f_n = U e_n$ velja
\[
\skl{f_n, f_m} = \skl{U e_n, U e_m} = \skl{e_n, e_m}.
\]
Sledi, da je tudi to ortonormiran sistem.
\end{proof}

\newpage

\subsection{Funkcijski račun}

\begin{definicija}
Naj bo $K$ kompakten sebiadjungiran operator,
$f \colon \sigma(K) \to \C$ pa omejena funkcija. Naj bo
\[
K = \sum_{\lambda \in \sigma_p(K)} \lambda \cdot P_\lambda.
\]
Predpisu
\[
f \mapsto
f(K) = \sum_{\lambda \in \sigma_p(K)} f(\lambda) \cdot P_\lambda
\]
pravimo \emph{funkcijski račun}\index{Funkcijski račun}.
\end{definicija}

\begin{opomba}
Ta definicija $f(K)$ za polinome sovpada s standardno.
\end{opomba}

\begin{izrek}
Naj bo $K \in \mathcal{K}(H)$ sebiadjungiran. Vsaki omejeni
funkciji $f \colon \sigma(K) \to \C$ lahko enolično priredimo
$f(K) \in \mathcal{B}(H)$, za katero velja naslednje:

\begin{enumerate}[i)]
\item Če je $f \equiv 1$, je $f(K) = I$.
\item Če je $\eval{f}{\sigma(K)}{} = \id$, je $f(K) = K$.
\item Predpis $f \mapsto f(K)$ je injektiven homomorfizem.
\item Velja $f(K)^* = \oline{f}(K)$.
\item Predpis je izometrija -- velja
$\norm{f}_\infty = \norm{f(K)}$.
\end{enumerate}
\end{izrek}

\begin{proof}
Tem lastnostim ustreza funkcijski račun. Prve tri lastnosti je
enostavno preveriti. Velja
\begin{align*}
\skl{f(K)^* u, v} &=
\skl{u, f(K) v}
\\
&=
\skl{u, \sum_{n=1}^\infty f(\lambda_n) \skl{v, e_n} e_n}
\\
&=
\sum_{n=1}^\infty \oline{f(\lambda_n)} \skl{e_n, v} \skl{u, e_n}
\\
&=
\skl{\oline{f}(K) u, v},
\end{align*}
zato je $f(K)^* = \oline{f}(K)$. Velja še
\[
\norm{f(K)u}^2 =
\sum_{n=1}^\infty \abs{f(\lambda_n)}^2 \abs{\skl{u, e_n}}^2 \leq
\norm{f}_\infty^2 \cdot \norm{u}^2,
\]
zato je $\norm{f}_\infty \geq \norm{f(K)}$. Ker pa je
$f(K) e_n = f(\lambda_n) e_n$, je
$\norm{f(K)} \geq \abs{f(\lambda_n)}$ in zato tudi
$\norm{f(K)} \geq \norm{f}_\infty$.

Enoličnost pokažimo za funkcije, ki so zvezne v $0$. Prve tri točke
enolično določajo predpis za vse polinome. S pomočjo
Stone-Weierstrassovega izreka se predpis razširi do vseh zveznih
funkcij.
\end{proof}

\begin{trditev}
Funkcijski račun ima naslednje lastnosti:

\begin{enumerate}[i)]
\item Če je $f(\sigma(K)) \subseteq \R$, je $f(K)$ sebiadjungiran.
\item Če je $f(\sigma(K)) \geq 0$, je $f(K) \geq 0$.
\item Če je $f(\sigma(K)) \subseteq \partial \dsk$, je $f(K)$
unitaren.
\item Če je
\[
\lim_{\lambda \to 0} f(\lambda) = f(0) = 0,
\]
je $f(K)$ kompakten.
\end{enumerate}
\end{trditev}

\begin{proof}
\phantom{a}
\begin{enumerate}[i)]
\item Posledica 4.~točke izreka.
\item Velja
\[
\skl{f(K) x, x} =
\sum_{n=1}^\infty f(\lambda_n) \abs{\skl{x, e_n}}^2 \geq
0.
\]
\item Velja
\[
\norm{f(K) x}^2 =
\sum_{n=1}^\infty \abs{\skl{x, e_n}}^2 =
\norm{x}^2,
\]
zato je $f(K)$ izometrija. Sledi, da je $f(K)^* \cdot f(K) = I$,
ker pa $f(K)$ komutira s $\oline{f}(K)$, pa velja tudi
$f(K) \cdot f(K)^* = I$.
\item Velja
\[
f(K) = \sum_{n=1}^\infty f(\lambda_n) \skl{x, e_n} e_n.
\]
Naj bo
\[
T_N = \sum_{n=1}^N f(\lambda_n) \skl{x, e_n} e_n.
\]
Tedaj velja
\[
\norm{f(K) x - T_N x}^2 =
\sum_{n=N+1}^\infty \abs{f(\lambda_n)}^2 \abs{\skl{x, e_n}}^2 \leq
\sup_{n > N} \abs{f(\lambda_n)}^2 \norm{x}^2.
\]
Sledi, da zaporedje $(T_n)_n$ konvergira k $f(K)$. Ker imajo ti
končen rang, je $f(K)$ kompakten. \qedhere
\end{enumerate}
\end{proof}
