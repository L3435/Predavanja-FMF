\section{Ring structure of holomorphic functions}
\begin{definicija}
    A divisor of $f \in \mathcal{M}^*(\Omega)$, denoted by $(f) : \Omega \rightarrow \Z$ by 
    $$
        (f)(z) = \begin{cases}
            0 & \text{$f$ has a pole in $z$} \\
            n & \text{$f$ has a zero of order $n$ in $z$} \\
            -n & \text{$f$ has a zero of order $n$ in $z$} \\
        \end{cases}.
    $$
\end{definicija}
\begin{opomba}
    The divisor of a product is the sum of divisors, i.e.\ $(f \cdot g) = (f) + (g)$.
\end{opomba}

\begin{definicija}
    Let $S \subseteq \mathcal{O}(\Omega)$ be a subset that contains a zero holomorphic function.

    To define $\gcd(S)$ we first define $d(z) = \min_{f \in S \setminus \set{0}} f(z) \in \N_0$. By Weierstraß product theorem there exists $g \in \mathcal{O}(\Omega)$ such that $(g) = d$. Then $g$ divides every $f \in S$.
\end{definicija}

\begin{lema}[Wedderburn]
    Let $\Omega \subseteq \C$ be a domain $f, g \in \mathcal{O}(|Omega)$ such that $\gcd(f, g) = 1$. Then There exist $a, b \in \mathcal{O}(\Omega)$ so that $a f + b g =1$. Morover, we can choose $a, b$ to be nonwhere vanishing. 
\end{lema}
\begin{proof}
    If $g = 0$ and $f \neq 0$ then $f$ can vanish by assumption on $\gcd$ and so $a = \frac{1}{g}$ and $g = 0$ suffice. Therefore, we can assume both $f, g$ and are nonzero, Then $Z(f) \cap Z(g)$ is empty, since in the other case $(z - p)$ would therefore divide $\gcd(f, g)$ for $p$ an element of the intersection. The set $Z(f) \cap Z(g)$ is thus discrete. Further, for each zero $p$ of $g$, there exists a disk of radius $\epsilon$ and a holomorphic function $f_p \in \mathcal{O}(D_{\epsilon}(p))$ such that $f = e^{f_p}$. By the Mittag-Leffler osculation theorem \ref{inf_prod:thr:MLOsculation} there exists a $h \in \mathcal{O}(\Omega)$ such that $\ord_p(h - f) > \ord_p(g)$.

    Here we stop for a short observation. Developing into the power series, we get that $e^{w^n} - 1 = w^n + O(w^{2 n})$.

    Then, 
    $$
        \ord_p{f - e^h} = \ord_p{e^h \cdot (e^{f_p - h} - 1)} = \ord_p{(e^{f_p - h} - 1)} = \ord_p(f_p - h) > \ord_p(g).
    $$

    Define $k \vcentcolon= \frac{f - e^h}{g} \in \mathcal{O}(\Omega)$. We claim that $a = e^{-h}$ and $b = -k e^{-h}$. Clearly $a$ doesnt vanish, and, since $k$ does not vanish by the previous calculation of the order, so does not $g$. Furthermore, 
    $$
        a f + b g = e^{-h}g  - k e ^{-h} g = e^{-h}(f - k g) = e^{- h}(f - \frac{f - e^h}{g}g) = e^{-h} e^h = 1.
    $$
\end{proof}

\begin{posledica}
    For holomorphic functions $f_j \in \mathcal{O}(\Omega), j = 1, 2, \ldots n$ then $f \gcd(f_1, f_2, \ldots, f_n)$ can be written as 
    $$
        f = \sum_{j = 1}^{n a_j f_j} 
    $$
\end{posledica}
\begin{proof}
    We proceed by induction. The base case is the lemma of Wedderburn. Now let $\hat{f} = \gcd(f_2, f_3, \ldots f_n)$, which can be written as $\hat{f} = \sum_{j = 2}^{n \hat{a}_j f_j}$. Then $\frac{f_1}{f}, \frac{\hat{f}}{f} \in \mathcal{O}(\Omega)$ and have $\gcd$ equal to $1$. Then we can apply lemma of Wedderburn to get $a, b$ such that $a \frac{f_1}{f} + b \frac{\hat{f}}{f} = 1$. We read coefficients $a_1 = a$ and $a_j = b \hat{a}_j$.
\end{proof}

\begin{izrek}
    If $I \subseteq \mathcal{O}(\Omega)$ be the ideal generated by holomoprhic functions $f_1, f_2, \ldots f_n$ on $\Omega$. Then there exists $f$ holomorphic that $I = (f)$. 
\end{izrek}
\begin{proof}
    Take $f$ the $\gcd$ of the $f_j$. This is in $I$ since it is a sum of $f_j$ by the previous corollary. Since $f | f_j$, this implies that $I = (f)$
\end{proof}

\begin{definicija}
    Let $\Omega \subseteq \C$ be a domain and $I \vartriangleleft \mathcal{O}(\Omega)$.
    \begin{itemize}
        \item We call $I$ closed if for every sequence $(f_n)_n \subseteq I$ that converges uniformly on compacts of $\Omega$ to some $f \in I$
        \item We dcall $p \in \Omega$ a zero of $I$ if $f(p) = 0$ for every $f \in I$
    \end{itemize}
\end{definicija}

\begin{lema}
    Let $\Omega \subseteq \C$ be a domain and $I \vartriangleleft \mathcal{O}(\Omega)$. Let $p \in \Omega$ be a point that is not a zero of $I$. Let $f, g \in \mathcal{O}(\Omega)$, where $f$ may have its only zero in $p$. If $f g \in I$, then $g \in I$.
\end{lema}
\begin{proof}
    Since $p$ is not a zero of $I$, then $h \in I$ such that $h(p) \neq0$. Let $n$ be $\ord_p(f)$. If $n = 0$, then $f$ is a unit, so $g \in I$. So assume $n \geq 1$. Then, 
    $$
        \frac{f(z)}{z - p}g = -\frac{1}{h(p)}\frac{h - h(p)}{z - p} fg - \frac{f g}{z - p} h \in I
    $$
    since $\frac{f}{z - p}$ is holomoprhic.
    
    Iterate this, replacing $f$ with $\frac{f}{z - p}$ in each step. Thus, $\frac{f}{(z - p)^n} g \in I$. Since $\frac{f}{(z - p)^n}$ is a unit $g$ must be an element of $I$.
\end{proof}

\begin{izrek}
    Let $\Omega \subseteq \C$ be a domain and $I \vartriangleleft \mathcal{O}(\Omega)$. If $I$ has no zeroes and is closed, then $I = \mathcal{O}(\Omega)$.
\end{izrek}
\begin{proof}
    Let $f$ be an arbitrary nonzero element of $I$. By Weierstraß product theorem, 
    $$
        f = \prod_{k = 1}^{\infty} f_k,
    $$ 
    where each $f_k$ has exactly one zero in $\Omega$, and the tails $\hat{f}_n \vcentcolon= \prod_{k = n}^{\infty} f_k$, converge to $1$ uniformly on compacts of $\Omega$. As $f = \hat{f}_1 = f_1 \hat{f}_2$, we can apply the preivous lemma, so $\hat{f}_2 \in I$. Inductively, $\hat{f}_n \in I$ and since the ideal $I$ is assumed to be closed, $1 = \lim_k \hat{f}_k \in I$. Thus $I = \mathcal{O}(\Omega)$.
\end{proof}

