\section{Theorems about holomorphic functions}

\subsection{Riemann mapping theorem}

\datum{2023-11-7}

\begin{definicija}
A domain $\Omega \subseteq \C$ is
\emph{simply connected}\index{simply connected} if every closed
path in $\Omega$ is homotopic to a constant path in $\Omega$.
\end{definicija}

\begin{lema}
\label{thm_hol:lm:inj_exis}
Let $\Omega \subset \C$ be a domain and $a \in \Omega$. Assume that
$\Omega$ admits a square root for all function
$g \in \mathcal{O}^*(\Omega)$. Then there exists a holomorphic
injection $f \colon \Omega \to \dsk$ such that $f(a) = 0$.
\end{lema}

\begin{proof}
Fix a point $p \in \C \setminus \Omega$. By our assumption, there
exists a function $v \in \mathcal{O}^*(\Omega)$ such that
$v(z)^2 = z-p$. Note that $v$ is injective. Similarly, we have
$v(\Omega) \cap -v(\Omega) = \emptyset$. Now choose a point
$b \in -v(\Omega)$. As $v$ is not constant, it is an open map.
Therefore, there exists some $r > 0 $ such that
$\dsk(b, r) \cap v(\Omega) = \emptyset$. The Möbius transformation
\[
h(w) = r \cdot \br{\frac{1}{w-b} - \frac{1}{v(a)-b}}
\]
thus maps $v(\Omega)$ into $\dsk$. The map $f$ is therefore given
as $f = h \circ v$.
\end{proof}

\begin{definicija}
An \emph{expansion}\index{expansion} if a map
$\kappa \colon \Omega \to \dsk$ where $0 \in \Omega \subset \dsk$
such that $\kappa(0) = 0$ and $\abs{\kappa(z)} > \abs{z}$ holds for
all $z \ne 0$.
\end{definicija}

\begin{lema}
Let $\Omega \subset \dsk$ be a domain with $0 \in \Omega$. Assume
that $\Omega$ admits a square root for all function
$g \in \mathcal{O}^*(\Omega)$. Choose $c \in \dsk$ such that
$c^2 \not \in \Omega$. For all $a \in \dsk$, let
\[
g_a = \frac{z - a}{\oline{a} z - 1}
\]
and choose $v \in \mathcal{O}(\Omega)$ such that
$v(z)^2 = g_{c^2}(z)$ and $v(0) = c$. Then the map
$\kappa = g_c \circ v$ is an expansion and
\[
g_{c^2} \circ (z \mapsto z^2) \circ g_c \circ \kappa =
\id_{\Omega}.
\]
\end{lema}

\begin{proof}
Note that $v$ is indeed well-defined. Also note that
\[
g_{c^2} \circ (z \mapsto z^2) \circ g_c \circ \kappa =
g_{c^2} \circ (z \mapsto z^2) \circ v =
g_{c^2} \circ g_{c^2} =
\id.
\]
We of course have $\kappa(0) = 0$. Denote
$\psi_c = g_{c^2} \circ (z \mapsto z^2) \circ g_c$. It remains to
check that $\abs{\kappa(z)} > \abs{z}$, which is equivalent to
$\abs{\psi_c(z)} < \abs{z}$ for $z \ne 0$ as
$\psi_c \circ \kappa = \id$. Note that
$\psi_c \colon \dsk \to \dsk$ is holomorphic. As it is not a
rotation (it is not injective), the conclusion follows from the
Schwarz lemma.
\end{proof}

\begin{lema}[Hurwitz]
\index{Hurwitz!lemma}
Let $\Omega \subseteq \C$ be a domain and let
$f_n \colon \Omega \to \C$ be holomorphic functions. Suppose that
the sequence $(f_n)_n$ converges uniformly on compacts of $\Omega$
to a non-constant function $f \colon \Omega \to \C$. Then for all
points $p \in \Omega$ there exists a sequence
$(p_n)_n \subseteq \Omega$ with limit $p$ such that
$f_n(p_n) = f(p)$ for all $n > N$.
\end{lema}

\begin{proof}
Let $w = f(p)$. There exists a disk $\dsk(p, \delta)$ such that
$f(z) \ne w$ for all points
$z \in \oline{\dsk(p, \delta)} \setminus \set{p}$. Note that we
have
\[
\min_{z \in \partial \dsk(p, \delta)} \abs{f(z) - w} >
\abs{f(p) - w} =
0.
\]
As $(f_n)_n$ converges uniformly on $\oline{\dsk(p, \delta)}$,
there exists some $n_0 \in \N$ such that for all $n \geq n_0$ we
have
\[
\min_{z \in \partial \dsk(p, \delta)} \abs{f_n(z) - w} >
\abs{f_n(p) - w}.
\]
By lemma~\ref{hol:lm:root}, $f_n(z) - w$ has a root
$p_n \in \dsk(p, \delta)$. For any convergent subsequence
$(p_{n_k})_k$ with limit $q$ we have
\[
f(p) = \lim_{k \to \infty} f_{n_k}(p_{n_k}) = f(q),
\]
therefore $p = q$.
\end{proof}

\begin{posledica}\label{thm_hol:cor:hurw}
Let $\Omega \subseteq \C$ be a domain and
$f_n \colon \Omega \to \C$ be holomorphic functions such that
$(f_n)_n$ converges uniformly on compacts of $\Omega$ to
$f \colon \Omega \to \C$. If all the $f_n$ are nowhere vanishing
and $f \ne 0$, then $f$ is nowhere vanishing.
\end{posledica}

\obvs

\begin{izrek}[Hurwitz]
\index{Hurwitz!theorem}
Let $\Omega, \Omega' \subseteq \Omega$ be domains and
$f_n \colon \Omega \to \Omega'$ be holomorphic functions that
converge uniformly on compacts of $\Omega$ to
$f \colon \Omega \to \Omega'$. Assume that $f$ is not constant.

\begin{enumerate}[i)]
\item If $f_n \colon \Omega \to \Omega'$ is injective, $f$ is also
injective.
\item We have $f(\Omega) \subseteq \Omega'$.
\end{enumerate}
\end{izrek}

\begin{proof}
\phantom{a}
\begin{enumerate}[i)]
\item Let $p \in \Omega$ and observe the functions
$g_n(z) = f_n(z) - f_n(p)$. This is a sequence of nowhere vanishing
functions. As $f$ is not constant, $f(z) - f(p)$ is nowhere
vanishing as well. It follows that $f$ is injective.
\item Suppose otherwise and apply the Hurwitz lemma for a point
$p$ with $f(p) \not \in \Omega'$. \qedhere
\end{enumerate}
\end{proof}

\begin{izrek}[Riemann mapping]
\index{Riemann!mapping theorem}
For a proper domain $\Omega \subset \C$ the following are
equivalent:

\begin{enumerate}[i)]
\item $\Omega$ is simply connected.
\item $\Omega$ admits a logarithm for any
$f \in \mathcal{O}^*(\Omega)$.
\item $\Omega$ admits a square root for any
$f \in \mathcal{O}^*(\Omega)$.
\item $\Omega$ is biholomorphic to $\dsk$.
\end{enumerate}
\end{izrek}

\begin{proof}
Note that if $\Omega$ is biholomorphic to $\dsk$, it is of course
simply connected. Suppose that $\Omega$ is simply connected. Then
\[
F(z) = a + \int_{z_0}^z \frac{f'(z)}{f(z)}\,dz
\]
defines a logarithm for any $f \in \mathcal{O}^*(\Omega)$. Given a
logarithm of a function, we can of course construct a square root
with $\sqrt{f} = e^{\frac{1}{2} \ln f}$. It remains to check that
all domains admitting square roots are biholomorphic to $\dsk$.

By lemma~\ref{thm_hol:lm:inj_exis} we can assume that
$\Omega \subseteq \dsk$ and $0 \in \Omega$. Now define the family
of functions
\[
\mathcal{F} = \setb{f \colon \Omega \to \dsk}
{f \in \mathcal{O}(\Omega) \land
f(0) = 0 \land \text{$f$ is injective}}.
\]
If $\mathcal{F}$ has no biholomorphic map, it is infinite. Note
that $\mathcal{F}$ is bounded, so it is normal by Montel.

Choose a point $p \in \Omega$ with $p \ne 0$. We claim that if
$h \in \mathcal{F}$ and
\[
\abs{h(p)} = \sup_{f \in \mathcal{F}} \abs{f(p)},
\]
we have $h(\Omega) = \dsk$. Indeed, if that were not the case, we'd
reach a contradiction with the expansion $\kappa$ of $\Omega$ as
\[
\abs{\kappa(h(p))} > \abs{h(p)}
\]
and $\kappa \circ h \in \mathcal{F}$.

Let
\[
M = \sup_{f \in \mathcal{F}} \abs{f(p)}
\]
and find a sequence $(f_n)_n \subseteq \mathcal{F}$ with
\[
\lim_{n \to \infty} \abs{f_n(p)} = M.
\]
As $\mathcal{F}$ is a normal family, there exists a convergent
subsequence. The limit is not constant as $f(p) \ne 0$. By Hurwitz,
$f$ is injective and $f(\Omega) \subseteq \dsk$. By the above
claim, we have $f(\Omega) = \dsk$.
\end{proof}

\newpage

\subsection{Bloch's theorem}

\datum{2023-11-8}

\begin{lema}
Let $\Omega \subseteq \C$ be a bounded domain and
$f \colon \oline{\Omega} \to \C$ a continuous map such that
$\eval{f}{\Omega}{}$ is an open map. Let $a \in \Omega$ be a point
such that
\[
s = \min_{z \in \partial \Omega} \abs{f(z) - f(a)} > 0.
\]
Then $f(\Omega)$ contains the disk $\dsk(f(a), s)$.
\end{lema}

\begin{proof}
By compactness, there exists a $w_0 \in \partial f(\Omega)$ such
that $d(\partial f(\Omega), f(a)) = \abs{w_0 - f(a)}$. Let
$(z_k)_k \subseteq \Omega$ be a sequence, convergent to $z_0$, such
that
\[
\lim_{k \to \infty} f(z_k) = w_0.
\]
Of course $f(z_0) = w_0$. Note that, as $\eval{f}{\Omega}{}$ is
open, we have $z_0 \in \partial \Omega$. But then
\[
d(\partial f(\Omega), f(a)) = \abs{f(z_0) - f(a)} \geq s. \qedhere
\]
\end{proof}

\begin{lema}
\label{thm_hol:lm:bloch}
Let $f$ be a non-constant function, holomorphic in a neighbourhood
of $\oline{\dsk(a, r)}$. Assume that
\[
\sup_{z \in \oline{\dsk(a,r)}} \abs{f'(z)} \leq 2 \abs{f'(a)}.
\]
Then $\dsk(f(a), R) \subseteq f(\dsk(a, r))$, where
\[
R = \br{3 - 2 \sqrt{2}} \cdot r \cdot \abs{f'(a)}.
\]
\end{lema}

\begin{proof}
Without loss of generality assume that $a = f(a) = 0$. Define
\[
A(z) = f(z) - f'(0) z = \int_0^1 \br{f'(tz) - f'(0)} z \,dt.
\]
Note that
\[
f'(v) - f'(0) =
\frac{1}{2 \pi i} \olint_{\partial \dsk(a, r)}
f'(\xi) \cdot \br{\frac{1}{\xi-v} - \frac{1}{\zeta}}\,d\xi,
\]
therefore
\[
\abs{f'(v) - f'(0)} \leq
\frac{1}{2 \pi} \cdot \abs{v} \cdot
\frac{\norm{f'}_{\dsk(a,r)}}{r \cdot (r - \abs{v})} \cdot 2 \pi r =
\abs{v} \cdot \frac{\norm{f'}_{\dsk(a,r)}}{r - \abs{v}}.
\]
It follows that
\begin{align*}
\abs{A(z)} &\leq
\int_0^1 \abs{z} \cdot \abs{f'(tz) - f'(0)}\,dt
\\
&\leq
\abs{z} \cdot \int_0^1
\abs{tz} \cdot \frac{\norm{f'}_{\dsk(a,r)}}{r - \abs{tz}}\,dt
\\
&\leq
\abs{z}^2 \cdot \norm{f'}_{\dsk(a,r)} \cdot \int_0^1
t \cdot \frac{1}{r - \abs{z}}
\\
&=
\abs{z}^2 \cdot
2 \frac{\abs{f'(0)}}{r - \abs{z}}.
\end{align*}
Now, using the triangle inequality, we get
\[
\abs{f(z)} \geq \abs{z} \cdot \abs{f'(0)} - \abs{A(z)}.
\]
Let $\abs{z} = \rho \in (0, r)$. We get
\[
\abs{f(z)} \geq
\rho \cdot \abs{f'(0)} - \abs{A(z)} \geq
\rho \cdot \abs{f'(0)} -
\frac{\rho^2}{r - \rho} \cdot \abs{f'(0)} \geq
\abs{f'(0)} \cdot \br{\rho - \frac{\rho^2}{r - \rho}}.
\]
Note that there exists a $\rho_0$ such that
\[
\rho_0 - \frac{\rho_0^2}{r - \rho_0} =
r \cdot \br{3 - 2 \sqrt{2}}.
\]
Therefore, we get
\[
\abs{f(z)} \geq \abs{f'(0)} \cdot r \cdot \br{3 - 2 \sqrt{2}}.
\]
Now just apply the previous lemma to the disk $\dsk(0, \rho_0)$.
\end{proof}

\begin{izrek}[Bloch]
\index{Bloch's theorem}
Let $f$ be a function, holomorphic in a neighbourhood of
$\oline{\dsk}$, with $f'(0) = 1$. Then $f(\dsk)$ contains a disk of
radius $\frac{3}{2} - \sqrt{2}$.
\end{izrek}

\datum{2023-11-14}

\begin{proof}
Define $h(z) = \abs{f'(z)} (1-\abs{z}) \geq 0$. Not that
$h \not \equiv 0$ as $f$ is not constant. Therefore $h$ attains a
maximum in a point $p \in \oline{\dsk}$. In particular, as
$\eval{h}{\partial \dsk}{} = 0$, we have $p \in \dsk$. Observe
$\Omega = \dsk \br{p, t}$ for
$t = \frac{1}{2} \cdot (1 - \abs{p})$.
For all $z \in \Omega$, we have $1 - \abs{z} \geq t$ and
\[
\abs{f'(z)} \cdot (1 - \abs{z}) \leq
\abs{f'(p)} \cdot (1 - \abs{p}) =
\abs{f'(p)} \cdot 2t \leq
\abs{f'(p)} \cdot 2 \cdot (1 - \abs{z}).
\]
Now, applying lemma~\ref{thm_hol:lm:bloch}, we have
$\dsk(f(p), R) \subseteq f(\dsk)$ with
\[
R =
\br{3 - 2 \sqrt{2}} \cdot \frac{1}{2}
\cdot (1 - \abs{p}) \cdot \abs{f'(p)} \geq
\frac{3}{2} - \sqrt{2}
\]
by choice of $p$.
\end{proof}

\begin{opomba}
Let
\[
\mathcal{F} =
\setb{\text{$f$ holomorphic on a neighbourhood of $\oline{\dsk}$}}
{f'(0) = 1}.
\]
For $f \in \mathcal{F}$, denote by $L_f$ the supremum of radii of
disks contained in $f(\dsk)$, and by $B_f$ the supremum of radii of
disks contained in $f(\dsk)$ that is a biholomorphic image of some
subdomain of $\dsk$. We then define the \emph{Landau's constant}
\[
L = \inf_{f \in \mathcal{F}} L_f
\]
and the \emph{Bloch's constant}
\[
B = \inf_{f \in \mathcal{F}} B_f.
\]
The current known bounds for the constants are
\[
0.5 < L < 0.544
\quad \text{and} \quad
\frac{\sqrt{3}}{4} + 10^{-14} < B \leq
\sqrt{\frac{\sqrt{3} - 1}{2}} \cdot
\frac{\Gamma \br{\frac{1}{3}} \cdot \Gamma \br{\frac{11}{12}}}
{\Gamma \br{\frac{1}{4}}}.
\]
\end{opomba}

\begin{posledica}
\label{thm_hol:cor:bloch}
Let $\Omega \subseteq \C$ be a domain, $f \in \mathcal{O}(\Omega)$
a function and $p \in \Omega$. Let $r = d(p, \partial \Omega)$.
Then $f(\Omega)$ contains a disk of radius
\[
\br{\frac{3}{2} - \sqrt{2}} \cdot r \cdot \abs{f'(p)}.
\]
\end{posledica}

\obvs

\begin{opomba}
Liouville's theorem follows from this corollary.
\end{opomba}

\begin{lema}\label{thm_hol:lm:arcos}
Let $\Omega \subseteq \C$ be a simply connected domain and
$1, -1 \not \in f(\Omega)$. Then there exists a function
$F \in \mathcal{O}(\Omega)$ such that $f = \cos(F)$.
\end{lema}

\begin{proof}
Note that, as $\Omega$ is simply connected, we can define
\[
F(z) =
\frac{1}{i} \cdot \ln \br{f(z) + \sqrt{f(z)^2-1}}. \qedhere
\]
\end{proof}

\begin{izrek}
Let $\Omega \subseteq \C$ be a simply connected domain and let
$f \in \mathcal{O}(\Omega)$. Suppose that
$0, 1 \not \in f(\Omega)$. Then the following statements are true:

\begin{enumerate}[i)]
\item There exists a function $g \in \mathcal{O}(\Omega)$ such that
\[
f = \frac{1}{2} \br{1 + \cos(\pi \cdot \cos(\pi \cdot g))}.
\]
\item If any $g \in \mathcal{O}(\Omega)$ satisfies the above
equality, then $g(\Omega)$ contains no disk of radius $1$.
\end{enumerate}
\end{izrek}

\begin{proof}
\phantom{a}
\begin{enumerate}[i)]
\item Apply the previous lemma twice.
\item Define
\[
A =
\setb{m \pm \frac{i}{\pi} \ln \br{n + \sqrt{n^2-1}}}
{m \in \Z \land n \in \N}.
\]
We claim that $g(\Omega) \cap A = \emptyset$. Indeed, for $a \in A$
we have
\[
f(a) = \frac{1}{2} \br{1 + \cos(\pm \pi \cdot n)} \in \set{0,1}.
\]
Now note that
\begin{align*}
\ln \br{n+1 + \sqrt{n^2 + 2n}} - \ln \br{n + \sqrt{n^2-1}} &=
\ln \br{\frac{n+1 + \sqrt{n^2 + 2n}}{n + \sqrt{n^2 - 1}}}
\\
&\leq
\ln \br{\frac{2n + 2}{n}}
\\
&\leq \ln(4)
\\
&<
\pi.
\end{align*}
It's straightforward to check that every disk of radius $1$
intersects $A$. \qedhere
\end{enumerate}
\end{proof}

\begin{izrek}[Picard's little theorem]
\index{Picard's little theorem}
Every non-constant entire function omits at most one complex value.
\end{izrek}

\begin{proof}
Without loss of generality assume that $f$ omits $0$ and $1$.
Applying the above theorem, we can write
\[
f = \frac{1}{2} \br{1 + \cos(\pi \cdot \cos(\pi \cdot g))}.
\]
Recall that $g(\C)$ contains no disk of radius $1$. If $g$ is
not constant, $g(\C)$ contains arbitrarily large disks by
corollary~\ref{thm_hol:cor:bloch}, which is a contradiction.
\end{proof}

\begin{posledica}
Suppose that $f \in \mathcal{M}(\C)$ is a non-constant function.
Then $f$ omits at most $2$ values.
\end{posledica}

\begin{proof}
Suppose that $f$ omits distinct values $a$, $b$ and $c$. Then
\[
g(z) = \frac{1}{f(z) - a}
\]
is an entire function that omits values $\frac{1}{b-a}$ and
$\frac{1}{c-a}$, therefore it is constant.
\end{proof}

\begin{izrek}
Let $f \in \mathcal{O}(\C)$ be an entire function. Then either
$f \circ f$ has a fixed point of $f(z) = z + c$.
\end{izrek}

\begin{proof}
If $f \circ f$ has no fixed point, the same holds for $f$. We can
therefore define an entire holomorphic function $g$ with
\[
g(z) = \frac{f(f(z)) - z}{f(z) - z}.
\]
Note that $g$ omits both $0$ and $1$, therefore it is constant. But
then
\[
f(f(z)) - z = \lambda (f(z) - z)
\]
for some $\lambda \not \in \set{0, 1}$ by Picard's little theorem.
Taking the derivative, we get
\[
f'(f(z)) \cdot f'(z) - 1 = \lambda (f'(z) - 1),
\]
or equivalently
\[
f'(z) \cdot \br{f'(f(z)) - \lambda} = 1 - \lambda \ne 0.
\]
Note that $f' \circ f$ omits both $\lambda$ and $0$, therefore it
is constant. But then $f'$ is constant as well. The only option is
$f'(z) = 1$.
\end{proof}

\datum{2023-11-15}

\begin{lema}
For all $w \in \C$ there exists a $v \in \C$ such that
$\cos(\pi v) = w$ and $\abs{v} \leq 1 + \abs{w}$.
\end{lema}

\begin{proof}
Let $v = \alpha + i \beta$ and note that
\[
\abs{w}^2 =
\cos(\pi \alpha)^2 + \sinh(\pi \beta)^2 \geq
\pi^2 \beta^2.
\]
Observe that we can choose some $\alpha$ such that
$\abs{\alpha} \leq 1$, therefore
\[
1 + \abs{w} \geq
1 + \pi \cdot \abs{\beta} \geq
\abs{\alpha} + \abs{\beta} \geq
\abs{v}. \qedhere
\]
\end{proof}

\begin{izrek}
Let $f$ be a function, holomorphic on a neighbourhood of
$\oline{\dsk}$, such that $0, 1 \not \in f(\Omega)$. There exists a
function $g$, holomorphic on a neighbouhood of $\oline{\dsk}$, such
that

\begin{enumerate}[i)]
\item the equality
\[
f = \frac{1}{2} \br{1 + \cos(\pi \cdot \cos(\pi \cdot g))}
\]
holds with $\abs{g(0)} \leq 3 + 2 \abs{f(0)}$, and
\item the inequality
\[
\abs{g(z)} \leq \abs{g(0)} + \frac{\theta}{\gamma (1 - \theta)}
\]
holds for all $\abs{z} \leq \theta$.
\end{enumerate}
\end{izrek}

\begin{proof}
Again, apply lemma~\ref{thm_hol:lm:arcos} and let
\[
2f - 1 = \cos(\pi \cdot F).
\]
Using the above lemma, we can transform $F$ such that
$\abs{F(0)} \leq 1 + \abs{2f(0) - 1}$. Applying
lemma~\ref{thm_hol:lm:arcos} again, we define $g$ such that
\[
F = \cos(\pi g).
\]
Again, using the above lemma, set $\abs{g(0)} \leq 1 + \abs{F(0)}$.
We therefore have
\[
\abs{g(0)} \leq
1 + \abs{F(0)} \leq
2 + \abs{2f(0) - 1} \leq
3 + 2 \abs{f(0)}.
\]
Recall that $g(\dsk)$ does not contain a disk of radius $1$. Let
$z \in \oline{\dsk(\theta)}$. Then, by Bloch's theorem, $g(\dsk)$
contains a disk of radius
$R = \gamma \cdot \abs{g'(z)} \cdot (1 - \theta)$. Therefore, we
must have
\[
\abs{g'(z)} < \frac{1}{\gamma (1 - \theta)}.
\]
It follows that
\[
\abs{g(z)} =
\abs{g(0) + \int_0^z g'(\xi)\,d\xi} \leq
\abs{g(0)} + \int_0^z \abs{g'(\xi)}\,d\xi \leq
\abs{g(0)} + \abs{z} \cdot \frac{1}{\gamma (1 - \theta)}. \qedhere
\]
\end{proof}

\begin{definicija}
For $r \geq 0$, let
\[
S(r) =
\setb{\text{$f$ holomorphic on a neighbourhood of $\oline{\dsk}$}}
{0, 1 \not \in f \br{\oline{\dsk}} \land \abs{f(0)} \leq r}.
\]
For $\theta \in (0, 1)$ and $r > 0$, let
\[
L(\theta, r) =
\exp \br{\pi \cdot
\exp \br{3 + 2r + \frac{\theta}{\gamma (1 - \theta)}}},
\]
where $\gamma$ is any constant such that Bloch's theorem holds,
e.g.~$\gamma = \frac{3}{2} - \sqrt{2}$.
\end{definicija}

\begin{izrek}[Schottky]
\index{Schottky's theorem}
Let $f \in S(r)$. Then for all $z \in \dsk$ such that
$\abs{z} < \theta$ we have
\[
\abs{f(z)} \leq L(\theta, r).
\]
\end{izrek}

\begin{proof}
Let $g$ be a holomorphic function as in the previous theorem. Note
that $\abs{\cos(w)} \leq e^{\abs{w}}$. We must therefore also have
\[
\frac{1}{2} \cdot \abs{1 + \cos(w)} \leq e^{\abs{w}}.
\]
Using this inequality, we get
\[
\abs{f(z)} \leq
\exp \br{\pi \cdot \exp \br{\pi \cdot \abs{g(z)}}} \leq
L(\theta, r). \qedhere
\]
\end{proof}

\newpage

\subsection{The great Picard theorem}

\datum{2023-11-21}

\begin{lema}
Let $\Omega \subseteq \C$ be a domain, $\omega \in \Omega$ and
$r \in (0, \infty)$. Let
\[
\mathcal{F} =
\setb{f \in \mathcal{O}(\Omega)}{0, 1 \not \in f(\Omega)}.
\]
and $\mathcal{F}_{\omega,r} \subseteq \mathcal{F}$ a subfamily with
$\abs{f(\omega)} \leq r$ for all $f \in \mathcal{F}_{\omega,r}$.

\begin{enumerate}[i)]
\item There exists some $t > 0$ such that
$\eval{\mathcal{F}_{\omega,r}}{\dsk(\omega,t)}{}$ is bounded.
\item The family $\mathcal{F}_{\omega,1}$ is locally bounded in
$\Omega$.
\end{enumerate}
\end{lema}

\begin{proof}
\phantom{a}
\begin{enumerate}[i)]
\item Choose a $t > 0$ such that
$\oline{\dsk(\omega, 2t)} \subseteq \Omega$ and set
$\varphi(z) = 2tz + \omega$. By Schottky's theorem, we have
\[
\abs{f \circ \varphi(z)} \leq L \br{\frac{1}{2}, r}
\]
for $\abs{z} < \frac{1}{2}$, or equivalently
\[
\sup_{v \in \dsk(w, t)} \abs{f(v)} \leq L \br{\frac{1}{2}, r}.
\]
The family $\mathcal{F}_{\omega, r}$ is therefore bounded.

\item Let
\[
\mathcal{U} =
\setb{u \in \Omega}
{\text{$\mathcal{F}_{\omega, 1}$ is bounded in a neighbourhood of
$u$}}.
\]
Note that $\omega \in \mathcal{U}$, therefore the set is non-empty.
Also observe that $\mathcal{U}$ is open. Suppose that
$\mathcal{U} \ne \Omega$ and let
$v \in \partial \mathcal{U} \cap \Omega$. Then there exists a
sequence
$(f_n)_n \subseteq \mathcal{F}_{\omega, 1}$ such that
\[
\lim_{n \to \infty} \abs{f_n(v)} = \infty.
\]
Define $g_n = \frac{1}{f_n}$. These functions are holomorphic and
omit both $0$ and $1$ by definition, therefore
$g_n \in \mathcal{F}$. Applying the item i) for the sequence
$(g_n)_n$ at point $v$, the sequence is bounded in a neighbourhood
of $v$. By Montel's theorem, there exists a subsequence
$\br{g_{n_k}}_k$ that converges to a function $g$ uniformly on
compacts of $\dsk(v, s)$. By corollary~\ref{thm_hol:cor:hurw},
the function $g$ is constant. But then
\[
\lim_{k \to \infty} \abs{f_{n_k}(z)} = \infty
\]
for all $z \in \dsk(v, s)$, which is not possible as $v$ is a
boundary point. It follows that $\mathcal{U} = \Omega$. \qedhere
\end{enumerate}
\end{proof}

\begin{definicija}
Let $\Omega \subseteq \C$ be a domain and
$f_n \colon \Omega \to \C$ a sequence of functions. We say that
$f_n$ \emph{converges to $\infty$}\index{convergence to $\infty$}
if
\[
\lim_{n \to \infty} \norm{f_n}_K = \infty
\]
for every compact $K \subset \Omega$.
\end{definicija}

\begin{izrek}[Montel -- sharp]
\index{Montel's theorem}
Let $\Omega \subseteq \C$ be a domain and
\[
\mathcal{F} =
\setb{f \in \mathcal{O}(\Omega)}{0, 1 \not \in f(\Omega)}.
\]
Then $\mathcal{F}$ is normal in $\Omega$ where we also allow
convergence to $\infty$.
\end{izrek}

\begin{proof}
Let $\Omega \subseteq \C$ be a domain and $p \in \Omega$. Consider
the family $\mathcal{F}_{p,1}$. Let $(f_n)_n \subseteq \mathcal{F}$
be a sequence. If there exists a subsequence
$\br{f_{n_k}}_k \subseteq \mathcal{F}_{p,1}$, we can apply the
above lemma. By the classical Montel's theorem, this subsequence
has a convergent subsequence.

Suppose now that no such subsequence exists, that is $(f_n)_n$ has
only finitely many terms in $\mathcal{F}_{p,1}$. But then there
exists a subsequence
$\br{\frac{1}{f_{n_k}}}_k \subseteq \mathcal{F}_{p,1}$. As before,
this sequence has a convergent subsequence with limit $g$. If $g$
is nowhere-vanishing, then $\frac{1}{g}$ is the limit of a
subsequence of $(f_n)_n$. Otherwise, by
corollary~\ref{thm_hol:cor:hurw}, we have $g = 0$ and therefore
$(f_n)_n$ converges to $\infty$.
\end{proof}

\begin{definicija}
Let $\Omega \subseteq \C$ be an open set and $p \in \Omega$. A
function $f \in \mathcal{O}(\Omega \setminus \set{p})$ has an
\emph{essential singularity}\index{essential singularity} in $p$ if
the limit
\[
\lim_{z \to p} f(z)
\]
does not exist and
\[
\lim_{z \to p} \abs{f(z)} \ne \infty.
\]
\end{definicija}

\begin{izrek}[Picard's great theorem]
\index{Picard's great theorem}
Let $\Omega \subseteq \C$ be an open set $p \in \Omega$ a point and
$f \in \mathcal{O}(\Omega \setminus \set{p})$ a function. If $f$
has an essential singularity at $p$, then $f$ assumes every complex
number as a value infinitely many times with at most one exception.
\end{izrek}

\begin{proof}
Without loss of generality assume that $p = 0$ and consider
$\Omega = \dsk(\varepsilon)$. Suppose that $f$ omits two values on
$\dsk(\varepsilon)$, without loss of generality $0$ and $1$.

We now claim that $f$ or $\frac{1}{f}$ is bounded in a
neighbourhood of $0$. Define the sequence of holomorphic functions
$(f_n)_n$ with $f_n(z) = f \br{\frac{z}{n}}$. This sequence also
omits $0$ and $1$, therefore either $(f_n)_n$ or
$\br{\frac{1}{f_n}}_n$ has a convergent subsequence that converges
uniformly on compacts by the sharp version of Montel's theorem.
Denote the subsequence by $\br{g_{n_k}}_k$ and set $g = f$ or
$g = \frac{1}{f}$ accordingly.

Observe that there exists a constant $M$ such that
\[
\norm{g_{n_k}}_{\partial \dsk \br{\frac{\varepsilon}{2}}} \leq M
\]
holds for all $k \in \N$. This is equivalent to
\[
\abs{g(z)} \leq M
\]
for $\abs{z} = \frac{1}{n_k} \cdot \frac{\varepsilon}{2}$. By the
maximum principle, we have
\[
\abs{g(z)} \leq M
\]
for all $z$ such that
\[
\frac{\varepsilon}{2} \cdot \frac{1}{n_k} \leq
\abs{z} \leq
\frac{\varepsilon}{2}.
\]
But as $(n_k)_k$ diverges, the inequality $g(z) \leq M$ holds for
all $z$ such that $\abs{z} \leq \frac{\varepsilon}{2}$, therefore
$f$ or $\frac{1}{f}$ is bounded near $0$.

Observe that $f$ is not bounded in a neighbourhood of $0$, as
otherwise $0$ is a removable singularity, which is not possible.
Similarly, if $\frac{1}{f}$ is bounded, then $f$ has either a
removable singularity or a pole at $0$, which is again a
contradiction.
\end{proof}
