\section{Infinite products}

\subsection{Definition and convergence}

\begin{definicija}
\index{infinite product}
Let $(a_k)_k$ be a sequence of complex numbers. The sequence
\[
n \mapsto \prod_{k=1}^n a_k
\]
is called the
\emph{sequence of partial products}\index{partial products} with
factors $a_k$. We denote
\[
p_{m,n} = \prod_{k=m}^n a_k.
\]
We say that the infinite product is \emph{convergent} if there
exists an index $m \in \N$ such that the limit
\[
\widehat{a}_m = \lim_{n \to \infty} p_{m,n}
\]
exists and is non-zero. We then define
\[
\prod_{k=1}^\infty a_k = p_{1, m-1} \cdot  \widehat{a}_m.
\]
as the limit of the infinite product.
\end{definicija}

\begin{opomba}
The limit is uniquely defined.
\end{opomba}

\begin{opomba}
An infinite product is convergent if and only if the product of all
its non-zero factors has a non-zero limit and only finitely many
factors are non-zero.
\end{opomba}

\datum{2023-11-22}

\begin{lema}
Let $(a_k)_k \subseteq \R_{\geq 0}$ be a sequence such that
\[
\sum_{k=1}^\infty (1 - a_k) = \infty.
\]
Then
\[
\lim_{n \to \infty} \prod_{k=p}^n a_k = 0
\]
for all $p \in \N$. In particular, the infinite product is
divergent.
\end{lema}

\begin{proof}
Observe that
\[
0 \leq
\prod_{k=p}^n a_k \leq
\prod_{k=p}^n e^{a_k - 1},
\]
which converges to $0$.
\end{proof}

\begin{definicija}
Let $X \subseteq \C$ be a set.

\begin{enumerate}[i)]
\item A series
\[
\sum_{k=1}^\infty g_k
\]
of continuous functions $g_k \in \mathcal{C}(X)$ is
\emph{normally convergent}\index{normal convergence} if for every
compact $K \subseteq X$ the series
\[
\sum_{k=1}^\infty \norm{g_k}_K
\]
converges.

\item A product
\[
\prod_{k=1}^\infty f_k
\]
of continuous functions $f_k = 1 + g_k \in \mathcal{C}(X)$ is
\emph{normally convergent} if the series
\[
\sum_{k=1}^\infty g_k
\]
is normally convergent.
\end{enumerate}
\end{definicija}

\begin{definicija}
Let $X \subseteq \C$ be a set and $f_k \in \mathcal{C}(X)$ be
continuous functions. Denote
\[
p_{m,n} = \prod_{k=m}^n f_k.
\]
We say that the infinite product
\[
\prod_{k=1}^\infty f_k
\]
converges \emph{uniformly}\index{uniform convergence} on a set
$L \subseteq X$ if there exists an index $m \in \N$ such that
$\eval{f_k}{L}{}$ has no zeroes for $k \geq m$ and
\[
\lim_{n \to \infty} p_{m,n} = \widehat{f}_k
\]
exists, is uniform on $L$ and has no zeroes on $L$. We define
\[
\prod_{k=1}^\infty f_k = p_{1, m-1} \cdot \widehat{f}_m
\]
on $L$.
\end{definicija}

\begin{izrek}[Reordering of infinite products]
\index{reordering theorem}
Let
\[
\prod_{k=1}^\infty f_k
\]
be a normally convergent product in $X \subseteq \C$. Then there
exists a functions $f \colon X \to \C$ such that for all bijections
$\tau \colon \N \to \N$ the product
\[
\prod_{k=1}^\infty f_{\tau(k)}
\]
converges to $f$ uniformly on compacts of $X$. In particular, the
infinite product converges uniformly on compacts.
\end{izrek}

\begin{proof}
Recall that, for $w \in \dsk$, we can define
\[
\log(1+w) = \sum_{k=1}^\infty \frac{(-1)^{k+1}}{k} w^k.
\]
Then,
\[
\abs{\log(1+w)} \leq
\abs{w} \cdot \sum_{k=0}^\infty \abs{w}^k =
\frac{\abs{w}}{1 - \abs{w}}.
\]
In particular, if $\abs{w} \leq \frac{1}{2}$, we have
\[
\abs{\log(1+w)} \leq 2 \abs{w}.
\]
Let $L \subseteq X$ be a compact and write $f_k = 1 + g_k$. For
all $k > N$ we have $\norm{g_k}_L \leq \frac{1}{2}$, therefore we
can write
\[
\log f_k =
\log(1 + g_k) =
\sum_{\ell=1}^\infty \frac{(-1)^{\ell+1}}{\ell} g_k^\ell.
\]
But then
\[
\norm{\log f_k}_L \leq 2 \norm{g_k}_L.
\]
It follows that the series
\[
\sum_{k=N}^\infty \norm{\log f_k}_L
\]
converges. But then the series
\[
h_N = \sum_{k=N}^\infty \log f_k
\]
converges absolutely, and therefore all reorderings of the series
converge as well to the same limit $h_N$.

Observe that
\[
e^{h_N} = \prod_{k=N}^\infty e^{\log f_k} = \prod_{k=N}^\infty f_k.
\]
This product therefore converges uniformly on $L$, independently of
reorderings. We now define
\[
f = \prod_{k=1}^{N-1} f_k \cdot e^{h_N}.
\]
Note that this holds for all reorderings, as they differ from a
suitable one by only finitely many transpositions.
\end{proof}

\newpage

\subsection{Zeroes of infinite products}

\datum{2023-11-28}

\begin{definicija}
Let $\Omega \subseteq \C$ be an open set and
$f \in \mathcal{O}(\Omega)$. The \emph{zero set}\index{zero set} of
$f$ is the set
\[
Z(f) = \setb{z \in \Omega}{f(z) = 0}.
\]
For all $c \in \Omega$, define the
\emph{zero order}\index{zero order} of $f$ in $c$ as follows: if
\[
f(z) = (z - c)^k \cdot g(z)
\]
where $g(c) \ne 0$ is a holomorphic function, then $\ord_c(f) = k$.
\end{definicija}

\begin{opomba}
For non-zero $f \in \mathcal{O}(\Omega)$, the set $Z(f)$ is
discrete in $\Omega$.
\end{opomba}

\begin{opomba}
We have
\[
\ord_c \br{\prod_{k=1}^n f_k} = \sum_{k=1}^n \ord_c(f_k).
\]
\end{opomba}

\begin{lema}
Let $\Omega \subseteq \C$ be a domain and
\[
f = \prod_{k=1}^\infty f_k
\]
be a normally convergent product in $\Omega$, where
$f_k \in \mathcal{O}(\Omega)$ are non-zero holomorphic functions.
Then $f$ is a non-zero function with
\[
Z(f) = \bigcup_{k=1}^\infty Z(f_k)
\]
and
\[
\ord_c(f) = \sum_{k=1}^\infty \ord_c(f_k).
\]
\end{lema}

\begin{proof}
Recall that normally convergent products converge uniformly on
compacts of $\Omega$. In particular, $f$ is a holomorphic function.

Pick a point $c \in \Omega$. By definition of convergence, there
exists some $m \in \N$ such that $\widehat{f}_m(c) \ne 0$. As
$\widehat{f}_m$ is holomorphic as well, we have
\[
f(c) = \br{p_{1, m-1} \cdot \widehat{f}_m}(c),
\]
but then
\[
\ord_c(f) =
\sum_{k=1}^{m-1} \ord_c(f_k) =
\sum_{k=1}^\infty \ord_c(f_k). \qedhere
\]
\end{proof}

\begin{lema}
Let $\Omega \subseteq \C$ be a domain. If
\[
f = \prod_{k=1}^\infty f_k
\]
is a normally convergent product in $\Omega$, where
$f_k \in \mathcal{O}(\Omega)$ are holomorphic functions, then the
sequence $\br{\widehat{f}_n}_n$ converges to $1$ uniformly on
compacts.
\end{lema}

\begin{proof}
Choose $m \in \N$ such that $\widehat{f}_m \ne 0$. Then the set
$Z \br{\widehat{f}_m}$ has no accumulation points in $\Omega$. We
can therefore write
\[
\widehat{f}_n = \frac{\widehat{f}_m}{p_{m,n-1}}
\]
on $\Omega \setminus Z \br{\widehat{f}_m}$. As $p_{m,n-1}$
converges to $\widehat{f}_m$ on compacts of $\Omega$,
\[
\lim_{n \to \infty} \widehat{f}_n = 1
\]
uniformly on compacts of $\Omega \setminus Z \br{\widehat{f}_m}$.
For any compact set $K \subseteq \Omega$, taking $m$ large enough,
we have $Z \br{\widehat{f}_m} \cap K = \emptyset$. The conclusion
follows.
\end{proof}

\begin{definicija}
Let $\Omega \subseteq \C$ be a domain and
$f \in \mathcal{O}(\Omega)$. The meromorphic function
$\frac{f'}{f}$ is called the
\emph{logarithmic derivative}\index{logarithmic derivative} of $f$.
\end{definicija}

\begin{opomba}
For holomorphic functions $f_1, \dots, f_n \in \mathcal{O}(\Omega)$
we have
\[
\br{\prod_{k=1}^n f_k}' \cdot \br{\prod_{k=1}^n f_k}^{-1} =
\sum_{k=1}^n \frac{f_k'}{f_k}.
\]
\end{opomba}

\begin{definicija}
Let $g_k \in \mathcal{M}(\Omega)$ be meromorphic functions. The
series
\[
\sum_{k=1}^\infty g_k
\]
is \emph{normally convergent}\index{normal convergence} in $\Omega$
if for every compact $L \subseteq \Omega$ there exists some
$m \in \N$ such that
\[
\sum_{k=m}^\infty \norm{g_k}_L
\]
converges.
\end{definicija}

\begin{izrek}[Logarithmic differentiation]
Let $\Omega \subseteq \C$ be a domain and
\[
f = \prod_{k=1}^\infty f_k
\]
be a normally convergent product in $\Omega$, where
$f_k \in \mathcal{O}(\Omega)$ are non-zero functions. Then
\[
\sum_{k=1}^\infty \frac{f_k'}{f_k}
\]
is normally convergent in $\Omega$ and
\[
\sum_{k=1}^\infty \frac{f_k'}{f_k} = \frac{f'}{f}.
\]
\end{izrek}

\begin{proof}
As $\widehat{f}_n$ converges to $1$ uniformly on compacts, the
sequence $\br{f_n'}_n$ converges to $0$ uniformly on compacts by
Cauchy estimates. Then for any compact $L$,
$\frac{\widehat{f}_n'}{\widehat{f}_n}$ converges to $0$ as
$\widehat{f}_n$ has no zeroes in $L$ for $n$ large enough. It
follows that
\[
\lim_{n \to \infty} \frac{f'}{f} - \sum_{k=1}^n \frac{f_k'}{f_k} =
\lim_{n \to \infty} \frac{\widehat{f}_{n+1}'}{\widehat{f}_{n+1}} =
0.
\]
Write $f_k = 1 + g_k$ and fix a compact set $L \subseteq \Omega$.
Choose an index $m$ such that we have
$Z \br{\widehat{f}_m} \cap L = \emptyset$ and
\[
\min_{z \in L} \abs{f_k(z)} \geq \frac{1}{2}.
\]
Choose $\varepsilon > 0$ such that
\[
L_\varepsilon =
\setb{z \in \C}{d(z, L) \leq \varepsilon} \subseteq
\Omega.
\]
By the Cauchy estimates, we have
$\norm{g_k'}_L \leq \frac{1}{\varepsilon} \norm{g_k}_L$. But then
\[
\sum_{k=m}^\infty \norm{\frac{f_k'}{f_k}}_L =
\sum_{k=m}^\infty \norm{\frac{g_k'}{f_k}}_L \leq
2 \cdot \sum_{k=m}^\infty \norm{g_k'}_L \leq
\frac{2}{\varepsilon} \cdot \sum_{k=m}^\infty \norm{g_k},
\]
which is convergent by our assumptions.
\end{proof}

\begin{lema}
Let $g$ be meromorphic on $\C$ with poles in $\Z$ with principal
parts $\frac{1}{z-m}$. Moreover, assume that $g$ is an odd
function that satisfies
\[
2 g(2z) = g(z) + g \br{z + \frac{1}{2}}.
\]
Then $g(z) = \pi \cdot \cot(\pi z)$.
\end{lema}

\begin{proof}
Simple calculations show that $\pi \cdot \cot(\pi z)$ is indeed a
solution of the functional equation. Define
$h(z) = g(z) - \pi \cdot \cot(\pi z)$. This another solution of the
functional equation, and an odd function. In particular,
$h(0) = 0$. Observe that the principal parts of $h$ are $0$,
therefore $h \in \mathcal{O}(\C)$ is an entire function.

Suppose that $h$ is not constant. In particular, there exists some
$c \in \partial \dsk(2)$ such that
\[
\abs{h(z)} < \abs{h(c)}
\]
for all $z \in \dsk(2)$. As
$\frac{c}{2}, \frac{c+1}{2} \in \dsk(2)$ , we can write
\[
2 \abs{h(c)} =
\abs{h \br{\frac{c}{2}} + h \br{\frac{c+1}{2}}} \leq
\abs{h \br{\frac{c}{2}}} + \abs{h \br{\frac{c+1}{2}}} <
2 \abs{h(c)},
\]
which is a contradiction. It follows that $h = 0$.
\end{proof}

\begin{posledica}
We have
\[
\pi \cdot \cot(\pi z) =
\frac{1}{z} + \sum_{k=1}^\infty \frac{2z}{z^2 - k^2}.
\]
\end{posledica}

\begin{proof}
Note that
\[
\frac{1}{z} + \sum_{k=1}^\infty \frac{2z}{z^2 - k^2} =
\frac{1}{z} + \sum_{k=1}^\infty \br{\frac{1}{z-k} + \frac{1}{z+k}},
\]
therefore the series has poles in $\Z$ with principal parts
$\frac{1}{z-m}$. It is also an odd function. A calculation shows
that, for
\[
r_n(z) = \frac{1}{z} + \sum_{k=1}^n \frac{2z}{z^2 - k^2},
\]
we have
\[
r_n(z) + r_n \br{z + \frac{1}{2}} =
2 r_{2n}(2z) + \frac{2}{2z + 2n + 1}.
\]
Taking $n \to \infty$, the conclusion follows.
\end{proof}

\begin{izrek}
We have
\[
\sin(\pi z) =
\pi z \cdot \prod_{k=1}^\infty \br{1 - \frac{z^2}{k^2}}.
\]
\end{izrek}

\begin{proof}
The above product is obviously normally convergent, therefore we
can take its logarithmic derivative. A simple calculation shows
that it is equal to $\pi \cot(\pi z)$. As logarithmic derivatives
are equal only for scalar multiples, we only have to check equality
in one point.
\end{proof}

\newpage

\subsection{The Euler gamma function}

\datum{2023-11-29}

\begin{lema}
The infinite product
\[
\prod_{k=1}^\infty \br{1 + \frac{1}{k}} e^{-\frac{z}{k}}
\]
in normally convergent in $\C$.
\end{lema}

\begin{proof}
Write
\begin{align*}
\abs{1 - (1-\omega) e^\omega} &=
\abs{1 - e^\omega + \omega e^\omega}
\\
&=
\abs{-\sum_{k=1}^\infty \frac{\omega^k}{k!} +
\sum_{k=0}^\infty \frac{\omega^{k+1}}{k!}}
\\
&=
\abs{\omega^2 \cdot \sum_{k=1}^\infty
\br{\frac{1}{k!} - \frac{1}{(k+1)!}} \omega^{k-1}}
\\
&\leq
\abs{\omega}^2 \cdot \sum_{k=1}^\infty
\br{\frac{1}{k!} - \frac{1}{(k+1)!}}
\\
&=
\abs{\omega}^2
\end{align*}
for $\abs{\omega} \leq 1$. But then the sum
\[
\sum_{k = \ceil{\abs{z}}}^\infty
\abs{1 - \br{1 + \frac{z}{k}} e^{-\frac{z}{k}}} \leq
\sum_{k = \ceil{\abs{z}}}^\infty \abs{\frac{z^2}{k^2}}
\]
converges normally. The infinite product must then converge
normally in $\C$ as well.
\end{proof}

\begin{lema}
Let
\[
H(z) =
z \cdot \prod_{k=1}^\infty \br{1 + \frac{z}{k}} e^{-\frac{z}{k}}.
\]
Then $H(1) = e^{-\gamma}$, where $\gamma$ is the
\emph{Euler-Mascheroni constant}\index{Euler-Mascheroni constant},
that is
\[
\gamma = \lim_{n \to \infty} \sum_{k=1}^n \frac{1}{k} - \log(n).
\]
\end{lema}

\begin{proof}
First note that
\[
\prod_{k=1}^n \br{1 + \frac{1}{k}} =
\prod_{k=1}^n \frac{k+1}{k} =
n+1.
\]
We therefore have
\[
\prod_{k=1}^n \br{1 + \frac{1}{k}} e^{-\frac{1}{k}} =
\exp \br{\log(n+1) - \sum_{k=1}^n \frac{1}{k}},
\]
therefore
\[
H(1) =
\lim_{n \to \infty}
\exp \br{\log(n+1) - \sum_{k=1}^n \frac{1}{k}} =
e^{-\gamma}. \qedhere
\]
\end{proof}

\begin{lema}
Let $\Delta(z) = e^{\gamma z} H(z)$.

\begin{enumerate}[i)]
\item We have $\Delta(1) = 1$ and $\Delta(z) = z \Delta(z+1)$.
\item We have $\pi \cdot \Delta(z) \Delta(1-z) = \sin(\pi z)$.
\end{enumerate}
\end{lema}

\begin{proof}
Note that $\Delta(1) = 1$ by the previous lemma. Rewrite the
partial products as
\[
z \cdot \prod_{k=1}^n \br{1 + \frac{z}{k}} e^{-\frac{z}{k}} =
\frac{z}{n!} \cdot \prod_{k=1}^n (z+k) \cdot
\exp \br{-z \sum_{k=1}^n \frac{1}{k}}.
\]
We therefore have
\begin{align*}
\Delta(z) &=
\lim_{n \to \infty}
\frac{e^{\gamma z}}{n!} \cdot \prod_{k=0}^n (z+k) \cdot
\exp \br{-z \sum_{k=1}^n \frac{1}{k}}
\\
&=
\lim_{n \to \infty}
\frac{e^{\gamma z}}{n! \cdot n^z} \cdot \prod_{k=0}^n (z+k) \cdot
\exp \br{z \log(n) -z \sum_{k=1}^n \frac{1}{k}}
\\
&=
\lim_{n \to \infty}
\frac{1}{n! \cdot n^z} \cdot \prod_{k=0}^n (z+k).
\end{align*}
We can now calculate
\[
z \cdot \Delta(z+1) =
\lim_{n \to \infty}
z \cdot \frac{1}{n! \cdot n^{z+1}} \cdot \prod_{k=1}^{n+1} (z+k) =
\Delta(z) \cdot \lim_{n \to \infty} \frac{z + n + 1}{n} =
\Delta(z).
\]
It remains to check the equality
$\pi \cdot \Delta(z) \Delta(1-z) = \sin(\pi z)$. We have
\begin{align*}
\pi \cdot \Delta(z) \Delta(1-z) &=
\pi \cdot \Delta(z) \cdot \frac{\Delta(-z)}{-z}
\\
&=
\pi e^{\gamma z} \cdot z \cdot
\prod_{k=1}^\infty \br{1 + \frac{z}{k}} e^{-\frac{z}{k}} \cdot
e^{-\gamma z} \cdot \frac{-z}{-z} \cdot
\prod_{k=1}^\infty \br{1 - \frac{z}{k}} e^{\frac{z}{k}}
\\
&=
\pi z \cdot \prod_{k=1}^\infty \br{1 - \frac{z^2}{k^2}}
\\
&=
\sin(\pi z). \qedhere
\end{align*}
\end{proof}

\begin{definicija}
The \emph{Euler gamma function}\index{Euler gamma function} is
defined as
\[
\Gamma(z) = \frac{1}{\Delta(z)}.
\]
\end{definicija}

\begin{izrek}
The $\Gamma$ function satisfies the following properties:

\begin{enumerate}
\item The function $\Gamma$ is meromorphic with simple poles in
$-\N_0$.
\item We have $\Gamma(1) = 1$.
\item The function $\Gamma$ satisfies $\Gamma(z+1) = z \Gamma(z)$.
\item The function $\Gamma$ satisfies
\[
\Gamma(z) \cdot \Gamma(1-z) = \frac{\pi}{\sin(\pi z)}.
\]
\item We have
\[
\Gamma(z) =
\lim_{n \to \infty}
n! \cdot n^z \cdot \br{\prod_{k=0}^n (z+k)}^{-1}.
\]
\end{enumerate}
\end{izrek}

\obvs

\begin{izrek}
Let $F$ be holomorphic in $\setb{z \in \C}{\Re(z) > 0}$ and assume
$F(z+1) = z \cdot F(z)$. Furthermore, assume that $F$ is bounded
on the strip $1 \leq \Re(z) < 2$ and $F(1) = 1$. Then $F = \Gamma$.
\end{izrek}

\newpage

\subsection{Weierstraß factors}

\datum{2023-12-5}

\begin{definicija}
The \emph{Weierstraß factors}\index{Weierstraß!factors} are
functions
\[
E_n(z) = (1-z) \cdot \exp \br{\sum_{\ell=1}^n \frac{z^n}{n}}.
\]
\end{definicija}

\begin{lema}
The Weierstraß factors satisfy the following:

\begin{enumerate}[i)]
\item For $n \geq 1$ we have
\[
E_n'(z) = -z^n \cdot \exp \br{\sum_{\ell=1}^n \frac{z^n}{n}}.
\]
\item For $n \geq 0$ we have
\[
E_n(z) = 1 + \sum_{k=n+1}^\infty a_k z^k,
\]
where
\[
\sum_{k=n+1}^\infty \abs{a_k} = 1.
\]
\item For $n \geq 0$ and $\abs{z} \leq 1$ we have
\[
\abs{E_n(z) - 1} \leq \abs{z}^{n+1}.
\]
\end{enumerate}
\end{lema}

\begin{proof}
\phantom{i}
\begin{enumerate}[i)]
\item Evident.
\item Observing the derivative, we see that
$a_1 = a_2 = \dots = a_n = 0$, and $a_k \leq 0$ for $k > n$. But
then
\[
\sum_{k=n+1}^\infty \abs{a_k} =
-\sum_{k=n+1}^\infty a_k =
1 - E_n(1) =
1.
\]
\item We have
\[
\abs{E_n(z) - 1} =
\abs{\sum_{k=n+1}^\infty a_k z^k} \leq
\sum_{k=n+1}^\infty \abs{a_k} \cdot \abs{z}^k \leq
\abs{z}^{n+1}. \qedhere
\]
\end{enumerate}
\end{proof}

\begin{lema}
Let $(a_k)_k \subset \C^*$ be a sequence of complex numbers with no
accumulation point and let $(p_k)_k \subseteq \N_0$ be non-negative
integers with
\[
\sum_{k=1}^\infty \abs{\frac{r}{a_k}}^{p_k+1}
\]
converges for every $r > 0$. Then the
\emph{Weierstraß product}\index{Weierstraß!product}
\[
\prod_{k=1}^\infty E_{p_k} \br{\frac{z}{a_k}}
\]
converges normally on $\C$.
\end{lema}

\begin{proof}
Note that $\abs{a_k} > \abs{z}$ for all but finitely many $k$. Now
just apply the previous lemma.
\end{proof}

\begin{izrek}[Weierstraß factorization theorem]
\index{Weirestraß!factorization theorem}
For any sequence $(a_k)_k \subset \C$ with no accumulation point
there exists a Weierstraß product
\[
z^q \cdot \prod_{\substack{k=1 \\ a_k \ne 0}}^\infty
E_{p_k} \br{\frac{z}{a_k}}
\]
that converges normally on $\C$.
\end{izrek}

\begin{proof}
Set $p_k = k-1$. For any $r > 0$ choose $m \in \N_0$ such that
$\abs{a_k} > 2r$ for all $k \geq m$. We then have
\[
\sum_{k=m}^\infty \abs{\frac{r}{a_k}}^{p_k+1} \leq
\sum_{k=m}^\infty \frac{1}{2^k} \leq
2. \qedhere
\]
\end{proof}

\begin{izrek}[Weierstraß product theorem]
\index{Weirestraß!product theorem}
\label{inf_prod:thm:w_prod_1}
Let $f \in \mathcal{O}(\C) \setminus \set{0}$ be a holomorphic
function. Then there exists a function $g \in \mathcal{O}(\C)$ such
that
\[
f = e^g \cdot z^q \cdot \prod_{\substack{k=1 \\ a_k \ne 0}}^\infty
E_{k-1} \br{\frac{z}{a_k}},
\]
where $a_k$ are zeroes of $f$ on $\C \setminus \set{0}$, counted
with multiplicities, and $q = \ord_0(f)$.
\end{izrek}

\obvs

\begin{lema}
\label{inf_prod:lm:w_prod_omega}
Let $\Omega \subset \C$ be an open subset, $(a_k)_k \subset \Omega$
a sequence with no accumulation point in $\Omega$ and
$A = \setb{a_k}{k \in \N}$. Let
$(b_k)_k \subset \C \setminus \Omega$ and $(p_k)_k \subseteq \N$ be
sequences such that the series
\[
\sum_{k=1}^\infty \abs{r (a_k - b_k)}^{p_k+1}
\]
converges for all $r > 0$ and denote $B = \setb{b_k}{k \in \N}$.
Then the infinite product
\[
\prod_{k=1}^\infty E_{p_k} \br{\frac{a_k - b_k}{z - b_k}}
\]
converges normally on $\C \setminus \oline{B}$.
\end{lema}

\begin{proof}
Let $L \subseteq \C \setminus \oline{B}$ be a compact set and let
$\ell = d\br{L, \oline{B}} > 0$. We then have
$\abs{z - b_k} \geq \ell$ for all $z \in L$ and $k \in \N$.

We can now bound
\[
\norm{\frac{a_k - b_k}{z - b_k}}_L \leq
\frac{\abs{a_k - b_k}}{\ell}.
\]
By the assumption of convergence for $r = \frac{1}{\ell}$, we must
have
\[
\abs{r \cdot (a_k - b_k)} < 1
\]
for all $k \geq n(L)$, but then
\[
\sum_{k=n(L)}^\infty
\norm{E_{p_k} \br{\frac{a_k - b_k}{z - b_k}} - 1}_L \leq
\sum_{k=n(L)}^\infty 
\norm{\frac{a_k - b_k}{z - b_k}}_L^{p_k+1}\leq
\sum_{k=n(L)}^\infty \abs{r \cdot \br{a_k - b_k}}^{p_k+1},
\]
which converges.
\end{proof}

\begin{opomba}
The Weierstraß factor $E_{p_k} \br{\frac{a_k - b_k}{z - b_k}}$ is
zero if and only if $z = a_k$.
\end{opomba}

\begin{lema}
\label{inf_prod:lm:divideA}
Let $A \subset \C$ be a discrete set and define
$A' = \oline{A} \setminus A$. Suppose that $A' \ne \emptyset$ and
let
\[
A_1 = \setb{z \in A}{\abs{z} \cdot d(z, A') \geq 1}
\]
and $A_2 = A \setminus A_1$. Now let
\[
A_2(\varepsilon) = \setb{z \in A_2}{d(z, A') \geq \varepsilon}.
\]
Then $A_1$ is a closed set and $A_2(\varepsilon)$ is finite for any
$\varepsilon > 0$.
\end{lema}

\begin{proof}
Assume $A_1$ has an accumulation point $a$ and let
$(a_k)_k \subseteq A$ be a sequence, converging to $a$. But then
\[
\lim_{k \to \infty} \abs{a_k} \cdot d(a_k, A') = 0,
\]
which is a contradiction.

Note that, for all $z \in A_2(\varepsilon)$, we have
$\abs{z} < \frac{1}{\varepsilon}$. If the set is infinite, it has
an accumulation point, which is impossible as
$d(z, A') \geq \varepsilon$.
\end{proof}

\begin{opomba}
If $A \subset \C$ is a discrete set, then $A'$ is a closed set in
$\C$.
\end{opomba}

\datum{2023-12-6}

\begin{izrek}[Weierstraß product theorem]
\index{Weierstraß!product theorem}
Let $\Omega \subseteq \C$ be an open subset. Let
$(a_k)_k \subset \Omega$ be a sequence without accumulation points
in $\Omega$ and denote $A = \setb{a_k}{k \in \N}$ and
$A' = \oline{A} \setminus A$. Then there exists a Weierstraß
product for $(a_k)_k$ that converges normally in $\C \setminus A'$.
This product has zeros precisely in $(a_k)_k$, counted with
multiplicities.
\end{izrek}

\begin{proof}
Assume that $\Omega \ne \C$ and
$A' \ne \emptyset$.\footnote{Otherwise just apply
theorem~\ref{inf_prod:thm:w_prod_1}.} Write $A = A_1 \cup A_2$ as
in the above lemma. Recall that $A_1$ has no accumulation points,
therefore we can apply theorem~\ref{inf_prod:thm:w_prod_1} for
$A_1$. It remains to construct a Weierstraß product for $A_2$.

Observe that $A' = A_2'$. As this is a closed space, for all
$a_k \in A_2$ there exists some $b_k \in A_2'$ such that
\[
\abs{a_k - b_k} = d(a_k, A_2').
\]
Observe that
\[
\lim_{\substack{k \to \infty \\ a_k \in A_2}} \abs{a_k - b_k} = 0,
\]
as the sets $A_2(\varepsilon)$ are finite. Now set $p_k = k$ and
apply lemma~\ref{inf_prod:lm:w_prod_omega}.
\end{proof}

\begin{posledica}[Blaschke products]
\index{Blaschke products}
Let $(a_k)_k \subset \dsk \setminus \set{0}$ be a sequence without
accumulation points in $\dsk$. If the series
\[
\sum_{k=1}^\infty \br{1 - \abs{a_k}}
\]
converges, then the product
\[
\prod_{k=1}^\infty
E_0 \br{\frac{a_k - \frac{1}{\oline{a}_k}}
{z - \frac{1}{\oline{a}_k}}}
\]
converges normally in $\dsk$ and has zeros precisely in $(a_k)_k$,
counted with multiplicities.
\end{posledica}

\begin{proof}
Note that
\[
\abs{a_k - b_k} =
\abs{a_k - \frac{1}{\oline{a}_k}} =
\abs{\frac{1}{\oline{a}_k}} \cdot \abs{\abs{a_k}^2 - 1} =
\abs{\frac{1}{\oline{a}_k}} \cdot
(1 - \abs{a_k}) (1 + \abs{a_k}) \leq
\frac{2}{m} \cdot \br{1 - \abs{a_k}},
\]
where
\[
m = \min \setb{\abs{a_k}}{k \in \N}.
\]
It follows that the series
\[
\sum_{k=1}^\infty r \cdot \abs{a_k - b_k}
\]
converges, therefore we can apply
lemma~\ref{inf_prod:lm:w_prod_omega}.
\end{proof}

\begin{izrek}
Let $\Omega \subseteq \Omega$ be a domain and
$f \in \mathcal{O}(\Omega) \setminus \set{0}$. Then we can write
\[
f = g \cdot \prod_{k=1}^\infty f_k,
\]
where $g \in \mathcal{O}^*(\Omega)$ and $f_k$ are Weierstraß
factors.
\end{izrek}

\obvs

\begin{izrek}
Let $\Omega \subseteq \C$ be a domain and
$f \in \mathcal{M}(\Omega)$. Then we can write $f = \frac{g}{h}$,
where $g, h \in \mathcal{O}(\Omega)$.
\end{izrek}

\begin{proof}
Define $h$ as the Weierstraß product of the poles of $f$.
\end{proof}

\begin{opomba}
Let $\Omega \subseteq \C$ be a domain. Then $\mathcal{O}(\Omega)$
is not a factorial ring,\footnote{``Kolobar z enolično
faktorizacijo.''} but $\gcd(f, g) \in \mathcal{O}(\Omega)$ exists.
\end{opomba}

\datum{2023-12-12}

\begin{definicija}
Let $\Omega$ be an open subset and $\set{a_k}_k$ be a sequence
without accumulation pints and without repetition. Let
\[
q_k(z) = \sum_{n=1}^\infty c_{k, m} (z - a)^{-m}
\]
be a principal part in $a_k$ for each $k$.

If there exist functions $g_k \in \mathcal{O}(\Omega)$ for each $k$
such that
\[
\sum_{k=1}^\infty q_k - g_k
\]
converges normally in $\Omega$, we call it the
\emph{Mittag-Leffler series}\index{Mittag-Leffler!series} for the
distribution of principal part $(a_k, q_k)$.
\end{definicija}

\begin{opomba}
We adopt the following conventions: If $0 \in \set{a_k}_k$, then
$a_1 = 0$.
\end{opomba}

\begin{izrek}[Mittag-Leffler for $\C$]
\label{inf_prod:thm:MLforC}
\index{Mittag-Leffler!theorem}
For every distribution of principal parts in $\C$ there exists a
corresponding Mittag-Leffler series.
\end{izrek}

\begin{proof}
Let $g_k$ be the Taylor series of $q_k$ about $0$ in the disk
$\dsk(\abs{a_k})$ such that the inequality
$\norm{q_k - g_k}_{\oline{\dsk(\frac{1}{2} \abs{a_k})}} < 2^{-k}$ holds for each $k \geq 2$. Note that
\[
\lim_{k \to \infty} \abs{a_k} = \infty
\]
as the points don't accumulate. For each $r > 0$ we can therefore
find an integer $n$ such that $r < \frac{1}{2} \abs{a_k}$ for all
$k \geq n$. Then
\[
\sum_{k = n}^{\infty} \norm{q_k - g_k}_{\oline{\dsk(r)}} \leq 1.
\qedhere
\]
\end{proof}

\begin{opomba}
The above series
$f \in \mathcal{O}(\C \setminus \set{a_1, a_2, \dots})$ with
principal parts $q_k$ in $a_k$ for each $k \in \N$. If the
principal part are finite, then $f \in \mathcal{M}$.
\end{opomba}

\begin{lema}
Let $a \in \C$, $q \in \mathcal{O} (\Omega \setminus \set{a})$ be a
principal part and $b \in \C \setminus \set{a}$. Then $q$ has a
Laurent series expansion about $b$ in the annulus
$\setb{z \in \C}{\abs{z - b} > \abs{a - b}}$ of the form
\[
q(z) = \sum_{m=1}^\infty c_m (z-b)^{-m}
\]
that converges uniformly for $\abs{z - b} \geq r > \abs{a - b}$.
\end{lema}

\begin{proof}
Choose a path $\gamma_r$ that goes around the circle centered at
$b$ of radius $r$. We claim that
\[
c_m =
\frac{1}{2 \pi} \lint_{\gamma_r} \frac{q(z)}{(z-b)^{-m+1}}\,dz
\]
for $m \in \Z$ suffice. We can estimate
\[
\abs{c_m} \leq
\frac{1}{2 \pi} \cdot 2 \pi \frac{\norm{q}_{\gamma_r}}{r^{- m}} =
\frac{\norm{q}_{\gamma_r}}{r^{-m}}.
\]
We know that $q(z)$ is of the form
\[
q(z) = \sum_{m = 1}^{\infty} d_m (z - a)^{-m}
\]
for some $d_m \in \C$ when developed into a Laurent series around
$a$. It is trivial to show that
\[
\lim_{\abs{z}\to \infty}q(z) = 0.
\]
Thus, $\norm{q}_{\gamma_r}$ approaches zero as $r$ goes to
infinity. If $m \leq 0$, then
\[
\lim_{r \to \infty} \norm{q}_{\gamma_r} r^{m} = 0.
\]
Therefore, $c_m = 0$ for $m \leq 0$ and
\[
q(z) = \sum_{m = 1}^{\infty} c_m (z - b)^{ - m}
\]
is indeed a power series in $z-b$ which converges uniformly for
$\abs{z - b}^{-1} \leq r$.
\end{proof}

\begin{definicija}
The partial sums of
\[
q_\ell(z) = \sum_{m = 1}^{\ell} c_m (z - b)^{ - m}
\]
are called the
\emph{$\ell$-th Laurent terms}\index{Laurent terms} of $q$ about
$b$.
\end{definicija}

\begin{lema}
\label{inf_prod:lm:sufficientForML}
Let $(a_k, q_k)_k$ be a distribution of principal parts in an open
set $\Omega \subseteq \C$, $A = \setb{a_k}{k \in \N}$ and
$A' = \oline{A} \setminus A$.\footnote{The closure is taken in
$\C$.} Assume there exists a sequence $(b_k)_k \subseteq A'$ with
\[
\lim_{k \to \infty} \abs{a_k - b_k} = 0.
\]
Let $q_{k, \ell}$ be the $\ell$-th Laurent term of $q_k$ about
$b_k$. Then there exists a sequence $(\ell_k)_k \subseteq \N_0$
such that
\[
\sum_{k=1}^{\infty}(q_k - q_{k, \ell_k})
\]
is a Mittag-Leffler series for $(a_k, q_k)_k$.
\end{lema}

\begin{proof}
For a principal part $q_k$ the Laurent series converges uniformly
on $\abs{z - b_k} > r$ for any $r > \abs{a_k - b_k}$ by the
previous lemma. Thus, we can choose $\ell_k$ large enough such that
\[
\abs{q_k(z) - q_{k, \ell}(z)} < 2^{-k}
\]
for all $z$ such that $\abs{z - b_k} \geq 2 \abs{a_k - b_k}$.

For any compact set $L \subseteq \C \setminus A'$, the distance to
$A'$ is strictly positive. Since
\[
\lim_{k \to \infty} \abs{a_k - b_k} = 0,
\]
the point $b_k$ must lie outside $L$ for large enough $k$. Thus,
there exists some $n(L) \in \N$ such that
\[
L \subseteq
\bigcap_{k \geq n(L)}
\setb{z \in \C}{\abs{z - b_k} \geq 2 \abs{a_k - b_k}}.
\]
Therefore, we can use the previous estimate on $L$, to get 
\[
\sum_{k \geq n(L)} \norm{q_k - q_{k, \ell_k}}_L \leq
\sum_{k \geq n(L)} 2^{- k} \leq
2. \qedhere
\]
\end{proof}

\begin{izrek}[Mittag-Leffler for open subsets]
\label{inf_prod:lm:MLTOpen}
\index{Mittag-Leffler!theorem}
Let $\Omega \subseteq \C$ be an open subset. Let $(a_k, q_k)_k$ be
a distribution of principal parts in $\Omega \subseteq$ and
$A = \setb{a_k}{k \in \N}$. Then there exists a Mittag-Leffler
series for $(a_k, q_k)$ that converges normally in
$\C \setminus A'$ where $A' = \oline{A} \setminus A$.
\end{izrek}

\begin{proof}
By lemma~\ref{inf_prod:lm:divideA}, $(A_1)'$ is empty and
$(A_2)' = A'$. If $A'$ is empty, then
\[
\lim_{k \to \infty} \abs{a_k} = \infty
\]
and we can apply theorem~\ref{inf_prod:thm:MLforC}. Similarly, we
can assume $\Omega \neq \C$. Again by the lemma, $A_2(\epsilon)$ is
finite. Hence, there exist $(b_k)_k \in A'$ such that
\[
\lim_{k \to \infty} \abs{a_k - b_k} = 0.
\]
We can apply lemma~\ref{inf_prod:lm:sufficientForML} to obtain a
Mittag-Leffler series. Now we apply
theorem~\ref{inf_prod:thm:MLforC} for $\C$ to $A_1$. Sum up this
two series to get the series from the statement.
\end{proof}

\begin{izrek}[Mittag-Leffler osculation theorem]
\label{inf_prod:thm:MLOsculation}
\index{Mittag-Leffler!osculation theorem}
Let $\Omega \subseteq \C$ be an open subset and
$(a_k)_k \subseteq \Omega$ be a sequence without accumulation
points and without repetition. Furthermore, let
\[
f_k(z) =
\sum_{\ell=-\infty}^{n(k)} c_{k, \ell}(z - a_k)^\ell,
\]
where $n(k) \in \N_0$, be normally convergent on $\C \setminus A$,
where $A$ is the set of $a_k$. Then there exists a function
$f \in \mathcal{O}(\Omega \setminus A)$ such that
$\ord_{a_k}(f - f_k) > n(k)$ for all $k \in \N$.
\end{izrek}

\begin{proof}
By the Weierstraß product theorem, there exists a function
$h \in \mathcal{O}(\Omega)$ such that $\ord_{a_k}(h) > n(k)$ and
$h$ has no zeroes on $\C \setminus A$. Then
$\br{a_k, \frac{f_k}{h}}_k$ is a distribution of principal parts.
By theorem~\ref{inf_prod:lm:MLTOpen}, there exists a
$g \in \mathcal{O}(\Omega \setminus A)$ with these principal parts.

Now define $f = g \cdot h$. Then
\[
f - f_k = g \cdot h - f_k = h \cdot \br{g - \frac{f_k}{h}},
\]
which vanishes to order larger than $n(k)$ in $a_k$.
\end{proof}

\begin{posledica}
For every sequence $(a_k)_k \subseteq \Omega$ without accumulation
points and without repetition and every sequence
$(c_k)_k \subseteq \C$ there exists a function
$f \in \mathcal{O}(\Omega)$ such that $f(a_k) = c_k$ for each
$k \in \N$.
\end{posledica}

\newpage

\subsection{Ring structure of holomorphic functions}

\begin{definicija}
Let $\Omega \subseteq \C$ be an open set. A
\emph{divisor}\index{divisor} of a meromorphic function
$f \in \mathcal{M}^*(\Omega)$ is the function
$(f) \colon \Omega \to \Z$, given by
\[
(f)(z) = \begin{cases}
 n & \text{$f$ has a zero of order $n$ in $z$} \\
-n & \text{$f$ has a pole of order $n$ in $z$} \\
 0 & \text{otherwise}.                         \\
\end{cases}
\]
\end{definicija}

\begin{opomba}
The divisor of a product is the sum of divisors,
i.e.~$(f \cdot g) = (f) + (g)$.
\end{opomba}

\begin{definicija}
Let $S \subseteq \mathcal{O}(\Omega)$ be a subset that contains a
non-zero holomorphic function.

Define
\[
d(z) = \min_{f \in S \setminus \set{0}} (f)(z) \in \N_0.
\]
By Weierstraß product theorem there exists a function
$g \in \mathcal{O}(\Omega)$ such that $(g) = d$. We define
$\gcd(S) = g$\index{greatest common divisor}.\footnote{There are of
course multiple possible functions that satisfy this condition, but
their quotients are invertible.}
\end{definicija}

\datum{2023-12-18}

\begin{lema}[Wedderburn]
\index{Wedderburn!lemma}
Let $\Omega \subseteq \C$ be a domain and
$f, g \in \mathcal{O}(\Omega)$ be functions with $\gcd(f, g) = 1$.
Then there exist functions $a, b \in \mathcal{O}(\Omega)$ so that
$a f + b g = 1$. Moreover, we can choose $a$ to be nonwhere
vanishing. 
\end{lema}

\begin{proof}
If $g = 0$ and $f \neq 0$ then $f$ cannot vanish by assumption on
$\gcd$, therefore $a = \frac{1}{f}$ and $b = 1$ suffice. Therefore,
we can assume both $f, g$ and are nonzero. Note that
$Z(f) \cap Z(g)$ is empty, since $(z - p)$ divides $\gcd(f, g)$ for
any $p \in Z(f) \cap Z(g)$. The set $Z(f) \cup Z(g)$ is thus
discrete. Further, for each zero $p$ of $g$, there exists a disk of
radius $\varepsilon$ and a holomorphic function
$f_p \in \mathcal{O}(\dsk(p, \varepsilon))$ such that
\[
f = e^{f_p}.
\]
By the Mittag-Leffler osculation theorem there exists a function
$h \in \mathcal{O}(\Omega)$ such that
$\ord_p(h - f_p) > \ord_p(g)$.

Here we stop for a short observation. Developing into the power
series, we get that $e^{w^n} - 1 = w^n + O(w^{2 n})$. Then, 
\[
\ord_p \br{f - e^h} =
\ord_p \br{e^h \cdot \br{e^{f_p - h} - 1}} =
\ord_p \br{\br{e^{f_p - h} - 1}} =
\ord_p \br{f_p - h} >
\ord_p(g).
\]

Define $k = \frac{f - e^h}{g} \in \mathcal{O}(\Omega)$. We claim
that $a = e^{-h}$ and $b = -k e^{-h}$ satisfy the conditions.
Clearly, $a$ doesn't vanish, and
\[
a f + b g =
e^{-h} f - k e ^{-h} g =
e^{-h}(f - k g) =
e^{- h} \br{f - \frac{f - e^h}{g} g} =
e^{-h} e^h =
1. \qedhere
\]
\end{proof}

\begin{posledica}
For holomorphic functions $f_j \in \mathcal{O}(\Omega)$, where
$j \leq n$, we can write $f = \gcd(f_1, f_2, \dots, f_n)$ as 
\[
f = \sum_{j=1}^n a_j f_j
\]
\end{posledica}

\begin{proof}
We proceed by induction. The base case is just Wedderburn's lemma.
Now let $\hat{f} = \gcd(f_2, f_3, \dots f_n)$, which can be
written as
\[
\hat{f} = \sum_{j=2}^n \hat{a}_j f_j
\]
by the induction hypothesis. Then
$\frac{f_1}{f}, \frac{\hat{f}}{f} \in \mathcal{O}(\Omega)$ are
holomorphic functions with $\gcd$ equal to $1$. We can therefore
apply Wedderburn' lemma to get functions $a$ and $b$ such that
\[
a \frac{f_1}{f} + b \frac{\hat{f}}{f} = 1.
\]
The conclusion follows
\end{proof}

\begin{izrek}
Let $I \edn \mathcal{O}(\Omega)$ be the ideal generated by
holomorphic functions $f_1, f_2, \dots f_n$ on $\Omega$. Then there
exists a holomorphic function $f$ such that $I = (f)$.
\end{izrek}

\begin{proof}
Take $f = \gcd(f_j)$. This function is an element of $I$ by the
previous corollary. Since $f \mid f_j$, this implies that
$I = (f)$.
\end{proof}

\begin{definicija}
Let $\Omega \subseteq \C$ be a domain and
$I \edn \mathcal{O}(\Omega)$ an ideal.

\begin{enumerate}[i)]
\item We call $I$ \emph{closed}\index{closed ideal} if for every
sequence $(f_n)_n \subseteq I$ that converges uniformly on compacts
of $\Omega$ to some function $f$, we also have $f \in I$.
\item We call $p \in \Omega$ a \emph{zero}\index{zero} of $I$ if
$f(p) = 0$ for every $f \in I$.
\end{enumerate}
\end{definicija}

\begin{lema}
Let $\Omega \subseteq \C$ be a domain and
$I \edn \mathcal{O}(\Omega)$ and ideal. Let $p \in \Omega$ be a
point that is not a zero of $I$. Let
$f, g \in \mathcal{O}(\Omega)$ be functions such that $f(z) \ne 0$
for all $z \ne p$. If $f g \in I$, then $g \in I$.
\end{lema}

\begin{proof}
Since $p$ is not a zero of $I$, then there exists a function
$h \in I$ such that $h(p) \ne 0$. Let $n = \ord_p(f)$. If $n=0$,
then $f$ is a unit, so $g \in I$. Otherwise, we have
\[
\frac{f(z)}{z - p}g =
-\frac{1}{h(p)} \cdot
\br{\frac{h - h(p)}{z - p} fg - \frac{f g}{z - p} h} \in I
\]
since $\frac{f}{z - p}$ is holomorphic.
    
We can iterate this process to find $\frac{f}{(z - p)^n} g \in I$.
Since $\frac{f}{(z - p)^n}$ is a unit, $g$ must be an element of
$I$.
\end{proof}

\begin{izrek}
Let $\Omega \subseteq \C$ be a domain and
$I \edn \mathcal{O}(\Omega)$ an ideal. If $I$ has no zeroes and is
closed, then $I = \mathcal{O}(\Omega)$.
\end{izrek}

\begin{proof}
Let $f$ be an arbitrary nonzero element of $I$. By the Weierstraß
product theorem, we can write
\[
f = \prod_{k = 1}^{\infty} f_k,
\]
where each $f_k$ has exactly one zero in $\Omega$, and the tails
\[
\widehat{f}_n = \prod_{k = n}^{\infty} f_k
\]
converge to $1$ uniformly on compacts of $\Omega$. As
$f = \widehat{f}_1 = f_1 \widehat{f}_2$, we can apply the preivous
lemma to find $\widehat{f}_2 \in I$. Inductively,
$\widehat{f}_n \in I$ and since the ideal $I$ is assumed to be
closed, we have
\[
1 = \lim_{k \to \infty} \widehat{f}_k \in I. \qedhere
\]
\end{proof}
