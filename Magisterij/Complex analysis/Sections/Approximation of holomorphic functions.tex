\section{Approximation of holomorphic functions}

\subsection{Runge's little theorem}

\datum{2023-12-19}

\begin{lema}\label{app:lm:seg}
Let $\Omega \subseteq \C$ be an open subset and
$K \subseteq \Omega$ a non-empty compact. Then there exist finitely
many horizontal or vertical line segments
$\sigma_1, \dots, \sigma_n$ of equal length in
$\Omega \setminus K$ such that for all $f \in \mathcal{O}(\Omega)$
we have
\[
f(z) = \frac{1}{2 \pi i} \sum_{k=1}^n
\lint_{\sigma_k} \frac{f(\xi)}{\xi - z}\,d\xi
\]
for all $z \in K$.
\end{lema}

\begin{proof}
Define
\[
\delta =
\begin{cases}
          1,           & \Omega = \C,   \\
d(K, \partial \Omega), & \Omega \ne \C.
\end{cases}
\]
Let $Q$ be a grid of squares, parallel to the coordinate axes, with
side length $d < \frac{\delta}{\sqrt{2}}$. As $K$ is compact, it
only intersects finitely many of them. Now just choose the boundary
of the union of those squares. It is clear that all those segments
are subsets of $\Omega \setminus K$.

Let $Q_k$ denote the above squares. Then,
\[
\frac{1}{2 \pi i} \sum_{k=1}^m \olint_{\partial Q_k}
\frac{f(\xi)}{\xi - z}\,d\xi =
\frac{1}{2 \pi i} \sum_{k=1}^n \lint_{\sigma_k}
\frac{f(\xi)}{\xi - z}\,d\xi.
\]
If $z \in \Int Q_\ell$, then
\[
f(z) =
\frac{1}{2 \pi i} \lint_{\partial Q_\ell}
\frac{f(\xi)}{\xi - z}\,d\xi
\]
and
\[
\frac{1}{2 \pi i} \lint_{\partial Q_k}
\frac{f(\xi)}{\xi - z}\,d\xi = 0
\]
for all $k \ne \ell$, therefore the lemma holds for all such $z$.
As both sides of the equations are continuous functions, they must
agree on the whole set $K$.
\end{proof}

\begin{lema}\label{app:lm:app_seg}
Let $\sigma$ be a compact line segment and $K \subseteq \C$ be a
compact set such that $K \cap \sigma = \emptyset$. Let
$h \in \mathcal{C}(\sigma)$ be a function. Then for all
$\varepsilon > 0$ there exist points $c_1, \dots, c_m \in \C$ and
$w_1, \dots, w_m \in \sigma$ such that
\[
\norm{\lint_\sigma \frac{h(\xi)}{\xi - z}\,d\xi -
\sum_{k=1}^m \frac{c_k}{z - w_k}}_K <
\varepsilon.
\]
\end{lema}

\begin{proof}
Let $\ell$ be the length of $\sigma$ and define the function
$v \colon \sigma \times K \to \C$ with
\[
v(\xi, z) = \frac{h(\xi)}{\xi - z}.
\]
It is clearly continuous, therefore it is uniformly continuous. In
particular, there exists a $\delta > 0$ such that
\[
\abs{v(\xi, z) - v(\xi', z)} < \frac{\varepsilon}{\ell}
\]
for all $z \in \C$ and $\abs{\xi - \xi'} < \delta$.

Let $\tau_1, \dots, \tau_m$ be the partition of $\sigma$ into line
segments of length $d < \sigma$. Choose points $w_k \in \tau_k$ and
set $c_k = - h(w_k) d$. We therefore have
\[
\abs{\lint_{\tau_k} v(\xi, z)\,d\xi - \frac{c_k}{z - w_k}} =
\abs{\lint_{\tau_k} v(\xi, z)\,d\xi - d \cdot v(w_k, z)} <
d \cdot \frac{\varepsilon}{\ell}.
\]
Summing up, we get the desired inequality.
\end{proof}

\begin{lema}\label{app:lm:app_by_rat}
Let $\Omega \subseteq \C$ be an open set and $K \subseteq \Omega$
be a non-empty compact. Then there exist finitely many line
segments $\sigma_1, \dots, \sigma_n$ in $\Omega \setminus K$ such
that for any holomorphic function $f \in \mathcal{O}(\Omega)$ and
$\varepsilon > 0$ there exists a rational function $q$ of the form
\[
q(z) = \sum_{k=1}^m \frac{c_k}{z - w_k},
\]
where $c_k \in \C$ and $w_k \in \sigma$, such that
\[
\norm{f - q}_K < \varepsilon.
\]
\end{lema}

\begin{proof}
By lemma~\ref{app:lm:seg} there exist line segments $\sigma_k$ such
that
\[
f(z) = \frac{1}{2 \pi i} \sum_{k=1}^n
\lint_{\sigma_k} \frac{f(\xi)}{\xi - z}\,d\xi
\]
for all $z \in K$. Now just apply lemma~\ref{app:lm:app_seg} to
each line segment separately.
\end{proof}

\datum{2023-12-20}

\begin{lema}[Shifting poles]\index{Shifting poles lemma}
Let $K \subseteq \C$ be a compact set, $Z$ a connected component of
$\C \setminus K$ and $a, b \in Z$. Then, for all $\varepsilon > 0$
there exists a polynomial $q$ such that
\[
\norm{\frac{1}{z-a} - q \br{\frac{1}{z-b}}}_K < \varepsilon.
\]
If $Z$ is the unbounded component, we can approximate
$\frac{1}{z-a}$ by $q(z)$ instead.
\end{lema}

\begin{proof}
Let $L_\omega$ be the family of all functions that are holomorphic
in a neighbourhood of $K$ that can be approximated uniformly on $K$
by polynomials in $\frac{1}{z-\omega}$. Note that, if
$\frac{1}{z-p} \in L_q$, then $L_p \subseteq L_q$.

Consider the set
\[
S = \setb{s \in Z}{\frac{1}{z-s} \in L_b}.
\]
We claim that $S = Z$. First note that $b \in S$, therefore
$S \ne \emptyset$. Take any point $p \in S$ and $\delta > 0$ such
that $\dsk(p, \delta) \subseteq Z$. For any
$s \in \dsk(p, \delta)$, we can write
\[
\frac{1}{z-s} =
\frac{1}{z-p} \cdot \frac{1}{1 - \frac{s-p}{z-p}} =
\frac{1}{z-p} \cdot \sum_{k=0}^\infty
\br{\frac{s-p}{z-p}}^k,
\]
which converges as $\abs{s-p} < \abs{z-p}$ by choice of $\delta$.
We conclude that $\dsk(p, \delta) \subseteq S$. In particular, $S$
is an open set. Take any point $p \in \partial S \cap Z$. Suppose
that $\dsk(p, 3 \delta) \subseteq Z$ and choose a point
$p' \in \dsk(p, \delta) \cap S$. Then,
$\dsk(p', 2\delta) \subseteq S$ and therefore
$\dsk(p, \delta) \subseteq S$, which is a contradiction and
therefore proves our claim.

Now suppose that $Z$ is the unbounded component and take a point
$d \in Z$ such that $K \subseteq \dsk(0, \abs{d})$. Then all
functions $\br{\frac{1}{z-d}}^n$ can be approximated uniformly on
$K$ by Taylor's polynomials about $0$.
\end{proof}

\begin{definicija}
For any set $P \subseteq \C$ denote by $\C_P[z]$ the family of
rational functions with poles in $P$.
\end{definicija}

\begin{izrek}[Runge approximation]
\index{Runge approximation theorem}
\label{app:thm:runge_app}
Let $K \subseteq \C$ be a compact. If $P$ intersects every bounded
connected component of $\C \setminus K$, then for every function
$f$, holomorphic in a neighbourhood of $K$, and every
$\varepsilon > 0$ there exists a function $q \in \C_P[z]$ such that
\[
\norm{f-q}_K < \varepsilon.
\]
\end{izrek}

\begin{proof}
Be lemma~\ref{app:lm:app_by_rat}, we can find a compact union
$\sigma$ of line segments such that every such $f$ can be
approximated by
\[
\widetilde{q} = \sum_{k=1}^m \frac{c_k}{z - w_k},
\]
where $c_k \in \C$ and $w_k \in \sigma$. Suppose then
\[
\norm{f - \widetilde{q}}_K < \frac{\varepsilon}{2}.
\]
Let $Z_k$ be the connected component of $\C \setminus K$ that
contains $w_k$.

If $Z_k$ is bounded, then choose $t_k \in P \cap Z_k$. By the pole
shifting lemma, we can find a polynomial $g_k$ such that
\[
\norm{\frac{c_k}{z - w_k} - g_k \br{1}{z - t_k}}_K <
\frac{\varepsilon}{2m}.
\]
If $Z_k$ is unbounded, we can instead approximate
$\frac{c_k}{z - w_k}$ by a polynomial instead. Choosing
\[
q = \sum_{k=1}^m g_k,
\]
we find
\[
\norm{f - q}_K \leq
\norm{f - \widetilde{q}}_K + \norm{\widetilde{q} - q}_K <
\frac{\varepsilon}{2} + m \cdot \frac{\varepsilon}{2m} =
\varepsilon. \qedhere
\]
\end{proof}

\begin{izrek}[Runge approximation]
\index{Runge approximation theorem}
Let $\Omega \subseteq \C$ be an open subset and let
$K \subseteq \Omega$ be a compact. If every bounded component of
$\C \setminus K$ intersects $\C \setminus \Omega$, then for every
function $f$ that is holomorphic on a neighbourhood of $K$ and
every $\varepsilon > 0$ there exists a function
$q \in \mathcal{O}(\Omega)$ such that
\[
\norm{f - q}_K < \varepsilon.
\]
\end{izrek}

\begin{proof}
Choose $P = \C \setminus \Omega$ in
theorem~\ref{app:thm:runge_app}.
\end{proof}

\begin{posledica}[Runge's little theorem]
\index{Runge's little theorem}
Let $K \subseteq \C$ be a compact set such that $\C \setminus K$ is
connected. Then for every holomorphic function on a neighbourhood
of $K$ and $\varepsilon > 0$ there exists a polynomial
$q \in \C[z]$ such that
\[
\norm{f - q}_K < \varepsilon.
\]
\end{posledica}

\begin{proof}
Choose $P = \emptyset$ in theorem~\ref{app:thm:runge_app}.
\end{proof}

\datum{2024-1-3}

\begin{definicija}
Let $V$ be a vector space over $\C$, equipped with a topology. Let
$T \colon V \to V$ be a linear map.

\begin{enumerate}[i)]
\item We call $T$ \emph{cyclic}\index{cyclic!map} if there exists
some $f \in V$, called a \emph{cyclic vector}\index{cyclic!vector},
such that
\[
\spn_\C \setb{T^n(f)}{n \in \N_0} = V.
\]
\item We call $T$ \emph{hypercyclic}\index{hypercyclic map} if
there exists some $f \in V$, such that
\[
\oline{\setb{T^n(f)}{n \in \N_0}} = V.
\]
\end{enumerate}
\end{definicija}

\begin{izrek}[Birkhoff]
\index{Birkhoff's theorem}
Let $\tau \colon \C \to \C$ be given by $\tau(z) = z + a$ for some
$a \ne 0$. Then the map
$T \colon \mathcal{O}(\C) \to \mathcal{O}(\C)$, given by
$T(f) = f \circ \tau$, is hypercyclic.
\end{izrek}

\begin{proof}
Set $K_n = \oline{\dsk(\ell_n \cdot a, n)}$ for a sequence
$(\ell_n)_n \subseteq \N$, such that all $K_n$ are pairwise
disjoint and $K_n \cap \dsk(0, n) = \emptyset$ for all $n \in \N$.
Choose a sequence $(\varepsilon_n)_n \subseteq \R^+$, converging to
$0$. Furthermore, let $(p_n)_n$ be a sequence of all polynomials
$(\Q \oplus i\Q)[z]$.

We first construct a sequence of holomorphic functions $(f_n)_n$,
such that the following conditions hold:

\begin{enumerate}[i)]
\item For all $n \in \N$, we have
\[
\norm{\sum_{k=1}^n f_k - \tau^{-\ell_n} \circ p_n}_{K_n} <
\frac{\varepsilon_n}{2^n}.
\]
\item For all $m < n$, we have
\[
\norm{f_n}_{K_m} < \frac{\varepsilon_m}{2^n}.
\]
\item For all $n \in \N$, we have
\[
\norm{f_n}_{\dsk(0,n)} < \frac{1}{2^n}.
\]
\end{enumerate}

Choose $\ell_n$ such that $\dsk(0,n) \cap K_m = \emptyset$ for all
$m < n$. Then the set
\[
\C \bigsetminus \br{\bigcup_{m=1}^n K_m \cup \dsk(0,n)}
\]
is obviously connected. We can now just apply Runge's little
theorem by choosing the function $0$ on $\dsk(0,n)$ and $K_m$ for
$m < n$, and
\[
\tau^{-\ell_n} \circ p_n - \sum_{k=1}^{n-1}
\]
on $K_n$.

Note that the series
\[
f = \sum_{n=1}^\infty f_n
\]
is uniformly convergent by the third condition. We now compute
\begin{align*}
\norm{f \circ \tau^{\ell_n} - p_n}_{\tau^{-\ell_n}(K_n)} &=
\norm{f - \tau^{-\ell_n} \circ p_n}_{K_n}
\\
&\leq
\sum_{k > n} \norm{f_k}_{K_n} +
\norm{\sum_{k=1}^n f_k - \tau^{-\ell_n} \circ p_n}_{K_n}
\\
&<
\varepsilon_n.
\end{align*}
It follows that we can approximate every rational polynomial with
iterations $T^n(f)$.
\end{proof}
