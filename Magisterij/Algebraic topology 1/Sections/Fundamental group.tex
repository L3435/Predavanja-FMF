\section{Fundamental group}

\subsection{Definition}

\begin{definicija}
Let $\alpha, \beta \colon I \to X$ be two paths with
$\alpha(1) = \beta(0)$. The
\emph{concatenation product}\index{concatenation product} is the
path
\[
(\alpha \cdot \beta)(t) =
\begin{cases}
\alpha(2t),    & t \leq \frac{1}{2}, \\
\beta(2t - 1), & t \geq \frac{1}{2}.
\end{cases}
\]
\end{definicija}

\begin{definicija}
Let $X$ be a topological space and $x_0 \in X$. The set of
\emph{loops}\index{loops} with starting point $x_0$ is the set
\[
\Omega(X, x_0) = \mathcal{C}((I, \partial I), (X, x_0)).
\]
\end{definicija}

\begin{izrek}
Concatenation on $\Omega(X, x_0)$ induces a group structure on
\[
\kvoc{\Omega(X, x_0)}{\simeq~\rel \partial I}.
\]
\end{izrek}

\begin{definicija}
The above group is called the
\emph{fundamental group}\index{fundamental group} and is denoted by
$\pi_1(X, x_0)$.
\end{definicija}

\datum{2023-10-16}

\begin{opomba}
Suppose that $C_{x_0}$ is the path-component of $X$ containing
$x_0$. Then we have $\pi_1(X, x_0) = \pi_1(C_{x_0}, x_0)$.
\end{opomba}

\begin{definicija}
Suppose $x_0, x_1 \in X$ are connected by a path
$\gamma \colon (I, 0, 1) \to (X, x_0, x_1)$. Define the map
$\tr_\gamma \colon \pi_1(X, x_0) \to \pi_1(X, x_1)$ as
$\tr_\gamma([\alpha]) =
[\oline{\gamma} \cdot \alpha \cdot \gamma]$.
\end{definicija}

\begin{trditev}
Suppose $x_0, x_1$ are connected by a path
$\gamma \colon (I, 0, 1) \to (X, x_0, x_1)$. Then the map
$\tr_\gamma$ is an isomorphism with inverse
$\br{\tr_\gamma}^{-1} = \tr_{\oline{\gamma}}$.
\end{trditev}

\obvs

\begin{trditev}
For paths $\gamma, \delta \colon (I, 0, 1) \to (X, x_0, x_1)$ we
have
\[
\tr_\delta = \tr_{\oline{\gamma} \delta} \circ \tr_\gamma.
\]
\end{trditev}

\obvs

\begin{opomba}
If $\pi_1(X, x_1)$ is commutative, the $\tr$-isomorphisms are
independent from the chosen paths. We write
$\pi_1(X) = \pi_1(X, x_0)$.
\end{opomba}

\begin{trditev}
Let $f \colon (X, x_0) \to (Y, y_0)$ be a map. Then $f$ induces a
homomorphism $f_* \colon \pi_1(X, x_0) \to \pi_1(Y, y_0)$.
\end{trditev}

\obvs

\begin{trditev}
The fundamental group is a functor
$\pi_1 \colon \catt{HoTop}_\bullet \to \catt{Grp}$.
\end{trditev}

\begin{posledica}
The following statements are true:

\begin{enumerate}[i)]
\item If $f \colon (X, x_0) \to (Y, y_0)$ is a homotopy
equivalence relative to $x_0$, the induced map
$f_* \colon \pi_1(X, x_0) \to \pi_1(X, x_1)$ is an isomorphism.
\item If $f \colon X \to Y$ is a homotopy equivalence, then the map
$f_* \colon \pi_1(X, x_0) \to \pi_1(Y, f(x_0))$ is an isomorphism.
\item If $X$ is contractible, then $\pi(X, x_0) \cong \set{1}$.
\item For $x_0 \in X$ and $y_0 \in Y$ we have that
\[
\br{(p_X)_*, (p_Y)_*} \colon
\pi_1(X \times Y, (x_0, y_0)) \to
\pi_1(X, x_0) \times \pi_1(Y, y_0)
\]
is an isomorphism.
\item If $A$ is a retract of $X$, then
$i_* \colon \pi_1(A, x_0) \to \pi_1(X, x_0)$ is a monomorphism for
all $x_0 \in A$. If $\pi_1(X, x_0)$ is commutative, the group
$\pi_1(A, x_0)$ is a direct summand.
\end{enumerate}
\end{posledica}

\datum{2023-10-19}

\begin{proof}
Assume that $A$ is a retract of $X$. As
$\pi_1 \colon \catt{HoTop}_\bullet \to \catt{Grp}$ is a functor, we
can apply it to the diagram
\[
\begin{tikzcd}
(A, x_0) \arrow[r, hook, "i"] &
(X, x_0) \arrow[r, "r"] &
(A, x_0).
\end{tikzcd}
\]
It follows that $i_* \colon \pi_1(A, x_0) \to \pi_1(X, x_0)$ is a
monomorphism.
\end{proof}

\begin{izrek}[Fundamental group of $S^1$]
The fundamental group of $S^1$ is isomorphic to $\Z$.
\end{izrek}

\begin{proof}
Let $\alpha \colon (I, 0, 1) \to (S^1, 1, 1)$ be a loop. We can
lift the map $\alpha$ to a unique map
$\widetilde{\alpha} \colon (I, 0) \to (\R, 0)$, as $\R$ is a
covering space of $S^1$. Observe the map
$\Phi_0 \colon \Omega(S^1, 1) \to \Z$, given by
$\Phi_0(\alpha) = \alpha(1)$. As any homotopy
$h \colon \alpha \simeq \beta~(\rel \partial I)$ can be lifted to a
unique homotopy $H \colon \widetilde{\alpha} \simeq
\widetilde{\beta}~(\rel \partial I)$, the map $\Phi_0$ is constant
on equivalence classes and therefore induces a map
$\Phi \colon \pi_1(S^1, 1) \to \Z$. Note that it is an isomorphism,
therefore, $\pi_1(S^1, 1) \cong \Z$.
\end{proof}

\begin{posledica}[Brouwer]
\index{Brouwer theorem}
Every map $f \colon B^2 \to B^2$ has a fixed point.
\end{posledica}

\begin{proof}
If $f$ had no fixed point, we could construct a retraction
$r \colon B^2 \to S^1$. This is of course not possible, as $\Z$ is
not a subgroup of the trivial group.
\end{proof}

\begin{trditev}
Let $q \colon I \to S^1$ be the quotient map. Then, the induced map
$q^* \colon [(S^1, 1), (X, x_0)] \to \pi_1(X, x_0)$ is an
isomorphism of groups.
\end{trditev}

\begin{definicija}
The \emph{degree}\index{degree} of $\alpha$ is the number
$\deg \alpha$, where $\deg \colon [(S^1, 1), (S^1, 1)] \to \Z$ is
the above isomorphism.
\end{definicija}

\begin{izrek}[Borsuk-Ulam]
\index{Borsuk-Ulam theorem}
For any continuous map $f \colon S^1 \to \R$ there exists some
$x \in S^1$ such that $f(x) = f(-x)$.
\end{izrek}

\begin{proof}
Let $g(x) = f(x) - f(-x)$. Note that $g$ is an odd function and
assume that $g(x) \ne 0$ for all $x$. Observe that
\[
\frac{g}{\abs{g}} \colon S^1 \to \set{-1, 1}
\]
is a well defined, continuous odd function. This is of course
impossible, as $S^1$ is connected.
\end{proof}

\datum{2023-10-23}

\begin{lema}
For any odd function $f \colon S^1 \to S^1$ we have
$2 \nmid \deg(f)$.
\end{lema}

\begin{proof}
Without loss of generality assume $f(1) = 1$. Let
$q \colon I \to S^1$ be the quotient map. Note that we can lift the
map $f \circ q$ to a map $F \colon I \to \R$ with $F(0) = 0$. Then,
$\deg(f) = F(1)$. Note that
\[
f \circ q \br{\frac{1}{2}} = f(-1) = -1,
\]
so $F \br{\frac{1}{2}} = k + \frac{1}{2}$. As $f$ is odd, the map
\[
G(x) =
\begin{cases}
F(x), & x \leq \frac{1}{2}, \\
F \br{\frac{1}{2}} + F \br{x - \frac{1}{2}}, & x \geq \frac{1}{2}
\end{cases}
\]
is also a lift of $f \circ q$. By uniqueness, we have $F = G$ and
therefore $F(1) = 2k+1$.
\end{proof}

\begin{posledica}
There are no odd maps $f \colon S^n \to S^1$ for $n > 1$.
\end{posledica}

\begin{proof}
Assume that such a map $f$ exists. Note that the inclusion
$\iota \colon S^1 \hookrightarrow S^n$ is null-homotopic. But then
$f \circ \iota$ should also be null-homotopic, hence have degree
$0$.
\end{proof}

\begin{izrek}[Borsuk-Ulam]
\index{Borsuk-Ulam theorem}
For any map $f \colon S^2 \to \R^2$ there exists some $x \in S^2$
such that $f(x) = - f(-x)$.
\end{izrek}

\begin{proof}
Suppose otherwise and note that $g \colon S^2 \to S^1$, given by
\[
g(x) = \frac{f(x) - f(-x)}{\norm{f(x) - f(-x)}},
\]
is an odd map.
\end{proof}

\begin{opomba}
The theorem holds for all maps $f \colon S^n \to \R^n$.
\end{opomba}

\begin{izrek}[Stone-Tukey]
\index{Stone-Tukey theorem}
Let $A$, $B$ and $C$ be bounded measurable subsets of $\R^3$. Then
there exists a plane which bisects all of $A$, $B$ and
$C$.\footnote{Also called the \emph{Ham Sandwich theorem}.}
\end{izrek}

\begin{izrek}[Lusternik-Schnirelmann]
\index{Lusternik-Schnirelmann theorem}
Let $A$, $B$ and $C$ form a closed cover of $S^2$. Then, at least
one of them contains a pair of antipodal points.
\end{izrek}

\begin{izrek}[Fundamental theorem of algebra]
\index{Fundamental theorem of algebra}
Every non-constant polynomial $p \in \C[x]$ has a complex root.
\end{izrek}

\begin{proof}
Let $n = \deg p$ and assume $0 \not \in p(\C)$. Define
\[
H(z, t) = \sum_{i=0}^n a_i z^i t^{n-i}.
\]
Note that $H \colon S^1 \times I \to \C \setminus \set{0}$ is a
homotopy between $z^n$ and $p(z)$. Also, the map
\[
K(z, t) = p(zt)
\]
is a homotopy between $a_0$ and $p(z)$. It follows that $p$ is
null-homotopic, so $\deg p = 0$.
\end{proof}

\newpage

\subsection{Computation of the fundamental group}

\datum{2023-10-26}

\begin{definicija}
The \emph{coproduct}\index{coproduct} of groups $G$ and $H$ is the
group
\[
G * H =
\setb{\prod_{i=1}^n g_i h_i}
{n \in \N \land g_i \in G \land h_i \in H}.
\]
\end{definicija}

\begin{opomba}
The definition coincides with coproducts in the category
$\catt{Grp}$.
\end{opomba}

\begin{izrek}[Seifert-van Kampen]
\index{Seifert-van Kampen theorem}
Let $X$ and $Y$ be open subspaces in $X \cup Y$ and
$x_0 \in X \cap Y$. Suppose that $X$, $Y$ and $X \cap Y$ are
path-connected. Denote by $i_Z \colon X \cap Y \hookrightarrow Z$
and $j_Z \colon Z \hookrightarrow X \cup Y$ the inclusions for
$Z \in \set{X, Y}$. Then
\[
\varphi = ((j_X)_*, (j_Y)_*) \colon
\pi_1(X, x_0) * \pi_1(Y, x_0) \to \pi_1(X \cup Y, x_0)
\]
is an epimorphism with
\[
\ker \varphi =
\skl{\setb{(i_X)_*(\alpha) \cdot (i_Y)_* \br{\alpha^{-1}}}
{\alpha \in \pi_1(X \cap Y, x_0)}}.
\]
\end{izrek}

\begin{proof}
Let $\alpha \in \pi_1(X \cup Y, x_0)$ be an arbitrary loop. Note
that, using the Lebesgue number, we can split $\alpha$ into
finitely many paths in $X$ and $Y$. Using the path-connectedness of
$X \cap Y$, we can join the path segments into loops starting at
$x_0$, thus constructing an element of
$\pi_1(X, x_0) * \pi_1(Y, x_0)$ that maps to $\alpha$.

Let
\[
N = \skl{\setb{(i_X)_*(\alpha) \cdot (i_Y)_*\br{\alpha^{-1}}}
{\alpha \in \pi_1(X \cap Y, x_0)}}.
\]
Note that
\begin{align*}
\varphi \br{(i_X)_*(\alpha) \cdot (i_Y)_*\br{\alpha^{-1}}} &=
(j_X)_*((i_X)_*(\alpha)) \cdot (j_Y)_* \br{(i_Y)_*\br{\alpha^{-1}}}
\\
&=
(j_Y)_*((i_Y)_*(\alpha)) \cdot (j_Y)_* \br{(i_Y)_*\br{\alpha}}^{-1}
\\
&=
1,
\end{align*}
as $j_X \circ i_X = j_Y \circ i_Y$. It follows that
$N \leq \ker \varphi$. It remains to check that
$\ker \varphi \leq N$.
%TODO Copy proof from Hatcher, p.44
\end{proof}

\begin{definicija}
A path-connected topological space $X$ is
\emph{simply connected}\index{simply connected} if any two paths
$\alpha, \beta \colon I \to X$, such that
$\eval{\alpha}{\partial I}{} = \eval{\beta}{\partial I}{}$, are
homotopic relative to $\partial I$.
\end{definicija}

\begin{opomba}
Equivalently, $X$ is path-connected with $\pi_1(X) \cong \set{1}$.
\end{opomba}
