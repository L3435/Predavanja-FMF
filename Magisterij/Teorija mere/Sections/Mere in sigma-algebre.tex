\section{Mere in \texorpdfstring{$\sigma$}{sigma}-algebre}

\subsection{\texorpdfstring{$\sigma$}{Sigma}-algebre in Borelove množice}

\datum{2024-2-19}

\begin{definicija}
Naj bo $X$ neprazna množica. Družina $\mathcal{A}$ podmnožic $X$ je
\emph{$\sigma$-algebra}\index{sigma-algebra@$\sigma$-algebra}, če
ima naslednje lastnosti:

\begin{enumerate}[i)]
\item Velja $X \in \mathcal{A}$.
\item Če je $A \in \mathcal{A}$, je tudi
$A^{\mathsf{c}} \in \mathcal{A}$.
\item Če je $\br{A_n}_{n \in \N} \subseteq \mathcal{A}$, je tudi
\[
\bigcup_{n \in \N} A_n \in \mathcal{A}.
\]
\end{enumerate}

Pravimo, da je $(X, \mathcal{A})$
\emph{merljiv prostor}\index{merljiv prostor}.
\end{definicija}

\begin{opomba}
Velja tudi $\emptyset \in \mathcal{A}$, poleg tega pa $\mathcal{A}$
vsebuje vse števne preseke svojih elementov.
\end{opomba}

\begin{trditev}
Naj bo $\mathcal{B}$ družina podmnožic neprazne množice $X$. Tedaj
je presek vseh $\sigma$-algeber, ki vsebujejo $\mathcal{B}$,
$\sigma$-algebra. Označimo jo s $\sigma(B)$.
\end{trditev}

\obvs

\begin{definicija}
Naj bo $(X, \tau)$ topološki prostor.
\emph{Borelova $\sigma$-algebra}\index{Borelova sigma-algebra@Borelova $\sigma$-algebra}
prostora $X$ je $\sigma$-algebra $\mathcal{B}_X = \sigma(\tau)$.
\end{definicija}

\begin{trditev}
Borelova $\sigma$-algebra na $\R$ je generirana z vsako od množic
\begin{align*}
&\setb{(a,b)}{a < b}, &
&\setb{[a,b]}{a < b}, &
&\setb{(a,b]}{a < b}, &
&\setb{[a,b)}{a < b}, \\
&\setb{(a,\infty)}{a \in \R}, &
&\setb{(\infty,b)}{b \in \R}, &
&\setb{[a,\infty)}{a \in \R}, &
&\setb{(\infty,b]}{b \in \R}.
\end{align*}
\end{trditev}

\obvs

\newpage

\subsection{Pozitivne mere}

\begin{definicija}
Naj bo $(X, \mathcal{A})$ merljiv prostor. Preslikava
$\mu \colon \mathcal{A} \to [0, \infty]$ je
\emph{pozitivna mera}\index{pozitivna mera}, če velja naslednje:

\begin{enumerate}[i)]
\item Velja $\mu(\emptyset) = 0$.
\item Če so $\br{A_n}_{n \in \N} \subseteq \mathcal{A}$ disjunktne
množice, je
\[
\mu \br{\bigcup_{n \in \N} A_n} = \sum_{n \in \N} \mu(A_n).
\]
\end{enumerate}

Pravimo, da je $(X, \mathcal{A}, \mu)$
\emph{merljiv prostor s pozitivno mero}\index{merljiv prostor!z mero}.
\end{definicija}

\begin{definicija}
Naj bo $X \ne \emptyset$ in $x \in X$.
\emph{Diracova delta}\index{Diracova delta} je mera na
$\mathcal{P}(X)$, podana s predpisom
\[
\delta_x(A) =
\begin{cases}
1, & x \in A, \\
0, & x \not \in A.
\end{cases}
\]
\end{definicija}

\begin{opomba}
Prostor $(X, \mathcal{P}(X), \delta_x)$ je merljiv prostor s
pozitivno mero.
\end{opomba}

\begin{definicija}
Naj bo $(X, \mu)$ merljiv prostor s pozitivno mero.

\begin{enumerate}[i)]
\item Mera $\mu$ je \emph{končna}\index{končna mera}, če
je $\mu(X) < \infty$. Pravimo, da je prostor $X$
\emph{končen merljiv}\index{končen merljiv prostor}.
\item Mera $\mu$ je
\emph{$\sigma$-končna}\index{sigma-končna@$\sigma$-končna},
če velja
\[
X = \bigcup_{n=1}^\infty A_n,
\]
kjer je $\mu(A_n) < \infty$. Pravimo, da je $X$
\emph{$\sigma$-končen}\index{sigma-končen merljiv prostor@$\sigma$-končen merljiv prostor}
merljiv prostor.
\item Mera $\mu$ je \emph{verjetnostna}\index{verjetnostna mera},
če je $\mu(X) = 1$.
\end{enumerate}
\end{definicija}

\begin{lema}
Naj bo $\mu \colon \mathcal{A} \to [0, \infty]$ končno aditivna
funkcija. Če za množici $A, B \in \mathcal{A}$ velja
$A \subseteq B$, je $\mu(A) \leq \mu(B)$.
\end{lema}

\begin{proof}
Zapišemo lahko
\[
\mu(B) = \mu(A) + \mu \br{B \setminus A} \geq \mu(A). \qedhere
\]
\end{proof}

\begin{trditev}
Naj bo $(X, \mathcal{A}, \mu)$ merljiv prostor s pozitivno mero.
Tedaj za vse $(A_n)_{n \in \N} \subseteq A$ velja
\[
\mu \br{\bigcup_{n=1}^\infty A_n} \leq \sum_{n=1}^\infty \mu(A_n).
\]
\end{trditev}

\begin{proof}
Za množice
\[
B_n = A_n \bigsetminus \bigcup_{k=1}^{n-1} A_k
\]
velja
\[
\mu \br{\bigcup_{n=1}^\infty A_n} =
\mu \br{\bigcup_{n=1}^\infty B_n} =
\sum_{n=1}^\infty \mu(B_n) \leq
\sum_{n=1}^\infty \mu(A_n). \qedhere
\]
\end{proof}

\begin{trditev}
Za merljiv prostor $(X, \mu)$ s pozitivno mero so naslednje trditve
ekvivalentne:

\begin{enumerate}[i)]
\item Mera $\mu$ je $\sigma$-končna.
\item Zapišemo lahko
\[
X = \bigcup_{n=1}^\infty A_n,
\]
pri čemer je $A_n \subseteq A_{n+1}$ in $\mu(A_n) < \infty$ za vsak
$n \in \N$.
\item Zapišemo lahko
\[
X = \bigcup_{n=1}^\infty A_n
\]
za disjunktne množice $A_n$, za katere velja $\mu(A_n) < \infty$.
\end{enumerate}
\end{trditev}

\obvs

\begin{trditev}
Končno aditivna funkcija $\mu \colon \mathcal{A} \to [0, \infty]$
za merljiv prostor $(X, \mu)$ je pozitivna mera natanko tedaj, ko
za vsako naraščajoče zaporedje množic
$(A_n)_{n \in \N} \subseteq \mathcal{A}$ velja
\[
\mu \br{\bigcup_{n=1}^\infty A_n} = \lim_{n \to \infty} \mu(A_n).
\]
\end{trditev}

\begin{proof}
Denimo najprej, da je $\mu$ pozitivna mera. Tedaj je res
\[
\mu \br{\bigcup_{n=1}^\infty A_n} =
\mu \br{A_1 \cup \bigcup_{n=2}^\infty A_n \setminus A_{n-1}} =
\mu(A_1) + \sum_{n=2}^\infty \mu(A_n \setminus A_{n-1}) =
\lim_{n \to \infty} \mu(A_n).
\]
Predpostavimo sedaj, da velja drugi pogoj in naj bodo
$(A_n)_{n \in \N} \subseteq \mathcal{A}$ disjunktne množice. Tedaj
velja
\[
\sum_{n=1}^\infty \mu(A_n) =
\lim_{k \to \infty} \sum_{n=1}^k \mu(A_n) =
\lim_{k \to \infty} \mu \br{\bigcup_{n \leq k} A_n} =
\mu \br{\bigcup_{n=1}^\infty A_n},
\]
zato je $\mu$ res pozitivna.
\end{proof}

\begin{opomba}
Podobna trditev za preseke ne velja.
\end{opomba}
