\section{Mere in \texorpdfstring{$\sigma$}{sigma}-algebre}

\subsection{\texorpdfstring{$\sigma$}{Sigma}-algebre in Borelove množice}

\datum{2024-2-19}

\begin{definicija}
Naj bo $X$ neprazna množica. Družina $\mathcal{A}$ podmnožic $X$ je
\emph{$\sigma$-algebra}\index{sigma-algebra@$\sigma$-algebra}, če
ima naslednje lastnosti:

\begin{enumerate}[i)]
\item Velja $X \in \mathcal{A}$.
\item Če je $A \in \mathcal{A}$, je tudi
$A^{\mathsf{c}} \in \mathcal{A}$.
\item Če je $\br{A_n}_{n \in \N} \subseteq \mathcal{A}$, je tudi
\[
\bigcup_{n \in \N} A_n \in \mathcal{A}.
\]
\end{enumerate}

Pravimo, da je $(X, \mathcal{A})$
\emph{merljiv prostor}\index{merljiv prostor}, množicam
$\mathcal{A}$ pa \emph{merljive množice}\index{merljiva množica}.
\end{definicija}

\begin{opomba}
Velja tudi $\emptyset \in \mathcal{A}$, poleg tega pa $\mathcal{A}$
vsebuje vse števne preseke svojih elementov.
\end{opomba}

\begin{trditev}
Naj bo $\mathcal{B}$ družina podmnožic neprazne množice $X$. Tedaj
je presek vseh $\sigma$-algeber, ki vsebujejo $\mathcal{B}$,
$\sigma$-algebra. Označimo jo s $\sigma(B)$.
\end{trditev}

\obvs

\begin{definicija}
Naj bo $(X, \tau)$ topološki prostor.
\emph{Borelova $\sigma$-algebra}\index{Borelova sigma-algebra@Borelova $\sigma$-algebra}
prostora $X$ je $\sigma$-algebra $\mathcal{B}_X = \sigma(\tau)$.
\end{definicija}

\begin{trditev}
Borelova $\sigma$-algebra na $\R$ je generirana z vsako od množic
\begin{align*}
&\setb{(a,b)}{a < b}, &
&\setb{[a,b]}{a < b}, &
&\setb{(a,b]}{a < b}, &
&\setb{[a,b)}{a < b}, \\
&\setb{(a,\infty)}{a \in \R}, &
&\setb{(\infty,b)}{b \in \R}, &
&\setb{[a,\infty)}{a \in \R}, &
&\setb{(\infty,b]}{b \in \R}.
\end{align*}
\end{trditev}

\obvs

\newpage

\subsection{Pozitivne mere}

\begin{definicija}
Naj bo $(X, \mathcal{A})$ merljiv prostor. Preslikava
$\mu \colon \mathcal{A} \to [0, \infty]$ je
\emph{pozitivna mera}\index{pozitivna mera}, če velja naslednje:

\begin{enumerate}[i)]
\item Velja $\mu(\emptyset) = 0$.
\item Če so $\br{A_n}_{n \in \N} \subseteq \mathcal{A}$ disjunktne
množice, je
\[
\mu \br{\bigcup_{n \in \N} A_n} = \sum_{n \in \N} \mu(A_n).
\]
\end{enumerate}

Pravimo, da je $(X, \mathcal{A}, \mu)$
\emph{merljiv prostor s pozitivno mero}\index{merljiv prostor!z mero}.
\end{definicija}

\begin{definicija}
Naj bo $X \ne \emptyset$ in $x \in X$.
\emph{Diracova delta}\index{Diracova delta} je mera na
$\mathcal{P}(X)$, podana s predpisom
\[
\delta_x(A) =
\begin{cases}
1, & x \in A, \\
0, & x \not \in A.
\end{cases}
\]
\end{definicija}

\begin{opomba}
Prostor $(X, \mathcal{P}(X), \delta_x)$ je merljiv prostor s
pozitivno mero.
\end{opomba}

\begin{definicija}
Naj bo $(X, \mu)$ merljiv prostor s pozitivno mero.

\begin{enumerate}[i)]
\item Mera $\mu$ je \emph{končna}\index{končna mera}, če
je $\mu(X) < \infty$. Pravimo, da je prostor $X$
\emph{končen merljiv}\index{končen merljiv prostor}.
\item Mera $\mu$ je
\emph{$\sigma$-končna}\index{sigma-končna@$\sigma$-končna},
če velja
\[
X = \bigcup_{n=1}^\infty A_n,
\]
kjer je $\mu(A_n) < \infty$. Pravimo, da je $X$
\emph{$\sigma$-končen}\index{sigma-končen merljiv prostor@$\sigma$-končen merljiv prostor}
merljiv prostor.
\item Mera $\mu$ je \emph{verjetnostna}\index{verjetnostna mera},
če je $\mu(X) = 1$.
\end{enumerate}
\end{definicija}

\begin{lema}
Naj bo $\mu \colon \mathcal{A} \to [0, \infty]$ končno aditivna
funkcija. Če za množici $A, B \in \mathcal{A}$ velja
$A \subseteq B$, je $\mu(A) \leq \mu(B)$.
\end{lema}

\begin{proof}
Zapišemo lahko
\[
\mu(B) = \mu(A) + \mu \br{B \setminus A} \geq \mu(A). \qedhere
\]
\end{proof}

\begin{trditev}
Naj bo $(X, \mathcal{A}, \mu)$ merljiv prostor s pozitivno mero.
Tedaj za vse $(A_n)_{n \in \N} \subseteq A$ velja
\[
\mu \br{\bigcup_{n=1}^\infty A_n} \leq \sum_{n=1}^\infty \mu(A_n).
\]
\end{trditev}

\begin{proof}
Za množice
\[
B_n = A_n \bigsetminus \bigcup_{k=1}^{n-1} A_k
\]
velja
\[
\mu \br{\bigcup_{n=1}^\infty A_n} =
\mu \br{\bigcup_{n=1}^\infty B_n} =
\sum_{n=1}^\infty \mu(B_n) \leq
\sum_{n=1}^\infty \mu(A_n). \qedhere
\]
\end{proof}

\begin{trditev}
Za merljiv prostor $(X, \mu)$ s pozitivno mero so naslednje trditve
ekvivalentne:

\begin{enumerate}[i)]
\item Mera $\mu$ je $\sigma$-končna.
\item Zapišemo lahko
\[
X = \bigcup_{n=1}^\infty A_n,
\]
pri čemer je $A_n \subseteq A_{n+1}$ in $\mu(A_n) < \infty$ za vsak
$n \in \N$.
\item Zapišemo lahko
\[
X = \bigcup_{n=1}^\infty A_n
\]
za disjunktne množice $A_n$, za katere velja $\mu(A_n) < \infty$.
\end{enumerate}
\end{trditev}

\obvs

\begin{trditev}
Končno aditivna funkcija $\mu \colon \mathcal{A} \to [0, \infty]$
za merljiv prostor $(X, \mu)$ je pozitivna mera natanko tedaj, ko
za vsako naraščajoče zaporedje množic
$(A_n)_{n \in \N} \subseteq \mathcal{A}$ velja
\[
\mu \br{\bigcup_{n=1}^\infty A_n} = \lim_{n \to \infty} \mu(A_n).
\]
\end{trditev}

\begin{proof}
Denimo najprej, da je $\mu$ pozitivna mera. Tedaj je res
\[
\mu \br{\bigcup_{n=1}^\infty A_n} =
\mu \br{A_1 \cup \bigcup_{n=2}^\infty A_n \setminus A_{n-1}} =
\mu(A_1) + \sum_{n=2}^\infty \mu(A_n \setminus A_{n-1}) =
\lim_{n \to \infty} \mu(A_n).
\]
Predpostavimo sedaj, da velja drugi pogoj in naj bodo
$(A_n)_{n \in \N} \subseteq \mathcal{A}$ disjunktne množice. Tedaj
velja
\[
\sum_{n=1}^\infty \mu(A_n) =
\lim_{k \to \infty} \sum_{n=1}^k \mu(A_n) =
\lim_{k \to \infty} \mu \br{\bigcup_{n \leq k} A_n} =
\mu \br{\bigcup_{n=1}^\infty A_n},
\]
zato je $\mu$ res pozitivna.
\end{proof}

\datum{2024-2-26}

\begin{posledica}
Naj bo $(X, \mathcal{A}, \mu)$ merljiv prostor s pozitivno mero in
naj bo $(A_n)_n$ padajoče zaporedje merljivih množic. Če je
$\mu(A_1) < \infty$, velja
\[
\mu \br{\bigcap_{n=1}^\infty A_n} =
\lim_{n \to \infty} \mu(A_n).
\]
\end{posledica}

\begin{proof}
Uporabimo prejšnjo trditev na zaporedju $(A_1 \setminus A_n)_n$.
\end{proof}

\newpage

\subsection{Napolnitev prostora z mero}

\begin{definicija}
Merljiv prostor $(X, \mathcal{A}, \mu)$ s pozitivno mero je
\emph{poln}\index{poln prostor z mero}, če za vsako množico
$A \in \mathcal{A}$, za katero je $\mu(A) = 0$, tudi za vse
$B \subseteq A$ velja $\mu(B) = 0$.
\end{definicija}

\begin{definicija}
Množica $A \in \mathcal{A}$ je
\emph{$\mu$-ničelna}\index{mu-ničelna@$\mu$-ničelna množica}, če je
$\mu(A) = 0$.
\end{definicija}

\begin{opomba}
Števna unija $\mu$-ničelnih množic je spet $\mu$-ničelna.
\end{opomba}

\begin{izrek}
Naj bo $(X, \mathcal{A}, \mu)$ merljiv prostor s pozitivno mero in
naj bo
\[
\oline{\mathcal{A}} =
\setb{B = A \cup S}
{A \in \mathcal{A} \land
\exists N \in \mathcal{A} \colon \mu(N) = 0 \land S \subseteq N}.
\]
Tedaj je $\oline{\mathcal{A}}$ $\sigma$-algebra. S predpisom
$\oline{\mu}(A \cup S) = \mu(A)$ postane
$(X, \oline{\mathcal{A}}, \oline{\mu})$ poln merljiv prostor s
pozitivno mero.
\end{izrek}

\begin{proof}
Preverimo najprej, da je $\oline{\mathcal{A}}$ $\sigma$-algebra.
Očitno je $X \in \oline{\mathcal{A}}$. Ker je
\[
\br{A \cup S}^{\mathsf{c}} =
A^{\mathsf{c}} \cap S^{\mathsf{c}} =
A^{\mathsf{c}} \cap \br{N^{\mathsf{c}} \cup N \setminus S} =
\br{A^{\mathsf{c}} \cap N^{\mathsf{c}}} \cup
\br{A^{\mathsf{c}} \cap N \setminus S},
\]
kar je očitno element $\oline{\mathcal{A}}$, je družina zaprta za
preseke. Zaprtost za števne unije je očitna.

Sedaj preverimo, da je $\oline{\mu}$ pozitivna mera. Mera je dobro
definirana, saj iz $A_1 \cup S_1 = A_2 \cup S_2$ sledi
\[
\mu(A_1) = \mu(A_1 \cup N_1) \geq \mu(A_2)
\]
in simetrično. Ker je
$\oline{\mu}(\emptyset) = 0$ in
\[
\oline{\mu} \br{\bigcup_{n=1}^\infty \br{A_n \cup S_n}} =
\oline{\mu} \br{\bigcup_{n=1}^\infty A_n \cup
\bigcup_{n=1}^\infty S_n} =
\mu \br{\bigcup_{n=1}^\infty A_n} =
\sum_{n=1}^\infty \mu(A_n) =
\sum_{n=1}^\infty \oline{\mu}(A_n \cup S_n),
\]
je mera pozitivna.

Denimo sedaj, da velja $\oline{\mu}(N) = 0$ za nek
$N \in \oline{\mathcal{A}}$ in $B \subseteq N$. Ker je
$N \subseteq M$ za nek $M \in \mathcal{A}$ z $\mu(M) = 0$, je tak
tudi $B$, zato je $\mu(B) = 0$. Prostor je zato res poln.
\end{proof}

\begin{definicija}
\emph{Lebesgueova $\sigma$-algebra}\index{Lebesgueova sigma-algebra@Lebesgueova $\sigma$-algebra}
$\mathcal{L}_\R$ na $\R$ je napolnitev Borelove.
\end{definicija}

\begin{izrek}
Naj bo $C$ Cantorjeva množica. Tedaj je
$\abs{\mathcal{L}_\R} = 2^{\abs{C}}$.
\end{izrek}

\begin{proof}
Cantorjeva množica je $\mu$-ničelna, zato je
$\abs{\mathcal{L}_\R} \geq 2^{\abs{C}}$. Ker pa je
$\mathcal{L}_R \subseteq \pot(\R)$ in je
$\abs{C} = \abs{\R}$, sledi
tudi obratna neenakost.
\end{proof}

\newpage

\subsection{Zunanje mere}

\begin{definicija}
\emph{Zunanja mera}\index{zunanja mera} na $X \ne \emptyset$ je
funkcija $\xi \colon \pot(X) \to [0, \infty]$, za katero velja
naslednje:

\begin{enumerate}[i)]
\item Velja $\xi(\emptyset) = 0$.
\item Če je $A \subseteq B$, je $\xi(A) \leq \xi(B)$.
\item Če so $(A_n)_n \subseteq X$ množice, velja
\[
\xi \br{\bigcup_{n=1}^\infty A_n} \leq
\sum_{n=1}^\infty \xi(A_n).
\]
\end{enumerate}
\end{definicija}

\begin{trditev}
Naj bo $\mathcal{S}$ družina podmnožic neprazne množice $X$, pri
čemer je $\emptyset, X \in \mathcal{S}$. Naj bo
$\mu \colon \mathcal{S} \to [0, \infty]$ funkcija, ki zadošča
$\mu(\emptyset) = 0$. Definiramo
$\mu^* \colon \pot(X) \to [0, \infty]$ s predpisom
\[
\mu^*(Y) =
\inf \setb{\sum_{n=1}^\infty \mu(A_n)}
{(A_n)_n \subseteq \mathcal{S} \land
Y \subseteq \bigcup_{n=1}^\infty A_n}.
\]
Tedaj je $\mu^*$ zunanja mera na $X$.
\end{trditev}

\begin{proof}
Očitno je $\mu^*(\emptyset) = 0$. Prav tako ni težko videti, da je
$\mu^*$ monotona. Sedaj za vsak $A_n$ izberimo pokritje, za
katerega je\footnote{Če je katera izmed $\mu^*(A_n) = \infty$, je
ta lastnost očitna.}
\[
\sum_{k=1}^\infty \mu(A_{n,k}) <
\mu^*(A_n) + \frac{\varepsilon}{2^n}.
\]
Tedaj je
\[
\mu^* \br{\bigcup_{n=1}^\infty A_n} \leq
\sum_{n,k \in \N} \mu(A_{n,k}) <
\sum_{n=1}^\infty \mu^*(A_n) +
\sum_{n=1}^\infty \frac{\varepsilon}{2^n} =
\sum_{n=1}^\infty \mu^*(A_n) + \varepsilon. \qedhere
\]
\end{proof}

\begin{definicija}
Naj bo $\xi$ zunanja mera na $X$. Množica $A \subseteq X$ je
\emph{$\xi$-merljiva}\index{merljiva množica}, če velja
\[
\xi(Y) = \xi(Y \cap A) + \xi(Y \cap A^{\mathsf{x}})
\]
za vse $Y \subseteq X$.
\end{definicija}

\begin{opomba}
Ekvivalentno iz $\xi(Y) < \infty$ sledi
\[
\xi(Y) \geq \xi(Y \cap A) + \xi(Y \cap A^{\mathsf{c}}).
\]
\end{opomba}

\begin{izrek}[Carathéodory]
\index{Carathéodoryjev izrek}
Naj bo $\xi$ zunanja mera za $X$. Tedaj je
\[
\mathcal{A}_\xi =
\setb{A \subseteq X}{\text{$A$ je $\xi$-merljiva}}
\]
$\sigma$-algebra na $X$, $\eval{\xi}{\mathcal{A}_\xi}{}$ pozitivna
mera in $(X, \mathcal{A}_\xi, \eval{\xi}{\mathcal{A}_\xi}{})$ poln
merljiv prostor.
\end{izrek}

\begin{proof}
Najprej opazimo, da je $X \in \mathcal{A}_\xi$, in da je
$\mathcal{A}_\xi$ zaprta za komplemente. Pokažimo, da je zaprta
tudi za končne unije -- naj bosta $A, B \in \mathcal{A}_\xi$. Tedaj
je
\begin{align*}
\xi \br{Y \cap \br{A \cup B}} +
\xi \br{Y \cap \br{A \cup B}^\mathsf{c}} &=
\xi \br{(Y \cap A) \cup (Y \cap B \cap A^\mathsf{c})} +
\xi \br{Y \cap A^\mathsf{c} \cap B^\mathsf{c}}
\\
&\leq
\xi(Y \cap A) + \xi(Y \cap B \cap A^\mathsf{c}) +
\xi \br{Y \cap A^\mathsf{c} \cap B^\mathsf{c}}
\\
&=
\xi(Y \cap A) + \xi(Y \cap A^\mathsf{c})
\\
&=
\xi(Y).
\end{align*}
%TODO
\end{proof}

%TODO dodaj to po definiciji Lebesgueove mere
%\begin{trditev}
%Obstajajo Lebesgueovo nemerljive množice.
%\end{trditev}
%
%\begin{proof}
%Na $\R$ vpeljemo ekvivalenčno relacijo $\sim$ kot
%$x \sim y \iff x-y \in \Q$. Za vsak $x \in [-1, 1]$ izberimo
%predstavnika njegovega ekvivalenčnega razreda iz
%$x+\Q \cap [-1, 1]$ in naj bo $S$ množica teh predstavnikov.
%Označimo $\Q_1 = \Q \cap [-2, 2]$. Množica
%$\mathcal{F} = \setb{r + S}{r \in \Q_1}$ je seveda števna. Hitro se
%prepričamo, da so elementi $\mathcal{F}$ disjunktni. Poleg tega je
%\[
%[-1, 1] \subseteq \bigcup_{r \in \Q_1} (r + S) \subseteq [-3, 3].
%\]
%Sledi, da $S$ ni merljiva.
%\end{proof}
