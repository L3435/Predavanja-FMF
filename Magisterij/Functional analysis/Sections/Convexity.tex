\section{Convexity}

\subsection{Locally convex spaces}

\datum{2023-10-4}

\begin{definicija}
\emph{A topological vector space}\footnote{Also linear topological
space.} $V$ is an $\F$-vector space that is also a topological
space, such both addition and scalar multiplication are continuous.
\end{definicija}

\begin{definicija}
Let $V$ be an $\F$-vectors pace. A map $p \colon V \to \R$ is a
\emph{seminorm}\index{seminorm} if the following holds:

\begin{enumerate}[i)]
\item $\forall x \in V \colon p(x) \geq 0$,
\item $\forall \lambda \in \F, x \in V \colon
p(\lambda x) = \abs{\lambda} p(x)$,
\item $\forall x, y \in V \colon p(x+y) \leq p(x) + p(y)$.
\end{enumerate}
\end{definicija}

\begin{definicija}
Let $V$ be an $\F$-vector space and $\mathcal{P}$ a family of
seminorms on $V$. We define a topology $\mathcal{T}$ on $V$ with
the sets
\[
U(x_0, p, \varepsilon) = \setb{x \in V}{p(x - x_0) < \varepsilon}
\]
as a subbasis.
\end{definicija}

\begin{definicija}
A topological vector space $X$ is a
\emph{locally convex space}\index{locally convex space} if its
topology is generated by a family of seminorms $\mathcal{P}$
satisfying
\[
\bigcap_{p \in \mathcal{P}} \setb{x \in X}{p(x) = 0} = \set{0}.
\]
\end{definicija}

\begin{trditev}
A locally convex space $X$ is Hausdorff.
\end{trditev}

\begin{proof}
Let $x, y \in X$ be distinct points. Let $p \in \mathcal{P}$ be a
seminorm such that $p(x-y) \ne 0$. Then the sets
\[
U = \setb{z \in X}{p(z-x) < \frac{\varepsilon}{2}}
\quad \text{and} \quad
V = \setb{z \in X}{p(z-y) < \frac{\varepsilon}{2}}
\]
split the points $x$ and $y$.
\end{proof}

\begin{opomba}
The converse is also true.
\end{opomba}

\begin{definicija}
A partially ordered set $I$ is
\emph{upward directed}\index{upward directed set} if for all
$i', i'' \in I$ there exists some $i \in I$ such that $i \geq i'$
and $i \geq i''$.
\end{definicija}

\begin{definicija}
A \emph{net}\index{net} is a pair $((I, \leq), x)$, where
$(I, \leq)$ is an upward directed set and $x \colon I \to X$ is a
function. We usually write $(x_i)_{i \in I}$.
\end{definicija}

\begin{opomba}
Let $(X, \mathcal{T})$ be a topological space and $x_0 \in X$.
Partially order the set
\[
\mathcal{U} =
\setb{U \subseteq X}{x_0 \in U \land \text{$U$ is open}}
\]
with reverse inclusion. Then any choice function defines a net
$(x_U)_{U \in \mathcal{U}}$.
\end{opomba}

\begin{definicija}
Let $X$ be a topological space. A net $(x_i)_{i \in I}$
\emph{converges}\index{converging net} so $x \in X$ if for all open
sets $U \subseteq X$ with $x \in U$ there exists some index
$i_0 \in I$ such that for all $i \geq i_0$ we have $x_i \in U$. We
write
\[
\lim_{i \in I} x_i = x.
\]
\end{definicija}

\begin{definicija}
A point $x \in X$ is a \emph{cluster point}\index{cluster point} of
a net $(x_i)_{i \in I}$ if for all open sets $U \subseteq X$ with
$x \in U$ and index $i_0 \in I$ there exists some index
$i \geq i_0$ such that $x_i \in U$.
\end{definicija}

\begin{trditev}
Let $X$ be a topological space and $A \subseteq X$. Then
$x \in \oline{A}$ if and only if there exists a net
$(a_i)_{i \in I}$ in $A$ such that
\[
\lim_{i \in I} a_i = x.
\]
\end{trditev}

\begin{proof}
Suppose a net $(a_i)_{i \in I}$ converges to $x$. For any
neighbourhood $U$ of $x$ and some $i_0 \in I$ we have
$a_{i_0} \in U$. Therefore, $U \cap A \ne \emptyset$.

Assume now that $x \in \oline{A}$. Again, define
\[
\mathcal{U} =
\setb{U \subseteq X}{x_0 \in U \land \text{$U$ is open}}.
\]
There is a choice function $a$ such that $a_U \in A$ for all $U$.
The net $(a_U)_{U \in \mathcal{U}}$ then converges to $x$.
\end{proof}

\begin{trditev}\label{prop:conv:cont_1}
Let $X$ and $Y$ be topological spaces and $f \colon X \to Y$ a map.
Then, $f$ is continuous in $x_0 \in X$ if and only if
\[
\lim_{i \in I} f(x_i) = f(x_0)
\]
for all nets $(x_i)_{i \in I}$ that converge to $x_0$.
\end{trditev}

\begin{proof}
Suppose that $f$ is continuous at $x_0$. Take an open neighbourhood
$U$ of $f(x_0)$. Then there must exist some $i_0 \in I$ such that
for all $i \geq i_0$ we have $x_i \in f^{-1}(U)$, therefore
$f(x_i) \in U$.

Now suppose $f$ is discontinuous at $x_0$. Let
\[
\mathcal{U} =
\setb{U \subseteq X}{x_0 \in U \land \text{$U$ is open}}
\]
and $V \subseteq Y$ be an open set such that $f(x_0) \in V$ and
$x_0$ is not an interior point of $f^{-1}(V)$. Now using the
discontinuity of $f$, for all $U \in \mathcal{U}$ choose
$x_U \in U$ such that $f(x_U) \not \in V$. Trivially the net
$(x_V)_{V \in \mathcal{V}}$ converges to $x_0$, but
\[
\lim_{V \in \mathcal{V}} f(x_V) \ne f(x_0). \qedhere
\]
\end{proof}

\begin{trditev}
The following statements are true:

\begin{enumerate}[i)]
\item A net $(x_i)_{i \in I}$ in a locally convex space converges
to $x_0$ if and only if the net $(p(x_i - x_0))_{i \in I}$
converges to $0$ for all $p \in \mathcal{P}$.
\item The topology in a locally convex space $X$ is the coarsest
topology in which all the maps $x \mapsto p(x - x_0)$ are
continuous for all $x_0 \in X$ and $p \in \mathcal{P}$.
\end{enumerate}
\end{trditev}

\begin{proof}
\phantom{a}
\begin{enumerate}[i)]
\item If $(x_i)_{i \in I}$ converges to $x_0$, just apply the
proposition~\ref{prop:conv:cont_1}. Suppose that all the nets
$(p(x_i - x_0))_{i \in I}$ converge to $0$. Choose an open set from
the local basis of $x_0$. It is given by
\[
U =
\setb{x \in X}{\forall k \leq n \colon p_k(x - x_0) < \varepsilon}.
\]
But as all nets $(p_k(x_i - x_0))_{i \in I}$ converge to $0$, there
is some index $i_k \in I$ such that for all $i \geq i_k$ we have
$p_k(x_i - x_0) < \varepsilon$. Now just take $i_0$ to be an upper
bound of $i_k$. For all $i \geq i_0$ we then have $x_i \in U$.
\item Obvious. \qedhere
\end{enumerate}
\end{proof}

\begin{definicija}
For all $f \in X^*$ define a seminorm $p_f \colon X \to \R$ as
$p_f(x) = \abs{f(x)}$. The family
$\mathcal{P} = \setb{p_f}{f \in X^*}$ induces the
\emph{weak topology}\index{weak topology} on $X$. We denote the
weak topology with $\sigma(X, X^*)$.
\end{definicija}

\begin{opomba}
The space $X$ with the topology $\sigma(X, X^*)$ is a locally
compact space by the Hahn-Banach theorem.\footnote{Introduction to
functional analysis, corollary 2.2.5.2.}
\end{opomba}

\begin{definicija}
Let $X$ be a normed space. For all $x \in X$ we define a seminorm
$p_x \colon X^* \to \R$ as $p_x(f) = \abs{f(x)}$. The family
$\mathcal{P} = \setb{p_x}{x \in X}$ induces the
\emph{weak-* topology}\index{weak-* topology} on $X^*$. We denote
the weak-* topology with $\sigma(X^*, X)$.
\end{definicija}

\begin{opomba}
The weak topology on $X^*$ is finer than the weak-* topology, as
$X$ can be isometrically mapped into $X^{**}$ with the map
$x \mapsto (f \mapsto f(x))$.
\end{opomba}

\newpage

\subsection{Banach-Alaoglu theorem}

\begin{izrek}[Banach-Alaoglu]
\index{Banach-Alaoglu theorem}
Let $X$ be a normed space. Then the closed unit ball in $X^*$
\[
(X^*)_1 = \setb{f \in X^*}{\norm{f} \leq 1}
\]
is compact in the weak-* topology on $X^*$.
\end{izrek}

\begin{proof}
Assign a disk to all $x \in X$ as
$D_x = \setb{z \in \F}{\abs{z} \leq \norm{x}}$ with the euclidean
topology. Define
\[
P = \prod_{x \in X} D_x
\]
with the product topology. The space $P$ is then compact by
Tychonoff's theorem. Now define the map $\Phi \colon (X^*)_1 \to P$
with $\Phi(f) = (f(x))_{x \in X}$. This map is injective.

Let $(f_i)_{i \in I}$ be a net in $(X^*)_1$ that weak-* converges
to $f \in X^*$. Equivalently, we have
\[
\lim_{i \in I} f_i(x) = f(x)
\]
for all $x \in X$. By the definition of the product topology we
have
\[
\lim_{i \in I} \Phi(f_i) = \Phi(f).
\]
Therefore, $\Phi$ is continuous. Analogously,
$\Phi^{-1} \colon \im \Phi \to (X^*)_1$ is continuous.

Suppose that $(\Phi(f_i))_{i \in I}$ converges to some $p \in P$.
By the definition of the product topology this means that
$f_i(x)$ converges to $p_x$ for all $x \in X$. Define a map
$f \colon X \to \F$ given by $f(x) = p_x$. Then, $f$ is linear and
bounded with $\norm{f} \leq 1$. Thus $p = \Phi(f) \in \im(\Phi)$,
therefore, $\Phi((X^*)_1)$ is closed. As $(X^*)_1$ is homeomorphic
to its image which is compact, it is also compact.
\end{proof}

\begin{posledica}
Every Banach space $X$ is isometrically isomorphic to a closed
subspace $\mathcal{C}(K)$ for some compact Hausdorff space $K$.
\end{posledica}

\begin{proof}
Choose $K = (X^*)_1$ with the weak-* topology. By Banach-Alaoglu,
$K$ is compact and Hausdorff. Now define the map
$\Delta \colon X \to K$ with $\Delta(x) = (f \mapsto f(x))$.
Now observe that
\[
\norm{\Delta(x)}_\infty =
\sup_{g \in K} \abs{\Delta(x)(g)} =
\sup_{g \in K} \abs{g(x)} =
\norm{x}
\]
by Hahn-Banach.\footnote{Introduction to functional analysis,
corollary 2.2.5.1.}
\end{proof}
