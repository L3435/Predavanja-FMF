\section{\texorpdfstring{$C^*$}{C*}-algebras and continuous functional calculus}

\subsection{Spectrum}

\begin{definicija}
Let $A$ be a complex algebra with unity $1$. Define the set
\[
\GL(A) = \setb{a \in A}{\exists b \in A \colon ab = ba = 1}.
\]
The \emph{spectrum}\index{spectrum} of $x \in A$ is the set
\[
\sigma_A(x) =
\setb{\lambda \in \C}{x - \lambda \cdot 1 \not \in \GL(A)}.
\]
\end{definicija}

\begin{trditev}
Let $A$ be a complex algebra with unity $1$ and $x, y \in A$. Then
\[
\sigma_A(xy) \cup \set{0} = \sigma_A(yx) \cup \set{0}.
\]
\end{trditev}

\begin{proof}
By scaling, it is enough to check that
$1 \in \sigma_A(xy) \iff 1 \in \sigma_A(yx)$. Suppose that
$1 - xy \in \GL(A)$. We can check that $1 - yx$ is invertible with
\[
(1-yx)^{-1} = 1 + y (1 - xy)^{-1} x. \qedhere
\]
\end{proof}

\newpage

\subsection{Banach and \texorpdfstring{$C^*$}{C*}-algebras}

\begin{definicija}
A \emph{Banach algebra}\index{Banach algebra} is a Banach space $A$
that is also an algebra, such that
$\norm{xy} \leq \norm{x} \cdot \norm{y}$ holds for all
$x, y \in A$. If a Banach algebra has an unity, we also demand
$\norm{1} = 1$.
\end{definicija}

\begin{definicija}
An \emph{involution}\index{involution} on a Banach algebra $A$ is a
skew-linear map $* \colon A \to A$, satisfying the following for
all $x, y \in A$:

\begin{enumerate}[i)]
\item $(xy)^* = y^* x^*$,
\item $(x^*)^* = x$,
\item $\norm{x^*} = \norm{x}$.
\end{enumerate}

A Banach algebra with involution is called a
\emph{Banach *-algebra}.
\end{definicija}

\begin{definicija}
A Banach *-algebra $A$ that also satisfies
$\norm{x^* x} = \norm{x}^2$ for all $x \in A$ is called a
\emph{$C^*$-algebra}\index{C star algebra@$C^*$-algebra}.
\end{definicija}

\begin{trditev}
Let $A$ be a Banach *-algebra. Then, for all $x \in A$ we have
$(x^*)^{-1} = (x^{-1})^*$ and
$\sigma_A(x^*) = \oline{\sigma_A(x)}$.
\end{trditev}

\begin{trditev}
Let $A$ be a Banach algebra. The following statements are true:

\begin{enumerate}[i)]
\item Let $x \in A$. If $\norm{x} < 1$, then $1 - x \in \GL(A)$ and
\[
(1-x)^{-1} = \sum_{n \in \N_0} x^n.
\]
\item The set $\GL(A)$ is an open subset of $A$ and the map
$x \mapsto x^{-1}$ is continuous on $\GL(A)$.
\end{enumerate}
\end{trditev}

\begin{proof}
Let $y \in \GL(A)$. If $\norm{x-y} < \frac{1}{\norm{y^{-1}}}$, then
\[
\norm{1-xy^{-1}} = \norm{(y-x) y^{-1}} < 1,
\]
therefore, $xy^{-1}$ is invertible. It follows that $x$ is also
invertible, so $\GL(A)$ is open.

Note that
\begin{align*}
\norm{(xy^{-1})^{-1}} &=
\norm{\br{1 - (1 - xy^{-1})}^{-1}}
\\
&\leq
\sum_{n \in \N_0} \norm{1 - xy^{-1}}^n
\\
&\leq
\sum_{n \in \N_0} \norm{y^{-1}}^n \cdot \norm{x-y}^n
\\
&=
\frac{1}{1 - \norm{y^{-1}} \norm{x-y}}.
\end{align*}
It follows that
\begin{align*}
\norm{x^{-1} - y^{-1}} &=
\norm{x^{-1} (y-x) y^{-1}}
\\
&\leq
\norm{y^{-1} (xy^{-1})^{-1}} \cdot
\norm{y^{-1}} \cdot \norm{x-y}
\\
&\leq
\frac{\norm{y^{-1}}^2}{1 - \norm{y^{-1}} \cdot \norm{x-y}} \cdot
\norm{x-y}. \qedhere
\end{align*}
\end{proof}

\begin{trditev}
Let $A$ be a Banach algebra and $x \in A$. Then $\sigma_A(x)$ is a
non-empty compact set.
\end{trditev}

\begin{proof}
Introduction to functional analysis, theorem~6.1.15.
\end{proof}

\begin{izrek}[Gelfald-Mazur]
\index{Gelfald-Mazur theorem}
If $A$ is a Banach algebra that is also a division ring, then
$A = \C$.
\end{izrek}

\begin{proof}
Let $x \in A$ and $\lambda \in \sigma_A(x)$. As
$x - \lambda \cdot 1 \not \in \GL(A) = A \setminus \set{0}$, we
have $x = \lambda \cdot 1 \in \C$.
\end{proof}

\begin{definicija}
If
\[
f(x) = \sum_{j=0}^n a_j x^j
\]
is a polynomial and $a \in A$ an element of an algebra, we define
\[
f(a) = \sum_{j=0}^n a_j a^j \in A.
\]
\end{definicija}

\begin{izrek}[Spectral mapping theorem for polynomials]
\index{spectral mapping theorem}
Let $A$ be an algebra and $f \in \C[x]$. Then
\[
f(\sigma_A(a)) = \sigma_A(f(a))
\]
holds for all $a \in A$.
\end{izrek}

\begin{proof}
Let $\lambda \in \sigma_A(a)$ and
\[
f(x) = \sum_{j=0}^n a_j x^j.
\]
We can write
\[
f - f(\lambda) =
(x - \lambda) \cdot
\sum_{j=1}^n a_j \sum_{k=0}^{j-1} x^k \lambda^{j-i-k}.
\]
It follows that
\[
f(a) - f(\lambda) =
(a - \lambda) \cdot
\sum_{j=1}^n a_j \sum_{k=0}^{j-1} a^k \lambda^{j-1-k}.
\]
As $a - \lambda$ is not invertible and commutes with the second
factor, it follows that $f(a) - f(\lambda)$ is also not invertible.

Conversely, if $\mu \not \in f(\sigma_A(a))$, we can write
\[
f - \mu = a_n \cdot \prod_{j=1}^n (x - \lambda_j).
\]
As $f(\lambda) - \mu \ne 0$ for all $\lambda \in \sigma_A(a)$, we
have $\lambda_i \not \in \sigma_A(a)$ for all $i$. Therefore, it
follows that $f(a) - \mu \in \GL(A)$.
\end{proof}

\begin{definicija}
Let $A$ be a Banach algebra and $x \in A$. The
\emph{spectral radius}\index{spectral radius} of $x$ is
\[
r(x) = \sup_{\lambda \in \sigma_A(x)} \abs{\lambda}.
\]
\end{definicija}

\begin{izrek}[Spectral radius formula]
\index{spectral radius formula}
Let $A$ be a Banach algebra and $x \in A$. Then, the limit
\[
\lim_{n \to \infty} \sqrt[n]{\norm{x^n}}
\]
exists and
\[
r(x) = \lim_{n \to \infty} \sqrt[n]{\norm{x^n}}.
\]
\end{izrek}

\begin{proof}
Introduction to functional analysis, theorem~6.1.20.
\end{proof}

\begin{definicija}
Let $A$ be a Banach *-algebra and $x \in A$.

\begin{enumerate}[i)]
\item The element $x$ is \emph{normal}\index{normal} if
$x x^* = x^* x$.
\item The element $x$ is \emph{selfadjoint}\index{selfadjoint} if
$x = x^*$.
\item The element $x$ is
\emph{skew selfadjoint}\index{skew selfadjoint} if $x = -x^*$.
\end{enumerate}
\end{definicija}

\begin{posledica}
Let $A$ be a Banach *-algebra and $x \in A$ a normal element. Then
\[
r(x^* x) \leq r(x)^2.
\]
If $A$ is a $C^*$-algebra, then $r(x^* x) = r(x)^2$.
\end{posledica}

\begin{proof}
Note that
\[
r(x^* x) =
\lim_{n \to \infty} \sqrt[n]{\norm{(x^* x)^n}} \leq
\lim_{n \to \infty} \sqrt[n]{\norm{x^n}}^2 = r(x)^2.
\]
If $A$ is a $C^*$-algebra, we have equality.
\end{proof}

\begin{trditev}
Let $A$ be a $C^*$-algebra and $x \in A$ a normal element. Then
\[
r(x) = \norm{x}.
\]
\end{trditev}

\begin{proof}
The statement holds for selfadjoint elements by Introduction to
functional analysis, corollary~6.1.20.1. We can therefore write
\[
\norm{x}^2 = \norm{x^* x} = r(x^* x) = r(x)^2. \qedhere
\]
\end{proof}
