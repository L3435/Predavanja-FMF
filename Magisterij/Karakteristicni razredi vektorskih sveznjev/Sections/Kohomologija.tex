\section{Kohomologija}

\subsection{Aksiomatična kohomologija}

\datum{2024-2-22}

\begin{definicija}
Naj bo $R$ komutativen kolobar in $\catt{Top_2}$ kategorija
topoloških parov z zveznimi preslikavami. Naj bo
$\kappa \colon \catt{Top_2} \to \catt{Top_2}$ funktor, ki objektu
$(X, A)$ priredi $(A, \emptyset)$, morfizmu
$f \colon (X, A) \to (Y, B)$ pa $\eval{f}{A}{}$.
\emph{Kohomološka teorija}\index{kohomološka teorija} z vrednostmi
v $\catt{_{R}Mod}$ je sestavljena iz zaporedja kofunktorjev
$h^n \colon \catt{Top_2} \to \catt{_RMod}$ in naravnih
transformacij $\delta^{n-1} \colon h^{n-1} \circ \kappa \to h^n$,
ki zadošča naslednjim aksiomom:

\begin{itemize}
\item Homotopija: Če je $F \colon (X, A) \times I \to (Y, B)$
homotopija parov, velja $h^n(F_0) = h^n(F_1)$ za vsak $n$.
\item Eksaktnost: Zaporedje
\[
\begin{tikzcd}[row sep=large]
\cdots \arrow[r] &
h^n(X, A) \arrow[r] &
h^n(X, \emptyset) \arrow[r] &
h^n(A, \emptyset) \arrow[lld, "\delta_{(X,A)}^n"'] \\&
h^{n+1}(X, A) \arrow[r] &
h^{n+1}(X, \emptyset) \arrow[r] &
h^{n+1}(A, \emptyset) \arrow[r] &
\cdots
\end{tikzcd}
\]
v $\catt{_RMod}$ je eksaktno.
\item Izrez: Naj bo $(X, A)$ par in $U \subseteq A$. Denimo, da
obstaja taka zvezna preslikava $\tau \colon X \to I$, da je
$U \subseteq \tau^{-1}(0) \subseteq \tau^{-1}([0,1)) \subseteq A$.
Tedaj je z inkluzijo inducirani morfizem
$h^n(X, A) \to h^n(X \setminus U, A \setminus U)$ izomorfizem za
vsak $n$.
\end{itemize}

Zaporedju modulov $\setb{h^n(\bullet)}{n \in \Z}$ pravimo
\emph{koeficienti teorije}\index{koeficienti kohomološke teorije}.
Kohomološko teorijo v celoti označimo s $h^*$. Pišemo
$h^n(X) = h^n(X, \emptyset)$.
\end{definicija}

\begin{opomba}
Ker je $\delta^{n-1}$ naravna transformacija, vsaka zvezna
preslikava parov $f \colon (A, X) \to (Y, B)$ inducira morfizem
eksaktnih zaporedij.
\end{opomba}

\begin{definicija}
Če velja $h^n(\bullet) = 0$ za vse $n \ne 0$, pravimo, da gre za
\emph{običajno kohomološko teorijo}\index{običajna kohomološka teorija}
s koeficienti v $R$-modulu $h^0(\bullet)$.
\end{definicija}

\begin{izrek}[Eksaktno zaporedje trojice]
Za dano trojico $B \subset A \subset X$ je zaporedje
\[
\begin{tikzcd}[row sep=large]
\cdots \arrow[r] &
h^n(X, A) \arrow[r] &
h^n(X, B) \arrow[r] &
h^n(A, B) \arrow[dll, "\delta_{(X,A,B)}^n"']
\\ &
h^{n+1}(X,A) \arrow[r] &
h^n(X, B) \arrow[r] &
h^n(A, B) \arrow[r] &
\cdots
\end{tikzcd}
\]
eksaktno, kjer je $\delta_{(X,A,B)}^n$ takšna, da diagram
\[
\begin{tikzcd}[row sep=large]
h^n(A,B) \arrow[r] \arrow[dr, "\delta_{(X,A,B)}^n"'] &
h^n(A, \emptyset) \arrow[d, "\delta_{(X,A)}^n"] \\ &
h^{n+1}(X,A)
\end{tikzcd}
\]
komutira.
\end{izrek}

\begin{proof}
Oglejmo si naslednji diagram:
\[
\begin{tikzcd}
h^{n-1}(B) \arrow[dr, blue] \arrow[dd, out=225, in=135, red]&
&
h^n(X,A) \arrow[dl, "\alpha"']
\arrow[dd, out=315, in=45, green!70!black]
\\
&
h^n(X,B) \arrow[dl, "\beta"'] \arrow[dr, blue]
\\
h^n(A,B) \arrow[dr, red]
\arrow[dd, out=225, in=135, "\delta_{(X,A,B)}^n"'] &
&
h^n(X) \arrow[dl, green!70!black] \arrow[dd, out=315, in=45, blue]
\\
&
h^n(A) \arrow[dl, green!70!black] \arrow[dr, red]
\\
h^{n+1}(X,A) \arrow[dr, "\gamma"] &
&
h^n(B) \arrow[dl, blue]
\\
&
h^{n+1}(X,B)
\end{tikzcd}
\]
Vsako izmed barvitih zaporedij je eksaktno po definiciji homološke
teorije. Diagram je komutativen, saj $\delta$ naravne
transformacije. Od tod takoj dobimo
$\delta_{(X,A,B)}^n \circ \beta = 0$ in
$\gamma \circ \delta_{(X,A,B)}^n = 0$.

Opazimo, da diagram inkluzij
\[
\begin{tikzcd}
(X,A) \arrow[r, hook] \arrow[d, hook] & (X,B) \arrow[d, hook] \\
(A,A) \arrow[r, hook] & (A,B)
\end{tikzcd}
\]
komutira, zato komutira tudi diagram
\[
\begin{tikzcd}
h^n(A,B) & h^n(X,B) \arrow[l, "\beta"'] \\
h^n(A,A) \arrow[u] & h^n(X,A). \arrow[u, "\alpha"'] \arrow[l]
\end{tikzcd}
\]
Ker je $h^n(A,A) = 0$, je tako tudi $\beta \circ \alpha = 0$.
Sledi, da je iskano zaporedje verižni kompleks. Eksaktnost sledi iz
lovljenja po diagramu.\footnote{Rezultatu pravimo tudi
\emph{lema o kiti}\index{lema o kiti}.}
\end{proof}

\datum{2024-2-23}

\begin{trditev}
Če je $f \colon (X, A) \to (Y, B)$ homotopska ekvivalenca parov, so
morfizmi $f^* = h^n(f) \colon h^n(Y, B) \to h^n(X, A)$ izomorfizmi.
\end{trditev}

\obvs

\begin{trditev}
Če je $f \colon (X, A) \to (Y, B)$ preslikava, za katero sta
$f$ in $\eval{f}{A}{}$ homotopski ekvivalenci, so
$f^* \colon h^n(Y, B) \to h^n(X, A)$ izomorfizmi.
\end{trditev}

\begin{proof}
Uporabimo lemo o petih na diagramu
\[
\begin{tikzcd}
\cdots \arrow[r] &
h^{n-1}(B) \arrow[r] \arrow[d, "f^*"', "\cong"] &
h^n(Y, B) \arrow[r] \arrow[d, "f^*"'] &
h^n(X) \arrow[r] \arrow[d, "f^*"', "\cong"] &
\cdots \\
\cdots \arrow[r] &
h^{n-1}(B) \arrow[r] &
h^n(Y, B) \arrow[r] &
h^n(X) \arrow[r] &
\cdots
\end{tikzcd}
\qedhere
\]
\end{proof}

\begin{trditev}
Obstaja izomorfizem\footnote{\emph{Suspenzijski izomorfizem}.}
\index{suspenzijski izomorfizem}
$\sigma \colon
h^{k-1}(S^{n-1}, \set{e_1}) \to h^k(S^n, \set{e_1})$.
\end{trditev}

\begin{proof}
Imamo diagram inkluzij
\[
\begin{tikzcd}[row sep=large]
(S^n, S^n \setminus \set{-e_{n+1}}) &
(S_-^n, S_-^n \setminus \set{-e_{n+1}}) \arrow[l, hook] \\
(S^n, S_+^n) \arrow[u, hook] &
(S_-^n, S^{n-1}) \arrow[u, hook] \arrow[l, hook],
\end{tikzcd}
\]
ki seveda komutira. Tako dobimo sledeč komutativen diagram:
\[
\begin{tikzcd}[row sep=large, column sep=small]
&
h^k(S^n, S^n \setminus \set{-e_{n+1}})
\arrow[r, "\alpha_1"] \arrow[d, "\alpha_2"'] &
h^k(S_-^n, S_-^n \setminus \set{-e_{n+1}}) \arrow[d, "\alpha_3"] \\
h^{k-1}(S_+^n, \set{e_1}) \arrow[r] &
h^k(S^n, S_+^n) \arrow[r, "\beta"'] \arrow[d] &
h^k(S_-^n, S^{n-1}) \arrow[r] &
h^k(S_-^n, \set{e_1})
\\
h^k(S_+^n, \set{e_1}) &
h^k(S^n, \set{e_1}) \arrow[l] &
h^{k-1}(S^{n-1}, \set{e_1}) \arrow[u] &
h^{k-1}(S_-^n, \set{e_1}). \arrow[l]
\end{tikzcd}
\]
Morfizem $\alpha_1$ je izomorfizem po lastnosti izreza, $\alpha_2$
in $\alpha_3$ pa sta izomorfizma po prejšnji trditvi. Sledi, da je
tudi $\beta$ izomorfizem. Ker pa je $h^l(S_*^n, \set{e_1}) = 0$ za
vse $l$ po prejšnji trditvi, dobimo
\[
h^k(S^n, \set{e_1}) \cong
h^k(S^n, S_+^n) \cong
h^k(S_-^n, S^{n-1}) \cong
h^{k-1}(S^{n-1}, \set{e_1}). \qedhere
\]
\end{proof}

\begin{posledica}
Velja $h^k(S^n, \set{e_1}) \cong h^{k-n}(\set{e_1})$.
\end{posledica}

\begin{definicija}
Triada $(X, A, B)$, pri čemer je $X = A \cup B$, je
\emph{izrezna}\index{izrezna triada}, če so morfizmi
$i^* \colon h^n(X, A) \to h^n(B, A \cap B)$ izomorfizmi.
\end{definicija}

\begin{opomba}
Izrezna triada $(X, A, B)$ inducira komutativno lestev
\[
\begin{tikzcd}[row sep=large, column sep=small]
\cdots \arrow[r] &
h^n(X,A) \arrow[r] \arrow[d, "i^*"', "\cong"] &
h^n(X) \arrow[r] \arrow[d] &
h^n(A) \arrow[r] \arrow[d] &
h^{n+1}(X,A) \arrow[r] \arrow[d, "i^*"', "\cong"] &
\cdots \\
\cdots \arrow[r] &
h^n(B,A \cap B) \arrow[r] &
h^n(B) \arrow[r] &
h^n(A \cap B) \arrow[r] &
h^{n+1}(B,A \cap B) \arrow[r] &
\cdots,
\end{tikzcd}
\]
in njeno pripadajoče
\emph{Mayer-Vietorisovo zaporedje}\index{Mayer-Vietorisovo zaporedje}.
Podobno v primeru, ko je $(A \cup B, A, B)$ izrezna triada za
$A \cup B \subset X$, dobimo komutativno lestev
\[
\begin{tikzcd}[row sep=large, column sep=small]
\cdots \arrow[r] &
h^n(X, A \cup B) \arrow[r] \arrow[d] &
h^n(X, A) \arrow[r] \arrow[d] &
h^n(A \cup B, A) \arrow[r] \arrow[d, "i^*"', "\cong"] &
h^{n+1}(X,A \cup B) \arrow[r] \arrow[d] &
\cdots \\
\cdots \arrow[r] &
h^n(X, B) \arrow[r] &
h^n(X, A \cap B) \arrow[r] &
h^n(B, A \cap B) \arrow[r] &
h^{n+1}(X, B) \arrow[r] &
\cdots
\end{tikzcd}
\]
in njeno pripadajoče Mayer-Vietorisovo zaporedje.
\end{opomba}
