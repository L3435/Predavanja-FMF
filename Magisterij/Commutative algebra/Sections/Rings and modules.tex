\section{Rings and modules}

\subsection{Rings and ring homomorphisms}

\datum{2025-2-17}

\begin{definicija}
Unless stated otherwise, rings always have a unit and are
commutative.
\end{definicija}

\begin{definicija}
Let $A$ be a ring. The set $A^\bullet$ denotes the set of
non-zero-divisors.
\end{definicija}

\begin{definicija}
A ring $A$ is a \emph{domain}\index{domain} if $0$ is the only
zero-divisor of $A$.
\end{definicija}

\begin{definicija}
Let $A \subseteq B$ be rings and $S \subseteq B$ a subset. The ring
\[
A[S] =
\bigcap_{\substack{A \subseteq A' \subseteq B \\ S \subseteq A'}}
A'
\]
is the subring of $B$ obtained by \emph{adjoining}\index{adjoining}
$S$ to $A$.
\end{definicija}

\begin{definicija}
The set $\Spec(A)$ denotes the prime ideals of $A$.
\end{definicija}

\begin{definicija}
The \emph{radical}\index{radical} of an ideal $I$ is defined as
\[
\sqrt{I} = \setb{a \in A}{\exists n \in \N \colon a^n \in I}.
\]
\end{definicija}

\begin{trditev}
The radical of an ideal is again an ideal.
\end{trditev}

\begin{proof}
It suffices to show that for any $a, b \in \sqrt{I}$ their sum is
also in $\sqrt{I}$. Suppose that $a^n, b^m \in I$. Then
\begin{align*}
(a+b)^{n+m-1} &=
\sum_{k=0}^{n+m-1} \binom{m+n-1}{k} a^k b^{n+m-1-k}
\\
&=
b^m \sum_{k=0}^{n-1} \binom{m+n-1}{k} a^k b^{n-1-k} +
a^n \sum_{k=n}^{n+m-1} \binom{m+n-1}{k} a^{k-n} b^{n+m-1-k} \in I.
\qedhere
\end{align*}
\end{proof}

\begin{definicija}
The \emph{nilradical}\index{nilradical} of $A$ is the set
$\mathcal{N}(A) = \sqrt{(0)}$.
\end{definicija}

\begin{definicija}
The \emph{Jacobson radical}\index{Jacobson radical}
$\mathcal{J}(A)$ is the intersection of all maximal ideals in $A$.
\end{definicija}

\begin{lema}
The nilradical is contained in the Jacobson radical.
\end{lema}

\begin{proof}
Let $a \in \mathcal{N}(A)$ and suppose that $a^n = 0$. For any
maximal ideal $M$, we know that $a^n \in M$. Since $M$ is prime, we
deduce $a \in M$.
\end{proof}

\begin{lema}
We have
\[
\mathcal{J}(A) =
\setb{a \in A}{\forall b \in A \colon 1 - ba \in A^\times}.
\]
\end{lema}

\begin{proof}
Let $a \in \mathcal{J}(A)$ and $b \in A$. Note that
$1 - ab \not \in M$ for any maximal ideal $M$, since $ab \in M$.
As $1 - ab$ is not contained in any maximal ideal, it follows that
$(1-ab) = A$, hence $1-ab$ is invertible.

Suppose now that $1 - ab \in A^\times$ for all $b \in A$. Let $M$
be a maximal ideal and suppose $a \not \in M$. Then $(M, a) = A$.
In particular, we can write $1 = m + xa$ with $m \in M$ and
$x \in A$. Rearranging, $m = 1 - xa$, which is a contradiction, as
$1-xa$ is invertible.
\end{proof}

\begin{lema}
The following statements hold:

\begin{enumerate}[i)]
\item Let $I \edn A$ and $P_1, \dots, P_n \in \Spec(A)$. If
$I \subseteq P_1 \cup \dots \cup P_n$, there exists some $k$ such
that $I \subseteq P_k$.
\item Let $I_1, \dots, I_n \edn A$ and $P \in \Spec(A)$. If
$I_1 \cap \dots \cap I_n \subseteq P$, then there exists some $k$
such that $I_k \subseteq P$.
\end{enumerate}
\end{lema}

\begin{proof}
\phantom{i}
\begin{enumerate}[i)]
\item We induct on $n$, noting that the statement trivially holds
for $n=1$.

Suppose the statement doesn't hold for $n$. By the induction
hypothesis we can find
\[
a_i \in I \bigsetminus \bigcup_{j \ne i} P_j
\]
for any $i$. Then $a_i \in P_i$. Consider the element
\[
a = \sum_{i=1}^n \prod_{j \ne i} a_j.
\]
Note that all but one of the above terms are an element of $P_i$.
But then $a$ is not an element of any $P_i$, which is a
contradiction.
\item Suppose the contrary and let $a_j \in I_j \setminus P$ for
all $j$. But then
\[
\prod_{j=1}^n a_j \subseteq
\prod_{j=1}^n I_j \subseteq
\bigcap_{j=1}^n I_j \subseteq
P,
\]
which is a contradiction. \qedhere
\end{enumerate}
\end{proof}

\begin{opomba}
The first statement is called
\emph{prime avoidance}\index{prime avoidance}.
\end{opomba}

\begin{trditev}
Let $f \colon A \to B$ be a ring homomorphism. If
$I \edn B$, then $f^{-1}(I) \edn A$. Furthermore, if
$P \in \Spec(B)$, then $f^{-1}(P) \in \Spec(A)$.
\end{trditev}

\begin{trditev}[Universal property]
\index{universal property for quotients}
Let $I \edn A$ and $\pi \colon A \to \kvoc{A}{I}$ be the canonical
epimorphism. For every ring homomorphism $f \colon B$ with
$I \subseteq \ker(f)$, there exists a unique ring homomorphism
$\hat{f} \colon \kvoc{A}{I} \to B$ such that
$f = \hat{f} \circ \pi$.
\[
\begin{tikzcd}[row sep=large, column sep=large]
A \arrow[r, "\pi"] \arrow[rd, "f"'] &
\kvoc{A}{I} \arrow[d, dashed, "\hat{f}"] \\ &
B
\end{tikzcd}
\]
\end{trditev}

\begin{posledica}
If $f \colon A \to B$ is a ring homomorphism, then
$\kvoc{A}{\ker f} \cong f(A)$.
\end{posledica}

\begin{izrek}[Isomorphism theorems]
\index{isomorphism theorems}
The following statements hold:

\begin{enumerate}[i)]
\item Let $I \edn A$. There is a bijective correspondence
\[
\setb{J \edn A}{I \subseteq J} \leftrightarrow
\set{\oline{J} \edn \kvoc{A}{I}},
\]
given by $J \mapsto \kvoc{J}{I}$ and
$\oline{J} \mapsto \pi^{-1} \br{\oline{J}}$.
\item If $I, J \edn A$ with $I \subseteq J$, then
\[
\kvoc{A}{J} \cong \kvoc{\kvoc{A}{I}}{\kvoc{J}{I}}.
\]
\item Let $B \subseteq A$ be a subring and $I \edn A$. Then
$I \cap B \edn B$ and
\[
\kvoc{B+I}{I} \cong \kvoc{B}{B \cap I}.
\]
\end{enumerate}
\end{izrek}
