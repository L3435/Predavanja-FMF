\section{Rings and modules}

\subsection{Rings and ring homomorphisms}

\datum{2025-2-17}

\begin{definicija}
Unless stated otherwise, rings always have a unit and are
commutative.
\end{definicija}

\begin{definicija}
Let $A$ be a ring. The set $A^\bullet$ denotes the set of
non-zero-divisors.
\end{definicija}

\begin{definicija}
A ring $A$ is a \emph{domain}\index{domain} if $0$ is the only
zero-divisor of $A$.
\end{definicija}

\begin{definicija}
Let $A \subseteq B$ be rings and $S \subseteq B$ a subset. The ring
\[
A[S] =
\bigcap_{\substack{A \subseteq A' \subseteq B \\ S \subseteq A'}}
A'
\]
is the subring of $B$ obtained by \emph{adjoining}\index{adjoining}
$S$ to $A$.
\end{definicija}

\begin{definicija}
The set $\Spec(A)$ denotes the prime ideals of $A$.
\end{definicija}

\begin{definicija}
The \emph{radical}\index{radical} of an ideal $I$ is defined as
\[
\sqrt{I} = \setb{a \in A}{\exists n \in \N \colon a^n \in I}.
\]
\end{definicija}

\begin{trditev}
The radical of an ideal is again an ideal.
\end{trditev}

\begin{proof}
It suffices to show that for any $a, b \in \sqrt{I}$ their sum is
also in $\sqrt{I}$. Suppose that $a^n, b^m \in I$. Then
\begin{align*}
(a+b)^{n+m-1} &=
\sum_{k=0}^{n+m-1} \binom{m+n-1}{k} a^k b^{n+m-1-k}
\\
&=
b^m \sum_{k=0}^{n-1} \binom{m+n-1}{k} a^k b^{n-1-k} +
a^n \sum_{k=n}^{n+m-1} \binom{m+n-1}{k} a^{k-n} b^{n+m-1-k} \in I.
\qedhere
\end{align*}
\end{proof}

\begin{definicija}
The \emph{nilradical}\index{nilradical} of $A$ is the set
$\mathcal{N}(A) = \sqrt{(0)}$.
\end{definicija}

\begin{definicija}
The \emph{Jacobson radical}\index{Jacobson radical}
$\mathcal{J}(A)$ is the intersection of all maximal ideals in $A$.
\end{definicija}

\begin{lema}
The nilradical is contained in the Jacobson radical.
\end{lema}

\begin{proof}
Let $a \in \mathcal{N}(A)$ and suppose that $a^n = 0$. For any
maximal ideal $M$, we know that $a^n \in M$. Since $M$ is prime, we
deduce $a \in M$.
\end{proof}

\begin{lema}
We have
\[
\mathcal{J}(A) =
\setb{a \in A}{\forall b \in A \colon 1 - ba \in A^\times}.
\]
\end{lema}

\begin{proof}
Let $a \in \mathcal{J}(A)$ and $b \in A$. Note that
$1 - ab \not \in M$ for any maximal ideal $M$, since $ab \in M$.
As $1 - ab$ is not contained in any maximal ideal, it follows that
$(1-ab) = A$, hence $1-ab$ is invertible.

Suppose now that $1 - ab \in A^\times$ for all $b \in A$. Let $M$
be a maximal ideal and suppose $a \not \in M$. Then $(M, a) = A$.
In particular, we can write $1 = m + xa$ with $m \in M$ and
$x \in A$. Rearranging, $m = 1 - xa$, which is a contradiction, as
$1-xa$ is invertible.
\end{proof}

\begin{lema}
The following statements hold:

\begin{enumerate}[i)]
\item Let $I \edn A$ and $P_1, \dots, P_n \in \Spec(A)$. If
$I \subseteq P_1 \cup \dots \cup P_n$, there exists some $k$ such
that $I \subseteq P_k$.
\item Let $I_1, \dots, I_n \edn A$ and $P \in \Spec(A)$. If
$I_1 \cap \dots \cap I_n \subseteq P$, then there exists some $k$
such that $I_k \subseteq P$.
\end{enumerate}
\end{lema}

\begin{proof}
\phantom{i}
\begin{enumerate}[i)]
\item We induct on $n$, noting that the statement trivially holds
for $n=1$.

Suppose the statement doesn't hold for $n$. By the induction
hypothesis we can find
\[
a_i \in I \bigsetminus \bigcup_{j \ne i} P_j
\]
for any $i$. Then $a_i \in P_i$. Consider the element
\[
a = \sum_{i=1}^n \prod_{j \ne i} a_j.
\]
Note that all but one of the above terms are an element of $P_i$.
But then $a$ is not an element of any $P_i$, which is a
contradiction.
\item Suppose the contrary and let $a_j \in I_j \setminus P$ for
all $j$. But then
\[
\prod_{j=1}^n a_j \subseteq
\prod_{j=1}^n I_j \subseteq
\bigcap_{j=1}^n I_j \subseteq
P,
\]
which is a contradiction. \qedhere
\end{enumerate}
\end{proof}

\begin{opomba}
The first statement is called
\emph{prime avoidance}\index{prime avoidance}.
\end{opomba}

\begin{trditev}
Let $f \colon A \to B$ be a ring homomorphism. If
$I \edn B$, then $f^{-1}(I) \edn A$. Furthermore, if
$P \in \Spec(B)$, then $f^{-1}(P) \in \Spec(A)$.
\end{trditev}

\begin{trditev}[Universal property]
\index{universal property!ring quotients}
Let $I \edn A$ and $\pi \colon A \to \kvoc{A}{I}$ be the canonical
epimorphism. For every ring homomorphism $f \colon B$ with
$I \subseteq \ker(f)$, there exists a unique ring homomorphism
$\hat{f} \colon \kvoc{A}{I} \to B$ such that
$f = \hat{f} \circ \pi$.
\[
\begin{tikzcd}[row sep=large, column sep=large]
A \arrow[r, "\pi"] \arrow[rd, "f"'] &
\kvoc{A}{I} \arrow[d, dashed, "\hat{f}"] \\ &
B
\end{tikzcd}
\]
\end{trditev}

\begin{posledica}
If $f \colon A \to B$ is a ring homomorphism, then
$\kvoc{A}{\ker f} \cong f(A)$.
\end{posledica}

\begin{izrek}[Isomorphism theorems]
\index{isomorphism theorems!rings}
The following statements hold:

\begin{enumerate}[i)]
\item Let $I \edn A$. There is a bijective correspondence
\[
\setb{J \edn A}{I \subseteq J} \leftrightarrow
\set{\oline{J} \edn \kvoc{A}{I}},
\]
given by $J \mapsto \kvoc{J}{I}$ and
$\oline{J} \mapsto \pi^{-1} \br{\oline{J}}$.
\item If $I, J \edn A$ with $I \subseteq J$, then
\[
\kvoc{A}{J} \cong \kvoc{\kvoc{A}{I}}{\kvoc{J}{I}}.
\]
\item Let $B \subseteq A$ be a subring and $I \edn A$. Then
$I \cap B \edn B$ and
\[
\kvoc{B+I}{I} \cong \kvoc{B}{B \cap I}.
\]
\end{enumerate}
\end{izrek}

\datum{2025-2-20}

\begin{izrek}[Chinese remainder theorem]
If $I_1, \dots, I_n \edn A$ are pairwise comaximal, then
\[
\kvoc{A}{I_1 \cap \dots \cap I_n} \cong
\prod_{k=1}^n \kvoc{A}{I_k}.
\]
\end{izrek}

\newpage

\subsection{Modules}

\begin{definicija}
Let $M$ be an $A$-module and $E \subseteq M$. The
\emph{$A$-module generated by $E$} is denoted by
\[
\skl{E}_A = \setb{\sum_{k=1}^n a_k m_k}{a_k \in A \land m_k \in E}.
\]
\end{definicija}

\begin{trditev}
Let $M$ be an $A$-module and $I \edn A$. Then
$\kvoc{M}{IM}$ is an $\kvoc{A}{I}$-module via the natural product.
\end{trditev}

\begin{opomba}
Categorically, $\kvoc{A}{I}$-modules are equivalent to $A$-modules
$M$ with $IM = 0$.
\end{opomba}

\begin{izrek}[Universal property]
\index{universal property!module quotients}
Let $N \leq M$ be $A$-modules and $\pi \colon M \to \kvoc{M}{N}$ be
the canonical epimorphism. If $f \colon M \to X$ is an $A$-module
homomorphism with $N \subseteq \ker f$, then there exists a unique
homomorphism $\hat{f} \colon \kvoc{M}{N} \to X$ such that
$f = \hat{f} \circ \pi$.
\[
\begin{tikzcd}[row sep=large, column sep=large]
M \arrow[r, "\pi"] \arrow[rd, "f"'] &
\kvoc{M}{N} \arrow[d, dashed, "\hat{f}"] \\ &
X
\end{tikzcd}
\]
\end{izrek}

\begin{izrek}[Isomorphism theorems]
\index{isomorphism theorems!modules}
The following statements hold:

\begin{enumerate}[i)]
\item We have $f(M) \cong \kvoc{M}{\ker(f)}$.
\item If $N \leq M$, then submodules $N \leq X \leq M$ are in
bijective correspondence with submodules of $\kvoc{M}{N}$.
\item If $N \leq X \leq M$, then
\[
\kvoc{M}{X} \cong \kvoc{\kvoc{M}{N}}{\kvoc{X}{N}}.
\]
\item If $N, N' \leq M$, then
\[
\kvoc{N+N'}{N} \cong \kvoc{N'}{N \cap N'}.
\]
\end{enumerate}
\end{izrek}

\begin{izrek}[Universal property]
\index{universal property!product and coproduct}
If $\br{f_i \colon M_i \to X}_{i \in I}$ is a family of $A$-module
homomorphisms, then there exists a unique homomorphism
\[
\hat{f} \colon \bigoplus_{i \in I} M_i \to X
\]
such that $f_i = \hat{f} \circ \varepsilon_i$ for all $i \in I$.

If $\br{g_i \colon X \to M_i}_{i \in I}$ is a family of $A$-module
homomorphisms, then there exists a unique homomorphism
\[
\hat{g} \colon X \to \prod_{i \in I} M_i
\]
such that $g_i = \pi_i \circ \hat{g}$ for all $i \in I$.
\end{izrek}

\datum{2025-2-24}

\begin{opomba}
Note that
\[
\bigoplus_{i \in I} M_i \subseteq \prod_{i \in I} M_i.
\]
If $I$ is finite, then $A$-modules form an abelian category.
\end{opomba}

\begin{definicija}
An $A$-module $M$ is \emph{free}\index{free module} if
\[
M \cong \bigoplus_{i \in I} A
\]
for some set $I$. A \emph{basis}\index{basis} of $M$ is a family
$(m_i)_{i \in I}$ such that the map
\[
\bigoplus_{i \in I} A \to M, \quad
(a_i)_{i \in I} \mapsto \sum_{i \in I} a_i m_i
\]
is an isomorphism.
\end{definicija}

\begin{lema}
The following statements hold:

\begin{enumerate}[i)]
\item Every module is a quotient of a free module.
\item A module $M$ is finitely generated if and only if there
exists an epimorphism $\varphi \colon A^k \to M$ for some integer
$k$.
\item A module $M$ is finitely generated and free if and only if
$M \cong A^k$ for some integer $k$.
\end{enumerate}
\end{lema}

\begin{lema}[Nakayama]
\index{Nakayama's lemma}
Let $M$ be a finitely generated module over $A$.

\begin{enumerate}[i)]
\item If $J(A)M = M$, then $M = (0)$.
\item If $N \leq M$ such that $M = N + J(A) M$, then $N = M$.
\end{enumerate}
\end{lema}

\begin{proof}
\phantom{i}
\begin{enumerate}[i)]
\item Assume $M \ne 0$ and let $m_1, \dots, m_r \in M$ be a minimal
generating set. Note that, as $M \ne 0$, $r \geq 1$. By our
assumptions, we can write
\[
m_r = \sum_{i=1}^r a_i m_i,
\]
where $a_i \in J(A)$. But as $1-a_r$ is invertible, we can express
$m_r$ as a linear combination of the other elements, which is a
contradiction.
\item Note that
\[
J(A) \kvoc{M}{N} = \kvoc{J(A) M + N}{N} = \kvoc{M}{N},
\]
hence $\kvoc{M}{N} = 0$. \qedhere
\end{enumerate}
\end{proof}

\begin{definicija}
A ring $A$ is \emph{local}\index{local ring} if $A \ne 0$ and $A$
has a unique maximal ideal. We denote it by $(A, \mathfrak{m})$,
where $\mathfrak{m}$ is the maximal ideal.
\end{definicija}

\begin{opomba}
If $(A, \mathfrak{m})$ is local, then $\kvoc{A}{\mathfrak{m}}$ is
a field and $\mathfrak{m} = J(A)$.
\end{opomba}

\begin{posledica}
Let $(A, \mathfrak{m})$ be a local ring and $M$ a finitely
generated module. If
$x_1 + \mathfrak{m} M, \dots, x_r + \mathfrak{m} M$ is a basis of
the $\kvoc{A}{\mathfrak{m}}$-vector space
$\kvoc{M}{\mathfrak{m} M}$, then $x_1, \dots, x_r \in M$ generate
$M$.
\end{posledica}

\begin{trditev}
If
$\begin{tikzcd}[column sep=small]
0 \arrow[r] &
M \arrow[r, "f"] &
N \arrow[r, "g"] &
P \arrow[r] &
0
\end{tikzcd}$
is a short exact sequence, then the diagram
\[
\begin{tikzcd}[row sep=large, column sep=large]
0 \arrow[r] &
M \arrow[r, "f"] \arrow[d, "f"', "\cong"] &
N \arrow[r, "g"] \arrow[d, "\id"'] &
P \arrow[r] \arrow[d, "\br{\hat{g}}^{-1}"', "\cong"] &
0 \\
0 \arrow[r] &
\im(f) \arrow[r, hook] &
N \arrow[r, "\pi", two heads] &
\kvoc{N}{K} \arrow[r] &
0
\end{tikzcd}
\]
commutes and has exact rows, where $\hat{g}(n+K) = g(n)$.
\end{trditev}

\begin{lema}
\label{rings:lm:hom_ex}
The following statements hold:

\begin{enumerate}[i)]
\item A sequence
$\begin{tikzcd}[column sep=small]
0 \arrow[r] &
N \arrow[r, "f"] &
M \arrow[r, "g"] &
P
\end{tikzcd}$
is exact if and only if the sequence
\[
\begin{tikzcd}
0 \arrow[r] &
\Hom(X, N) \arrow[r, "f_*"] &
\Hom(X, M) \arrow[r, "g_*"] &
\Hom(X, P)
\end{tikzcd}
\]
is exact for every $A$-module $X$.
\item A sequence
$\begin{tikzcd}[column sep=small]
N \arrow[r, "f"] &
M \arrow[r, "g"] &
P \arrow[r] &
0
\end{tikzcd}$
is exact if and only if the sequence
\[
\begin{tikzcd}
0 \arrow[r] &
\Hom(P, X) \arrow[r, "f^*"] &
\Hom(M, X) \arrow[r, "g^*"] &
\Hom(N, X)
\end{tikzcd}
\]
is exact for every $A$-module $X$.
\end{enumerate}
\end{lema}

%TODO proof lmao

\begin{definicija}
Let $M_1, \dots, M_n$ and $P$ be $A$-modules. A map
$f \colon M_1 \times \dots \times M_n \to P$ is
\emph{$A$-multilinear}\index{multilinear map} if it is linear in
every component.
\end{definicija}

\begin{definicija}
Let $M_1, \dots, M_n$ be $A$-modules. The
\emph{tensor product}\index{tensor product}
$M_1 \otimes \dots \otimes M_n$ is the $A$-module together with
the multilinear map
\[
\otimes \colon \prod_{i=1}^n M_i \to \bigotimes_{i=1}^n M_i
\]
defined by the following universal property:
\index{universal property!tensor product}
For every $A$-module
$P$ and every multilinear map
$f \colon M_1 \times \dots \times M_n \to P$ there exists a unique
$A$-module homomorphism
\[
\hat{f} \colon \bigotimes_{i=1}^n M_i \to P
\]
such that $\hat{f} \circ \otimes = f$.
\[
\begin{tikzcd}[row sep=large, column sep=large]
\displaystyle
\prod_{i=1}^n M_i \arrow[r, "\otimes"] \arrow[dr, "f"'] &
\displaystyle
\bigotimes_{i=1}^n M_i \arrow[d, dashed, "\hat{f}"] \\ &
P
\end{tikzcd}
\]
\end{definicija}

\begin{opomba}
The tensor product is associative and commutative. It is functorial
in each component.
\end{opomba}

\begin{izrek}[$\Hom$-$\otimes$ adjunction]
\index{Hom-$\otimes$ adjunction@$\Hom$-$\otimes$ adjunction}
Let $M$ be an $A$-module. Then $\cdot \otimes M$ is left-adjoint to
$\Hom(M, \cdot)$. That is, for any $A$-modules $M$, $N$ and $P$,
there are $A$-isomorphisms
$\Hom(N \otimes M, P) \to \Hom(N, \Hom(M, P))$, given by
$f \mapsto \br{n \mapsto \br{m \mapsto f(n \otimes m)}}$ with
inverse $g \mapsto \br{n \otimes m \mapsto g(n)(m)}$. These
isomorphisms are natural transformations in $N$, $M$ and $P$.
\end{izrek}

%TODO proof lmao

\begin{posledica}
The tensor product $M \otimes \cdot$ is right-exact. That is, for
every exact sequence
$\begin{tikzcd}[column sep=small]
N \arrow[r, "f"] &
P \arrow[r, "g"] &
Q \arrow[r] &
0
\end{tikzcd}$
the sequence
\[
\begin{tikzcd}
M \otimes N \arrow[r, "\id \otimes f"] &
M \otimes P \arrow[r, "\id \otimes g"] &
M \otimes Q \arrow[r] &
0
\end{tikzcd}
\]
is also exact.
\end{posledica}

\begin{proof}
Applying lemma~\ref{rings:lm:hom_ex}, we see that
\[
\begin{tikzcd}
0 \arrow[r] &
\Hom(Q, X) \arrow[r, "f^*"] &
\Hom(P, X) \arrow[r, "g^*"] &
\Hom(N, X)
\end{tikzcd}
\]
is exact. Applying lemma~\ref{rings:lm:hom_ex} again, we see that
\[
\begin{tikzcd}[column sep=scriptsize]
0 \arrow[r] &
\Hom(M, \Hom(Q, X)) \arrow[r, "(f^*)_*"] &
\Hom(M, \Hom(P, X)) \arrow[r, "(g^*)_*"] &
\Hom(M, \Hom(N, X))
\end{tikzcd}
\]
is exact as well. Applying the previous theorem and
lemma~\ref{rings:lm:hom_ex} again, we get the required sequence.
\end{proof}

\begin{posledica}
For any family $(N_i)_{i \in I}$ of $A$-modules we have
\[
M \otimes \br{\bigoplus_{i \in I} N_i} \subseteq
\bigoplus_{i \in I} \br{M \otimes N_i}.
\]
\end{posledica}

\begin{proof}
We can construct the isomorphisms using the universal properties.
\end{proof}

\datum{2025-2-27}

\begin{trditev}
Let $M$ be an $A$-module and $I \edn A$. Then
$\kvoc{M}{MI} \cong M \otimes_A \kvoc{A}{I}$.
\end{trditev}

\begin{proof}
As
$\begin{tikzcd}[column sep=small]
0 \arrow[r] &
I \arrow[r, hook] &
A \arrow[r] &
\kvoc{A}{I} \arrow[r] &
0
\end{tikzcd}$
is a short exact sequence, the sequence
\[
\begin{tikzcd}
I \otimes M \arrow[r] &
A \otimes M \arrow[r] &
\kvoc{A}{I} \otimes M \arrow[r] &
0.
\end{tikzcd}
\]
But as $A \otimes M \cong M$ under $\mu(a \otimes m) = am$ and
$\eval{\mu}{I \otimes M}{} = IM$, we get
\[
\kvoc{M}{IM} \cong
\kvoc{A \otimes M}{I \otimes M} \cong
\kvoc{A}{I} \otimes M. \qedhere
\]
\end{proof}

\begin{trditev}
If $A \ne 0$ and $A^{(I)} \cong A^{(J)}$, then
$\abs{I} = \abs{J}$.\footnote{
$\displaystyle A^{(I)} = \bigoplus_{i \in I} A$.}
\end{trditev}

\begin{proof}
Let $M$ be a maximal ideal in $A$, then $K = \kvoc{A}{M}$ is a
field. Then
\[
A^{(I)} \otimes \kvoc{A}{M} \cong
\br{A \otimes \kvoc{A}{M}}^{(I)} \cong
\br{\kvoc{A}{M}}^{(I)} \cong
K^{(I)}
\]
as $A$-modules and $\kvoc{A}{M}$-modules. Hence
$K^{(I)} \cong K^{(J)}$, therefore $\abs{I} = \abs{J}$.
\end{proof}

\begin{definicija}
If $M$ is a finitely generated free module, its
\emph{rank}\index{rank} is the unique $n \in \N$ such that
$M \cong A^n$.
\end{definicija}

\begin{definicija}
A module $M$ is

\begin{itemize}
\item \emph{projective}\index{projective module} if
$\Hom(M, \cdot)$ is exact.
\item \emph{injective}\index{injective module} if
$\Hom(\cdot, M)$ is exact.
\item \emph{flat}\index{flat module} if $M \otimes \cdot$ is exact.
\end{itemize}
\end{definicija}

\begin{izrek}
The following statements are equivalent for an $A$-module $P$:

\begin{enumerate}[i)]
\item The module $P$ is projective.
\item For every epimorphism $g \colon M \to N$, the map
$g_* \colon \Hom(P, M) \to \Hom(P, N)$ is an epimorphism.
\item For every epimorphism $f \colon M \to N$ and homomoprhism
$\varphi \colon P \to N$ there exists a homomorphism
$\psi \colon P \to M$ with $f \circ \psi = \varphi$.
\[
\begin{tikzcd}[row sep=large, column sep=large]
&
P \arrow[dl, dash, "\psi"'] \arrow[d, "\varphi"] \\
M \arrow[r, two heads, "f"'] &
N
\end{tikzcd}
\]
\item Every epimorphism $g \colon M \to P$ splits.
\item There exists an $A$-module $M$ such that $P \oplus M$ is
free.
\end{enumerate}
\end{izrek}

\begin{proof}
The first statement implies the second by definition.

Now assume that the second statement holds. The map
$f_* \colon \Hom(P, M) \to \Hom(P, N)$ is therefore an epimorphism.
By definition, we can construct a homomorphism
$\psi \colon P \to M$ that maps to $f$.

Assume now that the diagram condition holds. Then there exists
a homomorphism $s \colon P \to M$ such that $g \circ s = \id$.

Suppose every epimorphism $g \colon M \to P$ splits. In particular,
this holds for an epimorphism $g \colon A^{(I)} \to P$. But then
\[
\begin{tikzcd}
0 \arrow[r] &
\ker g \arrow[r] &
A^{(I)} \arrow[r, "g"] &
P \arrow[r] &
0
\end{tikzcd}
\]
is a short exact sequence. As it splits,
$P \oplus \ker g \cong A^{(I)}$.

Finally, suppose that $P \oplus C \cong A^{(I)}$ is a free module.
Take a short exact sequence
\[
\begin{tikzcd}
0 \arrow[r] &
M \arrow[r, "f"] &
N \arrow[r, "g"] &
Q \arrow[r] &
0.
\end{tikzcd}
\]
As $\Hom(P, \cdot)$ is left-exact, we only need to check that
$g_* \colon \Hom(P, N) \to \Hom(P, Q)$ is surjective. Let
$\varphi \in \Hom(P, Q)$. Let $\pi \colon A^{(I)} \to P$ be the
canonical projection and
$\varepsilon \colon P \hookrightarrow A^{(I)}$ the embedding. Then
$\pi \circ \varepsilon = \id_P$.

For each basis vector $e_i \in A^{(I)}$, choose $n_i \in N$ such
that $g(n_i) = \varphi \circ \pi(e_i)$, which is possible by
surjectivity of $g$.
\[
\begin{tikzcd}[row sep=large, column sep=large]
& A^{(I)} \arrow[d, "\pi"] \arrow[ddl, dashed, "\Psi"'] \\
& P \arrow[u, "\varepsilon"] \arrow[d, "\varphi"]
\arrow[dl, dashed, "\psi"] \\
N \arrow[r, "g"', two heads] &
Q \arrow[r] &
0
\end{tikzcd}
\]
Construct a homomorphism $\Psi \colon \Hom(A^{(I)}, N)$ by
$\Psi(e_i) = n_i$. Then $g \circ \Psi = \varphi \circ \pi$. Now
define $\psi = \Psi \circ \varepsilon$. Then, for every $p \in P$,
we have
\[
g \circ \psi(p) =
g \circ \Psi \circ \varepsilon(p) =
\varphi \circ \pi \circ \varepsilon(p) =
\varphi(p),
\]
hence $g_*(\psi) = \varphi$.
\end{proof}
